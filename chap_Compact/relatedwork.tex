\subsection{Comparison with OPRF-based Applications}
PrivacyPass and TrustToken \cite{privacypass,trusttoken} adopted the oblivious pseudo-random function (OPRF) protocol of HashDH \cite{oprf-proved,voprf-proved} to generate tokens, with which a user anonymously accesses resources.
A user generates a random number $e$ to blind an unsigned token $T$ into $[e]T$. After authenticating the user, a token server signs $[e]T$ using its private key $k$ and returns $[k e]T$ to the user, who uses $e$ to convert it into ($T, [k]T$),
        which is redeemed when the user accesses some protected resource
 (see \cite{privacypass,trusttoken,oprf-proved} for details).
In this procedure, the token server obliviously calculates a pseudo-random output as anonymous token,
    because ($T, [k]T$) and ($[e]T, [ke]T$) cannot be linked.

% RPs do not identify each user if we use such anonymous tokens in SSO services, because these tokens are indistinguishable.

\usso\ employs a similar cryptographic technique.\footnote{We designed and implemented \usso\ in 2020, and in 2023 someone reminded us that the identity transformations are similar to the HashDH OPRF.} 
 A user selects $t$ to transform $ID_{RP}$ to $PID_{RP} = [t]ID_{RP}$.
  % protecting $ID_{RP}$ from the IdP.
Then, the IdP uses the user's permanent identity $u$ to calculate $PID_U = [u]PID_{RP} = [ut]ID_{RP}$,
which is similar to token signing by the token server in PrivacyPass/TrustToken.
Hence, \emph{IdP untraceability} in \usso, i.e., the IdP cannot link $ID_{RP}$ and $[t]ID_{RP}$,
 roughly corresponds to {\em unlinkability of token signing-redemption} in PrivacyPass/TrustToken, i.e., the token server cannot link $T$ and $[e]T$.

\usso\ leverages this cryptographic technique in very different ways.
First, 
it works among three parties, unlike the two-party PrivacyPass/TrustToken protocols.
\usso\ shares the user-selected random number $t$ with the target RP,
enabling it to independently derive the user's permanent account, i.e., $Acct = [t^{-1}]PID_{U}$. This account is ``obliviously'' determined by the IdP when it calculates the user's pseudo-identity.
% based on the ``blinded'' input (i.e., the visited RP's pseudo-identity).
Second, $u$ and $ID_{RP}$, which correspond to $k$ and $T$ in PrivacyPass/TrustToken, are assigned as the permanent identities of the user and the RP, respectively, to establish the relationship among identities and accounts.
This is different from existing OPRF-based systems  \cite{privacypass, trusttoken, strong-oprf, oprf-bitcoin-wallet, pesto, oprf-ot-si, pp-ss, Private-Contact-Discovery, o-kms, oprf-deduplication} that always use $k$ as a server's secret key.
%The integration of SSO (pseudo-)identities and OPRF variables requires a deep understanding of both SSO and OPRF protocols.
Finally, \usso\ leverages randomness of HashDH OPRFs to provide \emph{RP unlinkability}, and this property is not utilized in PrivacyPass/TrustToken.
It ensures colluding RPs cannot link any logins across RPs,
even by sharing their knowledge about the pseudo-identities and permanent accounts in \usso.
This property offers desirable indistinguishability of \emph{different users} visiting colluding RPs.
It roughly corresponds to the indistinguishability of \emph{different private keys} for signing anonymous tokens, not considered in PrivacyPass/TrustToken. 

Although \usso\ mathematically employs the same technique in its identity transformations as the HashDH OPRF \cite{oprf-proved,voprf-proved}, we need more properties of these algorithms to ensure security and privacy. % in Theorems 4, 6, and 7.
\usso\ essentially 
depends on the \emph{deterministicness} property of \emph{pseudo-random} functions to derive a permanent account for any $t$, \emph{obliviousness} to ensure IdP untraceability, 
and \emph{pseudo-randomness} to provide RP unlinkability. 
In contrast, PrivacyPass and TrustToken \cite{privacypass,trusttoken} depend on only the properties of deterministicness and obliviousness.
Moreover, \usso\ needs \emph{more} properties for secure SSO services (i.e., RP designation and user identification; see Theorems \ref{thm-rp-des} and \ref{thm-u-id}
 in Section \ref{analysis-security}),
    not discussed in OPRFs \cite{sok-oprf,oprf-proved,voprf-proved}.
% in the case of user identity leakage, \emph{collision-freeness} of $PID_{RP}$ (see Appendix \ref{proof-rp-collision}).
So an OPRF protocol is not always ready to work as the identity transformations in \usso.
% unless no collision exists in the blinded inputs of the pseudo-random function \cite{oprf-proved,voprf-proved}.

