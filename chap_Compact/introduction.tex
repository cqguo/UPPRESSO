\section{Introduction}
\label{sec:intro}
Single sign-on (SSO) protocols such as OpenID Connect (OIDC) \cite{OpenIDConnect}, OAuth 2.0 \cite{rfc6749}, and SAML \cite{SAML, SAMLIdentifier}, are widely deployed for identity management and authentication.
SSO allows a user to log into a website,
 known as the \emph{relying party} (RP), using her account registered at a trusted web service, known as the \emph{identity provider} (IdP).
An RP delegates user identification and authentication to the IdP, which issues an \emph{identity token} (such as ``id token'' in OIDC and ``identity assertion'' in SAML) for a user to visit the RP.
For instance, an OIDC user requests to log into an RP,
which constructs an identity-token request with its identity (denoted as $ID_{RP}$) and posts (or redirects) it to a trusted IdP. After authenticating the user, the IdP issues an identity token binding the identities of the user and the target RP (i.e., $ID_U$ and $ID_{RP}$), which is returned to the user and then forwarded to the RP.
Finally, the RP verifies the identity token to determine if the token holder is allowed to log in. Thus, a user maintains only one credential for the IdP, instead of multiple credentials for different RPs.

OIDC also provides comprehensive services of identity management,
 by enabling an IdP to enclose more user attributes in identity tokens, along with the authenticated user's identity.
The attributes (e.g., age and hobby) are maintained at the IdP and enclosed in identity tokens after a user's authorization \cite{OpenIDConnect,rfc6749}.

The wide adoption of SSO raises concerns on user privacy, because it facilitates the tracking of a user's login activities by interested parties \cite{NIST2017draft, SPRESSO, BrowserID, maler2008venn}.
To issue identity tokens, an IdP should know the target RP to be visited by a user and the login time.
So a curious IdP could potentially track a user's all login activities over time
 \cite{BrowserID, SPRESSO},
called {\em IdP-based login tracing} in this paper.
Another privacy risk arises from the fact that RPs learn the user's identity from the identity tokens they receive.
If an identical user identity is enclosed in tokens for a user to visit different RPs, colluding RPs could link the logins across these RPs %and track the user's activities
to learn the user's online profile \cite{maler2008venn, FirefoxAccount}.
This risk is called {\em RP-based identity linkage}. %in this paper.

While preventing different privacy threats (i.e., IdP-based login tracing, RP-based identity linkage, or both),
privacy-preserving SSO schemes \cite{maler2008venn, NIST2017draft, BrowserID, SPRESSO,miso,POIDC} aim to implement authentication and identity management with the following features:
(\emph{a}) \emph{User authentication only to the IdP},
which eliminates the need for authentication between a user and any RP,
 and then requires a user to maintain only the credential for the IdP,
(\emph{b}) \emph{User identification at each RP}, which enables an RP to recognize each user by an account unique within the RP to provide customized services across multiple logins,
and (\emph{c}) \emph{IdP-confirmed attribute provision},
 where a user's attributes are maintained at a trusted IdP and provided to RPs after the user's authorization.
%Meanwhile, these privacy threats posed by different adversaries are considered, including \emph{an honest-but-curious IdP} and \emph{malicious RPs colluding with users}.
In Section \ref{sec:background}, we analyze existing privacy-preserving solutions for SSO and also identity federation.


We present \usso, an Untraceable and Unlinkable Privacy-PREserving Single Sign-On protocol.
It proposes {\em identity transformations} and integrates them in popular OIDC services.
In \usso, an RP and a user transform $ID_{RP}$ into ephemeral $PID_{RP}$, which is sent to a trusted IdP to transform $ID_U$ into an ephemeral user pseudo-identity $PID_U$.
The identity token issued by the IdP binds only $PID_U$ and $PID_{RP}$, instead of $ID_U$ and $ID_{RP}$. On receiving the token, % with a matching $PID_{RP}$,
 the RP transforms $PID_U$ into an account unique at each RP but identical across multiple logins to this RP.


\usso\ prevents both IdP-based login tracing and RP-based identity linkage, while existing privacy-preserving SSO solutions address only one of them \cite{BrowserID, SPRESSO, NIST2017draft, save-flow,POIDC} or need another fully-trusted server in addition to the IdP \cite{miso}.
Meanwhile, \usso\ implements SSO services
% works compatibly with widely-used SSO services \cite{OpenIDConnect, rfc6749, NIST2017draft}
 accessed from commercial-off-the-shelf (COTS) browsers without any plug-ins or extensions,
 and provides all desirable features of SSO protocols as listed above.
In contrast, privacy-preserving identity federation \cite{PseudoID, ELPASSO, UnlimitID, Opaak, uprov, hyperledge-idemix} supports only some but not all of these features,
    and always requires a browser plug-in or extension.

Our contributions are as follows.
\vspace{-\topsep}
\begin{itemize}
\setlength{\topsep}{0pt}
\setlength{\partopsep}{0pt}
\setlength{\itemsep}{0pt}
\setlength{\parsep}{0pt}
\setlength{\parskip}{0pt}
\item We proposed a novel identity-transformation approach for privacy-preserving SSO and designed identity-transformation algorithms with desirable properties.
\item We developed the \usso\ protocol based on the identity transformations with several designs specific to web applications, and proved that it satisfies the security and privacy requirements of SSO services.
\item We implemented a prototype of \usso\ on top of an open-source OIDC implementation. Through performance evaluations, we confirmed that \usso\ introduces reasonable overheads.
\end{itemize}


We present the background and related works in Section \ref{sec:background} and the identity-transformation approach in Section \ref{sec:challenge}, followed by the detailed designs in Section \ref{sec:UPPRESSO}.
Security and privacy are analyzed in Section \ref{sec:analysis}.
We explain the prototype implementation and evaluations in Section \ref{sec:implementation}, and discuss extended issues in Section \ref{sec:discussion}. Section \ref{sec:conclusion} concludes this work.
