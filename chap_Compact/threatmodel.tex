\section{The Designs of \usso}
\label{sec:UPPRESSO}

This section presents the threat model and assumptions, the identity-transformation algorithms, and finally the \usso\ protocol with designs specific for web applications.

\subsection{The Threat Model}
\label{subsec:threatmodel}
The system consists of an honest-but-curious IdP as well as several honest or malicious RPs and users. % or even collude with each other.
This threat model is in line with widely-used SSO services \cite{OpenIDConnect,rfc6749, SAML, SAMLIdentifier}.

%\vspace{0.5mm}
\noindent \textbf{Honest-but-curious IdP.} The IdP strictly follows the protocols,
 while remaining interested in learning about user activities.
For example, it could store received messages to infer the relationship between $ID_U$, $ID_{RP}$, $PID_{U}$, and $PID_{RP}$.
 % to track a user's login activities.
It never actively violates the protocols, so a script downloaded from the IdP also strictly follows the protocols (see Section \ref{sec:web-design} for specific designs for web applications).
The IdP maintains the private key well for signing identity tokens and RP certificates, %(see Section \ref{implementations} for details)
which prevents adversaries from forging such messages.

%\vspace{0.5mm}
\noindent \textbf{Malicious users.} Adversaries could control a set of users by stealing their credentials or registering Sybil users in the system.
 Their objective \cite{SPRESSO, FettKS14} is to (\emph{a}) impersonate a victim user at honest RPs or (\emph{b}) entice an honest user to log into an honest RP under another user's account.
A malicious user could modify, insert, drop, or replay messages or even behave arbitrarily in login flows.

\noindent \textbf{Malicious RPs.}
Adversaries could control a set of RPs by registering at the IdP as an RP or exploiting vulnerabilities to compromise some RPs.
Malicious RPs could behave arbitrarily, attempting to compromise the security or privacy guarantees of \usso.
For example, they could manipulate $PID_{RP}$ in a login, attempting to (\emph{a}) entice honest users to return an identity token that could be accepted by some honest RP or (\emph{b}) manipulate the generation of $PID_U$ to analyze the relationship between $ID_U$ and $PID_U$.

%\vspace{0.5mm}
\noindent \textbf{Colluding users and RPs.}
Malicious users and RPs could collude,
 attempting to break the security or privacy guarantees for honest users and RPs.
For example, a malicious RP could collude with malicious users to %steal another user's identity token and
 impersonate a victim at honest RPs or link an honest user's logins visiting colluding RPs.

We do \emph{not} consider the collusion of the IdP and RPs.
If the IdP were to collude with some RPs, a user would complete login flows \emph{entirely} with malicious entities. 
In principle, it would require a long-term user secret to protect (or transform) permanent accounts across these RPs.
This secret should be held \emph{only} by the user; otherwise, the colluding adversaries could always link these accounts.
However, the accounts derived from this user secret would need to be updated if the secret was lost or leaked (see discussions about IdP-RP collusive attacks in Section \ref{sec:discussion}).

Finally, the security and privacy properties of \usso\ are guaranteed against different combinations of the above adversaries.
In particular, it implements secure SSO services against colluding users and RPs,
        while preventing (\emph{a}) IdP-based login tracing by an honest-but-curious IdP
            and (\emph{b}) RP-based identity linkage by colluding RPs and users.

\subsection{Assumptions}
We assume secure communications between honest entities (e.g., HTTPS), and the cryptographic primitives are secure. The software stack of an honest entity is correctly implemented to transmit messages to receivers as expected.

\usso\ is designed for users who value privacy. We assume a user never authorizes the IdP to enclose any \emph{distinctive} attributes, such as telephone number, Email address, etc., in identity tokens or sets such attributes at any RP. Hence, privacy leakages due to re-identification by distinctive attributes across RPs are out of our scope.
In addition, we do not consider the tracking of a user's activities by network traffic analysis or crafted web pages, as they can be prevented by existing defenses.
For example, FedCM \cite{FedCM} attempts to disable iframe and third-party cookies in SSO, which could be exploited to track users. Our work focuses only on the privacy threats introduced by the flows of SSO protocols.


%When a user visits multiple RPs concurrently from one browser, a malicious RP may actively redirect his account to another RP server using crafted web pages. Meanwhile, traffic analysis is possible in SSO scenarios, where a user's activities can be tracked from network packets and active account linkage through malicious web pages. We do not consider these attacks in this work, as they can be prevented by existing defenses.

