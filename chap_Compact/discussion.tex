\section{Discussions}
\label{sec:discussion}

%\vspace{0.75mm}
%\noindent \textbf{Applicability of identity transformations.}
%The proposed identity-transformation algorithms %i.e., $\mathcal{F}_{PID_{RP}}()$, $\mathcal{F}_{PID_U}()$, and $\mathcal{F}_{Acct}()$,
%can be applied to a wide range of SSO scenarios, including web applications, mobile Apps, and native software.
%These algorithms follow the common model of popular SSO protocols and do not depend on any specific implementations.

%\vspace{0.75mm}
%\noindent \textbf{Scalability.}
%$ID_{RP}$ is generated uniquely during the initial registration of an RP with a capacity of $n$, which is the order of $G$. For the NIST P256 elliptic curve, $n$ is approximately $2^{256}$.
%$PID_{RP}$ is ensured to be unique in unexpired tokens.
%The probability of at least two identical $PID_{RP}$s among $\sigma$ unexpired tokens is $1-\prod_{i=0}^{\sigma-1}(1-i/n)$.
%If the system serves $10^{8}$ requests per second and each token has a validity period of 10 minutes, $\sigma$ is less than $2^{36}$. So the $PID_{RP}$-collision probability is negligible, i.e., less than $2^{-183}$ for the NIST P256 curve.
%
%The maximal amount of accounts at any RP is the same as the capacity of user identities at the IdP.
%At any RP a unique account is automatically assigned to each user because $Acct =  [ID_U]ID_{RP} = [u]ID_{RP}$.
%Finally, stronger elliptic curves accommodate more RPs and users, e.g., $n$ is about $2^{384}$ for the NIST P384 curve.


%\vspace{0.75mm}
\noindent \textbf{Alternatives for generating $ID_{RP}$ and binding $Enpt_{RP}$.}
In \usso\ the IdP generates random $ID_{RP}$ and uses an RP certificate to bind $ID_{RP}$ and $Enpt_{RP}$, which is verified by the IdP script. This ensures that the target RP has already registered itself at the IdP and prevents unauthorized RPs from accessing the IdP's services \cite{save-flow}.

An alternative method for binding $ID_{RP}$ and $Enpt_{RP}$ is
 to design a \emph{deterministic} scheme to calculate unique $ID_{RP}$ based on the RP's unambiguous name such as its domain.
This can be achieved by encoding the domain with a hashing-to-elliptic-curves function \cite{irtf-cfrg-hash-to-curve-16}, which provides collision resistance but not revealing the discrete logarithm of the output. It generates a point on the elliptic curve $\mathbb{E}$ as $ID_{RP}$ to ensure the \emph{uniqueness} of $ID_{RP} = [r]G$ while keeping $r$ unknown. %For example, using a hash function $Hs()$ to encode an RP's domain or the RP script's origin, e.g., verb+https://RP.com+)

In this case, the RP script sends only the endpoint but not its RP certificate in Step 2.2, and the IdP script calculates $ID_{RP}$ by itself. 
However, if the RP changes its domain, such as from  \verb+https://theRP.com+ to \verb+https://RP.com+, the account (i.e., $Acct = [ID_U]ID_{RP}$) will inevitably change.
As a result, the user is required to perform special operations to migrate her account to the updated RP system.
It is worth noting that the user operations cannot be eliminated in the migration from an RP to the updated one;
otherwise, it implies two colluding RPs could link a user's accounts across these RPs.


%\vspace{0.75mm}
\noindent \textbf{Restriction of the RP script's origin.}
When the IdP script forwards identity tokens to the RP script, the \verb+postMessage+ targetOrigin mechanism \cite{postm-targeto} is used to restrict the recipient of the tokens, to ensure that the tokens will be sent to the intended $Enpt_{RP}$, as specified in the RP certificate. The targetOrigin is specified as a combination of protocol, port (if not present, 80 for \verb+http+ and 443 for \verb+https+), and domain (e.g., \verb+RP.com+).
The RP script's origin must accurately match the targetOrigin to receive the tokens.

The targetOrigin mechanism does not check the whole URL path in $Enpt_{RP}$, but it introduces no {\em additional} risk.
%This assumes only one RP runs on a domain.
Consider two RPs in the same domain but receiving tokens through two different endpoints,
 e.g., \verb+https://RP.com/honest/tk+ and \verb+https://RP.com/malicious/tk+.
This mechanism cannot distinguish them.
%
% Following the same-origin policy (SOP) \cite{sop} for browser access control, the IdP script sends tokens to both endpoints using \verb+postMessage+.
%
Because a COTS browser controls access to web resources with the same-origin policy (SOP) \cite{sop},
    a user's resources in the honest RP server is always accessible to  the malicious one.
 For example, it could steal cookies using \verb+window.open('https://RP.com/honest').document.cookie+,
 even if the honest RP restricts only the HTTP requests to specific paths are allowed to access its cookies.
%
% an RP's resources in browsers could still be accessed maliciously by the other RP running on the same domain,
%  such as stealing cookies using the script \verb+window.open('https://RP.com/honest').document.cookie+,
%  even if it restricts that only HTTP requests to specific paths carry its cookies.
%
So, this risk is caused by the SOP model but not our designs.

%\vspace{0.75mm}
\noindent \textbf{Support for the authorization code flow.} In the authorization code flow of OIDC \cite{OpenIDConnect}, the IdP does not directly return identity tokens.
Instead, it sends an authorization code to the RP, which uses this code to request identity tokens. The identity-transformation algorithms, namely $\mathcal{F}_{PID_{U}}$, $\mathcal{F}_{PID_{RP}}$, and $\mathcal{F}_{Acct}$, can be integrated into this flow as below.

The IdP script can forward an authorization code to the RP script, and then to the RP.
This code only serves as an index to retrieve identity tokens from the IdP and does not disclose any more information about the user.
After receiving an authorization code, an RP uses it along with a secret credential issued by the IdP during the initial registration \cite{OpenIDConnect} to retrieve identity tokens from the IdP. However, to protect the RP identities from the IdP, anonymous tokens (e.g., ring or group signatures \cite{ring-sig,chaum1991group} and TrustToken \cite{trusttoken}) and anonymous communications (e.g., Tor \cite{tor} and oblivious proxies \cite{ODoH}) need to be adopted for RPs in the retrieval of identity tokens.

%\vspace{0.75mm}
\noindent \textbf{Quantum resistance.}
The current designs of \usso\ do not achieve quantum resistance.
In the future, we will investigate alternative algorithms for quantum-secure services.
We plan to adopt a quantum-resistant public-key algorithm to sign identity tokens and RP certificates,
    and study quantum-resistant OPRFs \cite{ideal-lattice-oprf,isogency-oprf}
 to investigate if they support collision-free $PID_{RP}$s (i.e., no collision exists in the blinded inputs of evaluated pseudo-random functions; see Appendix \ref{proof-rp-collision}).


%%%%%%%%%%%%%%%%%%%%% 几个方面的扩展
% 1. 解决IdP数据泄露
% 如果IdP的数据库泄露,用户列表u公开,则RP就可以,针对每一个u,计算[u]ID_{RP};然后,
% 2. 授权码模式
% 可以使用PKCE方式,直接在前端获取。通常,PKCS模式用在没有后端的RP(例如,纯客户端)。
% 对于有后端,可以将​code_verifier传给RP后端?也能够达到目标。
% 3. RP后端访问IdP,需要通过TOR
% 为了不传递RP ID和Secret,可以是:传递PKCE的code_verifier [user将code_verifier传递给RP],
% 也可以是群签名/环签名之类的。
% 4. 要求RP有授权
% 可以有2种方式:
%   去掉RP Cert;采取授权码方式 + 群签名/环签名之类凭证。
%   去掉RP Cert:采取授权码方式 + PrivacyPass之类匿名凭证(还可以有准确计费)。
% 5. 还有一种方式
% 隐式模式 + PrivacyPass之类匿名凭证(还可以有准确计费)。因为其它方式都需要通过TOR。