%usenix 2023
\documentclass[letterpaper,twocolumn,10pt]{article}
\usepackage{usenix-2020-09}



\usepackage[small]{titlesec}
% to be able to draw some self-contained figs
%\usepackage{tikz}
\usepackage{amsmath}
\usepackage{bm}

% inlined bib file
\usepackage{filecontents}

\usepackage{wasysym}


\usepackage{lipsum,cuted}
\usepackage{float}
\usepackage{caption}


\pagestyle{plain}


\usepackage{graphicx}
\usepackage{verbatim}
\usepackage{caption}
%

%\usepackage{algpseudocode}
\usepackage{amsmath,amssymb,amsthm}

%\usepackage{graphicx}
%\usepackage{geometry}
\usepackage{subfigure}
\usepackage{url}
\usepackage{multirow}
\usepackage{listings}
\usepackage{cite}
\usepackage{array}
\usepackage{enumerate}
\usepackage{booktabs}
\usepackage{color}
\usepackage{xcolor}
\usepackage{soul}
\usepackage{multicol}
\usepackage{macros}
\usepackage{xspace}
\usepackage{algorithm}
\usepackage{algorithmic}
%\usepackage{algcompatible}
%\usepackage[compatible]{algpseudocode}


\renewcommand{\thefootnote}{*}

\newcommand\usso{\textsf{UPPRESSO}}
\newcommand\dyu{{DYU}}

\newtheorem{theorem}{Theorem}
\newtheorem{thm}{\textsc{Theorem}}
\newtheorem{lemma}{\textsc{Lemma}}
%\newtheorem{theorem}{Theorem}
\newtheorem{definition}{\textsc{Definition}}
%\newtheorem{lemma}{Lemma}





\definecolor{Blue}{RGB}{0,0,255}
%\newcommand{\newc} {\color{Blue}}
%\newcommand{\oldc} {\color{black}}

% correct bad hyphenation here
\hyphenation{target-Origin js-rsa-sign op-tical net-works semi-conduc-tor}

\begin{document}
\date{}
\pagenumbering{arabic}
%\title{\Large \bf \usso: Untraceable and Unlinkable Privacy-PREserving\\Single Sign-On Services\footnotemark[1]}
\title{\Large \bf \usso: Untraceable and Unlinkable Privacy-PREserving\\Single Sign-On Services}


% Chengqian Guo, Jingqiang Lin, Quanwei Cai, Fengjun Li, Wentian Zhu, Wei Wang, Jiwu Jing, Qiongxiao Wang, Bin Zhao

%\author{
%{\rm Chengqian Guo$^{\sharp\nabla}$, \ \ Jingqiang Lin$^{\ddag}$, \ \ Quanwei Cai$^{\P}$, \ \ Fengjun Li$^{\S}$, \ \ Wentian Zhu$^{\ddag}$, \ \ Wei Wang$^{\nabla}$,}\\
%{\rm Jiwu Jing$^{\diamondsuit}$, \ \ Qiongxiao Wang$^{\nabla}$, \ \ Bin Zhao$^{\triangle}$}\\
%$\sharp$ Shenyang Aircraft Design \& Research Institute, China\\
%$\ddag$ School of Cyber Security, University of Science and Technology of China\\
%$\P$ Beijing Zitiao Network Technology Co., Ltd, China\\
%$\S$  Department of Electrical Engineering \& Computer Science, the University of Kansas, USA\\
%${\diamondsuit}$ School of Computer Science \& Technology, University of Chinese Academy of Sciences\\
%$\nabla$ Institute of Information Engineering, CAS\ \ \ \ \ \ \ \ \ \ \ \ \ \ \ \ \ \ \ \ \ \ \ 
%$\triangle$ JD.com Silicon Valley R\&D Center, USA}



\maketitle
\begin{abstract}
Single sign-on (SSO) allows a user to maintain only the credential at an identity provider (IdP) to log into multiple relying parties (RPs).
However, SSO introduces privacy threats, as ({\em a}) a curious IdP could potentially track a user's visits to all RPs, and ({\em b}) colluding RPs could learn a user's online profile by linking her identities across these RPs.
This paper presents a privacy-preserving SSO protocol, called \usso, to protect an honest user's online profile against (\emph{a}) an honest-but-curious IdP and (\emph{b}) malicious RPs and users who could collude.
\usso\ proposes an identity-transformation approach to generate untraceable \emph{ephemeral pseudo-identities} for an RP and a user (denoted as $PID_{RP}$ and $PID_U$, respectively) from which the target RP derives a \emph{permanent account} for the user (denoted as $Acct$), while the transformations provide unlinkability.
This approach protects the unique identities of the user and the RPs that she requests to log into while working compatibly with SSO services and satisfying all the security requirements.
%Compared with existing SSO protocols, \usso\ is among the first to prevent both types of privacy threats. Compared with existing privacy-preserving identity federation schemes, \usso\ supports all the desired SSO features while providing sufficient privacy protections.  % 省去,放到Introduction即可
% \usso\ proposes a novel identity-transformation approach. In each SSO login, an \emph{ephemeral pseudo-identity} of an RP, denoted as $PID_{RP}$, is negotiated between a user and the RP. $PID_{RP}$ is sent to the IdP and designated in the identity token, so that the IdP is not aware of the visited RP.
% The IdP then uses $PID_{RP}$ to transform the user's permanent identity $ID_U$, into an ephemeral user pseudo-identity $PID_U$, in the token.
% On receiving the identity token, %%%% the RP just waits for it.
% the RP transforms $PID_U$ into a \emph{permanent account} of the user, denoted as $Acct$. Given a user, the account at each RP is unique and different from $ID_U$, %%%% for different users, the accounts may be the same.
%so colluding RPs cannot link a user's identities across RPs.
We have built a prototype of \usso\ on top of MITREid Connect, an open-source SSO system. The extensive evaluations show that it fulfills the security and privacy requirements of SSO
while introducing reasonable overheads.
\end{abstract}

%\begin{IEEEkeywords}
%Single sign-on, security, privacy. %, trace, linkage
%\end{IEEEkeywords}

%\footnotetext[1]{Chengqian Guo participated in this work since his PhD study in Institute of Information Engineering, Chinese Academy of Sciences (CAS).}

\setcounter{footnote}{0}
\renewcommand{\thefootnote}{\arabic{footnote}}

\section{Introduction}
\label{sec:intro}
Single sign-on (SSO) protocols such as OpenID Connect (OIDC) \cite{OpenIDConnect}, OAuth 2.0 \cite{rfc6749}, and SAML \cite{SAML, SAMLIdentifier}, are widely deployed for identity management and authentication.
SSO allows a user to log into a website,
 known as the \emph{relying party} (RP), using her account registered at a trusted web service, known as the \emph{identity provider} (IdP).
An RP delegates user identification and authentication to the IdP, which issues an \emph{identity token} (such as ``id token'' in OIDC and ``identity assertion'' in SAML) for a user to visit the RP.
For instance, an OIDC user requests to log into an RP,
which constructs an identity-token request with its identity (denoted as $ID_{RP}$) and posts (or redirects) it to a trusted IdP. After authenticating the user, the IdP issues an identity token binding the identities of the user and the target RP (i.e., $ID_U$ and $ID_{RP}$), which is returned to the user and then forwarded to the RP.
Finally, the RP verifies the identity token to determine if the token holder is allowed to log in. Thus, a user maintains only one credential for the IdP, instead of multiple credentials for different RPs.

OIDC also provides comprehensive services of identity management,
 by enabling an IdP to enclose more user attributes in identity tokens, along with the authenticated user's identity.
The attributes (e.g., age and hobby) are maintained at the IdP and enclosed in identity tokens after a user's authorization \cite{OpenIDConnect,rfc6749}.

The wide adoption of SSO raises concerns on user privacy, because it facilitates the tracking of a user's login activities by interested parties \cite{NIST2017draft, SPRESSO, BrowserID, maler2008venn}.
To issue identity tokens, an IdP should know the target RP to be visited by a user and the login time.
So a curious IdP could potentially track a user's all login activities over time
 \cite{BrowserID, SPRESSO},
called {\em IdP-based login tracing} in this paper.
Another privacy risk arises from the fact that RPs learn the user's identity from the identity tokens they receive.
If an identical user identity is enclosed in tokens for a user to visit different RPs, colluding RPs could link the logins across these RPs %and track the user's activities
to learn the user's online profile \cite{maler2008venn, FirefoxAccount}.
This risk is called {\em RP-based identity linkage}. %in this paper.

While preventing different privacy threats (i.e., IdP-based login tracing, RP-based identity linkage, or both),
privacy-preserving SSO schemes \cite{maler2008venn, NIST2017draft, BrowserID, SPRESSO,miso,POIDC} aim to implement authentication and identity management with the following features:
(\emph{a}) \emph{User authentication only to the IdP},
which eliminates the need for authentication between a user and any RP,
 and then requires a user to maintain only the credential for the IdP,
(\emph{b}) \emph{User identification at each RP}, which enables an RP to recognize each user by an account unique within the RP to provide customized services across multiple logins,
and (\emph{c}) \emph{IdP-confirmed attribute provision},
 where a user's attributes are maintained at a trusted IdP and provided to RPs after the user's authorization.
%Meanwhile, these privacy threats posed by different adversaries are considered, including \emph{an honest-but-curious IdP} and \emph{malicious RPs colluding with users}.
In Section \ref{sec:background}, we analyze existing privacy-preserving solutions for SSO and also identity federation.


We present \usso, an Untraceable and Unlinkable Privacy-PREserving Single Sign-On protocol.
It proposes {\em identity transformations} and integrates them in popular OIDC services.
In \usso, an RP and a user transform $ID_{RP}$ into ephemeral $PID_{RP}$, which is sent to a trusted IdP to transform $ID_U$ into an ephemeral user pseudo-identity $PID_U$.
The identity token issued by the IdP binds only $PID_U$ and $PID_{RP}$, instead of $ID_U$ and $ID_{RP}$. On receiving the token, % with a matching $PID_{RP}$,
 the RP transforms $PID_U$ into an account unique at each RP but identical across multiple logins to this RP.


\usso\ prevents both IdP-based login tracing and RP-based identity linkage, while existing privacy-preserving SSO solutions address only one of them \cite{BrowserID, SPRESSO, NIST2017draft, save-flow,POIDC} or need another fully-trusted server in addition to the IdP \cite{miso}.
Meanwhile, \usso\ implements SSO services
% works compatibly with widely-used SSO services \cite{OpenIDConnect, rfc6749, NIST2017draft}
 accessed from commercial-off-the-shelf (COTS) browsers without any plug-ins or extensions,
 and provides all desirable features of SSO protocols as listed above.
In contrast, privacy-preserving identity federation \cite{PseudoID, ELPASSO, UnlimitID, Opaak, uprov, hyperledge-idemix} supports only some but not all of these features,
    and always requires a browser plug-in or extension.

Our contributions are as follows.
\vspace{-\topsep}
\begin{itemize}
\setlength{\topsep}{0pt}
\setlength{\partopsep}{0pt}
\setlength{\itemsep}{0pt}
\setlength{\parsep}{0pt}
\setlength{\parskip}{0pt}
\item We proposed a novel identity-transformation approach for privacy-preserving SSO and designed identity-transformation algorithms with desirable properties.
\item We developed the \usso\ protocol based on the identity transformations with several designs specific to web applications, and proved that it satisfies the security and privacy requirements of SSO services.
\item We implemented a prototype of \usso\ on top of an open-source OIDC implementation. Through performance evaluations, we confirmed that \usso\ introduces reasonable overheads.
\end{itemize}


We present the background and related works in Section \ref{sec:background} and the identity-transformation approach in Section \ref{sec:challenge}, followed by the detailed designs in Section \ref{sec:UPPRESSO}.
Security and privacy are analyzed in Section \ref{sec:analysis}.
We explain the prototype implementation and evaluations in Section \ref{sec:implementation}, and discuss extended issues in Section \ref{sec:discussion}. Section \ref{sec:conclusion} concludes this work.

\section{Background and Related Work}
\label{sec:background}

We describe %OIDC \cite{OpenIDConnect}, to describe
typical SSO services and discuss existing privacy-preserving solutions and other related works.

\subsection{OpenID Connect and SSO Services}
\label{subsec:OIDC}
OIDC is one of the most popular SSO protocols. It supports different login flows: implicit flow, authorization code flow, and hybrid flow (a mix of the other two). These flows differ in the steps for requesting and receiving identity tokens but have common security requirements for identity tokens. We present our designs in the implicit flow and discuss the support for the authorization code flow in Section \ref{sec:discussion}.

In OIDC, users and RPs register at an IdP with their identities
and other information such as user credentials %(e.g., passwords)
and RP endpoints. %(i.e., the URLs to receive tokens).
As shown in Figure \ref{fig:OpenID}, when an RP receives a login request, it constructs an identity-token request with its own identity and the scope of requested user attributes.
This request is redirected to the IdP. Once the IdP authenticates the user, it issues an identity token that encloses the identities (or pseudo-identities) of the user and the visited RP, the requested user attributes, a validity period, etc. The user then forwards the identity token to the RP's endpoint. The RP verifies the token and allows the holder to log in as the enclosed (pseudo-)identity.

% The user's operations, e.g., request redirection, authorization, and token forwarding, are performed by user agents, such as a browser for web applications.
% 这个浏览器相关的词句,放到后面的实现章节。

\begin{figure}[t]
  \centering
  \includegraphics[width=0.95\linewidth]{fig/OIDC.pdf}
  \caption{The implicit SSO login flow of OIDC}
  \label{fig:OpenID}
\end{figure}

\begin{table*}[tb]
\footnotesize
    \caption{Privacy-preserving solutions for SSO and identity federation}
    \centering
    \begin{tabular}{|c|c|c|c|c|c|c|}
  \hline
  \multirow{3}*{\textbf{~~Solution~~}} &
  \multicolumn{3}{c|}{\textbf{SSO Feature} - supported $\CIRCLE$, unsupported $\Circle$, or partially $\LEFTcircle$} & \multicolumn{3}{c|}{\textbf{Privacy Threat} - prevented $\CIRCLE$ or not $\Circle$} \\ \cline{2-7}
  & User Identification & User Authentication & IdP-confirmed Selective  & IdP-based & RP-based & Server \\
  & at each RP & only to the IdP &  Attribute Provision & Login Tracing & Identity Linkage & Collusion$^{\dag}$ \\\hline
  OIDC w/ PPID \cite{NIST2017draft} & $\CIRCLE$ & $\CIRCLE$ & $\CIRCLE$ & $\Circle$ & $\CIRCLE$ & $\Circle$ \\ \hline
  MISO \cite{miso} & $\CIRCLE$ & $\CIRCLE$ & $\CIRCLE$ & $\CIRCLE$ & $\CIRCLE$ & $\Circle$$^1$  \\ \hline 
  BrowserID \cite{BrowserID} & $\CIRCLE$ & $\CIRCLE$$^2$ & $\Circle$ & $\CIRCLE$ & $\Circle$ & $\Circle$ \\ \hline
  SPRESSO \cite{SPRESSO} & $\CIRCLE$ & $\CIRCLE$ & $\LEFTcircle$$^3$ & $\CIRCLE$ & $\Circle$ & $\Circle$ \\ \hline
  AIF-ZKP \cite{save-flow} & $\CIRCLE$ & $\CIRCLE$ & $\CIRCLE$ & $\CIRCLE$ & $\Circle$ & $\Circle$ \\ \hline
  PRIMA \cite{prima} & $\CIRCLE$ & $\Circle$ & $\CIRCLE$ & $\CIRCLE$ & $\Circle$ & $\Circle$ \\ \hline
  PseudoID \cite{PseudoID} & $\CIRCLE$ & $\Circle$ & $\LEFTcircle$$^4$ & $\CIRCLE$ & $\CIRCLE$ & $\CIRCLE$ \\ \hline
  Opaak \cite{Opaak} & $\LEFTcircle$$^5$ & $\Circle$ & $\Circle$ & $\CIRCLE$ & $\CIRCLE$ & $\CIRCLE$ \\ \hline
 % U-Prove \cite{uprov} & $\CIRCLE$ & $\Circle$ & $\LEFTcircle$$^6$ & $\CIRCLE$ & $\CIRCLE$ & $\CIRCLE$ \\ \hline
 % UnlimitID \cite{UnlimitID} & $\CIRCLE$ & $\Circle$ & $\CIRCLE$ & $\CIRCLE$ & $\CIRCLE$ & $\CIRCLE$ \\ \hline
 % EL PASSO \cite{ELPASSO} & $\CIRCLE$ & $\Circle$ & $\CIRCLE$ & $\CIRCLE$ & $\CIRCLE$ & $\CIRCLE$ \\ \hline
  \cite{ELPASSO,uprov,UnlimitID} & $\CIRCLE$ & $\Circle$ & $\CIRCLE$$^6$ & $\CIRCLE$ & $\CIRCLE$ & $\CIRCLE$ \\ \hline
  Fabric Idemix \cite{hyperledge-idemix} & $\LEFTcircle$$^7$ & $\Circle$ & $\CIRCLE$ & $\CIRCLE$ & $\CIRCLE$ & $\CIRCLE$ \\ \hline
 % PrivacyPass \cite{privacypass,trusttoken} & $\Circle$ & $\Circle$ & $\Circle$ & $\CIRCLE$ & $\CIRCLE$ & $\Circle$ \\ \hline
  \usso & $\CIRCLE$ & $\CIRCLE$ & $\CIRCLE$ & $\CIRCLE$ & $\CIRCLE$ & $\Circle$ \\ \hline
\end{tabular}
    \label{tbl:comparison-protocol}
\flushleft
{\footnotesize
$^{\dag}$. This threat happens when \emph{all} servers involved in the processing of identity tokens collude, usually the IdP and target RPs.\\
1. MISO is immune to collusive attacks by the IdP and RPs, but another \emph{fully-trusted} server called mixer is involved in the identity-token generation.\\
2. A BrowserID user generates an \emph{ephemeral} private key to sign subsidiary ``identity assertion'' tokens,
also verified by the RP.\\
3. SPRESSO can be extended to provide selective user attributes in the tokens, while the prototype does not implement this feature.\\
4. Blindly-signed user attributes can be selectively provided but not implemented in the prototype.\\
5. Opaak supports two exclusive pseudonym options: (\emph{a}) linkable within an RP but unlinkable across multiple RPs and (\emph{b}) unlinkable for any pair of actions.\\
6. Different from \cite{ELPASSO,UnlimitID}, a credential in U-Prove \cite{uprov} may contain some attributes that are \emph{invisible} to the IdP, in addition to the ones confirmed by the IdP.\\
7. In the original design of Idemix \cite{idemix}, every user logs into an RP with a unique account, but Fabric Idemix \cite{hyperledge-idemix} implements completely-anonymous services.}
\end{table*}

The following features are desired in SSO services and supported by popular SSO systems \cite{NIST2017draft, OpenIDConnect,rfc6749, SAML, SAMLIdentifier}.

\noindent \textbf{Unique user identification at an RP.}
An RP recognizes each user by a \emph{unique} identity (or account) at the RP to provide customized services across multiple logins.
Such a non-anonymous SSO system is much more desirable in various scenarios than anonymous services.

\noindent\textbf{User authentication only to the IdP.}
RPs only verify the identity tokens issued by an IdP, and the authentication between a user and the IdP is typically conducted \emph{independently} of the steps that deal with identity tokens.
This offers advantages. First, the IdP authenticates users by any appropriate means such as passwords, one-time passwords, or multi-factor authentication.
Meanwhile, a user only maintains her credential at the IdP; and if it is lost or leaked, the user only needs to renew it at the IdP.
However, if a user proves a \emph{non-ephemeral} secret to RPs that is valid across multiple logins, she will have to notify each RP in the events of loss or leakage, or additional revocation checking will be needed \cite{ELPASSO, UnlimitID}.

\noindent\textbf{Selective IdP-confirmed attribute provision.}
An IdP usually includes user attributes in identity tokens \cite{OpenIDConnect,rfc6749} along with user (pseudo-)identities.
A user maintains these attributes at the trusted IdP,
which obtains the user's authorization before enclosing attributes or provides only pre-selected attributes.

\subsection{Privacy-Preserving SSO and Identity Federation}
\label{subsec-solutions}

Privacy-preserving SSO is expected to offer the desired features listed in Section \ref{subsec:OIDC} while addressing different privacy threats.
In contrast, privacy-preserving identity federation offers more privacy protections but introduces extra complexity in the user authentication process.
Identity federation enables a user registered at a trusted IdP to be accepted by other parties, potentially with different accounts,
but \emph{additional user operations for the authentication steps between the user and RPs} are involved.\footnote{SSO protocols \cite{OpenIDConnect,rfc6749, SAML, SAMLIdentifier} allow a user to log into an RP \emph{without} maintaining an account at the RP by herself or holding a permanent secret to be verified by the RP. Although the same term ``single sign-on (SSO)'' was used in other schemes \cite{PseudoID, Opaak, ELPASSO, WangWS13, HanCSTW18, HanCSTWW20}, they are different from the widely-used SSO protocols because a user needs to maintain the accounts at different RPs and/or hold a permanent secret verified by the RPs. In this paper, we refer to them as \emph{identity federation} to emphasize this difference.}
As shown in Table \ref{tbl:comparison-protocol}, none of the existing solutions 
perfectly satisfies all expectations.


\noindent\textbf{Privacy-preserving SSO.}
Existing approaches \cite{BrowserID, SPRESSO, NIST2017draft} prevent either IdP-based login tracing or RP-based identity linkage, but not both.
Pairwise pseudonymous identifiers (PPIDs) are specified \cite{OpenIDConnect, SAMLIdentifier} and recommended \cite{NIST2017draft}
for protecting user privacy against curious RPs.
An IdP creates a unique PPID for a user to log into some RP and encloses it in identity tokens, so colluding RPs cannot link the user.
It does not prevent IdP-based login tracing because the IdP needs the RP's identity to assign PPIDs.

MISO \cite{miso} decouples the calculation of PPIDs from an IdP,
        and another mixer server calculates a user's PPID
    based on $ID_U$, $ID_{RP}$ and a secret,
    after it receives the authenticated user's identity from the IdP.
MISO prevents both RP-based identity linkag, %for the RPs receives only PPIDs,
    and also IdP-based login tracing because $ID_{RP}$ is disclosed to the mixer but not the IdP.
It protects a user's online profile against even collusive attacks by the IdP and RPs,
    but it requires an extra fully-trusted mixer.
  %      which is ensured by Intel SGX.

% 这个就是扯淡吧?为什么还要网络通信呢?直接把Mixer实现在IdP内部,就行了吧?

Other privacy-preserving SSO schemes prevent IdP-based login tracing but leave users vulnerable to RP-based identity linkage, due to the unique user identities enclosed in identity tokens.
For example, in BrowserID \cite{BrowserID} %(formerly known as Firefox Accounts \cite{FirefoxAccount} and Mozilla Persona \cite{persona}),
the IdP %(called the primary identity authority in BrowserID)
issues a ``user certificate'' token that binds a user identity to an \emph{ephemeral} public key. The user then signs a subsidiary ``identity assertion'' token that binds the target RP's identity and sends both tokens to the RP.
The RP creates a one-time pseudo-identity for each login in SPRESSO \cite{SPRESSO}, 
 or a user sends an identity-token request with a cryptographic commitment on the RP identity in AIF-ZKP \cite{save-flow,POIDC},
        which are enclosed in identity tokens along with the user's unique identity.

\noindent\textbf{Privacy-preserving identity federation.}
In PRIMA \cite{prima}, the IdP signs a credential
that binds user attributes and a verification key. Using the signing key, the user selectively provides attributes to the RPs. This verification key works as the user's identity but exposes her to RP-based identity linkage.

%To protect user privacy against more threats, 
PseudoID \cite{PseudoID} introduces a service in addition to the IdP,
 to blindly sign \cite{blind-sign}
an access token that binds a pseudonym and a user secret.
The user then unblinds this token and uses the secret to log into an RP. Approaches based on anonymous credentials \cite{anon-credential-2001, idemix, anon-credential} have been proposed to implement privacy-preserving identity federation \cite{hyperledge-idemix, Opaak, uprov, UnlimitID, ELPASSO}. For instance, the IdP signs anonymous credentials in Opaak \cite{Opaak}, UnlimitID \cite{UnlimitID}, EL PASSO \cite{ELPASSO}, and U-Prove \cite{uprov}, and binds them with non-ephemeral user secrets. %, with which users can authenticate to an RP.
Then the users prove ownership of the anonymous credentials using the secrets and disclose IdP-confirmed attributes in the credentials in most schemes except Opaak.
Similarly, Hyperledger Fabric \cite{hyperledge-idemix} integrates Idemix anonymous credentials \cite{idemix} for completely-unlinkable pseudonyms and IdP-confirmed attribute disclosure.


These approaches \cite{PseudoID,Opaak,ELPASSO,uprov,UnlimitID,hyperledge-idemix} prevent both IdP-based tracing and RP-based identity linkage for (\emph{a}) the RP's identity is not enclosed in the anonymous credentials and (\emph{b}) the user selects different pseudonyms when visiting different RPs.
They even protect user privacy against collusive attacks by the IdP and RPs, as \emph{user-managed} pseudonyms cannot be linked through anonymous credentials \cite{anon-credential-2001, idemix, anon-credential} even when the ownership of these credentials is proved to RPs using one user secret.
However, this privacy protection results in additional user operations in identity federation, compared with widely-used SSO.
Users are required to maintain not only the authentication credentials for the IdP but also the long-term secrets that are verified by RPs.
For example, EL PASSO \cite{ELPASSO} requires users to keep the secrets securely on their devices and coordinate the credential revocation process \cite{ELPASSO, UnlimitID}.
Besides, the users locally manage their accounts at different RPs, and it actually involves authentication steps between the user and RPs, which is referred to as \emph{asynchronous authentication} \cite{ELPASSO}.


\noindent\textbf{Anonymous identity federation.}
Such approaches offer strong privacy protections. They allow users to access RPs with pseudonyms that cannot be used to link any two actions.
Anonymous identity federation was formalized \cite{WangWS13} and implemented using cryptographic primitives such as group signature and zero-knowledge proof (ZKP) \cite{WangWS13, HanCSTWW20, HanCSTW18}. Special features including proxy re-verification \cite{HanCSTWW20}, designated verification \cite{HanCSTW18} and distributed IdP servers \cite{TSAPP}, are considered. %in these schemes.
%and proposed for different applications such as GSM communications \cite{ElmuftiWRR08},
These completely-anonymous authentication services only work for special scenarios and do not support user identification at an RP, a common requirement in most applications.

\subsection{Anonymous Tokens and OPRF-based Applications}
\label{sec:related}

PrivacyPass and TrustToken \cite{privacypass,trusttoken} adopted an oblivious pseudo-random function (OPRF) protocol \cite{oprf-proved,voprf-proved} to generate anonymous tokens with which authorized users could access resources anonymously.
A user generates a random number $e_i$ to blind an unsigned token $T_i$ into $[e_i]T_i$. After authenticating the user, a token server signs $[e_i]T_i$ using its private key $k$ and returns $[k e_i]T_i$ to the user, who converts it into an anonymous token  ($T_i, [k]T_i$) using $e_i$ (see \cite{privacypass,trusttoken,oprf-proved} for details). In this process, the server obliviously calculates a pseudo-random output (i.e., anonymous token).
RPs do not identify each user if we use such tokens in an SSO service, because these tokens are completely indistinguishable.

\usso\ applies a similar cryptographic technique in identity transformations  (details in Section \ref{sec:UPPRESSO}).
 A user selects a random number $t_i$ to transform $ID_{RP}$ to $PID_{RP} = [t_i]ID_{RP}$, protecting $ID_{RP}$ from the IdP.
Then, the IdP uses the user's permanent identity $u$ to calculate $PID_U = [u]PID_{RP} = [ut_i]ID_{RP}$,
which is similar to token signing by the token server in PrivacyPass/TrustToken. Hence, \usso\ provides the \emph{IdP-untraceability} property, i.e., the IdP cannot link $ID_{RP}$ and $[t_i]ID_{RP}$, which roughly corresponds to the {\em unlinkability of token signing-redemption} in PrivacyPass/TrustToken, i.e., the token server cannot link $T_i$ and $[e_i]T_i$ (or $[k]T_i$ and $[ke_i]T_i$). %($[e_i]T_i, [ke_i]T_i$) and  ($T_i, [k]T_i$).

To provide privacy-preserving SSO services, \usso\ works among three parties, unlike the two-party OPRF and PrivacyPass/TrustToken protocols.
So it leverages this cryptographic technique in different ways. %, unlike the completely anonymous service in PrivacyPass/TrustToken.
%This also requires extending the two-party OPRF protocol to work for three parties in SSO services.
First, \usso\ securely shares the user-selected random number $t_i$ with the target RP,
enabling it to independently derive the user's unique account, i.e., $Acct = [t_i^{-1}]PID_{U}$. This account is ``obliviously'' determined by the IdP when it calculates the user's pseudo-identity based on the ``blinded'' input (i.e., the RP's pseudo-identity). Second, $u$ and $ID_{RP}$, which roughly correspond to $k$ and $T_i$ in PrivacyPass/TrustToken, are designed as the permanent identities of the user and the RP, respectively, to establish the relationships between identities obliviously.
This is very different from existing OPRF-based systems  \cite{privacypass, trusttoken, strong-oprf, oprf-bitcoin-wallet, pesto, oprf-ot-si, pp-ss, Private-Contact-Discovery, o-kms, oprf-deduplication} that always use $k$ as the server's secret key.
The integration of SSO (pseudo-)identities  and OPRF variables requires a deep understanding of both SSO and OPRF protocols.
Finally, \usso\ leverages randomness of OPRF to provide \emph{RP unlinkability}, and this property is not utilized in PrivacyPass/TrustToken.
It ensures colluding RPs cannot link any logins across RPs, %i.e., ($ID_{RP}, t, [u]ID_{RP}$) and ($ID_{RP'}, t', [u']ID_{RP'}$),
even by sharing their knowledge about pseudo-identities and permanent accounts.
This property offers desirable indistinguishability of \emph{different users} logging into colluding RPs.
It corresponds to the indistinguishability of \emph{different private keys} for signing anonymous tokens, not considered in PrivacyPass/TrustToken. % but they intentionally distinguish different keys by configuring a unique public key for each user to verify the signed tokens.

Although \usso\ mathematical employs the same technique in its identity transformations as the OPRF protocols \cite{oprf-proved,voprf-proved}, we need more properties of these algorithms to ensure security and privacy. % in Theorems 4, 6, and 7.
\usso\ essentially depends on the \emph{obliviousness} property to ensure IdP untraceability, %prevent the IdP from learning any information about $ID_{RP}$ when receiving a token request for $[t_i]ID_{RP}$,
the \emph{deterministicness} property of \emph{pseudo-random} functions to derive a permanent account for any $t$, and the \emph{randomness} property to provide RP unlinkability. %when user identities (i.e., the key secret $k$ of pseudo-random functions) are unknown.
In contrast, PrivacyPass and TrustToken \cite{privacypass,trusttoken} actually depend on only the properties of obliviousness and deterministicness.
Besides, \usso\ needs an extra property for security in the case of leaked user identities,
     \emph{collision-freeness} of $PID_{RP}$ (see Appendix \ref{proof-rp-collision}).
So an OPRF protocol is not always ready for identity transformations in \usso, unless no collision exists in the blinded inputs of the pseudo-random function \cite{oprf-proved,voprf-proved}.
This property %, i.e., \emph{collision-freeness of blinded inputs},
is not explicitly required in OPRF-based solutions.

\begin{comment}
PrivacyPass and TrustToken \cite{privacypass,trusttoken}
adopted the oblivious pseudo-random function (OPRF) protocol \cite{oprf-proved,voprf-proved,oprf-bitcoin-wallet} to generate anonymous tokens for authorized users to access resources. A user generates a random number $e_i$ for each unsigned token $T_i$ and blinds $T_i$ into $[e_i]T_i$.
Once the user is authenticated, a token server signs $[e_i]T_i$ with a private key $k$ as $[k e_i]T_i$. Then, the user converts $[ke_i]T_i$ to $[k]T_i$ using $e_i$ and redeems the token ($T_i, T_i^k$) to anonymously access resources. \usso\ applies a similar cryptographic skill to design identity transformations. A user transforms $ID_{RP}$ to $PID_{RP} = [t_i]ID_{RP}$ using a random number $t_i$.
Then, the IdP calculates $PID_U = [u]PID_{RP}$ %= [ut]ID_{RP}$
from $PID_{RP}$ using the user's identity $u$. Finally, the visited RP calculates $Acct = [t_i^{-1}]PID_{U}$ from $PID_{U}$ using the shared $t_i$. The \emph{IdP-untraceability} property in \usso, i.e., $[t_i]ID_{RP}$ and $ID_{RP}$ cannot be linked by an IdP, roughly corresponds to the unlinkability between token signing and redemption in PrivacyPass/TrustToken, i.e., ($[e_i]T_i, [ke_i]T_i$) and  ($T_i, [k]T_i$) cannot be linked by the token server.


%%%%%%%%%%%%%%
%%% 以下的段落,我尽量不希望写成“我们直接借鉴OPRF协议来设计”。
%% 原因:1. 的确不是这样;2. 如果我们是直接借鉴OPRF协议来设计UPPRESSO,则后面的行文不应该是现在这样子。我们后面的全文行文,完全是在谈ID transformation的角度。
PrivacyPass/TrustToken adopted the OPRF protocol for token generation, i.e., the server obliviously calculates a pseudo-random output for any user input as tokens, and the anonymous tokens are indistinguishable.
If they are used to realize SSO, the RPs cannot identify each user. In contrast, \usso\ applies this cryptographic skill in different ways from anonymous tokens, to provide a non-anonymous but privacy-preserving SSO service. First, it works for three parties in SSO while OPRFs and PrivacyPass/TrustToken are two-party protocols. In \usso\ a user shares the random number $t$ with an RP, which is used for protecting $ID_{RP}$ and enables the RP to derive the user's account, %at this RP,
and the IdP ``obliviously'' determines the account based on the user identity and the RP's pseudo-identity (or ``blinded'' identity).
In this process, we actually apply pseudo-random functions but uses the \emph{secret key} of pseudo-random functions as a \emph{user identity} (i.e., $u$ in identity transformations), while it is used as \emph{secret keys} in PrivacyPass/TrustToken and other OPRF-based applications \cite{privacypass,trusttoken,strong-oprf,oprf-bitcoin-wallet, pesto,oprf-ot-si,pp-ss, Private-Contact-Discovery,o-kms,oprf-deduplication}.
This unusual application requires a special understanding of the variables in OPRFs and the (pseudo-)identities in SSO services.
%This extension is unlikely to occur in OPFRs that consider $k$ as the private key of the server, but it is reasonable in \usso\ as it uses random numbers as identities and trapdoors to design oblivious pseudo-random functions for identity transformations.
%Compared with the original OPRFs, we use the $k$ key (they are always private keys) as a user identity.

% 第一段:说明2个点:
% 1. 2 party -> 3 party: user把随机数分享给RP。
% 2. 把key用作uid,在SSO中。把prf的输入输出,用作PID和account。
% 他们的token,与uid/account无关
Moreover, \usso\ actually leverages the randomness property of OPRFs to provide \emph{RP unlinkability}, while this property is not needed in PrivacyPass/TrustToken. It ensures colluding RPs cannot link any two logins  across RPs, i.e., ($ID_{RP}, t, [u]ID_{RP}$) and ($ID_{RP'}, t', [u']ID_{RP'}$), even if they share the knowledge about the pseudo-identities and permanent accounts of the users initiating these logins.
In \usso\ this property results in the indistinguishability of \emph{different users} in colluding RPs' view, therefore corresponding to \emph{different keys} for signing anonymous tokens, which is not considered in PrivacyPass/TrustToken. % but they intentionally distinguish different keys by configuring a unique public key for each user to verify the signed tokens.

Although \usso\ employs the same mathematical algorithms in its identity transformations as the OPRF protocol \cite{oprf-proved,voprf-proved}, we prove more properties of these algorithms to ensure security and privacy. % in Theorems 4, 6, and 7.
\usso\ depends on the \emph{obliviousness} property to ensure IdP untraceability, %prevent the IdP from learning any information about $ID_{RP}$ when receiving a token request for $[t_i]ID_{RP}$,
the \emph{deterministicness} property of \emph{pseudo-random} functions to enable the RP to derive a permanent account for any $t$, and the \emph{randomness} property to provide RP unlinkability. %when user identities (i.e., the key secret $k$ of pseudo-random functions) are unknown.
Last but not least, \usso\ requires an additional property for SSO services, called \emph{RP designation}.
It is satisfied \emph{only if} no collision exists in RPs' pseudo-identities (see Lemma 1), which are actually the blinded inputs of the evaluated pseudo-random function \cite{oprf-proved,voprf-proved}.
This property %, i.e., \emph{collision-freeness of blinded inputs},
is not explicitly required in other OPRF-based solutions and therefore may not be supported by all OPRF protocols. Thus, an OPRF protocol is not always ready to implement identity transformations in \usso, unless no collision exists in the blinded inputs.

%For example, we plan to analyze this property of quantum-secure ORPF protocols \cite{ideal-lattice-oprf,isogency-oprf} in the future, to find out whether they work effectively with the identity transformations in \usso.

%\footnote{Due to the extension of three parties and the requirement of input-collisionlessness, we did not realize this cryptographic technique has been published in OPRFs until anonymous reviewers pointed it out.}

\end{comment}

\subsection{Extended Related Work}

%Cryptographic primitives are used for protecting user privacy.
\noindent\textbf{Cryptographic primitives for sign-on privacy.}
Distributed key generation and ZKPs are used to generate ring-signature private keys as untraceable pseudonyms \cite{crypto-book} or prove users' statements about their credentials without revealing the credential contents \cite{zklaim}.
One-time key-share tokens \cite{tandem} are generated in two-party threshold cryptography systems to protect users' key-usage patterns.
PESTO \cite{pesto} combines distributed partially-oblivious pseudo-random functions (dpOPRFs) and distributed signatures in proactively-secure SSO services, and signs a commitment of an identity token (but not the token itself) for better user privacy.
Ab-dSSO \cite{AttriB-SSO} enforces attribute-based policies,
     using an MPC scheme between the distributed IdP servers and attribute authorities.

%A user's authentication secrets are split between a browser extension and Larch log servers \cite{larch},  so that every sign-on is recorded, while sign-on privacy is protected against log servers by utilizing threshold signatures and ZKPs \cite{ZKBoo}.


%identity federation \cite{PrivSSO}

%\cite{DPP-SSO}



%\vspace{0.5mm}
\noindent\textbf{Formal analysis of SSO protocols.}
A formal analysis on SAML-based SSO \cite{ArmandoCCCT08} found some system does not bind RP identities correctly in the identity tokens.
Fett et al. \cite{FettKS16, FettKS17} formally analyzed OIDC and OAuth 2.0 using a Dolev-Yao style model \cite{FettKS14} and reported the 307 redirection and IdP mix-up attacks.
%We also developed a Dolev-Yao style model for \usso, to prove its security and privacy.


%\vspace{0.5mm}
%\noindent\textbf{Implementation vulnerabilities in SSO.}
%Various vulnerabilities have been found in several SSO systems for web applications, resulting in attacks %of impersonation and identity injection
%that break the confidentiality \cite{WangCW12,ccsSunB12, ArmandoCCCPS13, DiscoveringJCS,dimvaLiM16}, integrity \cite{WangCW12, SomorovskyMSKJ12, WangZLG16, MainkaMS16, MainkaMSW17,dimvaLiM16} or RP designation \cite{WangZLG16, MainkaMS16, MainkaMSW17, YangLCZ18,dimvaLiM16} of identity tokens.
%%In the SSO services of Google and Facebook, %from the view of browser-relayed traffics
%%    logic flaws of the IdPs and RPs were detected \cite{WangCW12}.  % to break the confidentiality and integrity of identity tokens.
%The integrity of identity tokens was violated %\cite{SomorovskyMSKJ12,WangCW12,WangZLG16,MainkaMS16, MainkaMSW17}
%due to software flaws such as defective verification by RPs \cite{WangCW12,WangZLG16,MainkaMSW17}, XML signature wrapping \cite{SomorovskyMSKJ12}, and IdP spoofing \cite{MainkaMS16,MainkaMSW17}.
%Meanwhile, the RP designation was broken because of incorrect binding by an IdP \cite{YangLCZ18, WangZLG16} or insufficient verification by RPs \cite{MainkaMS16, MainkaMSW17, YangLCZ18}.
%A defective IdP does not enclose an identifiable Email address in signed tokens \cite{WangCW12}, which breaks the user identification.




%Similar vulnerabilities have been found in Android Apps that break the confidentiality \cite{ChenPCTKT14, WangZLLYLG15, YangLS17, ShiWL19}, integrity \cite{ChenPCTKT14, YangLS17}, and RP designation \cite{ChenPCTKT14, ShiWL19, WangZLLYLG15} of SSO identity tokens.

%Navas et al. \cite{NavasB19} discussed the possible attack patterns against OIDC services.
%Automatic tools such as SSOScan \cite{ZhouE14}, OAuthTester \cite{YangLLZH16} and S3KVetter \cite{YangLCZ18},
% detect the violations of confidentiality, integrity, or RP designation of SSO identity tokens.
% Wang et al. \cite{ExplicatingSDK} detect the vulnerable applications
%     built with authentication/authorization SDKs,
%      due to the implicit but unsuitable assumptions of these SDKs.


%Furthermore, if a user is compromised, the attacker can log in to RPs on his behalf. So, we consider malicious users, malicious RPs, and colluding users and RPs in our threat model (see Section \ref{subsec:threatmodel}).





% In a mobile system,
% browsers, IdP Apps,
%     or IdP-provided SDKs %(e.g., an encapsulated WebView)
%          are responsible for forwarding identity tokens, %from the IdP App to RP Apps.
% but none of them ensures an identity token is sent to the designated RP only \cite{ChenPCTKT14, WangZLLYLG15}.
% %    because a WebView or the system browser cannot authenticate the RP Apps and the IdP App may be repackaged.
% %SSO protocols are modified for mobile Apps, but the modifications are not well understood by developers \cite{ChenPCTKT14, YangLS17}.
% Vulnerabilities were found in Android Apps,
%     to break confidentiality \cite{ChenPCTKT14,WangZLLYLG15,YangLS17,ShiWL19}, integrity \cite{ChenPCTKT14,YangLS17}, and RP designation \cite{ChenPCTKT14,ShiWL19} of identity tokens.
% A flaw was found in Google Apps \cite{ArmandoCCCPS13}, allowing a malicious RP to hijack a user's authentication attempt and inject a payload to steal the cookie or identity token belonging to another RP.

% If a user is compromised,
%     attackers will log in to RPs on his behalf.
% Single sign-off helps the victim user
%  to revoke all his tokens accepted and log out from the RPs  \cite{GhasemisharifRC18}.
% FedCM \cite{FedCM} attempts to disable iframe and third-party cookies in SSO, which could be exploited to track users.
% %UPRRSSO protects privacy in SSO through ID transformations and our prototype does NOT use either iframe or third-party cookies.

\section{The Identity-Transformation Approach}
\label{sec:challenge}

We discuss the security requirements of privacy-preserving SSO and present the identity-transformation approach.


\subsection{Security Requirements for SSO Services}
\label{subsec:basicrequirements}

Non-anonymous SSO services \cite{OpenIDConnect,rfc6749,SAML,SAMLIdentifier,NIST2017draft} are designed to allow a \emph{legitimate} user to log into an \emph{honest} RP with her account at this RP, %correlating multiple logins,
by presenting \emph{identity tokens} issued by a \emph{trusted} IdP.
To achieve this goal, the trusted IdP issues an identity token that specifies the RP being accessed (i.e., \emph{RP designation}) and identifies the authenticated user (i.e., \emph{user identification})
        by identities or pseudo-identities.
An honest RP checks the RP's identity designated in the token before accepting it and authorizes the token holder to log in as the specified user. This prevents malicious RPs from replaying any received tokens to gain unauthorized access to other honest RPs as victim users.
\emph{Confidentiality} and \emph{integrity} of identity tokens are also necessary to prevent eavesdropping and tampering. Identity tokens are forwarded to the target RPs by the authenticated user, so they are usually signed by the trusted IdP and transmitted over HTTPS \cite{OpenIDConnect, rfc6749, SAML}.

%The requirements for secure SSO services, i.e., RP designation, user identification, integrity, and confidentiality of identity tokens, have been extensively studied \cite{ArmandoCCCT08, SPRESSO, FettKS14, FettKS16, FettKS17}.
%Any vulnerabilities that undermine these properties result in various attacks \cite{SomorovskyMSKJ12, WangCW12, ArmandoCCCPS13, ZhouE14, WangZLLYLG15, WangZLG16, YangLLZH16, MainkaMS16, MainkaMSW17, YangLCZ18, YangLS17, ShiWL19, ChenPCTKT14, ccsSunB12, DiscoveringJCS, dimvaLiM16, CaoSBKVC14, TowardsShehabM14}.

\begin{table}[b]
\footnotesize
    \caption{The (pseudo-)identities in \usso}
    \centering
%    \begin{tabular}{|l|l|l|}
    \begin{tabular}{|p{0.93cm}|p{5.16cm}|p{1.13cm}|} \hline
    {\textbf{Notation}} & {\textbf{Description}} & {\textbf{Lifecycle}} \\ \hline
    {$ID_U$} & {The user's unique identity at the IdP.} & {Permanent} \\ \hline
    {$ID_{RP_j}$} & {The $j$-th RP's unique identity at the IdP.} & {Permanent} \\ \hline
    {$PID_{U,j}^i$} & {The user's pseudo-identity in her $i$-th login to the $j$-th RP.} & {Ephemeral} \\ \hline
    {$PID_{RP_j}^i$} & {The $j$-th RP's pseudo-identity in the user's $i$-th login to this RP.} & {Ephemeral} \\ \hline
    {$Acct_j$} & {The user's identity (or account) at the $j$-th RP.} & {Permanent} \\ \hline
    \end{tabular}
    \label{tbl:notations-dilemma}
\end{table}

\subsection{Identity Transformation}
\label{subsec:solutions}

\usso\ implements privacy-preserving SSO that ensures security properties, while preventing both IdP-based login tracing and RP-based identity linkage.
These requirements are satisfied through \emph{transformed identities} in the identity tokens. Table \ref{tbl:notations-dilemma} lists the notations,
and the subscript $j$ and/or the superscript $i$ may be ignored if it does not cause ambiguity.


In an SSO login flow, a user initiates the process by negotiating an \emph{ephemeral} pseudo-identity $PID_{RP}$  with the target RP and sending an identity-token request that encloses $PID_{RP}$ to the IdP.
After successfully authenticating the user as $ID_U$, the IdP calculates an \emph{ephemeral} $PID_U$ based on $ID_U$ and $PID_{RP}$ and issues an identity token that binds $PID_U$ and $PID_{RP}$. Upon receiving a valid token, the RP calculates the user's \emph{permanent} $Acct$ and authorizes the token holder to log in.
The relationships among the (pseudo-)identities are depicted in Figure \ref{fig:IDCorrelation}, where the labeled arrows denote the transformations of (pseudo-)identities, and the \emph{red} and \emph{green} blocks represent \emph{permanent} and \emph{ephemeral} ones, respectively.

\begin{figure}[tb]
  \centering
  \includegraphics[width=0.99\linewidth]{fig/IDCorrelation.pdf}
  \caption{Identity transformations in \usso} %privacy-preserving SSO}
  \label{fig:IDCorrelation}
\end{figure}

%An identity token binds the (pseudo-)identities of an authenticated user and an RP.
%Since an IdP authenticates users and then always knows the user's identity (i.e., $ID_U$),
%    to prevent the IdP-based login tracing,
%    we should not reveal the target RP's permanent identity (i.e., $ID_{RP}$) to the IdP.

For RP designation, $PID_{RP}$ should be associated \emph{uniquely} with the target RP.
For user identification, with \emph{ephemeral} $PID_{U}^i$ in each login, the designated RP should be able to derive a unique \emph{permanent} account  (i.e., $Acct$) of the user at this RP.
To prevent IdP-based login tracing, it is essential to ensure that the IdP does not obtain any information about $ID_{RP}$ from any $PID_{RP}^i$.
Thus, in a user's multiple logins to one RP,
 she should generate independent $PID_{RP}^i$s\footnote{The IdP should not be able to link multiple logins visiting a given RP, while the RP's identity is unknown to the IdP.} % the IdP-based login tracing is still effective, to correlate a user's multiple logins.
and independent $PID_U^i$s.\footnote{If $PID_U^i$ is not completely independent of each other, it implies that there is a possibility for the IdP to link multiple logins visiting a given RP.}
Finally, to prevent RP-based identity linkage,
% the IdP does not enclose $ID_U$ in identity tokens.
%a user pseudo-identity (i.e., $PID_U$) is bound instead:
%$PID_U$ is bound in identity tokens:
the RP should not obtain any information about $ID_U$ from any $PID_{U,j}$, which implies that $PID_{U,j}$ for different RPs should also be independent of each other.

We propose three identity transformations as below:
\vspace{-\topsep}\begin{itemize}
\setlength{\topsep}{0pt}
\setlength{\partopsep}{0pt}
\setlength{\itemsep}{0pt}
\setlength{\parsep}{0pt}
\setlength{\parskip}{0pt}
\item
$\mathcal{F}_{PID_{RP}}(ID_{RP}) = PID_{RP}$, calculated by the user and the RP.
In the IdP's view,
$\mathcal{F}_{PID_{RP}}()$ is a one-way function and $PID_{RP}$
is \emph{indistinguishable} from random variables.
\item
$\mathcal{F}_{PID_U}(ID_U, PID_{RP}) = PID_{U}$, calculated by the IdP.
In the target RP's view,
    $\mathcal{F}_{PID_U}()$ is a one-way function and $PID_{U}$ is \emph{indistinguishable} from random variables.
\item
$\mathcal{F}_{Acct}(PID_{U}, PID_{RP}) = Acct$, calculated by the target RP.
Given $ID_U$ and $ID_{RP}$, $Acct$ is %\emph{permanent} and
\emph{unique} to other accounts at this RP.
That is, in a user's two different logins to the RP,
 $\mathcal{F}_{Acct}(PID_{U}^i, PID_{RP}^i) = \mathcal{F}_{Acct}(PID_{U}^{i'}, PID_{RP}^{i'})$.
\end{itemize}




\section{The Designs of \usso}
\label{sec:UPPRESSO}

%This section presents the threat model and assumptions, the identity-transformation algorithms, and finally the \usso\ protocol with designs specific for web applications.

\subsection{Threat Model}
\label{subsec:threatmodel}
The system consists of an honest-but-curious IdP as well as several honest or malicious RPs and users. This threat model is in line with widely-used SSO services \cite{OpenIDConnect,rfc6749, SAML, SAMLIdentifier}.

%\vspace{0.5mm}
\noindent \textbf{Honest-but-curious IdP.} The IdP strictly follows the protocols,
 while remaining interested in learning about user activities.
For example, it could store received messages to infer the relationship between $ID_U$, $ID_{RP}$, $PID_{U}$, and $PID_{RP}$.
It never actively violates the protocols, so a script downloaded from the IdP also strictly follows the protocols (see Section \ref{sec:web-design} for specific designs for web applications).
The IdP maintains the private key well for signing identity tokens and RP certificates, %(see Section \ref{implementations} for details)
which prevents adversaries from forging such messages.

%\vspace{0.5mm}
\noindent \textbf{Malicious users.} Adversaries could control a set of users by stealing their credentials or registering Sybil users in \usso.
 Their objective \cite{SPRESSO, FettKS14} is to (\emph{a}) impersonate a victim user at honest RPs or (\emph{b}) entice an honest user to log into an honest RP under another user's account.
A malicious user could modify, insert, drop, or replay messages or even behave arbitrarily in login flows.

\noindent \textbf{Malicious RPs.}
Adversaries could control a set of RPs by registering at the IdP as an RP or exploiting vulnerabilities to compromise some RPs.
Malicious RPs could behave arbitrarily, attempting to compromise the security or privacy guarantees of \usso.
For example, they could manipulate $PID_{RP}$ and $t$ in a login, attempting to (\emph{a}) entice honest users to return an identity token that could be accepted by some honest RP or (\emph{b}) manipulate the generation of $PID_U$ to further analyze the relationship between $ID_U$ and $PID_U$.

%\vspace{0.5mm}
\noindent \textbf{Colluding users and RPs.}
Malicious users and RPs could collude,
 attempting to break the security or privacy guarantees for honest users and RPs.
For example, a malicious RP could collude with malicious users to impersonate a victim at honest RPs or link an honest user's logins visiting colluding RPs.

We do \emph{not} consider the collusion of the IdP and RPs.
In this case, a user would have to complete login flows \emph{entirely} with malicious entities. 
In principle, it would require a long-term user secret to protect permanent accounts across RPs.
%This secret should be held \emph{only} by the user; otherwise, colluding adversaries could eventually link these accounts.
If the relationship of these accounts is not masked by the user secret or in extra fully-trusted servers \cite{miso},
 the colluding IdP and RPs will eventually find a way to link them.
Privacy-preserving approaches in identity federation \cite{ELPASSO, UnlimitID, idemix, PseudoID, Opaak, uprov} prevent IdP-RP collusive attacks;
however, they require (\emph{a}) a long-term secret that is held only by the user and verified by the RPs and (\emph{b}) user-managed accounts for each user at different RPs.\footnote{While these accounts can be derived from an RP's domain and the user's secret \cite{ELPASSO, UnlimitID, Opaak, uprov,idemix},
 they still cause inconvenience to users.
For example, a user needs to pre-install a browser extension to handle the long-term secret.}
Moreover, if this secret is lost or leaked, the user must notify all the RPs to update her accounts derived from the secret.
% on the other hand, MISO protects user privacy against the IdP-RP collusive attack, but introducing an extra fully-trusted mixer server in the identity-token generation.
Unlike these schemes, \usso\ is not designed to prevent such collusive attacks,
 because a user is authenticated only \emph{once} in the login flow.
In \usso\ a user's identity at the IdP is obliviously transformed into accounts at RPs,
 which are not related to any user secrets or credentials such as passwords, one-time passwords, 
  and FIDO devices.

%Finally, the security and privacy properties of \usso\ are guaranteed against different combinations of the above adversaries.
%It implements secure SSO services against colluding users and RPs, while preventing (\emph{a}) IdP-based login tracing by an honest-but-curious IdP and (\emph{b}) RP-based identity linkage by colluding RPs and users.

\subsection{Assumptions}
We assume secure communications between honest entities (e.g., HTTPS), and the cryptographic primitives are secure. The software stack of an honest entity is correctly implemented to transmit messages to receivers as expected.

\usso\ is designed for users who value privacy. So a user never authorizes the IdP to enclose  \emph{distinctive} attributes, such as telephone number, Email address, etc., in tokens or sets such attributes at any RP. Hence, privacy leakages due to re-identification by distinctive attributes across RPs are out of our scope.
Moreover, our work focuses only on the privacy threats introduced by SSO protocols, and does not consider the tracking of user activities by network traffic analysis or crafted web pages, as they can be prevented by existing defenses.
For example, FedCM \cite{FedCM} proposes to disable iframe and third-party cookies in SSO, which could be exploited to track users.


%When a user visits multiple RPs concurrently from one browser, a malicious RP may actively redirect his account to another RP server using crafted web pages. Meanwhile, traffic analysis is possible in SSO scenarios, where a user's activities can be tracked from network packets and active account linkage through malicious web pages. We do not consider these attacks in this work, as they can be prevented by existing defenses.


\subsection{Identity-Transformation Algorithms}
\label{subsec:overview}

We design identity transformations on an elliptic curve $\mathbb{E}$,
and Table \ref{tbl:notations-protocol} lists the notations.
%The subscript $j$ and/or superscript $i$ may be omitted if there is no ambiguity.


\begin{table}[tb]
\footnotesize
    \caption{Notations used in the \usso\ protocol}
    \centering
%    \begin{tabular}{|c|c|c|}
    \begin{tabular}{|p{0.93cm}|p{6.71cm}|} \hline
    {\textbf{Notation}} & {\textbf{Description}} \\ \hline
    {$\mathbb{E}$, $G$, $n$} & {$\mathbb{E}$ is an elliptic curve over a finite field $\mathbb{F}_q$. $G$ is a base point (or generator) on $\mathbb{E}$, and the order of $G$ is a prime number $n$.} \\ \hline
    {$ID_{U_i}$} & {$ID_U = u \in \mathbb{Z}_n$ is the $i$-th user's unique identity at the IdP, which is known only to the IdP.} \\ \hline
   {$ID_{RP_j}$} & {$ID_{RP} = [r]G$ is the $j$-th RP's unique identity, which is publicly known; $r \in \mathbb{Z}_n$ is known to \emph{nobody}.} \\ \hline
    {$t$} & {$t \in \mathbb{Z}_n$ is a user-selected random integer in each login; $t$ is shared with the target RP and kept unknown to the IdP.} \\ \hline
    {$PID_{RP_j}^l$} & {$PID_{RP} = [t]{ID_{RP}} = [tr]G$ is the $j$-th RP's pseudo-identity, in the user's $l$-th login visiting this RP.} \\ \hline
    {$PID_{U_i,j}^l$} & {$PID_U = [{ID_U}]{PID_{RP}} = [utr]G$ is the $i$-th user's pseudo-identity, in the user's $l$-th login visiting the $j$-th RP.} \\ \hline
     {$Acct_{i,j}$} & {$Acct = [t^{-1}\bmod n]PID_{U} = [ID_U]ID_{RP} = [ur]G$ is the $i$-th user's locally-unique account at the $j$-th RP, publicly known.} \\ \hline
    {$SK$, $PK$} & {The IdP's private key and public key, used to sign and verify identity tokens and RP certificates.} \\ \hline
    {$Enpt_{RP_j}$} & {The $j$-th RP's endpoint for receiving the identity tokens.} \\ \hline
    {$Cert_{RP_j}$} & {The IdP-signed RP certificate binding $ID_{RP_j}$ and $Enpt_{RP_j}$.} \\ \hline
    \end{tabular}
    \label{tbl:notations-protocol}
\end{table}

\noindent {\bf $\boldsymbol{ID_{\boldsymbol{U}}}$, $\boldsymbol{ID_{\boldsymbol{RP}}}$ and $\boldsymbol{Acct}$.}
The IdP assigns a unique random integer $u \in \mathbb{Z}_n$ to a user (i.e., $ID_U = u$),
 and randomly selects unique $ID_{RP} = [r]G$ for a registered RP. % if it is a point on $\mathbb{E}$.
Here, $G$ is a base point on $\mathbb{E}$ of order $n$, and $[r]G$ denotes the addition of $G$ on the curve $r$ times.

$Acct = \mathcal{F}_{Acct\ast}(ID_U, ID_{RP_j})= [ID_U]ID_{RP_j} =[ur_j]G$ is automatically assigned 
        to a user at every RP,
and a user's accounts at different RPs are inherently different and unlinkable.

\noindent {\bf $\boldsymbol{ID_{\boldsymbol{RP}}}$-$\boldsymbol{PID_{\boldsymbol{RP}}}$ Transformation.} In each login, a user selects a random number $t \in \mathbb{Z}_n$ to calculate $PID_{RP}$.
\begin{equation}
PID_{RP} = \mathcal{F}_{PID_{RP}}(ID_{RP}) = [t]{ID_{RP}} = [tr]G
\label{equ:PIDRP}
\end{equation}


%In each login, the user selects $t$ and shares it with the RP to negotiate $PID_{RP}$. 
%--removed due to double-check discussion
%It makes no difference if the RP selects $t$ randomly and sends it to the user, as long as both of them calculate $PID_{RP}$ \emph{independently} and check if the received $PID_{RP}$ is equal to the calculated one.

\noindent {\bf $\boldsymbol{ID_U}$-$\boldsymbol{PID_U}$ Transformation.}
On receiving an identity-token request for $PID_{RP}$ from a user identified as $ID_U$, the IdP calculates $PID_{U}$ as below.
\begin{equation}
PID_{U} = \mathcal{F}_{PID_U}(ID_U, PID_{RP}) =
  [{ID_U}]{PID_{RP}} = [utr]G
 \label{equ:PIDU}
\end{equation}


\noindent {\bf $\boldsymbol{PID_U}$-$\boldsymbol{Acct}$ Transformation.}
The user sends $t$ to the target RP as a trapdoor to derive her account.
After verifying a token that encloses $PID_U$ and $PID_{RP}$, it calculates $Acct$ as follows.
\begin{equation}
Acct = \mathcal{F}_{Acct}(PID_{U})
   = [t^{-1} \bmod n]PID_{U}
   \label{equ:Account}
\end{equation}
From Equations \ref{equ:PIDRP}, \ref{equ:PIDU}, and \ref{equ:Account}, it is derived that
\begin{equation}
   Acct =  [t^{-1}utr]G = [ur]G = \mathcal{F}_{Acct\ast}(ID_U, ID_{RP})
   \label{equ:AccountNotChanged}
\end{equation}


With the help of $t$, the visited RP derives an identical account for a user in her different logins. It is exactly the user's \emph{permanent} account at this RP.

A user's identity $u$ is unknown to all entities except the honest IdP; otherwise, colluding RPs could calculate $[u]ID_{RP_j}$s for any known $u$ and link these accounts.
Meanwhile, $ID_{RP} = [r]G$ and $Acct = [ID_U]ID_{RP}$ are publicly-known,
 but $r$ is always kept secret;
otherwise, two colluding RPs with $ID_{RP_j} = [r]G$ and $ID_{RP_{j'}} = [r']G$ could link a user's accounts by checking whether $[r']Acct_j$ is equal to $[r]Acct_{j'}$ or not.

\subsection{The Designs Specific for Web Applications}
\label{sec:web-design}

%The user's operations, e.g., request redirection, authorization, and token forwarding, are performed by user agents, such as a browser for web applications.

In commonly-used SSO protocols \cite{OpenIDConnect,rfc6749, SAML, SAMLIdentifier},
an IdP needs to know the visited RP to ensure confidentiality of identity tokens. For instance, in OIDC services with redirect UX, an RP's endpoint to receive tokens is stored as the \texttt{redirect\_uri} parameter at the IdP.
The IdP employs HTTP 302 redirection to send tokens to the RP, by setting this parameter as the target URL in the HTTP response to a user's identity-token request \cite{OpenIDConnect}.
  % so the user agent (i.e., browser) forwards it to the designated RP.
Alternatively, in an OIDC system with pop-up UX  \cite{dimvaLiM16,GoogleIdIntegrate,uber}, the user-i script confirms the target RP's origin with the IdP, and then forwards tokens to the user-r script restricted by this origin.

In \usso\ the IdP does not know about the visited RP, requiring a user agent by itself to calculate $PID_{RP}$ and forward identity tokens to the RP.
We implement \usso\ compatibly in OIDC services with pop-up UX \cite{GoogleIdIntegrate,uber},
 and the user-agent functions are also implemented by the two scripts,
%        i.e., the user-i script and the user-r script,
    downloaded from the IdP and the visited RP, respectively.
%The scripts are responsible for communicating with the origin web servers and also the cross-origin communications within the browser.
% as most browsers do in secure OIDC settings \cite{de2014oauth,GoogleIdIntegrate}.
%\footnote{The user-i script is recommended in OIDC services for an in-context user experience of attribute authorization \cite{GoogleIdIntegrate,uber}, so the authorization of user attributes is also implemented through this user-i script in \usso.}
The user-i script is necessary in \usso\ and then it cannot be integrated in OIDC services with redirect UX, 
because the user-r script could leak its origin to the IdP web server (i.e., break IdP untraceability) due to the automatic inclusion of an HTTP \texttt{referer} header in HTTP requests it sends.


In original OIDC systems with pop-up UX,
    the user-i script implements attribute authorization and token receiving,
        while the user-r script receives token requests from the visited RP and forwards tokens to the target RP.
We improve these scripts as below to further support identity transformations.
In each login, the user-r script prepares $ID_{RP}$ and $Enpt_{RP}$ %for the user-i script,
 through an RP certificate, returned in the token request. %It binds the RP's identity and endpoint. %(i.e., $ID_{RP}$ and $Enpt_{RP}$).
The user-r script sends the certificate to the user-i script, which verifies it to extract $ID_{RP}$ and $Enpt_{RP}$.
Note that in \usso\ a user configures nothing locally for the IdP's public key is set in the user-i script, like in other popular SSO systems \cite{OpenIDConnect, rfc6749, SAML, SAMLIdentifier}.
Then, within a browser the user-i script calculates $PID_{RP} = [t]ID_{RP}$ based on $ID_{RP}$ extracted from the verified RP certificate, % that binds $ID_{RP}$ and $Enpt_{RP}$.
%This should be conducted by an \emph{honest} script that obtains $ID_{RP}$ correctly
%    and does not leak $t$, from which it could calculate $ID_{RP} = [t^{-1}\bmod n]PID_{RP}$.
 %otherwise, it could directly leak the RP's domain to the IdP.\footnote{To eliminate this assumption, we can implement the user-agent functions with trusted browser extensions, which are installed before a user visits RPs.}
to request a token for $PID_{RP}$.
On receiving an identity-token request, the IdP web server checks the included \texttt{referer} header to ensure it is sent by the user-i script.


After receiving a token from the IdP, the user-i script needs to ensure the user-r script will forward the token to $Enpt_{RP}$ %, which is bound with $ID_{RP}$
specified in the verified RP certificate.
As the communications between scripts occurs within COTS browsers using the \verb+postMessage+ HTML5 API, %To avoid the honest user sending the identity token to an adversary,
the \verb+postMessage+ targetOrigin mechanism \cite{postm-targeto} restricts the recipient. %(i.e., the RP script).
When the user-i script sends messages, the recipient's origin is set as a parameter, e.g., \verb+postMessage(tk, 'https://RP.com')+, including the protocol (i.e., \verb+https+), the domain (i.e., \verb+RP.com+), and a port if applicable.
Only the script downloaded from this targetOrigin is a legitimate recipient.
This design is commonly used to correctly forward tokens in popular OIDC systems \cite{SPRESSO,MITREid,BrowserID,de2014oauth,OpenIDConnect}.

%The \emph{RP certificates} deals with the problem of mapping an identity proof with its targeting RP.
%That is, the IdP script derives the RP's $ID_{RP}$ and origin from the RP certificate, while the $PID_{RP}$ is generated with this $ID_{RP}$. Thus, the IdP script always knows the targeting RP of identity proof, therefore, the \verb+postMessage+ mechanism can guarantee that the identity proof would not be sent to the adversary.


Besides, when a user attempts to visit an RP in \usso\ (and OIDC systems with pop-up UX), the RP window downloads the user-r script, which in turn ``pops up'' a new browser window to access the IdP and download the user-i script.
To prevent referer leakage during this access, we need to ensure the HTTP request does not carry a \texttt{referer} header, which reveals the visited RP's domain to the IdP server.\footnote{In original OIDC services with pop-up UX this leakage commonly exists, but does not matter because such services are not designed to prevent IdP-based login tracing.} %Generally, when a browser window visits another website not belonging to its opener's origin, the HTTP request to this website automatically carries a \texttt{referer} header (i.e., the opener's origin). Such an HTTP header leaks the visited RP's domain to the IdP.
Fortunately, in \usso\ this pop-up is an HTTP redirection from the new window to the IdP (Steps 1.2-1.3 in Figure \ref{fig:process}), but not a direct visit.
The HTTP redirection response from the RP server includes a \texttt{referrer-policy=no-referrer} header, which ensures that the HTTP request to access the IdP carries no \texttt{referer} header.
This method is specified by W3C \cite{referer_policy} and widely supported. We have tested it in various browsers such as Chrome, Safari, Edge, Opera, and Firefox, and confirmed no referer leakage.

\begin{figure*}[htb]
  \centering
  \includegraphics[height=0.491\textheight]{fig/process-js.pdf}
  \caption{The SSO login flow of \usso}
  \label{fig:process}
\end{figure*}



\subsection{The \usso\ protocol}
\label{implementations}

\noindent \textbf{System Initialization.}
$\mathbb{E}$, $G$ and $n$ are set up and publicly published.
An IdP generates a key pair ($SK$, $PK$) to sign and verify identity tokens and RP certificates.
%The IdP keeps $SK$ secret, and $PK$ is publicly known.

\vspace{1mm}
\noindent\textbf{RP Registration.}
Each RP registers itself at the IdP to obtain $ID_{RP}$ and its RP certificate $Cert_{RP}$ as follows.
This registration may be conducted by face-to-face means.
\vspace{-\topsep}\begin{enumerate}
\setlength{\topsep}{0pt}
\setlength{\partopsep}{0pt}
\setlength{\itemsep}{0pt}
\setlength{\parsep}{0pt}
\setlength{\parskip}{0pt}
\item[1.]
An RP pre-installs $PK$ by trusted means.
It sends a registration request, including the endpoint to receive identity tokens and other information.
\item[2.]
After examining the request,
the IdP randomly selects $r \in \mathbb{Z}_n$
        and assigns a \emph{unique} point $[r]G$ to the RP as its identity.
Note that $r$ is not processed any more and then known to \emph{nobody}
 due to the elliptic curve discrete logarithm problem (ECDLP).
%random number $r \in [1,n)$ until $ID_{RP} = [r]G$ is unique.
 %   but $r$ is kept unknown to the RP.
The IdP then signs $Cert_{RP} = [ID_{RP}, Enpt_{RP}, *]_{SK}$,
     where $[\cdot]_{SK}$ is a message signed using $SK$ and $*$ is supplementary information such as the RP's common name.
\item[3.]
The RP verifies $Cert_{RP}$ using $PK$, and accepts $ID_{RP}$ and $Cert_{RP}$ if they are valid.
\end{enumerate}

%\vspace{0.5mm}
\noindent\textbf{User Registration.}
%Each user registers once at the IdP. 
Each user sets up her unique username and the corresponding credential for the IdP.
The IdP assigns
a unique random identity $ID_U = u\in \mathbb{Z}_n$ to the user.

It is required that $ID_U$ is known \emph{only} to the IdP.
$ID_U$ is used only by the IdP \emph{internally},
 not enclosed in any message.
%so a user inputs her username in the authentication to the IdP and $ID_U$ is processed only by the IdP internally.
For example, a user's identity is generated and always restored by hashing her username concatenated with the IdP's private key.
Then,
 $ID_U$ is never stored in hard disks,
 and protected almost the same as the IdP's private key
because it is \emph{only} used to calculate $PID_{U}$ as the IdP is signing a token binding $PID_{U}=
  [{ID_U}]{PID_{RP}}$.

\vspace{1mm}
\noindent\textbf{SSO Login.} 
A registered user may log into any registered RP,
    after authenticated herself to the IdP.
A login flow %is typically launched through a browser,
%when a user attempts to visit an RP. It
involves four steps: script downloading, RP identity transformation, identity-token generation, and $Acct$ calculation. In Figure \ref{fig:process}, the IdP's and RP's operations are connected by two vertical lines, respectively. The user operations are split into two groups in different browser windows by vertical lines, one communicating with the IdP and the other with the RP. Solid horizontal lines indicate messages exchanged between a user and the IdP (or the RP), while dotted lines represent \verb+postMessage+ invocations between two scripts (or browser windows) within a browser.


\vspace{1mm}
\noindent 1. {\em Script Downloading.}
The browser downloads user-agent scripts from the visited RP and the IdP as below.
\vspace{-\topsep}
\begin{itemize}
\setlength{\topsep}{0pt}
\setlength{\partopsep}{0pt}
\setlength{\itemsep}{0pt}
\setlength{\parsep}{0pt}
\setlength{\parskip}{0pt}
\item[1.1]
When requesting any protected resources at the RP, the user downloads the user-r script.
\item[1.2]
The user-r script opens a window in the browser to visit the RP's login path, which is then redirected to the IdP.
\item[1.3]
The redirection to the IdP downloads the user-i script.
\end{itemize}


\noindent 2. {\em RP Identity Transformation.}
The user and the RP negotiate $PID_{RP} = [t]{ID_{RP}}$.
\vspace{-\topsep}
\begin{itemize}
\setlength{\topsep}{0pt}
\setlength{\partopsep}{0pt}
\setlength{\itemsep}{0pt}
\setlength{\parsep}{0pt}
\setlength{\parskip}{0pt}
\item[2.1] The user-i script chooses a random number $t \in \mathbb{Z}_n$ and sends it to the user-r script through \verb+postMessage+. The user-r script then forwards $t$ to the RP.
\item[2.2] The RP verifies if the received $t$ is an integer in $\mathbb{Z}_n$, and
%Upon receiving $t$, the RP verifies $1 \leq t < n$ and %calculates $PID_{RP}$.
%To acknowledge the negotiation of $PID_{RP}$, The RP
replies with $Cert_{RP}$ and the scope of requested attributes, forwarded by the user-r script to the user-i script.  % through \verb+postMessage+.
\item[2.3] The user-i script verifies $Cert_{RP}$, extracts $ID_{RP}$ and $Enpt_{RP}$ from $Cert_{RP}$, and calculates $PID_{RP}=[t]{ID_{RP}}$.

\end{itemize}


%\vspace{1mm}
\noindent 3. {\em Identity-Token Generation.}
The IdP calculates $PID_U = [ID_U]{PID_{RP}}$ and signs an identity token as follows. % The processes are as follows.
\vspace{-\topsep}
\begin{itemize}
\setlength{\topsep}{0pt}
\setlength{\partopsep}{0pt}
\setlength{\itemsep}{0pt}
\setlength{\parsep}{0pt}
\setlength{\parskip}{0pt}
\item[3.1]
The user-i script sends an identity-token request for $PID_{RP}$ on behalf of the user. %and the user attributes.
 %by checking whether this user is authenticated by IdP.

\item[3.2] The IdP authenticates the user, if not authenticated yet.

\item [3.3]
The user-i script displays the RP's common name,
 locally obtains the user's authorization for the requested attributes,
 and then sends the scope of the authorized attributes.
The IdP checks that $PID_{RP}$ is a point on $\mathbb{E}$,
calculates $PID_U = [ID_U]{PID_{RP}}$, and signs $[PID_{RP}, PID_U, Issuer, Validity, Attr]_{SK}$, where $Issuer$ is the IdP, $Validity$ indicates the validity period, and $Attr$ contains the authorized user attributes.
\item[3.4] The IdP replies with the identity token to the user-i script.
\end{itemize}

%\vspace{1mm}
\noindent 4. {\em $Acct$ Calculation.}
The RP receives the identity token and allows the user to log in.
\vspace{-\topsep}
\begin{itemize}
\setlength{\topsep}{0pt}
\setlength{\partopsep}{0pt}
\setlength{\itemsep}{0pt}
\setlength{\parsep}{0pt}
\setlength{\parskip}{0pt}
\item [4.1]
The user-i script forwards the identity token to the user-r script,
    which then sends it to the RP through $Enpt_{RP}$.
\item[4.2] The RP verifies the signature and the validity period of the token 
%Then, the RP extracts $PID_{RP}$ from the token, checks if it equals $[t]ID_{RP}$,
and calculates $Acct = [t^{-1}]{PID_U}$.

\item [4.3] The RP allows the user to log in as $Acct$.

\end{itemize}


If any verification fails, this flow will be terminated immediately.
For example, the user halts it when receiving an invalid $Cert_{RP}$.
The IdP rejects an identity-token request in Step 3.3 if the received $PID_{RP}$ is not a point on $\mathbb{E}$, and the RP rejects a token in Step 4.2 if the signature is invalid. 

%%%%%%%%%%%%%%%%%%%%%%%%%%%%%%%%%%%%%%%%%%%%%%%%%%%%%%%%%%
% u泄露,则RP有必要比较!否则,就会伤害security。
%or the enclosed $PID_{RP}$ is not equal to $[t]ID_{RP}$.
%%%%%%%%%%%%%%%%%%%%%%%%%%%%%%%%%%%%%%%%%%%%%%%%%%%%%%%%%%
% 分析结论是:
% 现有的简化协议(RP不比较$PID_{RP}$是否相等),则:
% 当u不泄露,u的security和privacy都没有问题;
% 当u泄露,u的security和privacy都有问题。
%
% 改进协议(RP比较$PID_{RP}$是否相等),则:
% 当u不泄露,u的security和privacy都没有问题;
% 当u泄露,u的security没有问题,u的privacy有问题。
%%%%%%%%%%%%%%%%%%%%%%%%%%%%%%%%%%%%%%%%%%%%%%%%%%%%%%%%%%

\subsection{Compatibility with OIDC}
\label{subsec:compatible}

An OIDC system works in two alternative modes \cite{dimvaLiM16,GoogleIdIntegrate,uber,de2014oauth}:
 pop-up UX with two web scripts, and redirect UX with only the user-r script.
\usso\ only works in OIDC systems with pop-up UX. %(i.e., the communication flows among the IdP, the browser and the visited RP are identical),
  %  while the downloaded scripts are modified to support identity transformations.
%Both \usso\ and OIDC work with COTS browsers. %Among the four steps of the login flow
%
When a user attempts to visit an RP in \usso,
        a user agent is prepared by two scripts in the \emph{script downloading} step.
These scripts are downloaded, % following the same flow 
like in OIDC systems with pop-up UX.
%In \usso\ two scripts implement the verifications of RP certificates 
%  and the user operations of identity transformations, in addition to attribute authorization and token receiving by the user-i script and token forwarding by the user-r script in original OIDC services.
%
%\usso\ employs web scripts to hide the RP's endpoint from the IdP, while securely forwarding identity tokens to the RP through $Enpt_{RP}$ extracted from the signed RP certificate.
%Thus, the IdP cannot set \texttt{redirect_uri} in the HTTP responses, which is different from OIDC where HTTP redirections are used to implement these communications.
%

The \emph{RP identity transformation} step in \usso\ take place within a browser,
            and the RP responds with its certificate after receiving $t$.
The RP response is constructed following the standard format of OIDC requests \cite{dimvaLiM16},
    but then processed by the user-i script to remove any information leaking the RP's identity.
%which can be viewed as a supplementary access to static resources on the RP server.
%The operations in the $PID_{RP}$ registration are almost identical to those in the RP Dynamic Registration of OIDC \cite{DynamicRegistration}, except that in OIDC the IdP assigns the RP's identity  while in UPPRESSO this (pseudo-)identity is generated by the registered entity. Besides, the $PID_{RP}$ registration has a validity period.

The operations of \emph{identity-token generation} in \usso\ are compatible with those in OIDC,
 because (\emph{a}) the calculation of $PID_U$ is actually a special method to generate PPIDs and (\emph{b}) $PID_{RP}$ can be viewed as a ``dynamic'' RP pseudo-identity
    so that \texttt{redirect\_uri} is missing in the responses by the IdP.
%
%Compared to the original OIDC protocol, \usso\ simplifies the IdP's operations in these two steps, while allowing ``dynamic'' RP pseudo-identities.
Finally, the operations of \emph{$Acct$ calculation} in \usso\ are identical to those in OIDC,
 because the calculation of $Acct$ is a mapping from the user pseudo-identity enclosed in tokens to a local account at the visited RP.

The compatibility is experimentally confirmed through our prototype implementation.
When we built the IdP prototype of \usso\ on top of MITREid Connect \cite{MITREid},
    only 23 lines of Java code are added:
        three lines to calculate $PID_U$ as a special PPID using Java cryptographic libraries,
    and 20 lines to support pop-up UX with identity transformations.
%    modify the method of forwarding identity tokens.

\section{Security and Privacy Analysis}
\label{sec:analysis}

%We define three adversarial scenarios under the threat model in Section \ref{subsec:threatmodel}, develop a Dolev-Yao style model \cite{BrowserID} to formalize the SSO login flow in \usso, and then integrate its conclusions to formally prove the security and privacy guarantees provided by \usso.

We formally prove the security and privacy guarantees provided by \usso.
It is worth noting that these guarantees are proved against different adversaries.


% \subsection{Adversarial Scenarios}

% Based on our design goals (i.e., the desired security and privacy guarantees) and the potential adversaries discussed in Section \ref{subsec:threatmodel}, we consider three adversarial scenarios as below.

% \noindent\textbf{Adversaries against security.} Malicious users could collude with each other and even with malicious RPs, attempting to (\emph{a}) impersonate an honest user to log into an honest RP or (\emph{b}) entice an honest user to log into an honest RP under a malicious user's account.

% \noindent\textbf{Adversaries against IdP untraceability.}
% The honest-but-curious IdP tries to infer the identities of the RPs that a user requests to access. %or link multiple logins to any RP initiated by a user.

% \noindent\textbf{Adversaries against RP unlinkability.}
% Malicious RPs could collude with each other and even with malicious users, attempting to link logins across these RPs that are initiated by honest users. 
%\oldc


% \subsection{The Dolev-Yao Style Model for \usso}
% \label{dy-model}

% We develop a Dolev-Yao style model \cite{BrowserID, SPRESSO, FettKS16, FettKS17} for \usso, referred to as the \emph{\dyu\ model}, to formalize the login flow of \usso.
% % Dolev-Yao style models abstract cryptographic concepts into an algebra of symbolic messages to discover structural flaws using simple formal logic. % which has been used in the formal analysis of SSO protocols such as OAuth 2.0 \cite{FettKS16} and OIDC \cite{FettKS17}.
% The model abstracts the entities in a web system, such as web servers and browsers, as atomic processes, %which communicate with each other through events. % such as HTTPS requests and responses.
% and defines scripting processes to formulate client-side scripts.
% %The script is dependently invoked by the browser to process the server-defined logic.%such as verifying $Certificate_{RP}$. %postmessage events; %atomic process <-> scripting process, communication. %Other events change self-trigger.
% The atomic processes of \usso\ include an IdP process, a finite set of web servers for honest RPs, a finite set of honest browsers, and a finite set of attacker processes that model malicious RPs and malicious users.
% A browser may invoke an honest IdP script and multiple RP scripts that could be honest or malicious.
% The processes communicate with each other through events such as HTTPS requests and responses,
% %Although the scripts coexist in the same browser, they are strictly separated.
% except that the scripting processes communicate with each other through \verb+postMessage+ which are modeled as transmitted-to-itself events of a browser process.
% %To clearly indicate the action of postMessage communication, we define it as the transmitting-to-itself event of the browser (which is not defined in SPRESSO).


% Applying the \dyu\ model, we trace the lifecycle of an identity token from its generation at the IdP to its acceptance at an RP, locate the places where $PID_U$, $PID_{RP}$, and other elements related to the identity token such as $t$ and $u$ are processed, and locate the places where $PK$ is transmitted and used in the IdP script.
% We confirm the following conclusions in the \dyu\ model:
% (\emph{a}) an identity token binding pseudo-identities of honest entities, cannot be leaked to any malicious process;
% (\emph{b}) pseudo-identities and other elements in verified identity tokens cannot be manipulated by any malicious process;
% (\emph{c}) the IdP's public key set in the IdP script cannot be replaced or tampered with by any malicious process, within an honest browser;
% (\emph{d}) the IdP receives nothing about $t$ shared between two honest processes;
% (\emph{e}) $r$ is not leaked to any malicious process as it never leaves the IdP;
% and (\emph{f}) the RPs cannot receive anything about $u$ shared between two honest processes.


\subsection{Security}
\label{analysis-security}

Based on the threat model in Section \ref{subsec:threatmodel}, the adversary against security w.r.t. authentication \cite{FettKS14,BrowserID,SPRESSO} compromises some users, colluding with malicious RPs, and attempts to (\emph{a}) impersonate an honest user to log into an honest RP (i.e., \emph{impersonation}), or (\emph{b}) break the login of an honest user to an honest RP so that the honest user is authenticated as another user (i.e., \emph{identity injection}). 

%could collude with each other and even with malicious RPs, attempting to (\emph{a}) impersonate an honest user to log into an honest RP or (\emph{b}) entice an honest user to log into an honest RP under a malicious user's account.


For the analysis of security w.r.t. authentication, we develop a formal model of \usso\ web systems (denoted as $\mathcal{U\!W\!S}^{Auth}$), which is defined in the appendix. It contains an IdP, a finite set of RPs, a finite set of users (i.e., browsers), and a network attacker. The attacker subsumes all web attackers and is assumed to be able to  corrupt a set of users and RPs.


%------- new Theorem 5, move to the beginning of the proof ---------
\begin{thm}
\textsc{(Security)} \emph{\usso\ provides a secure SSO service w.r.t. authentication}.
\label{thm-security}
\end{thm}

\noindent\textbf{\textsc{Proof.}} Every SSO system should satisfy two fundamental properties that are deduced from \cite{FettKS14} and summarized in \cite{BrowserID,SPRESSO}: {\bf (A)} the attacker should not be able to access services of an honest RP as an honest user, and {\bf (B)} the attacker should not be able to authenticate an honest user to an honest RP on behalf of another user, which are formally defined in Definitions 2 and 3 in Appendix \ref{appendix-security}. For the proofs, we demonstrate that both properties are fulfilled for every $\mathcal{UWS}^{Auth}$. 

Before proving Properties {\bf A} and {\bf B}, we first show that a network attacker cannot obtain any information from or manipulate the identity token request from a user (browser) to the IdP, referred to as Property {\bf C1}. Meanwhile, a network attacker cannot obtain any information from or alter the identity token from the IdP to the target RP through the user (browser), referred to as Property {\bf C2}. We adopted the general web system properties shown in the full version of \cite{FettKS14} and followed the approaches in \cite{BrowserID,SPRESSO}, which are based on a Dolev-Yao-style model, and prove them with Lemmas 2-6 in Appendix~\ref{appendix-security}. 

To prove Property {\bf A}, we show that the attacker cannot obtain any useful information from any login of an honest user to an honest RP, which would allow him to impersonate an honest user at some honest RP. In particular, he cannot know the identity-token request (and its $PID_{RP}$) according to Property {\bf C1} nor a valid identity token of the honest user if they are not initiated by a corrupted user or intended for a corrupted RP, according to Property {\bf C1}.  

Meanwhile, we prove that Property {\bf A} is still fulfilled in the following two adversarial scenarios. First, consider a malicious user initiating a login to an honest RP. He selects a random $t$ to share with this RP and then manipulates $PID_{RP}$ to be $[t']ID_{RP}$ instead of $[t]ID_{RP}$ when requesting for an identity token. We prove in Theorem \ref{thm-user-id} that, by showing the identity token with a manipulated $PID_{RP}$ to this honest RP, the malicious user cannot be recognized under the account of any existing honest user at this RP (i.e., {\em User Identification}). Secondly, we consider a malicious $RP$ sharing with a malicious user $U$ the identity token $TK$ that it received from an honest user $U'$. With $TK$, if the malicious $RP$ is able to find $t'$ satisfying $[t']ID_{RP'}=[t]ID_{RP}$ for any honest $RP'$, it could share $TK$ and $t'$ with the malicious user, enabling him to log into $RP'$ on behalf of $U$. We prove in Lemma \ref{lemma-rp-main} and Theorem \ref{thm-rp-designation} that a malicious RP receiving $t$ and $[t]ID_{RP}$ for $U$'s login cannot find such $t'$, since $PID_{RP}$ in $TK$ is uniquely associated with $t$ and the login to the designated RP (i.e., {\em RP Designation}).

%a network attacker cannot alter: (1) communications between an honest user and an honest RP through restricted \verb+postMessage+ messages between the honest IdP script and RP script (within the user's browser) and obtain or manipulate the secret element $t$;  

To prove property {\bf B}, we first prove in Theorem \ref{thm-user-id}: when an honest user logs into an honest RP, the identity token $TK$ that binds $PID_U$ and $PID_{RP}$ uniquely identifies a user account for this user at the designated RP (i.e., {\em User Identification}). Therefore, an honest user would not be authenticated by an honest RP as any other users in the system.

Meanwhile, we show that the attacker cannot manipulate any login of an honest user to an honest RP in a way that the honest user is mistakenly authenticated as another user. 
%In particular, we show a property of $\mathcal{UWS}^{Auth}$, i.e., a network attacker cannot tamper with $ID_{RP}$ extracted from the RP's certificate. 
In particular, we show that the network attacker cannot tamper with identity token requesting (i.e., Property \textbf{C1} and Lemma 7 in Appendix~\ref{appendix-security}), and therefore he cannot manipulate $ID_{RP}$, which is extracted from the RP's certificate, and $PID_{RP}$, which is computed as $PID_{RP}=[t]ID_{RP}$, to trigger a token for another user while returning it to the victim user. Besides, the attacker cannot manipulate the returned identity token directly based on Property \textbf{C2}. 
%Then, we show that the attacker cannot manipulate $PID_{RP}$ in the identity token request of the honest user (i.e., Property \textbf{C1}), nor the identity token $TK$ issued by the IdP to the honest user and the target honest RP (i.e., Property \textbf{C2}); that is,
%\emph{Token Integrity} is ensured. 
\hfill $\square$

%------- old Theorem 5 ---------
\begin{comment}
\begin{thm}
\textsc{(Security)} \emph{\usso\ provides secure authentication services.}
\end{thm}

\noindent\textbf{\textsc{Proof.}}
According to the formal analysis on SSO security \cite{SPRESSO, FettKS14},
    an SSO system satisfying the following two requirements provides \emph{secure} authentication services in the first adversarial scenario: (\emph{a}) an adversary never obtains a valid identity token issued for an honest user and presents it to an honest RP, and (\emph{b}) an honest user never presents a valid identity token that is not issued for herself to an honest RP.

Following the login flow of \usso, because the integrity and confidentiality of identity tokens are satisfied in Theorems \ref{thm-integrity} and \ref{thm-confidentiality},
 no adversary could obtain a valid identity token that is issued to an honest user for accessing an honest RP. %and accepted by an honest RP.
Meanwhile, if an adversary presents an identity token to an honest RP, which may be issued to the adversary for accessing another RP or to an honest user for accessing a colluding RP, RP designation and user identification, proved in Theorems \ref{thm-rp-designation} and \ref{thm-user-id}, ensure that the honest RP does not derive an honest user's account from this identity token.
%Meanwhile, $PID_U$ can only be calculated by the IdP and the user, since no one else knows or could intercept $u$ according to the DYU model. \oldc

According to the login flow and the \dyu\ model, an honest user always obtains identity tokens issued to herself, because $PID_U$ is calculated based on her identity. 
%% which is protected against adversaries. 这一句,对于security没有用,protecting u是用于privacy。
The confidentiality and integrity of identity tokens ensure that an honest user never presents any token issued for someone else to an honest RP. Finally, with user identification,\footnote{\newc RP designation is a precondition of user identification in \usso, but this is not always necessary for other SSO systems.} her identity tokens are always associated with her account at the target RP.
%According to the conclusions of the DYU model, $PID_U$ is calculated based on the user when the IdP issues an identity token for an authenticated user.
%Moreover, because the confidentiality and integrity of identity tokens are satisfied, an honest user never presents a valid token that is issued to other users.
%When the user identification is satisfied, this token always derives this honest user's account.
%So, an honest user never presents an identity token that is accepted by an honest RP to derive another user’s account at this RP.

Thus, \usso\ is a secure SSO system.
\hfill $\square$

%Finally, according to Theorems 1, 2, 3, 4, and 5, UPPRESSO is secure.

\end{comment}


%We prove that identity tokens in \usso\ and the enclosed pseudo-identities satisfy four properties, namely, \emph{RP designation}, \emph{user identification}, \emph{token confidentiality}, and \emph{token integrity}, which together ensure the security of \usso\ in the first adversarial scenario.

\vspace{1.5mm}
Next, we prove that \usso\ provides the ``RP Designation'' and ``User Identification'' properties that are used in the proof of Theorem \ref{thm-security}. Let us consider an identity token $TK$ binding $PID_{RP}$ and $PID_U$, which is generated by the IdP upon a request from an authenticated user with $ID_U$.

\begin{lemma}
\textsc{(RP Designation of $TK$)}  \emph{$PID_{RP}= [t]ID_{RP}$ in $TK$ uniquely designates only the RP with $ID_{RP}$ and receiving $t$ in \usso, where $t \in [1,n)$, $ID_{RP} = [r]G$, $r$ is a unknown random number, and $G$ is a generator on $\mathbb{E}$ of order $n$.}
\label{lemma-rp-designation}
\end{lemma}

\noindent\textbf{\textsc{Proof.}} In \usso, $PID_{RP}=[t]ID_{RP}$ is generated by a user based on the target RP's identity $ID_{RP}$ and a user-selected random number $t \in [1,n)$.
%The target RP with $ID_{RP}$ receives $t$, and it will also calculate $PID_{RP}=[t]ID_{RP}$ to match $PID_{RP}$ extracted from a token received.
%It is computationally easy for any party who knows $ID_{RP}$ and $t$ to validate the $PID_{RP}$ in an identity token. A valid
Thus, $PID_{RP}$ specifies an RP, i.e., %$PID_{RP}$ sent by a user in her identity-token request is calculated as $PID_{RP} = [t]ID_{RP}$, where $ID_{RP}$ is the target RP's identity and $t$ is a random number selected by the user and shared with this RP.
designates the target RP that knows $t$. 

Next, we prove that, given $PID_{RP} = [t]ID_{RP}$, the probability that $PID_{RP}$ designates another RP with $ID_{RP'}$ is \emph{negligible}, %This means that $PID_{RP}$ cannot be associated with any other RPs in the system.
so $PID_{RP}$ designates only the target RP with $ID_{RP}$ in the system.
Finding $t$ and $t'$ that satisfy $[t]ID_{RP_j} = [t']ID_{RP_{j'}}$, can be described as a $PID_{RP}$-collision game $\mathcal{G}_c$ between an adversary and a challenger: the adversary receives from the challenger a finite set of RP identities, i.e., $ID_{RP_1}$, ..., $ID_{RP_m}$, where $m$ is the  finite number of RPs in the system, and outputs $(a, b, t, t')$ where $a \neq b$. If $[t]ID_{RP_a}=[t']ID_{RP_b}$, which occurs with a probability ${\rm Pr}_s$, the adversary succeeds in this game.
%The attack success probability is defined as ${\rm Pr}_s$.

As depicted in Figure \ref{fig:ecdlp_algorithm}, we then design a probabilistic polynomial time (PPT) algorithm $\mathcal{D}^*_c$ based on $\mathcal{G}_c$, to solve the ECDLP: find a number $x \in \mathbb{Z}_n$ satisfying $Q = [x]G$, where $Q$ is a point on $\mathbb{E}$ and $G$ is a generator on $\mathbb{E}$ of order $n$.

%where ${\rm Pr}\{\}$ denotes the probability.
%where $k$ denotes the security parameter and $\epsilon_{c}(k)$ becomes negligible when $k$ is sufficiently large.
%For any sufficiently large $k$, $m \ll 2^k$ since $m$ is a finite integer.

\begin{figure}[tb]
  \centering
  \includegraphics[width=0.96\linewidth]{fig/ecdlp_algorithm.pdf}
  \caption{The PPT algorithm $\mathcal{D}^*_c$ constructed based on the $PID_{RP}$ collision game to solve the ECDLP.}
  \label{fig:ecdlp_algorithm}
\end{figure}

The algorithm $\mathcal{D}^*_c$ works as below.
The input of $\mathcal{D}^*_c$ is in the form of ($G, Q$). On receiving an input ($G$, $Q$), the challenger first randomly chooses $r_1, \cdots, r_m$ in $\mathbb{Z}_n$ to calculate $[r_1]G, \cdots, [r_m]G$.
Then, it randomly chooses $j \in [1,m]$, replaces $[r_j]G$ with $Q$, and sends $m$ RP identities to the adversary, which returns the result ($a$, $b$, $t$, $t'$). Finally, the challenger calculates $s = t^{-1}t'r_b \bmod n$ and returns $s$ as the output of $\mathcal{D}^*_c$.

If the adversary succeeds in $\mathcal{G}_c$ and $[r_a]G$ happens to be replaced with $Q$, then $\mathcal{D}^*_c$ outputs $s=t^{-1}t'r_b =x$ because $[tr_a]G = [t]Q = [t'r_b]G$. For the adversary, $Q$ is indistinguishable from any other RP identities in the input set, as $[r_j]G$ is randomly replaced by the challenger.
Hence, the probability of solving the ECDLP using $\mathcal{D}^*_c$ is formulated as:
\begin{equation*}
{\rm Pr}\{\mathcal{D}^*_c(G, [x]G)=x\} = {\rm Pr}\{s = x\}={\rm Pr}\{a=j\}{\rm Pr}_s=\frac{1}{m}{\rm Pr}_s
\end{equation*}

If the probability of finding $t$ and $t'$ satisfying $[t]ID_{RP_j} = [t']ID_{RP_{j'}}$ is non-negligible, the adversary would have advantages  in $\mathcal{G}_c$ and ${\rm Pr}_s$ is non-negligible regardless of the security parameter $\lambda$.
Thus, we would find that ${\rm Pr}\{\mathcal{D}^*_c(G, [x]G)=x\}$ also becomes non-negligible even when $\lambda$ is sufficiently large, because $m$ is a finite integer and $m \ll 2^\lambda$.
This violates the ECDLP assumption. Therefore, the probability of finding $t$ and $t'$ that satisfy $[t]ID_{RP_j} = [t']ID_{RP_{j'}}$ is negligible.
\hfill $\square$


%\begin{lemma}
%\emph{Given any two RPs in a finite set of RPs in \usso,
% the probability that an adversary finds different numbers $t$ and $t' \in [1,n)$ satisfying $[t]ID_{RP_j} = [t']ID_{RP_{j'}}$ is negligible, where $ID_{RP_j}=[r]G$, $ID_{RP_{j'}}=[r']G$, $r$ and $r'$ are different numbers unknown to the adversary, and $G$ is a generator on $\mathbb{E}$ of order $n$.}
%\label{lemma-rp-main}
%\end{lemma}





\begin{lemma}
\textsc{(User Identification of $TK$)} \emph{$PID_U= [ID_U]PID_{RP}$ in $TK$ uniquely identifies an account at the RP designated by $PID_{RP} = [t]ID_{RP}$, and this account is uniquely mapped to a user with $ID_U$.}
\label{lemma-user-id}
\end{lemma}

\noindent\textbf{\textsc{Proof.}}
To issue an identity token requested for $PID_{RP}$, the honest IdP authenticates the user with $ID_U$ and calculates $PID_U = [ID_U]PID_{RP}$.
The RP designated by $PID_{RP}$ receives $t$ from the user,
    and calculates $Acct = [t^{-1}]PID_{U} = [ID_U]ID_{RP}$, which is a \emph{permanent} identifier determined by $ID_U$ and $ID_{RP}$.
$ID_{RP} = [r]G$ is also a generator on $\mathbb{E}$ of order $n$, as $\mathbb{E}$ is a finite cyclic group. Therefore, given a user with $ID_U$, $Acct$ is a \emph{unique} point on $\mathbb{E}$ for any $u \in [1, n)$, and it is \emph{uniquely} associated with $ID_U=u$. 
%We first prove that $PID_{U}$ \emph{uniquely} identifies one account at the designated RP and one user in the system. 
This proves that $PID_U$ in $TK$ identifies an account $Acct$ at the designated RP, which is uniquely mapped to a user with $ID_U$ in the system.

Next, we consider adversarial scenarios where the attacker replays $TK$ for another user,
 to (\emph{a}) the designated RP but receiving $t'\neq t$ in this login, and (\emph{b}) any other honest RP with $ID_{RP'} = [r']G \neq [r]G$ (i.e., $r' \neq r$). In the first case, the designated RP calculates an account as $[t'^{-1}]PID_U = [t'^{-1}ut]ID_{RP}$.
Because a user's identity is randomly selected by the IdP in $\mathbb{Z}_n$ and known only to the honest IdP, the probability that $t'^{-1}ut$ happens to be the identity of another user is negligible, when $n$ is sufficiently large. As a result, $[t'^{-1}]PID_U$ does not identify any existing account at the RP and therefore will be treated as a new account. 
Secondly, the attacker presents $TK$ to another RP with $ID_{RP'} = [r']G \neq [r]G$.
This RP will calculate the account as $Acct' = [\tilde{t}^{-1}]PID_{U} = [\tilde{t}^{-1}utrr'^{-1}][r']G = [\tilde{t}^{-1}utrr'^{-1}]ID_{RP'}$. The probability that $\tilde{t}^{-1}utrr'^{-1}$ happens to be the identity of another registered user at this RP is also negligible, when $n$ is sufficiently large. 
%it identifies no account mapped to a user, at any RP not designated by $PID_{RP}$.
\hfill $\square$


Because user identification only works at the designated RP and
    no registered user will be derived at other RPs when $ID_U$ is kept unknown to adversaries,
 in Step 4.2 the RP does not need to check whether $PID_{RP}$ in the received token equals to $[t]ID_{RP}$ or not.


%---- remove Theorem 4 and 5 and move the proofs to Appendix
% \begin{thm}
% \textsc{(Token Integrity)} \emph{An identity token issued by the IdP cannot be forged or manipulated.}
% \label{thm-integrity}
% \end{thm}

% \noindent\textbf{\textsc{Proof.}} Identity tokens are generated and signed by the honest IdP using its private key, which is sufficiently protected at the IdP against adversaries.
% %Meanwhile, the IdP's public key $PK$ is sent from the IdP to the RPs.
% %and $PK$ which are pre-installed by an RP cannot be manipulated by adversaries.
% With the pre-installed public key, an RP verifies the identity tokens it receives and rejects any forged or manipulated identity tokens. Finally, according to the conclusions of the DYU model, the elements in a verified token cannot be manipulated by malicious entities.
% \hfill $\square$


% \begin{thm}
% \textsc{(Token Confidentiality)} \emph{An identity token is accessible only to its designated RP, besides the requesting user and the IdP.}
% \label{thm-confidentiality}
% \end{thm}

% \noindent\textbf{\textsc{Proof.}}
% An identity token is generated by the IdP and then sent to the requesting user (i.e., its IdP script).
% The IdP script verifies if $ID_{RP}$ specified in a verified RP certificate is designated by $PID_{RP}$ in the identity token and forwards the token to the correct RP script, which is downloaded from the origin of $Enpt_{RP}$ specified also in the RP certificate. %Because the IdP script calculates $PID_{RP} =[t]ID_{RP}$ based on $ID_{RP}$ in this verified RP certificate, this RP is designated in the identity token and will receive the token from the RP script.
% As all the communications between the IdP, RPs, and users are protected by HTTPS and two scripts communicate with each other through restricted \verb+postMessage+ HTML5 channels, according to the conclusions of the \dyu\ model, the identity tokens cannot be leaked to any other entities. \hfill $\square$

%--- End of Theorem 4 and 5

\subsection{Privacy against IdP-based Login Tracing}
\label{subsec:IdP-privacy}

As depicted in Section \ref{subsec:threatmodel}, the IdP would attempt to infer the RP's identity visited by a user. Since the IdP is considered \emph{honest-but-curious}, it only collects information from the login requests to infer the visited RPs' identities, but does not %tamper with the authentication process or
 deviate from the protocol or collude with others.

\begin{thm}
\textsc{(Privacy against the Curious IdP)} \emph{\usso\ protects user privacy against the IdP.}
\label{them-idp-untraceability}
\end{thm}


To show \usso\ preventing such \emph{IdP-based login tracing}, we first prove that in $\mathcal{UWS}^{Auth}$, the IdP cannot obtain any information related to the target RP in each login (e.g., $ID_{RP}$, $t$, RP's $EndPoint$) except its \emph{ephemeral} pseudo-identity $PID_{RP}$ (with Property {\bf C1}). Then, we prove in Theorem \ref{thm-idp-untraceability-main} that with these $PID_{RP}$s in the identity token requests from the users, the IdP cannot (\emph{a}) associate multiple logins to the same RP, or (\emph{b}) distinguish a login to one RP from other logins to another RP, because $PID_{RP}$ is \emph{indistinguishable} from any random variables to the IdP. So, \usso\ prevents \emph{an honest-but-curious IdP} from tracing a user's login activities.

\begin{lemma}
\textsc{(IdP Untraceability)} \emph{In \usso, the IdP cannot distinguish $PID_{RP} = [t]ID_{RP}$ from a random variable on $\mathbb{E}$, where $t$ is random in $\mathbb{Z}_n$ and unknown to the IdP.}
\label{lemma-idp-untraceability}
\end{lemma}

\noindent\textbf{\textsc{Proof.}}
Consider a finite cyclic group $\mathbb{E}$ where the number of points on $\mathbb{E}$ is $n$.
Because $G$ is a generator of order $n$, $ID_{RP} = [r]G$ is also a generator on $\mathbb{E}$ of order $n$. %According to the formal model in Appendix \ref{appendix-model}, 
When $t$ is randomly chosen in $\mathbb{Z}_n$ and kept unknown to the IdP, $PID_{RP} = [t]ID_{RP}$ is \emph{indistinguishable} from a point that is randomly chosen on $\mathbb{E}$ \cite{oprf-proved,voprf-proved}.\footnote{This property has been actually proved in OPRFs \cite{oprf-proved,voprf-proved}, where the OPRF server works similarly to the IdP and learns \emph{nothing} about a client's inputs or outputs of the evaluated pseudo-random function.} \hfill $\square$


\subsection{Privacy against RP-based Identity Linkage}
\label{subsec:RP-privacy}

As depicted in the threat model, malicious RPs,
which could collude with some malicious users,
aim to infer the identities of honest users
    or link an honest user's accounts across these RPs.
They attempt to (1) steal the user's identity $ID_U$ directly,
% (2) steal information from or tamper with pseudo-identity transformations in login sessions,
 or (2) link the pseudo-identities of an honest user that are used in her logins to multiple malicious RPs. 


In each login, an RP obtains \emph{only} a random number $t$ and an identity token enclosing $PID_{RP}$ and $PID_U$. From the token, it learns only an ephemeral pseudo-identifier $PID_{U} = [{ID_U}]{PID_{RP}}$ of the user, from which it deduces a permanent pseudo-identifier $Acct = [ID_U]ID_{RP}$. However, it cannot derive $ID_U$ from $PID_{U}$ or $Acct$ as it is an ECDLP hard problem.

\begin{thm}
\textsc{(Privacy against Colluding RPs)} \emph{\usso\ protects user privacy against the RPs.}
\label{them-rp-unlinkability}
\end{thm}

Next, we show that, with Property {\bf C1}, \usso\ prevents malicious RP(s) from reading or altering the identity token request sent by the IdP script that runs in an honest browser to the IdP. Therefore, the RP obtains no knowledge about the user identity $ID_U$. Meanwhile, following the formal model in Appendix \ref{appendix-model}, we see that the RP script running in an honest browser is always honest, no matter if the RP would be corrupted at a later time or not. A script download from a malicious RP would not be installed by an honest user. 

Finally, we prove in Theorem 5 that even if malicious RPs collude with each other and malicious users by sharing $PID_U$s and other information observed in all the logins, they still cannot link any login from an honest user to any other logins from any other honest users to malicious RPs.

%\usso\ prevents the privacy threats of \emph{RP-based identity linkage} against \emph{malicious RPs which collude with malicious users}, attempting to link the logins initiated by an honest user but visiting different colluding RPs. 
%The IdP does not collude with the adversaries in this scenario.

%Therefore, the colluding adversaries share all logins visiting malicious RPs, and attempt to leverage the logins initiated by malicious users to link the other logins by honest users.


With the trapdoor $t$, $PID_{RP}$ and $PID_U$ can be easily transformed into $ID_{RP}$ and $Acct$, respectively, and vice versa. Therefore, we denote the information that an RP learns from a login as a tuple $L$, where $L =(ID_{RP}, t, Acct)=(ID_{RP}, t, [ID_{U}]ID_{RP})=([r]G, t, [ur]G)$.


%Hence, an $RP_j$ with $w$ logins knows $\{L_{1,j},...,L_{w,j}\}$.

When $c$ malicious RPs collude with each other, they create a shared view of all their logins, denoted as $\mathbb{L}$.
%some of which are initiated by honest users and denoted as $\mathbb{L}^h$, and the others by $v$ malicious users  are $\mathbb{L}^m = \mathbb{L} \setminus \mathbb{L}^h$.
When they collude further with $v$ malicious users, the logins initiated by these malicious users are picked out and linked together as
$\mathfrak{L}^m=\left \{ \begin{matrix}
L^m_{1,1},&L^m_{1,2},&\cdots,&L^m_{1,c}\\
L^m_{2,1},& L^m_{2,2},&\cdots,&L^m_{2,c}\\
\cdots,&\cdots,&L^m_{i,j},&\cdots\\
L^m_{v,1},&L^m_{v,2},&\cdots,&L^m_{v,c}
\end{matrix}\right\}$,
where $L^m_{i, j}=([r_j]G, t_{i,j}, [u_ir_j]G)$ for $1 \le i \le v$ and $1 \le j \le c$, and $L^m_{i,j} \in \mathbb{L}$. Any login in $\mathbb{L}$ but not linked in $\mathfrak{L}^m$ is initiated by an honest user to one of the $c$ malicious RPs.


\begin{lemma}
\textsc{(RP Unlinkability)} \emph{In \usso, given $\mathbb{L}$ and $\mathfrak{L}^m$, $c$ malicious RPs and $v$ malicious users cannot link any login from an honest user to some malicious RP to any subset of logins from honest users to any other malicious RPs.}
\end{lemma}


\noindent\textbf{\textsc{Proof.}} From the logins in $\mathbb{L}$,
 we randomly choose one login $L' \neq L^m_{i,j}$,
 which is from an (unknown) honest user with $ID_{U'}=u'$ to a malicious $RP_a$ and $a \in [1,c]$.
Then, we randomly choose another malicious $RP_b$, where $b \in [1,c]$ and $b \neq a$.
Consider any subset $\mathbb{L}''$ of $w$ logins visiting $RP_b$ by unknown honest users,
 we denote the identities of the honest users who initiate these logins as $\mathbf{u}_w=\{{u''_1}, {u''_2}, \cdots, {u''_w}\}$.
Next, we prove that the colluding adversaries cannot decide if $u'$ is in $\mathbf{u}_w$ or randomly selected from the universal user set.
This indicates the colluding adversaries cannot link $L'$ to another login visiting $RP_b$
    or to another subset of logins visiting $RP_b$.

We define an RP-based linkage game $\mathcal{G}_r$ between an adversary and a challenger, which describes this login linkage privacy threat: the adversary receives $\mathfrak{L}^m$, $L'$, and $\mathbb{L}''$ from the challenger and outputs $s$, where $s = 1$ if it decides $u'$ is in $\mathbf{u}_w$ %$\{{U''_1}, {U''_2}, \cdots, {U''_w}\}$
and $s=0$ if it believes $u'$ is randomly chosen from the universal user set.
Thus, the adversary succeeds in $\mathcal{G}_r$ with an advantage $\mathbf{Adv}$:
\begin{align*}
%&{\rm Pr}_1={\rm Pr}\{\mathcal{G}_r(\mathfrak{L}, L', \mathbb{L}'' | u' \in \{{u''_1}, {u''_2}, \cdots, {u''_w}\})=1\} \\
&{\rm Pr}_1={\rm Pr}(\mathcal{G}_r(\mathfrak{L}^m, L', \mathbb{L}'')=1 \;| \; u' \in \mathbf{u}_w)  \\
%&{\rm Pr}_2={\rm Pr}\{\mathcal{G}_r(\mathfrak{L}, L', \mathbb{L}'' | u' \in \mathbb{Z}_n)=1\}\\
&{\rm Pr}_2={\rm Pr}(\mathcal{G}_r(\mathfrak{L}^m, L', \mathbb{L}'')=1 \; | \; u' \in \mathbb{Z}_n)\\
&{\mathbf{Adv}}=|{\rm Pr}_1-{\rm Pr}_2|
\end{align*}

As depicted in Figure \ref{fig:dalgorithm}, we then design a PPT algorithm $\mathcal{D}^*_r$ based on $\mathcal{G}_r$ to solve the elliptic curve decisional Diffie-Hellman (ECDDH) problem: given $(G, [x]G$, $[y]G$, $[z]G)$, decide whether $z$ is equal to $xy$ or randomly chosen in $\mathbb{Z}_n$, where $G$ is a point on an elliptic curve $\mathbb{E}$ of order $n$, and $x$ and $y$ are integers randomly and independently chosen in $\mathbb{Z}_n$.


\begin{figure}[tb]
  \centering
  \includegraphics[width=1.0\linewidth]{fig/rp-linkage-game.pdf}
  \caption{The PPT algorithm $\mathcal{D}^*_r$ constructed based on the RP-based linkage game to solve the ECDDH problem.}
  \label{fig:dalgorithm}
\end{figure}


The algorithm $\mathcal{D}^*_r$ works as below. (1) Upon receiving an input $(G, Q_1=[x]G, Q_2=[y]G, Q_3=[z]G)$, %of $\mathcal{D}^*_r$
the challenger
chooses random numbers in $\mathbb{Z}_n$ to construct $\{u_i\}$, $\{r_j\}$, and $\{t_{i, j}\}$ for $1 \le i \le v$ and $1 \le j \le c$, with which it assembles $L^m_{i, j}=([r_j]G, t_{i,j}, [u_ir_j]G)$.
In this process, it ensures $[r_{j}]G \neq Q_2$ so that $r_j \neq y$.  % 这个应该反过来讲;因为y是离散对数。
(2) It randomly chooses $a \in [1, c]$ and $t' \in \mathbb{Z}_n$, to assemble $L' = ([r_{a}]G, t', [r_{a}]Q_1) = ([r_{a}]G, t', [xr_{a}]G)$.
(3)
% Here, $L'$ represents the knowledge of the login visiting $RP_{j'}$ by a user with $ID_U = x$.
Next, the challenger randomly chooses $b \in [1, c]$ and $b \neq a$, and replaces $ID_{RP_b}$ with $Q_2 = [y]G$.
Hence, for $1 \le i \le v$, the challenger replaces $L^m_{i, b}=([r_b]G, t_{i,b}, [u_ir_b]G)$ with $(Q_2, t_{i,b}, [u_i]Q_2) = ([y]G, t_{i,b}, [u_iy]G)$, and then constructs $\mathfrak{L}^m$.
(4) The challenger chooses random numbers in $\mathbb{Z}_n$ to construct $\{u''_k\}$ and $\{t''_k\}$ for $1 \leq k \leq w$,
 with which it assembles $\mathbb{L}'' = \{L''_{k; 1\leq k \leq w}\} = \{(Q_2, t''_k, [u''_k]Q_2)\} = \{([y]G, t''_k, [u''_ky]G)\}$.
In this process, it ensures that $[u''_k]G \neq Q_1$ (i.e., $u''_k \neq x$) and $u''_k \neq u_i$,
 for $1 \le i \le v$ and $1 \le k \le w$.
Finally, it randomly chooses $d \in [1, w]$ and replaces $L''_{d}$ with $(Q_2, t''_d, Q_3) = ([y]G, t''_d, [z]G)$.
 Thus, $\mathbb{L}'' = \{L''_{k;1\leq k \leq w}\}$ represents the logins initiated by $w$ honest users, i.e., $\mathbf{u}_w=\{u''_1, u''_2, \cdots, u''_{d-1}, z/y, u''_{d+1}, \cdots, u''_w\}$.
 (5) When the adversary of $\mathcal{G}_r$ receives $\mathfrak{L}^m$, $L'$, and $\mathbb{L}''$ from the challenger, it returns $s$ as the output of $\mathcal{D}^*_r$.

According to the above construction, % of $\mathfrak{L}$, $L'$ and $\mathbb{L}''$,
$x$ is embedded as $ID_{U'}$ in the login $L'$ visiting the RP with $ID_{RP_{a}} = [r_{a}]G$,
and $z/y$ is embedded as $ID_{U''_d}$ in $\mathbb{L}''$ visiting the RP with $ID_{RP_{b}}=[y]G$,
together with $\{u''_1, \cdots, u''_{d-1}, u''_{d+1}, \cdots, u''_w\}$.
Meanwhile, $[r_{a}]G$ and $[y]G$ are two malicious RPs' identities in $\mathfrak{L}^m$.
Because $x \neq u''_{k; 1\leq k \leq w, k \neq d}$ and then $x$ is not in $\{u''_1, \cdots, u''_{d-1}, u''_{d+1}, \cdots, u''_w\}$, the adversary outputs $s=1$ and succeeds in the game \emph{only if} $x = z/y$.
% 这里不是if and only if. "if, 就变成了the adversary必胜了;并不是,而是“有显著的概率”"
% 当“the adversary outputs s=1 且 succeeds in the game”,=> "x = z/y"
% 但是,"x = z/y"  => 不能推导得到“the adversary outputs s=1 且 succeeds in the game”。因为adversary有时候fail、不总是succeed
 Therefore, using $\mathcal{D}^*_r$ to solve the ECDDH problem, we have an advantage $\mathbf{Adv}^*=|{\rm Pr}^*_1 - {\rm Pr}^*_2|$, where
\begin{align*}
&{\rm Pr}^*_1 =  {\rm Pr}(\mathcal{D}^*_r(G, [x]G, [y]G, [xy]G)=1) \\
=&{\rm Pr}(\mathcal{G}_r(\mathfrak{L}^m, L', \mathbb{L}'')=1 \; | \; u' \in \mathbf{u}_w) = {\rm Pr}_1 \\
&{\rm Pr}^*_2= {\rm Pr}(\mathcal{D}^*_r(G, [x]G, [y]G, [z]G)=1) \\
=&{\rm Pr}(\mathcal{G}_r(\mathfrak{L}^m, L', \mathbb{L}'')=1 \; | \; u' \in \mathbb{Z}_n) = {\rm Pr}_2 \\
&\mathbf{Adv}^*=|{\rm Pr}^*_1-{\rm Pr}^*_2|=|{\rm Pr}_1-{\rm Pr}_2|={\mathbf{Adv}}
\end{align*}

If in $\mathcal{G}_r$ the adversary has a non-negligible advantage, then $\mathbf{Adv}^*={\mathbf{Adv}}$ is also non-negligible regardless of the security parameter $\lambda$. This violates the ECDDH assumption. Therefore, the adversary has no advantage in $\mathcal{G}_r$ and cannot decide whether $L'$ is initiated by some user with an identity in $\mathbf{u}_w$ or by a user in the universal user set.
Because $RP_b$ is any malicious RP, this proof can be easily extended from $RP_b$ to more colluding malicious RPs.
\hfill $\square$




%\input{chap_Compact/uid.tex}

\section{Implementation and Evaluation}
\label{sec:implementation}

We implemented a prototype of \usso\footnote{The prototype is open-sourced at \url{https://github.com/uppresso/}.} and conducted experimental comparisons with two open-source SSO systems:
 (\emph{a}) MITREid Connect \cite{MITREid}, a PPID-enhanced OIDC system \cite{NIST2017draft} to prevent RP-based identity linkage,
 and (\emph{b}) SPRESSO \cite{SPRESSO}, which prevents only IdP-based login tracing.
All these solutions work with COTS browsers as user agents.

\subsection{Prototype Implementation}
\label{subsec:proto-imple}

The \usso\ prototype implemented identity transformations on the NIST P256 elliptic curve where $n \approx 2^{256}$, with RSA-2048 and SHA-256 serving as the digital signature and hash algorithms, respectively. The IdP and RP scripts consist of approximately 160 and 140 lines of JavaScript code, respectively.  %to provide the functions in Steps 2.1, 2.3, and 4.3.
The cryptographic computations such as $Cert_{RP}$ verification and $PID_{RP}$ negotiation are performed using jsrsasign \cite{jsrsasign}, an open-source JavaScript cryptographic library.

The IdP was developed on top of MITREid Connect\cite{MITREid}, a Java implementation of OIDC, %certificated by the OpenID Foundation \cite{OIDF},
with minimal code modifications. Only three lines of code were added for calculating $PID_U$ and 20 lines were added to modify the method of forwarding identity tokens.
The calculations for $ID_{RP}$ and $PID_U$ were implemented using Java cryptographic libraries.

We developed a Java-based RP SDK with about 500 lines of code on the Spring Boot framework.
Two functions encapsulate the \usso\ protocol operations: one for requesting identity tokens and the other for deriving accounts. The cryptographic computations are finished using the Spring Security library.
An RP can easily integrate \usso\ by  adding less than 10 lines of Java code to invoke necessary functions.

\subsection{Performance Evaluation}
\label{sec:evaluation}
%\textcolor{red}{Three machines connected in an isolated 1Gbps LAN,
%    build the experimental SSO environment.
%The CPUs are Intel Core i7-4770 3.4 GHz for the IdP,
%    Intel Core i7-4770S 3.1 GHz for the RP, and Intel Core i5-4210H 2.9 GHz for users.
%Each machine is configured with 8 GB RAM and
%    installs Windows 10 as the operating system.
%The user agent is Chrome v75.0.3770.100.}

%RP(不包括特定方案的SDK)大约需要230行JAVA代码
%OIDC(MITREid)的SDK需要大约20行的JAVA代码,需要额外添加一个HTML文件(包含大约20行JavaScript代码)
%UPPRESSO的SDK需要大约1100行代码,不需要添加额外的HTML文件
%OIDC和UPPRESSO的SDK只需要RP提供两个网络接口(网址),然后在对应的网络接口中引用对应的API(每个接口对应一个API,分别命名为tokenRequestGenerate和userAccountAchieve),其他的处理流程均由SDK完成
%SPRESSO由于结构与OIDC完全不同,所以使用了SPRESSO提供的RP的开源代码
%For better evaluation, we build one RP for both UPPRESSO and MITREid Connect which is also implemented  based on Spring Boot framework, as well as the identity token transmission from user to RP in MITREid Connect is implemented by JavaScript running in RP's web page. The IdP in MITREid Connect is achieved from GitHub \cite{MITREid}. However, the SPRESSO system is downloaded from \cite{spressome} containing IdP, RP, and FWD.

\noindent {\bf Experiment setting.} We compared \usso\ comprehensively with MITREid Connect and SPRESSO.
MITREid Connect supports the implicit flow of OIDC, while SPRESSO follows a similar approach to forward the identity token from a user to the RP.
SPRESSO encrypts the RP's domain in identity tokens and keeps the symmetric key known only to the RP and the user. All the systems employ RSA-2048 and SHA-256 for token generation.
SPRESSO implements all entities by JavaScript based on node.js, while MITREid Connect provides Java implementations of IdP and RP SDK.
Thus, for \usso\ and MITREid Connect, we implemented RPs based on Spring Boot by integrating the respective SDKs. In all three schemes, the RPs provide the same function of obtaining the user's account from verified identity tokens.

The IdP and RP servers were deployed on Alibaba Cloud Elastic Compute Service,
each of which ran Windows 10 with 8 vCPUs and 32 GB RAM. The forwarder of SPRESSO ran Ubuntu 20.04.4 with 16 vCPUs and 16 GB RAM, also on the Alibaba Cloud. 

We conducted experiments in two scenarios: (\emph{a}) a browser, Chrome 104.0.5112.81, ran on a virtual machine on Alibaba Cloud with 8 vCPUs and 32 GB memory, and (\emph{b}) a browser running locally on a PC with Core i7-8700 CPU and 32 GB memory, remotely accessed the servers.
In the cloud scenario, all entities were deployed in the same virtual private cloud and connected to one vSwitch, which minimized the impact of network delays. In both scenarios, the IdP server never directly communicated with the RPs.

\noindent {\bf Comparisons.} We split the login flow into three phases for detailed comparisons: (1)
{\em identity-token requesting} (Steps 1 and 2 of the \usso\ protocol), to construct an identity-token request and send it to the IdP server; (2) {\em identity-token generation} (Step 3 of \usso), to generate an identity token at the IdP server, while the user authentication and the user-attribute authorization are excluded; and (3) {\em identity-token acceptance} (Step 4), where the RP receives, verifies, and parses the identity token.


We compared the average time required for an SSO login in three schemes based on 1,000 measurements. As shown in Figure \ref{fig:evaluation},
MITREid Connect, \usso, and SPRESSO require (\emph{a}) 63 ms, 179 ms, and 190 ms, respectively when all entities were deployed on Alibaba Cloud,
 or (\emph{b}) 312 ms, 471 ms, and 510 ms, respectively when the user browser ran locally to visit the cloud servers.

Regarding identity-token requesting, %the user browser loads the RP's webpage and starts the login request.
the RP of MITREid Connect immediately constructs an identity-token request. %MITREid Connect only needs 10 ms but UPPRESSO requires 271 ms.
In SPRESSO the RP needs to obtain information about the IdP % 's public key %(SPRESSO allows a user to assign any IdPs before login without initial registrations)
and encrypt its domain using an ephemeral key, resulting in extra overheads.
In contrast, \usso\ incurs extra overheads in opening a new browser window and downloading scripts. This overhead may be reduced %by silently conducting these operations when the user visits the RP website, or
by implementing a user agent with browser extensions.
% but users need to install the extension before visiting RPs.
We have implemented such a browser extension while keeping the IdP and RPs unmodified, and the supplemental experiments showed a reduction of (\emph{a}) about 90 ms in the virtual private cloud setting and (\emph{a}) 260 ms when accessed remotely.

\begin{figure}[tb]
  \centering
	\subfigure[In a virtual private cloud]{
  		\begin{minipage}[b]{0.439\textwidth}
			\includegraphics[width=0.995\linewidth]{fig/evaluation-lan.pdf}
		\end{minipage}}
	\subfigure[With a remotely-visiting browser]{
  		\begin{minipage}[b]{0.439\textwidth}
			\includegraphics[width=0.995\linewidth]{fig/evaluation-internet.pdf}
		\end{minipage}}
  \caption{The time costs of SSO login in MITREid Connect, SPRESSO, and \usso}
  \label{fig:evaluation}
\end{figure}

Compared with the two schemes, \usso\ requires less time for identity token generation since it retrieves the token from the IdP without any additional processing.
In contrast,
MITREid Connect requires more time as the user is required to download a script from the RP to process the token retrieved from the IdP, which is carried with a URL following the fragment identifier \verb+#+ instead of \verb+?+ due to security considerations \cite{de2014oauth}.
SPRESSO takes slightly more time to generate a token, as it implements the IdP using node.js and uses a JavaScript cryptographic library that is less efficient than the Java library used by the other two schemes.

%transmission & extraction
In the identity-token acceptance phase, MITREid Connect and \usso\ take similar amounts of time to send an identity token to the RP and verify it.
In contrast, SPRESSO takes the longest time due to its complex process at the user agent.
After receiving the identity tokens from the IdP, the browser needs to download JavaScript code from the trusted forwarder server, decrypt the RP endpoint, and then send the identity tokens to this endpoint.


\section{Discussions}
\label{sec:discussion}

%\vspace{0.75mm}
%\noindent \textbf{Applicability of identity transformations.}
%The proposed identity-transformation algorithms %i.e., $\mathcal{F}_{PID_{RP}}()$, $\mathcal{F}_{PID_U}()$, and $\mathcal{F}_{Acct}()$,
%can be applied to a wide range of SSO scenarios, including web applications, mobile Apps, and native software.
%These algorithms follow the common model of popular SSO protocols and do not depend on any specific implementations.

%\vspace{0.75mm}
%\noindent \textbf{Scalability.}
%$ID_{RP}$ is generated uniquely during the initial registration of an RP with a capacity of $n$, which is the order of $G$. For the NIST P256 elliptic curve, $n$ is approximately $2^{256}$.
%$PID_{RP}$ is ensured to be unique in unexpired tokens.
%The probability of at least two identical $PID_{RP}$s among $\sigma$ unexpired tokens is $1-\prod_{i=0}^{\sigma-1}(1-i/n)$.
%If the system serves $10^{8}$ requests per second and each token has a validity period of 10 minutes, $\sigma$ is less than $2^{36}$. So the $PID_{RP}$-collision probability is negligible, i.e., less than $2^{-183}$ for the NIST P256 curve.
%
%The maximal amount of accounts at any RP is the same as the capacity of user identities at the IdP.
%At any RP a unique account is automatically assigned to each user because $Acct =  [ID_U]ID_{RP} = [u]ID_{RP}$.
%Finally, stronger elliptic curves accommodate more RPs and users, e.g., $n$ is about $2^{384}$ for the NIST P384 curve.


%\vspace{0.75mm}
\noindent \textbf{Alternatives for generating $ID_{RP}$ and binding $Enpt_{RP}$.}
In \usso\ the IdP generates random $ID_{RP}$ and uses an RP certificate to bind $ID_{RP}$ and $Enpt_{RP}$, which is verified by the IdP script. This ensures that the target RP has already registered itself at the IdP and prevents unauthorized RPs from accessing the IdP's services \cite{save-flow}.

An alternative method for binding $ID_{RP}$ and $Enpt_{RP}$ is
 to design a \emph{deterministic} scheme to calculate unique $ID_{RP}$ based on the RP's unambiguous name such as its domain.
This can be achieved by encoding the domain with a hashing-to-elliptic-curves function \cite{irtf-cfrg-hash-to-curve-16}, which provides collision resistance but not revealing the discrete logarithm of the output. It generates a point on the elliptic curve $\mathbb{E}$ as $ID_{RP}$ to ensure the \emph{uniqueness} of $ID_{RP} = [r]G$ while keeping $r$ unknown. %For example, using a hash function $Hs()$ to encode an RP's domain or the RP script's origin, e.g., verb+https://RP.com+)

In this case, the RP script sends only the endpoint but not its RP certificate in Step 2.2, and the IdP script calculates $ID_{RP}$ by itself. 
However, if the RP changes its domain, such as from  \verb+https://theRP.com+ to \verb+https://RP.com+, the account (i.e., $Acct = [ID_U]ID_{RP}$) will inevitably change.
As a result, the user is required to perform special operations to migrate her account to the updated RP system.
It is worth noting that the user operations cannot be eliminated in the migration from an RP to the updated one;
otherwise, it implies two colluding RPs could link a user's accounts across these RPs.


%\vspace{0.75mm}
\noindent \textbf{Restriction of the RP script's origin.}
When the IdP script forwards identity tokens to the RP script, the \verb+postMessage+ targetOrigin mechanism \cite{postm-targeto} is used to restrict the recipient of the tokens, to ensure that the tokens will be sent to the intended $Enpt_{RP}$, as specified in the RP certificate. The targetOrigin is specified as a combination of protocol, port (if not present, 80 for \verb+http+ and 443 for \verb+https+), and domain (e.g., \verb+RP.com+).
The RP script's origin must accurately match the targetOrigin to receive the tokens.

The targetOrigin mechanism does not check the whole URL path in $Enpt_{RP}$, but it introduces no {\em additional} risk.
%This assumes only one RP runs on a domain.
Consider two RPs in the same domain but receiving tokens through two different endpoints,
 e.g., \verb+https://RP.com/honest/tk+ and \verb+https://RP.com/malicious/tk+.
This mechanism cannot distinguish them.
%
% Following the same-origin policy (SOP) \cite{sop} for browser access control, the IdP script sends tokens to both endpoints using \verb+postMessage+.
%
Because a COTS browser controls access to web resources with the same-origin policy (SOP) \cite{sop},
    a user's resources in the honest RP server is always accessible to  the malicious one.
 For example, it could steal cookies using \verb+window.open('https://RP.com/honest').document.cookie+,
 even if the honest RP restricts only the HTTP requests to specific paths are allowed to access its cookies.
%
% an RP's resources in browsers could still be accessed maliciously by the other RP running on the same domain,
%  such as stealing cookies using the script \verb+window.open('https://RP.com/honest').document.cookie+,
%  even if it restricts that only HTTP requests to specific paths carry its cookies.
%
So, this risk is caused by the SOP model but not our designs.

%\vspace{0.75mm}
\noindent \textbf{Support for the authorization code flow.} In the authorization code flow of OIDC \cite{OpenIDConnect}, the IdP does not directly return identity tokens.
Instead, it sends an authorization code to the RP, which uses this code to request identity tokens. The identity-transformation algorithms, namely $\mathcal{F}_{PID_{U}}$, $\mathcal{F}_{PID_{RP}}$, and $\mathcal{F}_{Acct}$, can be integrated into this flow as below.

The IdP script can forward an authorization code to the RP script, and then to the RP.
This code only serves as an index to retrieve identity tokens from the IdP and does not disclose any more information about the user.
After receiving an authorization code, an RP uses it along with a secret credential issued by the IdP during the initial registration \cite{OpenIDConnect} to retrieve identity tokens from the IdP. However, to protect the RP identities from the IdP, anonymous tokens (e.g., ring or group signatures \cite{ring-sig,chaum1991group} and TrustToken \cite{trusttoken}) and anonymous communications (e.g., Tor \cite{tor} and oblivious proxies \cite{ODoH}) need to be adopted for RPs in the retrieval of identity tokens.

%\vspace{0.75mm}
\noindent \textbf{Quantum resistance.}
The current designs of \usso\ do not achieve quantum resistance.
In the future, we will investigate alternative algorithms for quantum-secure services.
We plan to adopt a quantum-resistant public-key algorithm to sign identity tokens and RP certificates,
    and study quantum-resistant OPRFs \cite{ideal-lattice-oprf,isogency-oprf}
 to investigate if they support collision-free $PID_{RP}$s (i.e., no collision exists in the blinded inputs of evaluated pseudo-random functions; see Appendix \ref{proof-rp-collision}).


%%%%%%%%%%%%%%%%%%%%% 几个方面的扩展
% 1. 解决IdP数据泄露
% 如果IdP的数据库泄露,用户列表u公开,则RP就可以,针对每一个u,计算[u]ID_{RP};然后,
% 2. 授权码模式
% 可以使用PKCE方式,直接在前端获取。通常,PKCS模式用在没有后端的RP(例如,纯客户端)。
% 对于有后端,可以将​code_verifier传给RP后端?也能够达到目标。
% 3. RP后端访问IdP,需要通过TOR
% 为了不传递RP ID和Secret,可以是:传递PKCE的code_verifier [user将code_verifier传递给RP],
% 也可以是群签名/环签名之类的。
% 4. 要求RP有授权
% 可以有2种方式:
%   去掉RP Cert;采取授权码方式 + 群签名/环签名之类凭证。
%   去掉RP Cert:采取授权码方式 + PrivacyPass之类匿名凭证(还可以有准确计费)。
% 5. 还有一种方式
% 隐式模式 + PrivacyPass之类匿名凭证(还可以有准确计费)。因为其它方式都需要通过TOR。
\section{Conclusion}
\label{sec:conclusion}
This paper presents \usso, an untraceable and unlinkable privacy-preserving SSO system for protecting a user's online profile across RPs against both a curious IdP and colluding RPs.
We propose an identity-transformation approach for privacy-preserving SSO and design algorithms that satisfy the requirements, where (\emph{a}) $\mathcal{F}_{PID_{RP}}$ protects an RP's identity from the curious IdP, (\emph{b}) $\mathcal{F}_{PID_{U}}$ prevents colluding RPs from linking a user across different RPs, and (\emph{c}) $\mathcal{F}_{Acct}$ enables an RP to derive a permanent account for a user in multiple logins. These identity transformations are integrated into the widely-adopted OIDC protocol, maintaining user convenience and security guarantees of SSO services. Our experimental evaluations of the \usso\ prototype demonstrate its efficiency, with an average login taking 174 ms when the IdP, the visited RP, and user agents are deployed in a virtual private cloud, or 421 ms when a user visits remotely.

%\section*{Acknowledgments}
%We would like to thank the anonymous shepherd and reviewers for their valuable suggestions and comments.
%Prof. Xianhui Lu at Institute of Information Engineering, CAS 
%    helped us to improve the analysis of security and privacy.





% conference papers do not normally have an appendix


% use section* for acknowledgement
%\section*{Acknowledgment}

% trigger a \newpage just before the given reference
% number - used to balance the columns on the last page
% adjust value as needed - may need to be readjusted if
% the document is modified later
%\IEEEtriggeratref{8}
% The "triggered" command can be changed if desired:
%\IEEEtriggercmd{\enlargethispage{-5in}}

% references section

% can use a bibliography generated by BibTeX as a .bbl file
% BibTeX documentation can be easily obtained at:
% http://www.ctan.org/tex-archive/biblio/bibtex/contrib/doc/
% The IEEEtran BibTeX style support page is at:
% http://www.michaelshell.org/tex/ieeetran/bibtex/
%\bibliographystyle{IEEEtranS}
% argument is your BibTeX string definitions and bibliography database(s)
%\bibliography{IEEEabrv,../bib/paper}
%
% <OR> manually copy in the resultant .bbl file
% set second argument of \begin to the number of references
% (used to reserve space for the reference number labels box)
%\begin{thebibliography}{1}

%\end{thebibliography}
\bibliographystyle{plain}
\bibliography{ref}

\renewcommand{\algorithmicrequire}{\textbf{Input:}}
\newcommand{\deflet}{\textbf{let}}
\newcommand{\mystate}[1]{\STATE \textbf{let} {{}#1}}
\newcommand{\mystop}[1]{\STATE \textbf{stop} \myss{\myangle{{{}#1}}, s'}}
\newcommand{\mystopp}[1]{\STATE \textbf{stop} \myss{\myangle{{{}#1}}}}
\newcommand{\myss}[1]{${{}#1}$}
\newcommand{\myangle}[1]{\langle {{}#1} \rangle}
\newcommand{\myif}[1]{\IF{\myss{{{}#1}}}}
\newcommand{\myelse}[1]{\ELSIF{\myss{{{}#1}}}}

\newcommand{\aaa}[1]{\STATE \textbf{if} #1 \textbf{then} \begin{ALC@g}}
\newcommand{\bbb}[1]{\end{ALC@g} \STATE \textbf{else if} #1 \textbf{then} \begin{ALC@g}}
\newcommand{\ccc}{\end{ALC@g} \STATE \textbf{else} \textbf{then} \begin{ALC@g}}
\newcommand{\ddd}{\end{ALC@g} \STATE \textbf{endif}}

\newcommand{\SWITCH}[1]{\STATE \textbf{switch} #1\ \textbf{do} \begin{ALC@g}}
\newcommand{\ENDSWITCH}{\end{ALC@g}\STATE \textbf{end switch}}
\newcommand{\CASE}[1]{\STATE \textbf{case} #1\textbf{:} \begin{ALC@g}}
\newcommand{\ENDCASE}{\end{ALC@g}}
\newcommand{\CASELINE}[1]{\STATE \textbf{case} #1\textbf{:} }
\newcommand{\DEFAULT}{\STATE \textbf{default:} \begin{ALC@g}}
\newcommand{\ENDDEFAULT}{\end{ALC@g}}
\newcommand{\DEFAULTLINE}[1]{\STATE \textbf{default:} }

\setcounter{section}{0}
\renewcommand{\thesection}{\Alph{section}}

\section{Preparation}

%所有和SPRESSO的不同点放在同一个章节中一起说明。

Our formal security analysis of UPPRESSO is based on 
the general Dolev-Yao web model in SPRESSO. 
To facilitate the definition of UPPRESSO, however, 
we have some difference from SPRESSO. 
In particular, we remove some processes and 
add some function symbols for asymmetric encryption/decryption.

\subsection{Functions Symbols}

Since our model is using ECC(Elliptic Curve Cryptography) to encrypt/decrypt the data,
we add the following symbols to the signature $\Sigma$ for the terms and messages:

\begin{itemize}
  \item $\mathbb{E}$ is an elliptic curve over a finite field $\mathbb{F}_q$, $G$ is a base point(or generator) of $\mathbb{E}$ and the order of $G$ is a prime number n.
  \item $[t]P$ means using asymmetric key $t$ to encrypt the point $P=[p]G$ on the elliptic curve where $p$ is the actual plaintext.
  \item $[t^{-1}]C$ means using the reverse of $t$ to decrypt the point $C=[c]G=[tm]G$ on the elliptic curve where $c$ is the cipertext.  
  \item $\str{isValid}(P)$ checks whether $P$ is a valid point on the elliptic curve. That is to say whether $P=[m]G$ for the base point $G$ and some nonce $m$.
\end{itemize}

\subsection{DNS servers}

%如果是自己编写的脚本引入了DNS请求,那就需要在模型中考虑DNS服务器,UPPRESSO可以不需要考虑。

In SPRESSO, when receiving an e-mail address, 
RP needs to send DNS requests to DNS servers manually 
to fetch the information of the IdP server. 
Since there may be various DNS servers in SPRESSO, 
DNS server security issues need to be given special consideration.
As a result, DNS servers are added into the formal model of SPRESSO.

In UPPRESSO, however, we only have one centralized IdP server, and 
all RPs know the relevant information of the IdP in advance.
So all DNS requests are generated spontaneously by the browser, 
not introduced by our scripts.
Therefore, we remove DNS servers from the formal model of UPPRESSO.

\section{Formal Model of UPPRESSO}
\label{app:model-uppresso}
We here present the full details of our formal model of UPPRESSO. For our analysis regarding our authentication and privacy properties below, we will further restrict this generic model to suit the setting of respective analysis.\par
We model UPPRESSO as a web system. We call a web system $\uppressowebsystem=(\bidsystem, \scriptset, \mathsf{script}, E^0)$ an UPPRESSO web system if it is of the form described in what follows.

\subsection{Outline}\label{app:outlineuppressomodel}
The system $\bidsystem=\mathsf{Hon}\cup \mathsf{Web} \cup \mathsf{Net}$ consists of 
web attacker processes (in $\mathsf{Web}$), network attacker processes (in $\mathsf{Net}$), 
a finite set $\fAP{B}$ of web browsers, 
a finite set $\fAP{RP}$ of web servers for the relying parties, 
a finite set $\fAP{IDP}$ of web servers containing only one identity provider 
with $\mathsf{Hon} := \fAP{B} \cup \fAP{RP} \cup \fAP{IDP}$. 
More details on the processes in $\mathpzc{W}$ are provided below. 
%
Figure~\ref{fig:scripts-in-w} shows the set of scripts $\scriptset$ 
and their respectice string representations that are defined by the 
mapping $\mathsf{script}$. 
%
The set $E^0$ contains only the trigger events.

\begin{figure}[htb]
    \centering
    \begin{tabular}{|@{\hspace{1ex}}l@{\hspace{1ex}}|@{\hspace{1ex}}l@{\hspace{1ex}}|}\hline 
      \hfill $s \in \scriptset$\hfill  &\hfill $\mathsf{script}(s)$\hfill  \\\hline\hline
      $\Rasp$ & $\str{att\_script}$  \\\hline
      $\mi{script\_rp}$ & $\str{script\_rp}$  \\\hline
      $\mi{script\_idp}$ &  $\str{script\_idp}$  \\\hline
    \end{tabular}
    
    \caption{List of scripts in $\scriptset$ and their respective string
      representations.}
    \label{fig:scripts-in-w}
  \end{figure}
  
  This outlines $\uppressowebsystem$. We will define the DY processes in 
  $\uppressowebsystem$ and their addresses, domain names, and secrets in more detail. 
  The scripts are defined in detail in Appendix~\ref{app:uppresso-scripts}
  
  \subsection{Addresses and Domain Names}
  The set $\addresses$ contains for every web attacker in $\fAP{Web}$, 
  every network attacker in $\fAP{Net}$, 
  every relying party in $\fAP{RP}$, 
  the only one identity provider in $\fAP{IDP}$, 
  and every browser in $\fAP{B}$ a finite set of addresses each. 
  By $\mapAddresstoAP$ we denote the corresponding
  assignment from a process to its address. 
  The set $\dns$ contains a finite set of domains for 
  every relying party in $\fAP{RP}$, 
  the only one identity provider in $\fAP{IDP}$, 
  every web attacker in $\fAP{Web}$, and 
  every network attacker in $\fAP{Net}$. 
  Browsers (in $\fAP{B})$ do not have a domain.
  
  By $\mapAddresstoAP$ and $\mapDomain$ we denote the assignments from
  atomic processes to sets of $\addresses$ and $\dns$, respectively.
  
  %需不需要为椭圆曲线的点单独设立一个集合?
  \subsection{Keys and Secrets} 
  The set $\nonces$ of nonces is partitioned into four sets, 
  an infinite sequence $N$, 
  an infinite set $K_\text{SSL}$, 
  an infinite set $K_\text{sign}$, 
  an infinite set $K_\text{id}$, 
  an infinite set $K_\text{point}$, 
  and a finite set $\RPSecrets$. 
  We thus have
  \begin{align*}
  \def\hereMaxHeightPhantom{\vphantom{K_{\text{p}}^\bidsystem}}
  \nonces = 
  \underbrace{N\hereMaxHeightPhantom}_{\text{infinite sequence}} 
  \dot\cup \underbrace{K_{\text{SSL}}\hereMaxHeightPhantom}_{\text{finite}} 
  \dot\cup \underbrace{K_{\text{sign}}\hereMaxHeightPhantom}_{\text{finite}}
  \dot\cup \underbrace{K_{\text{point}}\hereMaxHeightPhantom}_{\text{finite}}  
  \dot\cup \underbrace{\RPSecrets\hereMaxHeightPhantom}_{\text{finite}}\ .
  \end{align*}
  The set $N$ contains the nonces that are available for each DY process
  in $\bidsystem$ (it can be used to create a run of $\bidsystem$). 
  
  The set $K_\text{SSL}$ contains the keys that will be used for SSL
  encryption. Let $\mapTLSKey\colon \dns \to K_\text{SSL}$ be an injective
  mapping that assigns a (different) private key to every domain.
  
  The set $K_\text{sign}$ contains the keys that will be used by IdPs
  for signing IDToken. Let $\mapSignKey\colon \fAP{IdPs} \to K_\text{sign}$
  be an injective mapping that assigns a (different) private key to every identity
  provider.
  
  The set $K_\text{point}$ contains all valid points on the curve. 
  The set $K_\text{point}$ will be used to generate identities of $\fAP{B}$ and $\fAP{RP}$.
  
  The set $\RPSecrets$ is the set of passwords (secrets) 
  the browsers share with the identity providers. 
  
  %用户的id表示:<id, username, idp-domain, <acct1, rp1-domain>, <acct2, rp2-domain>, <acct3, rp3-domain>, ...>
  %RP的id表示:,<id, rp-commonname, idp-domain>
  
  %def1.用户的id表示:<id, username, idp-domain>
  %def2.用户的account:<<acct1, rp1-domain>, <acct2, rp2-domain>, <acct3, rp3-domain>, ...>
  %def3.RP的id表示,<id, rp-commonname, idp-domain>
  
  %直接使用password
  \subsection{Identities}\label{app:uppresso-identities}
  There are many different types of identities in UPPRESSO. 
  The first is browsers' identities at the IdP. Browsers share
  a $username\in\mathbb{S}$ with IdP to identify an user 
  uniquely, and not like SPRESSO, UPPRESSO doesn't need a 
  domain to identify the IdP.
  
  By $\NToS:\mathbb{S} \to \RPSecrets$ we denote the bijective 
  mapping that assigns secrets to all usernames. 
  
  Let $\mapPLItoOwner: \RPSecrets \to \fAP{B}$ denote the 
  mapping that assigns to each secret a browser that 
  \emph{owns} this secret. Now, we define the mapping 
  $\mapIDtoOwner: \mathbb{S} \to \fAP{B}$, $username \mapsto
  \mapPLItoOwner(\NToS(username))$, which assigns to each 
  identity the browser that owns this identity (we say that the 
  identity belongs to the browser).
  
  Besides, the browsers' identities also have IDs 
  $\mi{ID_u}:=u\in[1,n)$ which is only known to the IdP. By $\NToID:\mathbb{S} \to N$ we denote 
  the bijective mapping that assigns IDs to all usernames. 
  
  The second type of identities is relying parties' identities 
  at the IdP, which are IDs 
  $\mi{ID_{rp}}:=[r]G \in K_{\text{point}}$ in which $r\in[1,n)$ 
  is unknown to any party.
  
  The third type of identities is browsers' identities at the
  Relying Parties which is called $\mi{Acct} := [\mi{ID_u}]
  \mi{ID_{rp}} = [ur]G \in K_{\text{point}}$. 
  
  \subsection{Tags, Identity Tokens and Service Tokens}\label{app:identity-assertions}
  
  \begin{definition}\label{def:tag}
    A \emph{tag} is a term of the form $\mi{PID_{rp}}=[t]ID_{rp}=[tr]G$ for a nonce 
    (here used as a asymmetric key) $t$.
  \end{definition}
  \begin{definition}
    An \emph{identity Tokens (IDToken)} is a term of the form 
    $\an{PID_{rp}, PID_u, ver}$ for a tag $PID_{rp}$, an encrypted identity 
    $PID_u=[u]PID_{rp}=[utr]G$ and a signature $ver=\sig{\an{PID_{rp},PID_u}}{k}$ 
    for a nonce $k\in K_{\text{sign}}$.
  \end{definition}
  %使用<n,i>来定义 service token;Acct or pidu?
  \begin{definition}
    A \emph{service token} is a term of the form 
    $\myangle{\mi{nonce}, \mi{Acct}}$ with 
    $\mi{Acct} = [t^{-1}]PID_u=[t^{-1}][utr]G=[ur]G$ 
    for a nonce $t$.
  \end{definition}
  
  %是否足够描述semi-honest的状态?
  %\subsection{Curiosity}
  %同样,也会有人给idp发送命令,让他进入curiosity状态
  %进入curiosity状态之后,就可以开始把历史上所有的信息,开始用来推导各种事情。
  %或者IdP始终honest-but-curious的状态,在安全性证明时不考虑curious的状态。
  
  \subsection{Corruption}
  RPs can become corrupted: If they receive the message
  $\corrupt$, they start collecting all incoming messages in their state
  and (upon triggering) send out all messages that are derivable from
  their state and collected input messages, just like the attacker
  process. We say that an RP is \emph{honest} if the according
  part of their state ($s.\str{corrupt}$) is $\bot$, and that they are
  corrupted otherwise. IdP is always honest-but-curious and never corrupted.
  
  We are now ready to define the processes in $\websystem$ as well as
  the scripts in $\scriptset$ in more detail. 
  
  \subsection{Processes in $\bidsystem$ (Overview)}
  
  We first provide an overview of the processes in $\bidsystem$. All
  processes in $\websystem$ contain in their initial states all public
  keys and the private keys of their respective domains (if any). We
  define $I^p=\mapAddresstoAP(p)$ for all $p\in \mathsf{Hon} \cup \mathsf{Web}$.
  
  %两个attacker是否足够描述corrupt的程度?
  \subsubsection{Web Attackers.}  Each $\mi{wa} \in \mathsf{Web}$  is a
  web attacker who uses only his own addresses for sending and listening. 
  
  \subsubsection{Network Attackers.}  Each $\mi{na} \in \mathsf{Net}$  is a
  network attacker who uses all addresses for sending and listening. 
  
  \subsubsection{Browsers.} Each $b \in \fAP{B}$ is a web browser. 
  The initial state contains all secrets owned by $b$, stored under the origin of the
  respective IdP. See Appendix~\ref{app:browsers-uppresso} for details.
  
  \subsubsection{Relying Parties.} 
  A relying party $r \in \fAP{RP}$ is a web server. RP knows four distinct paths: 
  $\mathtt{/script}$, where it serves $\str{script\_rp}$ to open a new window 
  and facilitate the login flow.
  $\mathtt{/loginSSO}$, where it only accepts GET requests and sends 
  redirect response to redirect the browser to the IdP to download $\str{script\_IdP}$
  $\mathtt{/startNegotiation}$, where it only accepts POST requests logically sent 
  from $\str{script\_rp}$ using postMessge and checks whether the data $t\in K_\text{id}$.
  If the request valid, it send back a certificate.
  $\mathtt{/uploadToken}$ running in the browser. It checks the ID token and, 
  if the data is deemed ``valid'', it issues a service token (again, for details, see below). 
  Intuitively, a client having such a token can use the service of the RP 
  (for a specific identity record along with the token). 
  Just like IdPs, RPs can become corrupted.
  
  \subsubsection{Identity Providers.} Each IdP is a web server, 
  users can authenticate to the IdP with their credentials. 
  IdP tracks the state of the users with sessions. 
  Authenticated users can receive IDTokens from the IdP. 
  %When receiving a special message ($\corrupt$) IdPs can become corrupted. 
  %Similar to the definition of corruption for the browser,
  %IdPs then start sending out all messages that are derivable from their state.
  
  %\subsubsection{DNS.} Each $\mi{dns} \in \fAP{DNS}$ is a DNS server.
  %Their state contains the allocation of domain names to IP addresses.
  
  \subsection{TLS Key Mapping}\label{app:common-data-structures}
  Before we define the atomic DY processes in more detail, we first
  define the common data structure that holds the mapping of domain
  names to public TLS keys: For an atomic DY process $p$ we define
  \[\mi{tlskeys}^p = \an{\left\{\an{d, \mapTLSKey(d)} \mid d \in \mapDomain(p)\right\}}.\]
  %ssl改成tls
  
  \subsection{Web Attackers}\label{app:webattackers-uppresso}
  Each $\mi{wa} \in \fAP{Web}$ is a web attacker. 
  The initial state of each $\mi{wa}$ is 
  $s_0^\mi{wa} = \an{\mi{attdoms}, \mi{tlskeys}, \mi{signkeys}}$, 
  where $\mi{attdoms}$ is a sequence of all domains along with 
  the corresponding private keys owned by $\mi{wa}$, 
  $\mi{tlskeys}$ is a sequence of all domains and 
  the corresponding public keys, and 
  $\mi{signkeys}$ contains the public signing key for the IdP. 
  %All other parties use the attacker as a DNS server.
  
  \subsection{Network Attackers}\label{app:networkattackers-uppresso}
  As mentioned, each network attacker $\mi{na}$ is modeled to 
  be a network attacker. We allow it to listen to/spoof all 
  available IP addresses, and hence, define 
  $I^\mi{na} = \addresses$. 
  The initial state is $s_0^\mi{na} = 
  \an{\mi{attdoms}, \mi{tlskeys}, \mi{signkeys}}$, 
  where $\mi{attdoms}$ is a sequence of all domains along with 
  the corresponding private keys owned by the attacker 
  $\mi{na}$, $\mi{tlskeys}$ is a sequence of all domains 
  and the corresponding public keys, and 
  $\mi{signkeys}$ contains the public signing key for the IdP. 
  
  \subsection{Browsers}\label{app:browsers-uppresso} 
  Each $b \in \fAP{B}$ is a web browser with 
  $I^b := \mapAddresstoAP(b)$ being its addresses.
  
  To define the inital state, first let $U^b := 
  \mapIDtoOwner^{-1}(b)$ be the set of all usernames of $b$, 
  %$\mi{ID}^{b,d} := \{i \mid \exists\, x,n:\ i = \an{id, n, d} \in \mi{ID}^b\}$ 
  %be the set of IDs of $b$ for a domain $d$, and 
  %$\mi{SecretDomains}^b := \{d \mid \mi{ID}^{b,d} \neq \emptyset \}$ 
  %be the set of all domains that $b$ owns identities for.
  Then, the initial state $s_0^b$ is defined as follows: the key mapping
  maps every domain to its public (tls) key, according to the mapping
  $\mapTLSKey$; the DNS address is $\mapAddresstoAP(p)$ with $p \in \bidsystem$;
  $\mi{ids}$ is $\an{U^b}$; $\mi{sts}$ is empty.
  
  \subsection{Relying Parties} \label{app:relying-parties-uppresso}
  
  A relying party $r \in \fAP{RP}$ is a web server modeled as an atomic
  DY process $(I^r, Z^r, R^r, s^r_0)$ with the addresses $I^r :=
  \mapAddresstoAP(r)$. Its initial state $s^r_0$ contains its domains,
  the private keys associated with its domains.
  The full state additionally contains the sets of service tokens and login 
  session identifiers the RP has issued. RP only accepts HTTPS requests.
  
  RP manages two kinds of sessions: The \emph{login sessions}, which are
  only used during the login phase of a user, and the \emph{service
    sessions} (we call the session identifier of a service session a
  \emph{service token}). Service sessions allow a user to use RP's
  services. The ultimate goal of a login flow is to establish such a
  service session.
  
  In a typical flow with one client, $r$ will first receive an HTTP GET
  request for the path $\str{/script}$. In this case, $r$ returns the script
  $\str{script\_rp}$ (see below).
  
  After the user loaded the script in his browser, $r$ will receive an 
  HTTP GET request for the path $\str{/loginSSO}$ sent from the new window opened
  by $\str{script\_rp}$. In this request, $r$ will send back a redirect response  
  for downloading $\str{script\_IdP}$ from IdP.
  
  When the IdP document in the browser generates a number $t$,
  $r$ will receive the third request for the path $\str{/startNegotiate}$.
  $r$ will verify $t$ and if valid, $r$ will create the corresponding 
  login session with a $\mi{loginSessionToken}$ as the identifier. After that,
  $r$ will use $t$ to generate $PID_{rp}$ and bind it with the login session.
  After all these are down, $r$ send its certificate signed by the specific IdP that browser selected.
  
  Finally, $r$ receives a last request in the login flow. This POST request 
  contains the IDToken. To conclude the login, $r$ looks up the user's login session, 
  compare the $\mi{IDToken}.\str{PID_{rp}}$ with the $\mi{PID_{rp}}$ in the login session, and checks 
  whether $\mi{IDToken}.\str{PID_{ver}}$ is a correct signature. If successful, $r$ calculates the 
  service token and returns it, which is also stored in the state of $r$.
  
  If $r$ receives a corrupt message, it becomes corrupt and acts like
  the attacker from then on.
  
  We now provide the formal definition of $r$ as an atomic DY process
  $(I^r, Z^r, R^r, s^r_0)$. As mentioned, we define $I^r =
  \mapAddresstoAP(r)$. Next, we define the set $Z^r$ of states of
  $r$ and the initial state $s^r_0$ of $r$.
  
  %和tag是否一致?
  \begin{definition}
    A \emph{login session record} is a term of the form 
    $\an{\mi{t}, \mi{PID_{rp}}}$ with 
    $\mi{t}\in N, \mi{PID_{rp}}\in K_{\text{point}}$.
  \end{definition}
  
  \begin{sloppypar}
    \begin{definition}\label{def:relying-parties}
      A \emph{state $s\in Z^r$ of an RP} is a term of the form
      $\langle\mi{keyMapping}$, 
      $\mi{tlskeys}$, 
      $\mi{loginSessions}$, 
      $\mi{serviceTokens}$, 
      $\mi{corrupt}$, 
      $\mi{IdPConfig}$, 
      $\mi{rp}\rangle$ where 
      $\mi{keyMapping} \in \dict{\mathbb{S}}{\nonces}$,
      $\mi{tlskeys}=\mi{tlskeys}^r$,
      $\mi{serviceTokens} \in \dict{\nonces}{K_{\text{point}}}$,
      $\mi{loginSessions} \in \dict{\nonces}{\terms}$ 
      is a dictionary of login session records,
      $\mi{corrupt} \in \terms$,
      $\mi{IdPConfig} \in \terms$ 
      is the configuration retrieved from IdP server,
      $\mi{rp} \in K_{\text{point}}$ is the identity of the RP, 
      see details in Appendix~\ref{app:uppresso-identities}.
  
      The \emph{initial state $s^r_0$ of $r$} is a state of 
      $r$ with $s^r_0.\str{serviceTokens} = 
      s^r_0.\str{loginSessions} = \an{}$,
      $s^r_0.\str{corrupt} = \bot$, 
      $s^r_0.\str{keyMapping}$ 
      is the same as the keymapping for browsers above,
      $s^r_0.\str{IdPConfig} = \an{\mi{pubkey},\mi{scriptUrl},\mi{Cert_{rp}}}$ and
      $s^r_0.\str{rp} = [r]G$ with $r\in N$.
    \end{definition}
  \end{sloppypar}
  
  We now specify the relation $R^r$. We describe this relation by a non-deterministic algorithm. 
  
  \captionof{algorithm}{\label{alg:rp} Relation of a Relying Party $R^r$}
  \begin{algorithmic}[1]
    \REQUIRE \myss{\myangle{a, b, m}, s}
    \mystate{\myss{s':=s}}
    \myif{s'.\str{corrupt} \not\equiv \bot \vee m \equiv \corrupt}
      \mystate{\myss{s'.\str{corrupt} := \an{\an{a, f, m}, s'.\str{corrupt}}}}
      \mystate{\myss{m' := d_{V}(s')}\label{line:usage-of-signkey-corrupt-uppresso}}
      \mystate{\myss{a' := \addresses}}
      \mystop{a',a,m'}
    \ENDIF
    \mystate{\myss{m_{dec},k,k',\mi{inDomain}} \textbf{such that} \breakalgohook{0}
      \myss{\an{m_{\text{dec}}, k} \equiv \dec{m}{k'} \wedge \an{inDomain,k'} \in s'.\str{sslkeys}}\breakalgohook{0}
      \textbf{if possible; otherwise stop} \myss{\myangle{}, s'}}
    \mystate{\myss{n, method, path, parameters, headers, body} \textbf{such that} \breakalgohook{0}
      \myss{\myangle{\mathtt{HTTPReq},n,method,path,parameters,headers,body} \equiv m_{dec}}\breakalgohook{0}
      \textbf{if possible; otherwise stop} \myss{\myangle{}, s'}}
    \myif{path \equiv /script}\label{line:rp-script}
      \mystate{\myss{m':=\encs{\myangle{\mathtt{HTTPResp},n,200, \myangle{}, \str{script\_rp}}}{k}}}
      \mystop{b, a, m'}
    \myelse{path \equiv /loginSSO}\label{line:rp-loginSSO}
      \mystate{\myss{m':=\encs{\myangle{\mathtt{HTTPResp},n,302,\myangle{\myangle{\mathtt{Location}, s'.\str{IdPConfig}.\mi{scriptUrl}}}, \myangle{}}}{k}}}
      \mystop{b, a, m'}
    \myelse{path \equiv /startNegotiation}\label{line:rp-startNegotiation}
      \mystate{\myss{\mi{loginSessionToken} := \nu_1}}
     %\mystate{\myss{cookie := headers[Cookie]}}
     %\mystate{\myss{session := s'.SessionList[cookie]}}
      \mystate{\myss{\mi{t} := body[t]}}\label{line:gen-t}
     %\mystate{\myss{t^{-1}:= \mathtt{Inverse}(t)}}
      \mystate{\myss{\mi{ID_{rp}} := s'.\str{rp}}}
      \mystate{\myss{\mi{PID_{rp}} := [\mi{t}]\mi{ID_{rp}}}}
      \mystate{\myss{\mi{state} := \str{expectToken}}}
      \mystate{\myss{\mi{Cert_{rp}} := s'.\str{IdPConfig}.\mi{Cert_{RP}}}}
      \mystate{\myss{s'.\str{loginSessions}[\mi{loginSessionToken}] := \an{\mi{t}, \mi{PID_{rp}}, \mi{state}}}}
     %\mystate{\myss{session[t] := t}}
     %\mystate{\myss{session[t^{-1}] := t^{-1}}}
     %\mystate{\myss{session[state] := expectToken}}
      \mystate{\myss{\mi{setCookie} := \myangle{\cSetCookie, \myangle{\myangle{\str{sessionid}, \mi{loginSessionToken}, \True, \True, \True}}}}}
      \mystate{\myss{m' := \encs{\myangle{\mathtt{HTTPResp}, n, 200, \myangle{\mi{setCookie}}, \myangle{\mathtt{Cert_{RP}}, \mi{Cert_{rp}}}}}{k}}}
      \mystop{b, a, m'}
    \myelse{path \equiv /uploadToken}\label{line:rp-uploadToken}
      \mystate{\myss{\mi{cookie} := headers[\str{Cookie}]}}
      \myif{\mi{headers}[\str{Origin}] \not\equiv \an{\mi{inDomain}, \https} \vee \mi{cookie}[\str{sessionid}] \equiv \myangle{}}
        \mystop{}\label{line:alg-rp-stop1}
      \ENDIF
      \mystate{\myss{\mi{loginSessions} := s'.\str{loginSessions}[\mi{cookie}[\str{sessionid}]]}}
      %\mystate{\myss{cookie := headers[Cookie]}}
      %\myif{\mi{loginSessions} \equiv \an{}}
      %  \mystop{}
      %\ENDIF
      %\mystate{\myss{session := s'.SessionList[cookie]}}
      %\myif{session[state] \not\equiv expectToken}
      \myif{\mi{loginSessions}.\str{state} \not\equiv expectToken}
        \mystate{\myss{m' := \encs{\myangle{\mathtt{HTTPResp}, n, 200, \myangle{}, \mathtt{Fail}}}{k}}}
        \mystop{b, a, m'}\label{line:alg-rp-stop2}
      \ENDIF
      \mystate{\myss{s'.\str{loginSessions} := s'.\str{loginSessions} - body[\mi{loginSessionToken}]}}
      \mystate{\myss{\mi{IDToken} := body[\str{IDToken}]}}
      \myif{\mi{IDToken}.\str{PID_{rp}} \not\equiv \mi{loginSessions}.\str{PID_{rp}}}
        \mystate{\myss{m' := \encs{\myangle{\mathtt{HTTPResp}, n, 200, \myangle{}, \mathtt{Fail}}}{k}}}
        \mystop{b, a, m'}\label{line:alg-rp-stop3}
      \ENDIF
      \myif{\checksigThree{\mi{IDToken}.\str{ver}}{\an{\mi{IDToken}.\str{PID_{rp}}, \mi{IDToken}.\str{PID_{u}}}}{s'.\str{IdPConfig}.\mi{pubkey}} \equiv \bot}
        \mystate{\myss{m' := \encs{\myangle{\mathtt{HTTPResp}, n, 200, \myangle{}, \mathtt{Fail}}}{k}}}
        \mystop{b, a, m'}\label{line:alg-rp-stop4}
      \ENDIF
     %\mystate{\myss{Time := \mathtt{CurrentTime}()}}
     %\mystate{\myss{PIDValidity := session[PIDValidity]}}
     %\mystate{\myss{Content := Token.Content}}
     %\myif{Time>Content.Validity}
       %\mystate{\myss{m' := \myangle{\mathtt{HTTPResp}, n, 200, \myangle{}, \mathtt{Fail}}}}
       %\mystop{b, a, m'}
     %\ENDIF
      \mystate{\myss{\mi{PID_u} := \mi{IDToken}.\str{PID_{u}}}}
      \mystate{\myss{\mi{Acct} := [\mi{loginSessions}.\str{t}]\mi{PID_u}}}\label{line:gen-acct}
     %\mystate{\myss{Acct := \mathtt{Multiply}(PID_U, t^{-1})}}
     %\myif{Acct \not\in \mathtt{ListOfUser}()}
       %\mystate{\myss{\mathtt{AddUser}(Acct)}}
     %\ENDIF
     %\mystate{\myss{session[user] := Acct}}
      \mystate{\myss{\mi{nonce} := \nu_2}}
      \mystate{\myss{s'.\str{serviceTokens} := s'.\str{serviceTokens} + ^{\myangle{}}\myangle{\mi{nonce}, \mi{Acct}}}}\label{line:add-service-token}
     %\mystate{\myss{s'.serviceTokens := s'.serviceTokens + ^{\myangle{}}\myangle{IDToken, Acct}}}
      \mystate{\myss{\mi{setCookie} := \myangle{\cSetCookie, \myangle{\myangle{\str{sessionid}, \mi{nonce}, \True, \True, \True}}}}}
      \mystate{\myss{m' := \encs{\myangle{\mathtt{HTTPResp}, n, 200, \myangle{\mi{setCookie}}, \mathtt{LoginSuccess}}}{k}}}
      \mystop{b, a, m'}
    \ENDIF
    \mystop{}
  \end{algorithmic}\setlength{\parindent}{1em}
  
  \subsection{Identity Providers} \label{app:idps}
  
  The identity provider $\mathsf{IdP}$ is a web server 
  modeled as an atomic process $(I, Z, R, s_0)$ with 
  the addresses $I := \mapAddresstoAP(\mathsf{IdP})$. 
  Its initial state $s_0$ contains a list of its 
  domains and (private) TLS keys, 
  a list of users and identites, and a private key 
  for signing IDTokens. Besides this, 
  the full state of $\mathsf{IdP}$ further contains a list 
  of used nonces, and information about active sessions.
  
  $\mathsf{IdP}$ react to four types of requests:
  
  First, they provide the $\str{script\_idp}$, where a $t$ 
  will be chosen and following requests to $\mathsf{IdP}$ 
  will be sent. $\mathsf{IdP}$ will transfer the data
  to RP by the communicating between two scripts $\str{script\_idp}$ 
  and $\str{script\_rp}$ using $\tPostMessage$.
  
  Second, they provide $\mi{IDToken}$ when receiving $\mi{PID_{rp}}$ and this 
  $\mi{PID_{rp}}$ has already first. If not, IdPs will redirect to the login dialog.
  
  After the user enter his username and password(secret) in the login dialog, a login
  request will send to $\str{/authentication}$. IdPs will check the parameters and 
  set the login session.
  
  The last type of requests IdPs react to is authorize requests with $\mi{PID_{rp}}$ and attribute
  scopes as parameters. After receving consent from browsers, IdPs will calculate 
  $\mi{PID_{u}}$ and construct $\mi{IDToken}$.
  
  \subsubsection{Formal description.} In the following, we 
  will first define the (initial) state of IdP formally and 
  afterwards present the definition of the relation $R$.
  
  To define the initial state, we will need a term that 
  represents the ``user database'' of the IdP. We will 
  call this term $\mi{userset}$. This database defines, 
  which secret and $\mi{ID_u}$ is valid for which identity. 
  It is encoded as a mapping of username to secrets and $\mi{ID_u}$. 
  For example, if the secret $\mi{secret}_1$ and $\mi{ID_{u_1}}$ is 
  valid for the username $u_1$ and the secret $\mi{secret}_2$ 
  and $\mi{ID_{u_2}}$ is valid for the identity $u_2$, the 
  $\mi{userset}^i$ looks as follows:
  \begin{align*}
  \mi{userset} = [u_1{:}\myangle{\mi{ID_{u_1}}, \mi{secret}_1}, 
    u_2{:}\myangle{\mi{ID_{u_2}}, \mi{secret}_2}]
  \end{align*}
  
  We define $\mi{userset}$ as $\mi{userset} = \an{\{\an{\mi{username}, \myangle{\mi{id}=\NToID(\mi{username}), \mi{secret}=\NToS(\mi{username})}}\, |\, username \in \mathbb{S}\}}$.
  
  \begin{definition}\label{def:initial-state-idp}
    A \emph{state $s\in Z$ of the IdP} is a term of the form
    $\langle\mi{tlskeys}$, $\mi{users}$, $\mi{signkey}$,
    $\mi{sessions}$, $\mi{corrupt}\rangle$ where 
    $\mi{tlskeys} = \mi{tlskeys} $, 
    $\mi{users} = \mi{userset}$, 
    $\mi{signkey} \in \nonces$ 
    (the key used by the IdP to sign IDTokens),
    $\mi{sessions}\in\dict{\nonces}{\terms}$, $\mi{corrupt} \in \terms$.
  
    An \emph{initial state $s_0$ of IdP} is a state of the form 
    $\an{\mi{tlskeys}, \mi{userset}, \mapSignKey(\mathsf{IdP}), \an{}, \bot}$.
  \end{definition}
  
  The relation $R$ that defines the behavior of the IdP is defined as follows:
  
  %和代码基本保持一致
  \captionof{algorithm}{\label{alg:idp} Relation of IdP $R$}
  \begin{algorithmic}[1]
    \REQUIRE \myss{\myangle{a, b, m}, s}
    \mystate{\myss{s':=s}}
    \myif{s'.\str{corrupt} \not\equiv \bot \vee m \equiv \corrupt}
      \mystate{\myss{s'.\str{corrupt} := \an{\an{a, f, m}, s'.\str{corrupt}}}}
      \mystate{\myss{m' := d_{V}(s')}\label{line:usage-of-signkey-corrupt-uppresso}}
      \mystate{\myss{a' := \addresses}}
      \mystop{a', a, m'}
    \ENDIF
    \mystate{\myss{m_{dec},k,k',\mi{inDomain}} \textbf{such that} \breakalgohook{0}
      \myss{\an{m_{\text{dec}}, k} \equiv \dec{m}{k'} \wedge \an{inDomain,k'} \in s'.\str{sslkeys}}\breakalgohook{0}
      \textbf{if possible; otherwise stop} \myss{\myangle{}, s'}}
    \mystate{\myss{n, method, path, parameters, headers, body} \textbf{such that} \breakalgohook{0}
      \myss{\myangle{\mathtt{HTTPReq},n,method,path,parameters,headers,body} \equiv m_{dec}}\breakalgohook{0}
      \textbf{if possible; otherwise stop} \myss{\myangle{}, s'}}
    \myif{path \equiv /script}\label{line:idp-script}
      \mystate{\myss{m':=\encs{\myangle{\mathtt{HTTPResp},n,200, \myangle{}, \str{script\_idp}}}{k}}}
      \mystop{b, a, m'}
    \myelse{path \equiv /authentication}\label{line:idp-authentication}
      \mystate{\myss{\mi{username} := \mi{body}[\str{username}]}}
      \mystate{\myss{\mi{password} := \mi{body}[\str{password}]}}
      \myif{\mi{password} \not\equiv s'.\str{userset}[\mi{username}].\mi{secret}}\label{line:check-username-password}
        \mystate{\myss{m':=\encs{\myangle{\mathtt{HTTPResp},n,200,\myangle{},\mathtt{LoginFailure}}}{k}}}
        \mystop{b, a, m'}
      \ENDIF
      \mystate{\myss{\mi{sessionid} := \nu_3}}
      \mystate{\myss{s'.\str{sessions}[\mi{sessionid}] := \mi{username}}}
      \mystate{\myss{\mi{setCookie} := \myangle{\cSetCookie, \myangle{\myangle{\str{sessionid}, \mi{sessionid}, \True, \True, \True}}}}}
      \mystate{\myss{m' :=\myangle{\mathtt{HTTPResp},n,200,\myangle{\mi{setCookie}},\mathtt{LoginSucess}}}}
      \mystop{b, a, m'}
    \myelse{path \equiv /reqToken}\label{line:idp-reqToken}
      \mystate{\myss{\mi{cookie} := headers[\str{Cookie}]}}
      \myif{\mi{cookie}[\str{sessionid}] \equiv \myangle{}}\label{line:check-sessionid}
        \mystate{\myss{m' := \encs{\myangle{\mathtt{HTTPResp},n,200,\myangle{},\mathtt{Unauthenticated}}}{k}}}
        \mystop{b, a, m'}
      \ENDIF
      \mystate{\myss{\mi{sessionid} := \mi{cookie}[\str{sessionid}]}}
      \mystate{\myss{\mi{PID_{rp}} := \mi{parameters}[\str{PID_{rp}}]}}
      \myif{s'.\str{sessions}[\mi{sessionid}].\mi{IDToken}[\mi{PID_{rp}}] \equiv \myangle{}}\label{line:check-session-pidrp}
        \mystate{\myss{m' := \encs{\myangle{\mathtt{HTTPResp},n,200,\myangle{},\mathtt{Unauthorized}}}{k}}}
        \mystop{b, a, m'}
      \ENDIF
      \mystate{\myss{\mi{IDToken} := s'.\str{sessions}[\mi{sessionid}].\mi{IDToken}[\mi{PID_{rp}}]}}
      \mystate{\myss{m' := \encs{\myangle{\mathtt{HTTPResp},n,200,\myangle{}, \mi{IDToken}}}{k}}}
      \mystop{b, a, m'}
    \myelse{path \equiv /authorize}\label{line:idp-authorize}
      \mystate{\myss{\mi{cookie} := headers[\str{Cookie}]}}
      \myif{\mi{cookie}[\str{sessionid}] \equiv \myangle{}}\label{line:uppresso-idp-check-login-state}
        \mystate{\myss{m' := \encs{\myangle{\mathtt{HTTPResp},n,200,\myangle{},\mathtt{Unauthenticated}}}{k}}}
        \mystop{b, a, m'}
      \ENDIF
      \mystate{\myss{\mi{sessionid} := \mi{cookie}[\str{sessionid}]}}
      \mystate{\myss{\mi{PID_{RP}} := \mi{parameters}[\str{PID_{RP}}]}}
      \myif{\mathtt{IsValid}(PID_{RP}) \equiv \bot}
        \mystate{\myss{m' := \encs{\myangle{\mathtt{HTTPResp}, n, 200, \myangle{}, \mathtt{Fail}}}{k}}}
        \mystop{b, a, m'}
      \ENDIF
      \myif{\mathtt{IsInScope}(uid, \mi{body}[\str{Attr}]) \equiv \bot}
        \mystate{\myss{m' := \encs{\myangle{\mathtt{HTTPResp}, n, 200, \myangle{}, \mathtt{Fail}}}{k}}}
        \mystop{b, a, m'}
      \ENDIF
      
      \mystate{\myss{\mi{username} := s'.\str{sessions}[\mi{sessionid}].username}}
      \mystate{\myss{\mi{ID_u} := s'.\str{userset}[\mi{username}].\mi{id}}}\label{line:get-idu}
      \mystate{\myss{\mi{PID_u} := [\mi{ID_u}]\mi{PID_{rp}}}}\label{line:uppresso-idp-set-pidu}
      %\mystate{\myss{Validity := \mathtt{CurrentTime} ()+ s'.Validity}}
      %\mystate{\myss{Content := \myangle{PID_{RP}, PID_U, s'.Issuer, Validity}}}
      \mystate{\myss{\mi{content} := \myangle{PID_{rp}, PID_u}}}
      \mystate{\myss{\mi{ver} := \sig{\mi{content}}{s'.\str{signkey}}}}\label{line:sign-token}
      \mystate{\myss{\mi{IDToken} := \myangle{\mi{content}, \mi{ver}}}}
      \mystate{\myss{s'.\str{sessions}[\mi{IDToken}]:=s'.\str{sessions}[\mi{IDTokens}]+^{\myangle{}}\myangle{PID_{rp}, IDToken}}}
      \mystate{\myss{m':=\encs{\myangle{\mathtt{HTTPResp}, n, 200, \myangle{}, \mi{IDToken}}}{k}}}
      \mystop{b, a, m'}
    \ENDIF
    \mystop{}
  \end{algorithmic}\setlength{\parindent}{1em}
  
  \subsection{UPPRESSO Scripts}\label{app:uppresso-scripts}
  As already mentioned in Appendix~\ref{app:outlineuppressomodel}, the set $\scriptset$ 
  of the web system $\uppressowebsystem=(\bidsystem, \scriptset, \mathsf{script}, E^0)$ 
  consists of the scripts $\Rasp$, $\mi{script\_rp}$, $\mi{script\_idp}$, and with their 
  string representations being $\str{att\_script}$, $\str{script\_rp}$, $\str{script\_idp}$, 
  and (defined by $\mathsf{script}$). 
  
  In what follows, the scripts $\mi{script\_rp}$ and $\mi{script\_idp}$ are
  defined formally.
  
  \subsubsection{Relying Party Page (script\_rp).}\label{app:uppresso-script-rp}
  As defined in SPRESSO, a script is a relation that takes a termas input and outputs 
  a new term. The input term is provided by the browser. It contains the current 
  internal state of the script (which we call \emph{scriptstate} in what follows) and
  additional information containing all browser state information the
  script has access to, such as the input the script has obtained so far
  via \xhrs and \pms, information about windows, etc. The browser
  expects the output term to contain, among other information, the new internal \emph{scriptstate}.
  
  We first describe the structure of the internal scriptstate
  of the script $\mi{script\_rp}$.
  
  \begin{definition} \label{def:scriptstaterp} 
  A \emph{scriptstate $s$ of $\mi{script\_rp}$} is a term of the form $\langle 
  \mi{phase}$, 
  %$\mi{loginSessionToken}$, 
  $\mi{refXHR}\rangle$, 
  where $phase \in \mathbb{S}$, 
  %$\mi{loginSessionToken}$,
  $\mi{refXHR}\in \nonces \cup \{\bot\}$. 
  
  The \emph{initial scriptstate $\mi{initState_{rp}}$} of $\mi{script\_rp}$ is 
  $\an{\str{start},
  %\bot,
  \bot}$.
  \end{definition}
  
  We now specify the relation $\mi{script\_rp}$ formally. We describe this relation
  by a non-deterministic algorithm.
  
  \captionof{algorithm}{\label{alg:uppresso-script-rp} Relation of $\mi{script\_rp}$}
  \begin{algorithmic}[1]
  \REQUIRE \myss{\langle\mi{tree},\mi{docnonce},\mi{scriptstate},\mi{scriptinputs},\mi{cookies},\mi{localStorage},\mi{sessionStorage},}
  \breakalgohook{-1}\myss{\mi{ids},\mi{secret}\rangle}
  \mystate{\myss{ s' := \mi{scriptstate}}}
  \mystate{\myss{\mi{command} := \myangle{}}}
  \mystate{\myss{\mi{origin} := \mathsf{GETORIGIN}(\mi{tree},\mi{docnonce})}}
  \mystate{\myss{\mi{RPDomain} := \mi{origin}.\str{host}}}
  \SWITCH{\myss{s'.\str{phase}}}
  \CASE{\myss{\str{start}}}
    \mystate{\myss{\mi{url} := \an{\tUrl, \https, \mi{RPDomain}, \str{/loginSSO}, \myangle{}}}}
    \mystate{\myss{\mi{command} := \an{\tHref,\mi{url},\wBlank,\an{}}}}
    \mystate{\myss{s'.\str{phase} := \str{expectt}}}
  \ENDCASE
  \CASE{\myss{\str{expectt}}}
    \mystate{\myss{\mi{pattern} := \myangle{\tPostMessage, target, *, \myangle{\str{t}, *}}}}
    \mystate{\myss{\mi{input} := \textsf{CHOOSEINPUT}(\mi{scriptinputs}, \mi{pattern})}}
    \myif{\mi{input} \not\equiv \bot}
      \mystate{\myss{t := \pi_2(\pi_4(\mi{input}))}}\label{line:receive-t}
      %\mystate{\myss{\mi{url} := \myangle{\tUrl, \https, \mi{RPDomain}, \str{/startNegotiation}, \myangle{}}}}
      \mystate{\myss{\mi{body} := \myangle{\myangle{\str{t},t}}}}
      \mystate{\myss{\mi{command} := \langle\tXMLHTTPRequest,\textsf{URL}^{\mi{RPDomain}}_\str{/startNegotiation},\mPost,\mi{body},}\breakalgohook{0}\myss{s'.\str{refXHR}\rangle}}
      \mystate{\myss{s'.\str{phase} := \str{expectCert}}}
    \ENDIF
  \ENDCASE
  \CASE{\myss{\str{expectCert}}}
    \mystate{\myss{pattern := \myangle{\tXMLHTTPRequest,*,s'.\str{refXHR}}}}
    \mystate{\myss{\mi{input} := \textsf{CHOOSEINPUT}(\mi{scriptinputs}, \mi{pattern})}}
    \myif{\mi{input} \not\equiv \bot}
      \mystate{\myss{\mi{Cert_{rp}} := \pi_2(\mi{input}).\str{Cert_{rp}}}}
      \mystate{\myss{\mi{IdPWindowNonce} := \pi_1(\textsf{SUBWINDOWS}(\mi{tree},\mi{docnonce})).\str{nonce}}}
      \mystate{\myss{\mi{IdPOrigin} := \mathsf{GETORIGIN}(\mi{tree}, \mi{IdPWindowNonce})}}
      \mystate{\myss{\mi{command} := \langle\tPostMessage, \mi{IdPWindowNonce}, \myangle{\str{Cert}, \mi{Cert_{rp}}},}\breakalgohook{0}\myss{\mi{IdPOrigin}\rangle}}
      \mystate{\myss{s'.\str{phase} := \str{expectToken}}}
    \ENDIF
  \ENDCASE
  \CASE{\myss{\str{expectToken}}}
    \mystate{\myss{\mi{pattern} := \myangle{\tPostMessage, target, *, \myangle{\str{IDToken}, *}}}}
    \mystate{\myss{\mi{input} := \textsf{CHOOSEINPUT}(\mi{scriptinputs}, \mi{pattern})}}
    \myif{input \not\equiv \bot}
      \mystate{\myss{\mi{IDToken} := \pi_2(\pi_4(\mi{input}))}}
      %\mystate{\myss{\mi{url} := \myangle{\tUrl, \https, \mi{RPDomain}, \str{/uploadToken}, \myangle{}}}}
      \mystate{\myss{\mi{body} := \myangle{\myangle{\str{IDToken},\mi{IDToken}}}}}
      \mystate{\myss{\mi{command} := \langle\tXMLHTTPRequest,\textsf{URL}^{\mi{RPDomain}}_\str{/uploadToken},\mPost,\mi{body},}\breakalgohook{0}\myss{s'.\str{refXHR}\rangle}}
      \mystate{\myss{s'.\str{phase} := \str{expectLoginResult}}}
    \ENDIF
  \ENDCASE
  \CASE{\myss{\str{expectLoginResult}}}
    \mystate{\myss{pattern := \myangle{\tXMLHTTPRequest,*,s'.\str{refXHR}}}}
    \mystate{\myss{\mi{input} := \textsf{CHOOSEINPUT}(\mi{scriptinputs}, \mi{pattern})}}
    \myif{input \not\equiv \bot}
      \myif{\pi_2(input) \equiv \str{LoginSuccess}}
      \mystate{Load Homepage}
      \ENDIF
    \ENDIF
  \ENDCASE
  \ENDSWITCH\\
  \mystopp{s',\mi{cookies},\mi{localStorage},\mi{sessionStorage},\mi{command}}
  \end{algorithmic}\setlength{\parindent}{1em}
  
  \subsubsection{Identity Provider Page (script\_idp).}\label{app:uppresso-script-Idp}
  
  \begin{definition}\label{def:scriptstateidp}
    A \emph{scriptstate $s$ of $\mi{script\_idp}$} is a term of the form
    $\langle \mi{phase}$, $\mi{user}$, $\mi{parameters} \rangle$ with $\mi{phase} \in
    \mathbb{S}$, $\mi{user} \in \IDs \cup \{\an{}\} \in \gterms$ and $\mi{parameters} \in \dict{\mathbb{S}}{\terms}$,. The 
    \emph{initial scriptstate} of $\mi{script\_idp}$ is $\an{\str{start},*,\myangle}$.
  \end{definition}
  
  We now formally specify the relation of $\mi{script\_idp}$
  
  \captionof{algorithm}{\label{alg:uppresso-script-idp} Relation of $\mi{script\_idp}$ }
  \begin{algorithmic}[1]
  \REQUIRE \myss{\langle\mi{tree},\mi{docnonce},\mi{scriptstate},\mi{scriptinputs},\mi{cookies},\mi{localStorage},\mi{sessionStorage},}
  \breakalgohook{-1}\myss{\mi{ids},\mi{secret}\rangle}
  \mystate{\myss{s' := scriptstate}}
  \mystate{\myss{\mi{command} := \myangle{}}}
  \mystate{\myss{\mi{target} := \textsf{OPENERWINDOW}(\mi{tree},\mi{docnonce})}}
  \mystate{\myss{\mi{origin} := \mathsf{GETORIGIN}(\mi{tree},\mi{docnonce})}}
  \mystate{\myss{\mi{IdPDomain} := \mi{origin}.\str{host}}}
  \SWITCH{\myss{s'.\str{phase}}}
  \CASE{\myss{start}}
    \mystate{\myss{t := \str{random}()}}
    \mystate{\myss{\mi{command} := \myangle{\tPostMessage, \mi{target}, \myangle{\str{t}, t}, \myangle{}}}}\label{line:send-t}
    \mystate{\myss{s'.\str{parameters}[t] := t}}
    \mystate{\myss{s'.\str{phase} := \str{expectCert}}}
  \ENDCASE
  \CASE{\myss{\str{expectCert}}}
    \mystate{\myss{\mi{pattern} := \myangle{\tPostMessage, target, *, \myangle{\str{Cert}, *}}}}
    \mystate{\myss{\mi{input} := \textsf{CHOOSEINPUT}(\mi{scriptinputs}, \mi{pattern})}}
    \myif{input \not\equiv \bot}
      \mystate{\myss{\mi{Cert_{rp}} := \pi_2(\pi_4(input))}}
      \myif{\checksigThree{\mi{Cert_{rp}}.\str{ver}}{\mi{Cert_{rp}}.\str{content}}{s'.\str{IdPConfig}.\mi{pubkey}} \equiv \True}\label{alg:script-idp-verify-cert}
        \mystate{\myss{s'.\str{parameters}[\mi{cert}] := \mi{Cert_{rp}}}}
        \mystate{\myss{t := s'.\str{parameters}[t]}}
        \mystate{\myss{\mi{PID_{rp}} := [t]\mi{Cert_{rp}}.\str{content}[\mi{ID_{rp}}]}}\label{line:gen-pidrp}
        \mystate{\myss{s'.\str{parameters}[\mi{PID_{rp}}] := \mi{PID_{rp}}}}
        \mystate{\myss{\mi{body} := \myangle{\myangle{\str{PID_{rp}},\mi{PID_{rp}}}}}}
        \mystate{\myss{\mi{command} := \langle\tXMLHTTPRequest,\textsf{URL}^{\mi{IdPDomain}}_\str{/reqToken},\mPost,\mi{body},}\breakalgohook{0}\myss{s'.\str{refXHR}\rangle}}
        \mystate{\myss{s'.\str{phase} := \str{expectReqToken}}}
      \ENDIF
    \ENDIF
  \ENDCASE
  \CASE{\myss{\str{expectReqToken}}}
    \mystate{\myss{pattern := \myangle{\tXMLHTTPRequest,*,s'.\str{refXHR}}}}
    \mystate{\myss{\mi{input} := \textsf{CHOOSEINPUT}(\mi{scriptinputs}, \mi{pattern})}}
    \myif{input \not\equiv \bot}
      \myif{\pi_2(input) \equiv \str{Unanthenticated}}
        \mystate{\myss{s'.\str{user} \gets \mi{ids}}}
        \mystate{\myss{\mi{username} := s'.\str{user}.\mi{name}}}
        \mystate{\myss{\mi{password} := \textsf{secretOfID}(s'.\str{user})}}
        \mystate{\myss{\mi{body} := \myangle{\myangle{\str{username}, \mi{username}}, \myangle{\str{password}, \mi{password}}}}}
        \mystate{\myss{\mi{command} := \langle\tXMLHTTPRequest,\textsf{URL}^{\mi{IdPDomain}}_\str{/authentication},\mPost,\mi{body},}\breakalgohook{0}\myss{s'.\str{refXHR}\rangle}}
        \mystate{\myss{s'.\str{phase} := \str{expectLoginResult}}}
      \myelse{\pi_2(input) \equiv \str{Unauthorized}}
        \mystate{\myss{\mi{PID_{rp}} := s'.\str{parameters}[\mi{PID_{rp}}]}}
        \mystate{\myss{\mi{Attr} := \textsf{GETPARAMETERS}(\mi{tree}, \mi{docnonce})[\str{iaKey}]}}
        \mystate{\myss{\mi{body} := \myangle{\myangle{\str{PID_{rp}}, \mi{PID_{rp}}}, \myangle{\str{Attr}, \mi{Attr}}}}}
        \mystate{\myss{\mi{command} := \langle\tXMLHTTPRequest,\textsf{URL}^{\mi{IdPDomain}}_\str{/authorize},\mPost,\mi{body},}\breakalgohook{0}\myss{s'.\str{refXHR}\rangle}}
        \mystate{\myss{s'.\str{phase} := \str{expectToken}}}
      \myelse{}
        \mystate{\myss{IDToken := \pi_2(input)[\str{IDToken}]}}
        \mystate{\myss{RPOringin := \myangle{s'.\str{parameters}[\mi{cert}].\mi{Content}[\str{Enpt}], \mathtt{S}}}}
        \mystate{\myss{\mi{command} := \myangle{\tPostMessage, \mi{target}, \myangle{\str{IDToken},\mi{IDToken}}, RPOrigin}}}
        \mystate{\myss{s'.\str{phase} := \str{stop}}}
      \ENDIF
    \ENDIF
  \ENDCASE
  \CASE{\myss{\str{expectLoginResult}}}
    \mystate{\myss{pattern := \myangle{\tXMLHTTPRequest,*,s'.\str{refXHR}}}}
    \mystate{\myss{\mi{input} := \textsf{CHOOSEINPUT}(\mi{scriptinputs}, \mi{pattern})}}
    \myif{input \not\equiv \bot}
      \myif{\pi_2(input) \equiv \str{LoginSuccess}}
        \mystate{\myss{\mi{PID_{rp}} := s'.\str{parameters}[\mi{PID_{rp}}]}}
        \mystate{\myss{\mi{Attr} := \textsf{GETPARAMETERS}(\mi{tree}, \mi{docnonce})[\str{iaKey}]}}
        \mystate{\myss{\mi{body} := \myangle{\myangle{\str{PID_{rp}}, \mi{PID_{rp}}}, \myangle{\str{Attr}, \mi{Attr}}}}}\label{line:send-pidrp}
        \mystate{\myss{\mi{command} := \langle\tXMLHTTPRequest,\textsf{URL}^{\mi{IdPDomain}}_\str{/authorize},\mPost,\mi{body},}\breakalgohook{0}\myss{s'.\str{refXHR}\rangle}}
        \mystate{\myss{s'.\str{phase} := \str{expectToken}}}
      \ENDIF
    \ENDIF
  \ENDCASE
  \CASE{\myss{\str{expectToken}}}
    \mystate{\myss{pattern := \myangle{\tXMLHTTPRequest,*,s'.\str{refXHR}}}}
    \mystate{\myss{\mi{input} := \textsf{CHOOSEINPUT}(\mi{scriptinputs}, \mi{pattern})}}
    \myif{input \not\equiv \bot}
      \mystate{\myss{IDToken := \pi_2(input)[\str{IDToken}]}}
      \mystate{\myss{RPOringin := \myangle{s'.\str{parameters}[\mi{cert}].\mi{Content}[\str{Enpt}], \mathtt{S}}}}
      \mystate{\myss{\mi{command} := \myangle{\tPostMessage, \mi{target}, \myangle{\str{IDToken},\mi{IDToken}}, RPOrigin}}}\label{line:token-send}
      \mystate{\myss{s'.\str{phase} := \str{stop}}}
    \ENDIF
  \ENDCASE
  \ENDSWITCH
  \mystopp{s',\mi{cookies},\mi{localStorage},\mi{sessionStorage},\mi{command}}
  \end{algorithmic}\setlength{\parindent}{1em}
  
  
  \section{Proof of Security}
  
  To state the security properties for \uppresso, we first
  define an \emph{\uppresso web system for authentication analysis}. This
  web system is based on the \uppresso web system and only considers one
  network attacker (which subsumes all web attackers and further network
  attackers).
  
  \begin{definition}
    Let $\uppressoauthwebsystem = (\bidsystem, \scriptset, \mathsf{script}, E^0)$
    an \uppresso web system. We call $\uppressoauthwebsystem$ an
    \emph{\uppresso web system for authentication analysis} iff
    $\bidsystem$ contains only one network attacker process
    $\fAP{attacker}$ and no other attacker processes (i.e.,
    $\mathsf{Net} = \{\fAP{attacker}\}$, $\mathsf{Web} = \emptyset$).
    %Further, $\bidsystem$ contains no DNS servers. DNS servers are
    %assumed to be dishonest, and hence, are subsumed by
    %$\fAP{attacker}$. In the initial state $s_0^b$ of each browser $b$
    %in $\bidsystem$, the DNS address is
    %$\mapAddresstoAP(\fAP{attacker})$. Also, in the initial state
    %$s_0^r$ of each relying party $r$, the DNS address is
    %$\mapAddresstoAP(\fAP{attacker})$.
  \end{definition}
  
  The security properties for \uppresso are formally defined 
  as follows. First note that every $Acct$ recorded in RP was
  calculated by RP as the result of an HTTPS $\mPost$ request 
  $m$. We refer to $m$ as the 
  \emph{request corresponding to $Acct$}. 
  
  In the following definition, when we say a browser 
  $b\in \fAP{B}$ owns an $Acct$, we holds that for some relying 
  party $rp\in \fAP{RP}$ that calculate the $Acct$ and 
  a $\mi{username} \in \mathbb{S}$ with $\mapIDtoOwner(username) = b$.
  \[\mi{Acct}=[\NToID(username)]ID_{rp}=[ID_u]ID_{rp}=[ur]G\]

  \newc
  Here we can treat $Acct$ a \emph{permanent} identifier 
  determined by $ID_u$ and $ID_{rp}$. 
  $ID_{rp} = [r]G$ is a generator on $\mathbb{E}$ of order $n$, 
  as $\mathbb{E}$ is a finite cyclic group. 
  Therefore, given a user with $ID_u$ owned by browser $b$, $Acct$ is a \emph{unique} point on 
  $\mathbb{E}$ for any $u \in [1, n)$, and it is \emph{uniquely} 
  associated with $ID_u=u$.
  That's why we can say a browser $b$ owns an $Acct$.
  \oldc
  
  We now define the similar security properties as the definition 52 in SPRESSO. 
  
  \begin{definition}\label{def:uppresso-security-property} 
    Let $\uppressoauthwebsystem$ be an \uppresso web system for authentication analysis. 
    We say that \emph{$\uppressoauthwebsystem$ is secure} if for every run $\rho$ of
    $\uppressoauthwebsystem$, every state $(S^j, E^j, N^j)$ in $\rho$,
    every $r\in \fAP{RP}$ that is honest in $S^j$, every RP service token of the form 
    $\myangle{nonce, Acct}$ recorded in $S^j(r).\str{serviceTokens}$, the following two conditions are
    satisfied:
  
    \textbf{(A)} If $\myangle{nonce, Acct}$ is derivable from the attackers knowledge
    in $S^j$ (i.e., $\myangle{nonce, Acct} \in d_{\emptyset}(S^j(\fAP{attacker}))$),
    then it follows that the browser $b$ owning $Acct$ is fully corrupted
    in $S^j$ (i.e., the value of $\mi{isCorrupted}$ is $\fullcorrupt$).
  
    \textbf{(B)} If the request corresponding to $\myangle{nonce, Acct}$ was sent by
    some $b\in \fAP{B}$ which is honest in $S^j$, then $b$ owns $Acct$.
  \end{definition}
  
  %First note that the RP service token should be defined as $\langle IDToken$, $Acct \rangle$ 
  %which is $\langle n$, $i \rangle$ in SPRESSO. That is,  
  
  %let  $\mathcal{U\!W\!S}^{auth}$ be an UPPRESSO web system for authentication analysis. We say that $\mathcal{U\!W\!S}^{auth}$ is secure if for every run $\rho$ of $\mathcal{U\!W\!S}^{auth}$, every state ($S^j$, $E^j$, $N^j$) in $\rho$, every $r \in$ $\mathtt{RP}$ that is honest, every RP service token of the form $\langle IDToken$, $Acct \rangle$ recorded in $S^j$($r$).$\mathtt{serviceTokens}$, the following two conditions are satisfied:
  
  %(A) If $\langle IDToken$, $Acct \rangle$ is derivable from the attackers knowledge in $S^j$ (i.e., $\langle IDToken$, $Acct \rangle \in d_{\emptyset}$($S^j$($\mathtt{attacker}$))), then it follows that the browser b owning $Acct$ is fully corrupted in $S^j$ (i.e., the value of $isCorrupted$ is $\mathtt{FULLCORRUPT}$) or $\mathtt{governor}$($Acct$) is not an honest IdP (in $S^j$).
  
  %(B) If the request corresponding to $\langle IDToken$, $Acct \rangle$ was sent by some $b \in \mathtt{B}$ which is honest in $S^j$, then b owns the $ID_U$ which satisfies $Acct=[ID_U]S^j(r).ID_{RP}$.
  
  \begin{theorem}\label{thm:authentication}
    Let $\uppressoauthwebsystem$ be an \uppresso web system as
    defined above. Then $\uppressoauthwebsystem$ is secure
    w.r.t.~authentication.
  \end{theorem}
  
  To prove Theorem~\ref{thm:authentication}, we are going to prove the following Lemmas.
  
  \begin{lemma}\label{lemma:k-does-not-leak-from-honest-rp} 
    If in the processing step $s_i \rightarrow s_{i+1}$ of a run $\rho$
    of $\uppressoauthwebsystem$ an honest relying party $r$ (I) emits an HTTPS
    request of the form
  
    \[ m = \ehreqWithVariable{\mi{req}}{k}{\pub(k')} \]
  %
    (where $\mi{req}$ is an HTTP request, $k$ is a nonce (symmetric
    key), and $k'$ is the private key of some other DY process $u$), and (II) in the
    initial state $s_0$ the private key $k'$ is only known to $u$, and
    (III) $u$ never leaks $k'$, then all of the following
    statements are true:
    \begin{enumerate}
    \item There is no state of $\uppressoauthwebsystem$ where any party except
      for $u$ knows $k'$, thus no one except for $u$ can
      decrypt $\mi{req}$.
      \label{prop:attacker-cannot-decrypt-spresso}
    \item If there is a processing step $s_j \rightarrow s_{j+1}$ where
      the RP $r$ leaks $k$ to $\bidsystem \setminus \{u, r\}$ there
      is a processing step $s_h \rightarrow s_{h+1}$ with $h < j$
      where $u$ leaks the symmetric key $k$ to $\bidsystem \setminus
      \{u,r\}$ or $r$ is corrupted in
      $s_j$. \label{prop:k-doesnt-leak-spresso}
    \item The value of the host header in $\mi{req}$ is the domain that
      is assigned the public key $\pub(k')$ in RP's keymapping
      $s_0.\str{keyMapping}$ (in its initial
      state). \label{prop:host-header-matches-spresso}
    \item If $r$ accepts a response (say, $m'$) to $m$ in a processing step $s_j
      \rightarrow s_{j+1}$ and $r$ is honest in $s_j$ and $u$ did not
      leak the symmetric key $k$ to $\bidsystem \setminus \{u,r\}$ prior
      to $s_j$, then $u$ created the HTTPS response $m'$ to the HTTPS
      request $m$, i.e., the nonce of the HTTP request $\mi{req}$ is not known to
      any atomic process $p$, except for the atomic DY processes $r$ and
      $u$.\label{prop:only-owner-answers-spresso}
    \end{enumerate}
  \end{lemma}
  
  %\begin{lemma}\label{lemma:wkcache-never-lies}
  %  For every honest relying party $r \in \fAP{RP}$, every $s \in \rho$, every
  %  $\an{\mi{host}, \mi{wkDoc}} \inPairing S(r).\str{wkCache}$ it holds
  %  that $\mi{wkDoc}[\str{signkey}] \equiv
  %  \pub(\mathsf{signkey}(\mapDomain^{-1}(\mi{host})))$ if
  %  $\mapDomain^{-1}(\mi{host})$ is an honest IdP.
  %\end{lemma}
  
  \begin{lemma}\label{lemma:uppresso-request-exists}
    In a run $\rho$ of $\uppressoauthwebsystem$, for every 
    state $s_j \in\rho$, every RP $r \in \fAP{RP}$ that is 
    honest in $s_j$, every $\myangle{nonce, Acct} \inPairing 
    S^j(r).\str{serviceTokens}$, the following properties hold:
  
    \begin{enumerate}
    \item There exists exactly one $l' < j$ such that there exists a
      processing step in $\rho$ of the form
      \[ s_{l'} \xrightarrow[r \rightarrow \an{\an{a',f',m'}}]{e'
        \rightarrow r} s_{l'+1}\]
      with $e'$ being some events, $a'$ and $f'$
      being addresses and $m'$ being a service token response for $Acct$.
  
    \item There exists exactly one $l < j$ such that there exists a
      processing step in $\rho$ of the form 
      \[ s_{l} \xrightarrow[r \rightarrow e]{\an{a,f,m} \rightarrow r}
      s_{l+1} \] with $e$ being some events, $a$ and $f$ being
      addresses and $m$ being a service token request for $Acct$.
  
    \item The processing steps from (1) and (2) are the same, i.e., $l = l'$.
  
    \item \label{lemma:item:form}The service token request for $Acct$, $m$ in (2), is an HTTPS message of the following form:
      \[ \mathsf{enc}_\mathsf{a}(\langle \hreq{ 
            nonce=n_\text{req}, 
            method=\mPost,
            xhost=d_r,
            path=\str{/authorize}, 
            parameters=x, 
            headers=h,
            xbody=b}, k\rangle, \pub(\mapTLSKey(d_r))) \]  
      for $d_r \in \mapDomain(r)$, some terms $x$, $h$, $n_\text{req}$, and a dictionary $b$ such that 
      \[ b[\str{IDToken}] \equiv \myangle{\mi{PID_{rp}, \mi{PID_u}, ver}} \]
      with 
      \[ \mi{PID_{rp}} \equiv [S^l(r).\str{loginSessions}[t]]S^l(r).\str{rp}, \]
      \[ \mi{PID_{u}} \equiv [u]\mi{PID_{rp}}, \]
      \[ \mi{ver} \equiv \sig{\an{PID_{rp},PID_u}}{k_{sign}} \]
      for some nonces $u$, and $k_\text{sign}$.
    \item If the IdP $i$ is honest, we have that $k_\text{sign} = S^l(i).\str{signkey}$.
    \end{enumerate}
  \end{lemma}
  
  We define the Lemma~\ref{lemma:k-does-not-leak-from-honest-rp} 
  and ~\ref{lemma:uppresso-request-exists}, which prove 
  that the data transmitted through HTTPS is secure and the 
  IdP's public key used for generating IDToken is secure. 
  In UPPRESSO, only the single IdP is trusted, so that the 
  public key is guaranteed to be always trusted. Therefore, 
  we can also follow the proofs in SPRESSO.
  
  %补充证明逻辑的描述
  %lemma8的描述需要明确malicious party是哪一个,honest party是哪一个
  %lemma8采用纯数学描述,去除UPPRESSO表述
  \subsection{Proof of Property A}
  Then we prove the Property $A$ is satisfied in UPPRESSO.
  As stated above, the Property $A$ is defined as follows:
  \begin{definition}\label{def:uppresso-security-property} 
    Let $\uppressoauthwebsystem$ be an \uppresso web system 
    for authentication analysis. We say that 
    \emph{$\uppressoauthwebsystem$ is secure 
    (with respect to Property A)} if for every run $\rho$ of 
    $\uppressoauthwebsystem$, every state $(S^j, E^j, N^j)$ in 
    $\rho$, every $r\in \fAP{RP}$ that is honest in $S^j$, 
    every RP service token of the form $\myangle{n, Acct}$ 
    recorded in $S^j(r).\str{serviceTokens}$ and derivable 
    from the attackers knowledge in $S^j$ (i.e., 
    $\myangle{n, Acct} \in 
    d_{\emptyset}(S^j(\fAP{attacker}))$), it follows that the 
    browser $b$ owning $Acct$ is fully corrupted in $S^j$ 
    (i.e., the value of $\mi{isCorrupted}$ is $\fullcorrupt$). 
  \end{definition}
  
  %\begin{definition}
  %Let $\mathcal{U\!W\!S}^{auth}$  be an UPPRESSO web system for authentication analysis. We say that $\mathcal{U\!W\!S}^{auth}$  is secure (with respect to Property A) if for every run $rho$ of $\mathcal{U\!W\!S}^{auth}$ , every state ($S^j$, $E^j$, $N^j$) in $rho$, every $r \in \mathtt{RP}$ that is honest in
  %$S^j$, every RP service token of the form $\langle IDToken$, $Acct \rangle$ recorded in $S^j$($r$).$\mathtt{serviceTokens}$ and derivable from the attackers knowledge in $S^j$ (i.e., $\langle IDToken$, $Acct \rangle \in d_{\emptyset}$($S^j$($\mathtt{attacker}$))), it follows that the browser b owning $Acct$ is fully corrupted in $S^j$ (i.e., the value of $isCorrupted$ is $\mathtt{FULLCORRUPT}$) or $\mathtt{governor}$($Acct$) is not an honest IdP (in $S^j$).
  %\end{definition}
  
  Same as the proof in SPRESSO, we want to show that every UPPRESSO web system is secure with regard to Property A and therefore assume that there exists an UPPRESSO web system that is not secure. We will lead this to a contradication and thereby show that all UPPRESSO web systems are secure (with regard to Property A).
  
  %In detail, we assume: \emph{There is an UPPRESSO web system for authentication analysis $\mathcal{U\!W\!S}^{auth}$. We say that $\mathcal{U\!W\!S}^{auth}$  is secure (with respect to Property A) if for every run $rho$ of $\mathcal{U\!W\!S}^{auth}$ , every state ($S^j$, $E^j$, $N^j$) in $rho$, every $r \in \mathtt{RP}$ that is honest in
  %$S^j$, every RP service token of the form $\langle IDToken$, $Acct \rangle$ recorded in $S^j$($r$).$\mathtt{serviceTokens}$ and derivable from the attackers knowledge in $S^j$ (i.e., $\langle IDToken$, $Acct \rangle \in d_{\emptyset}$($S^j$($\mathtt{attacker}$))), it follows that the browser b owning $Acct$ is not fully corrupted in $S^j$ and $\mathtt{governor}$($Acct$) is an honest IdP (in $S^j$).}
  
  In detail, we assume: \emph{There exists an \uppresso web 
  system $\uppressoauthwebsystem$, a run $\rho$ of 
  $\uppressoauthwebsystem$, a state $s_j = (S^j, E^j, N^j)$ 
  in $\rho$, a RP $r\in \fAP{RP}$ that is honest in $S^j$, 
  an RP service token of the form $\myangle{n, Acct}$
  recorded in $S^j(r).\str{serviceTokens}$ and derivable from 
  the attackers knowledge in $S^j$ (i.e., $\myangle{n, Acct} \in
  d_{\emptyset}(S^j(\fAP{attacker}))$), and the browser $b$ 
  owning $Acct$ is not fully corrupted (in $S^j$).}
  
  We now proceed to prove that this is a contradiction. 
  First, we can see that for $\an{nonce, Acct}$ and $s_j$, 
  the conditions in Lemma~\ref{lemma:uppresso-request-exists} 
  are fulfilled, i.e., a service token request $m$ and a 
  service token response $m'$ to/from $r$ exist, and $m'$ is 
  of form shown in Property~\ref{lemma:item:form} where there is an IDToken.
  Let $I$ be the Identity Provider. 
  We know that $I$ is an honest IdP.
  As such, it never leaks its signing key (see Algorithm~\ref{alg:idp}). 
  Therefore, the signed subterm $\mi{IDToken} := \myangle{\mi{Content}, \mi{ver}}$ in which
  $\mi{Content} := \myangle{PID_{rp}, PID_u}$ and 
  $\mi{ver} := \sig{\an{PID_{rp},PID_u}}{k_{sign}}$ 
  had to be created by the IdP $I$. 
  An (honest) IdP creates signatures only in Line~\ref{line:sign-token} of Algorithm~\ref{alg:idp}.
  
  %nonce从哪些获得?
  %当且仅当从系统中的任何一个角色获得token,攻击者可以实现攻击。
  %假设获得token,但无法获得,推翻假设。
  %spresso definition
  \begin{lemma}\label{lemma:b-trigger-request}%(Same as Lemma 4 in SPRESSO) 
    Under the assumption above, only the browser b can issue a 
    request $req$ (say, $m_{auth}$)that triggers the IdP I to 
    create the signed term IDToken. The request was sent by b 
    over HTTPS using I's public HTTPS key.
  \end{lemma}
  \begin{proof}
    We have to consider two cases for the request $m_{auth}$:
  
    \textbf{(A).} First, if the user is not logged in with the 
    $username$ at $I$ (i.e., the browser $b$ has no session 
    cookie that carries a nonce which is a session id at $I$ for 
    which the identitiy is marked as being logged in, compare 
    Line~\ref{line:uppresso-idp-check-login-state} of 
    Algorithm~\ref{alg:idp}), then the request has to carry (in
    the request body) the password matching the $username$ 
    ($\NToS(username)$) to the path $\str{/authentication}$ to 
    retrieve the session cookie. This secret is only known to 
    $b$ initially. Depending on the corruption status of $b$, we 
    can now have two cases:
    \begin{enumerate}
    \item[a)] If $b$ is honest in $s_j$, it has not sent the 
      password to any party except over HTTPS to $I$ (as defined 
      in the definition of browsers). 
    \item[b)] If $b$ is close-corrupted, it has not sent it to 
      any other party while it was honest (case a). When 
      becoming close-corrupted, it discarded the secret.
    \end{enumerate}
  
    I.e., the password has been sent only to $I$ over HTTPS or 
    to nobody at all. The IdP $I$ cannot send it to any other 
    party. Therefore we know that only the browser $b$ can send 
    the request $m_\text{attr}$ in this case.
  
    \textbf{(B).} Second, if the user is logged in at $I$, the 
    browser provides a session id to $I$ that refers to a 
    logged in session at $I$. This session id can only be 
    retrieved from $I$ by logging in, i.e., case (A) applies, 
    in particular, $b$ has to provide the proper password, 
    which only itself and $I$ know (see above). The session id 
    is sent to $b$ in the form of a cookie, which is set to 
    secure (i.e., it is only sent back to $I$ over HTTPS, and
    therefore not derivable by the attacker) and httpOnly 
    (i.e., it is not accessible by any scripts). The browser $b$ 
    sends the cookie only to $I$. The IdP $I$ never sends the 
    session id to any other party than $b$. The session id 
    therefore only leaks to $b$ and $I$, and never to the 
    attacker. Hence, the browser $b$ is the only atomic DY 
    process which can send the request $m_\text{auth}$ in this case.
  
    We can see that in both cases, the request was sent by $b$ 
    using HTTPS and $I$'s public key: If the browser would 
    intend to sent the request without encryption, the request 
    would not contain the password in case (A) or the cookie in 
    case (B). The browser always uses the ``correct'' encryption 
    key for any domain (as defined in $\uppressoauthwebsystem$).
  %The proof is same as the Lemma 4's proof in SPRESSO.
  %It can be proved that the $IDToken$ only contains the $PID_U:=[ID_U]PID_{RP}$ while $PID_U$ is provided by $b$, and $b$ owns the password of $ID_U$.
  \end{proof}
  
  \begin{lemma}\label{lemma:script-idp-trigger-request} %(Same as Lemma 5 in SPRESSO) 
    In the browser $b$, the request $m_{auth}$ was triggered by $\mi{script\_idp}$ 
    loaded from the origin $\myangle{d, S}$ for some $d \in \mathtt{dom}(I)$.
  \end{lemma}
  \begin{proof}
    First, $\an{d,\https}$ for some $d \in \mapDomain(I)$ is the 
    only origin that has access to the password $\NToS(u)$ for 
    the username $u$.
    (as defined in Appendix~\ref{app:browsers-uppresso}).
  
    With the general properties defined in~\cite{BrowserID} and the
    definition of Identity Providers in Appendix~\ref{app:idps}, in
    particular their property that they only send out one script,
    $\mi{script\_idp}$, we can see that this is the only script that can
    trigger a request containing the password.
  %The proof follows the Lemma 5's proof in SPRESSO.
  %It can be proved that only the IdP's script $script\_idp$ owns the password of $ID_U$ can request the $IDToken$ from $I$.
  \end{proof}

  \newc
  We now know that only the $\mi{script\_idp}$ in an honest browser 
  loaded from the IdP can issue a request for the IDToken. 
  That is to say, the attacker cannot request for the correct IDToken 
  by himself. Therefore, next we are going to discuss whether the attacker
  can steal the IDToken stored in honest parties.

  Obviously, the attacker is unable to get the IDToken from IdP because IdP 
  is always honest and never leak it.
  \oldc
  
  \begin{lemma} \label{lemma:idp-to-script-idp} %(Same as Lemma 6 in SPRESSO)
    In the browser $b$, the script $\mi{script\_idp}$ receives 
    the response to the request $m_{auth}$ (and no other script), 
    and at this point, the browser is still honest.
  \end{lemma}
  \begin{proof}
    From the definition of browser corruption, we can see that 
    the browser $b$ discards any information about pending 
    requests in its state when it becomes close-corrupted, in 
    particular any TLS keys. It can therefore not decrypt the 
    response if it becomes close-corrupted before receiving the 
    response.
  
    The rest follows from the general properties defined
    in~\cite{BrowserID}.
  %The proof follows Lemma 6's proof in SPRESSO.
  %It is proved that only the closed-corrupted browser cannot receive the $IDToken$ responding to the $req$ started by the honest browser $b$.
  \end{proof}
  
  We now know that only the script $\mi{script\_idp}$ received 
  the response containing the IDToken. For the following lemmas, 
  we will assume that the browser $b$ is honest. In the other 
  case (the browser is close-corrupted), the IDToken and any 
  information would be discarded from the browser's state 
  (as seen in the proof for Lemma~\ref{lemma:idp-to-script-idp}). 
  This would be a contradiction to the assumption 
  (which requires that the IDToken arrived at the RP).
  
  %Lemma 7 in SPRESSO is not useful here because there is no FWD server in UPPRESSO.
  
  \begin{lemma}\label{lemma:script-idp-to-script-rp} %(Same as Lemma 8 in SPRESSO) 
    The script $\mi{script\_idp}$ forwards the IDToken only to 
    the script $\mi{script\_rp}$ loaded from the origin 
    $\langle d_r, \https\rangle$.
  \end{lemma}
  \begin{proof}
    It is clear that, the IDToken held by the honest 
    $script\_idp$ is only sent to the origin 
    $\langle Cert_{rp}.Enpt_{rp}, \https\rangle$, 
    while the $IDToken.PID_{rp} \equiv [t]Cert_{rp}.ID_{rp}$, 
    and $t$ is the one-time random number. The relation of 
    $Cert_{rp}.ID_{rp}$ and $Cert_{rp}.Enpt_{rp}$ is guaranteed 
    by the signature $Cert_{rp}.ver$ generated by IdP $I$. 
    The process is shown at Line~\ref{line:token-send}
    Algorithm~\ref{alg:uppresso-script-idp}.
  %The proof is same as proof of Lemma 8 in SPRESSO.
  %It can be proved that, the $IDToken$ held by the honest $script\_idp$ is only sent to the origin $\langle Cert_{RP}.Enpt_{RP}, S \rangle$, while the $IDToken.PID_{RP} \equiv [t]Cert_{RP}.ID_{RP}$, and $t$ is the one-time random number.  The relation of $ID_RP$ and $Enpt$ is guaranteed by the signature generated by IdP $I$. The process is shown at Line 9, 16, 19, 21, 38, 39, 59, 60  in Algorithm~\ref{alg:script_idp}.
  \end{proof}
  
  \begin{lemma}\label{lemma:script-rp-to-rp} %(Same as Lemma 9 in SPRESSO) 
    From the RP document, the IDToken is only sent to the RP r 
    and over HTTPS
  \end{lemma}
  \begin{proof}
    It is proved that $script\_rp$ of the origin 
    $\langle Cert_{rp}.Enpt_{rp}, \https\rangle$ 
    would only sent to the corresponding RP $r$, 
    which is shown in Algorithm~\ref{alg:uppresso-script-rp}.
  %The proof follows the proof of Lemma 9 in SPRESSO.
  \end{proof}
  
  \newc
  The proofs show that the IDToken is only sent to the honest 
  browser and target RP. It cannot be known to the attacker or 
  any corrupted party, as none of the listed parties leak it to 
  any corrupted party or the attacker.
  \oldc
  
  These proofs are enough for SPRESSO system to show its 
  security, however, they are not enough for UPPRESSO. So far, 
  the proofs only guarantee that the $IDToken$ is never leaked 
  to the attacker. Obviously, the $tag$ in SPRESSO can be only 
  decrypted to a unique domain of RP. However, in UPPRESSO, this 
  statement is not easy to see, an attacker may misuse a wrong 
  $IDToken$ to retrieve the service token from an honest RP, 
  i.e., the attacker can use an IDToken from a corrupted RP. 
  He may try find the $t^{adversary}$ 
  satisfied $IDToken.PID_{RP} \equiv [t^{adversary}]ID_{RP}^{honest}$.
  Therefore, the following Lemmas should be proved.

  \begin{lemma}
    The $t^{adversary}$ is not derivable from the attackers knowledge in $S^j$ (i.e., $t^{adversary} \in d_{\emptyset}$($S^j$($\mathtt{attacker}$))), which satisfies that $IDToken.PID_{RP} \equiv [t^{adversary}]ID_{RP}^{honest}$.
  \end{lemma}
  \begin{proof}
    In UPPRESSO, $PID_{RP}=[t]ID_{RP}$ is generated by a user based on the target RP's identity $ID_{RP}$ and a user-selected random number $t \in [1,n)$.
    %The target RP with $ID_{RP}$ receives $t$, and it will also calculate $PID_{RP}=[t]ID_{RP}$ to match $PID_{RP}$ extracted from a token received.
    %It is computationally easy for any party who knows $ID_{RP}$ and $t$ to validate the $PID_{RP}$ in an identity token. A valid
    Thus, $PID_{RP}$ always specifies an RP, i.e., %$PID_{RP}$ sent by a user in her identity-token request is calculated as $PID_{RP} = [t]ID_{RP}$, where $ID_{RP}$ is the target RP's identity and $t$ is a random number selected by the user and shared with this RP.
    designates the target RP that knows $t$. 
    Moreover, according to Lemma \ref{lemma-rp}, given $PID_{RP} = [t]ID_{RP}$, the probability that $PID_{RP}$ designates another RP with $ID_{RP'}$ is \emph{negligible}. %This means that $PID_{RP}$ cannot be associated with any other RPs in the system.
    Therefore, $PID_{RP}$ designates only the target RP with $ID_{RP}$ in the system, 
    so attacker cannot find a number $t^{adversary}$\hfill
  \end{proof}
  
  \begin{lemma}\label{lemma-rp}
    Given any two points on an elliptic curve denoted by $[r]G$ and $[r']G$ 
    where $r$ and $r'$ are different numbers unknown to an adversary, 
    and $G$ is a generator on $\mathbb{E}$ of order $n$, 
    the probability that the adversary finds different numbers $t$ and $t' \in [1,n)$ 
    satisfying $[tr]G = [t'r']G$ is negligible.
  \end{lemma}
  \begin{proof}
    Finding $t$ and $t'$ that satisfy $[tr]G = [t'r']G$ can be described as 
    a collision game $\mathcal{G}_c$ between an adversary and a challenger: 
    the adversary receives from the challenger a finite set of points, 
    i.e., $[r_1]G$, ..., $[r_m]G$, where $m$ is the number of points, 
    and outputs $(a, b, t, t')$ where $a \neq b$. If $[tr_a]G=[t'r_b]G$, 
    which occurs with a probability ${\rm Pr}_s$, the adversary succeeds in this game.
    
    As depicted in Figure \ref{fig:ecdlp_algorithm}, 
    we design a probabilistic polynomial time (PPT) algorithm $\mathcal{D}^*_c$ based on $\mathcal{G}_c$, 
    to solve the ECDLP: find a number $x \in \mathbb{Z}_n$ satisfying $Q = [x]G$, 
    where $Q$ is a point on $\mathbb{E}$ and $G$ is a generator on $\mathbb{E}$ of order $n$.
  
    \begin{figure}[tb]
      \centering
      \includegraphics[width=0.96\linewidth]{fig/ecdlp_algorithm.pdf}
      \caption{The PPT algorithm $\mathcal{D}^*_c$ constructed based on the $PID_{RP}$ collision game to solve the ECDLP.}
      \label{fig:ecdlp_algorithm}
    \end{figure}
  
    The algorithm $\mathcal{D}^*_c$ works as below.
    The input of $\mathcal{D}^*_c$ is in the form of ($G, Q$). 
    On receiving an input ($G$, $Q$), 
    the challenger first randomly chooses $r_1, \cdots, r_m$ in $\mathbb{Z}_n$ to calculate $[r_1]G, \cdots, [r_m]G$.
    Then, it randomly chooses $j \in [1,m]$, replaces $[r_j]G$ with $Q$, and sends $m$ points to the adversary, 
    which returns the result ($a$, $b$, $t$, $t'$). 
    Finally, the challenger calculates $s = t^{-1}t'r_b \bmod n$ and returns $s$ as the output of $\mathcal{D}^*_c$.
  
    If the adversary succeeds in $\mathcal{G}_c$ and $[r_a]G$ happens to be replaced with $Q$, 
    then $\mathcal{D}^*_c$ outputs $s=t^{-1}t'r_b =x$ because $[tr_a]G = [t]Q = [t'r_b]G$. 
    For the adversary, $Q$ is indistinguishable from any other points in the input set, 
    as $[r_j]G$ is randomly replaced by the challenger.
    Hence, the probability of solving the ECDLP using $\mathcal{D}^*_c$ is formulated as:
    \begin{equation*}
      {\rm Pr}\{\mathcal{D}^*_c(G, [x]G)=x\} = {\rm Pr}\{s = x\}={\rm Pr}\{a=j\}{\rm Pr}_s=\frac{1}{m}{\rm Pr}_s
    \end{equation*}
  
    If the probability of finding $t$ and $t'$ satisfying $[tr]G = [t'r']G$ is non-negligible, 
    the adversary would also have non-negligible advantages in $\mathcal{G}_c$ and ${\rm Pr}_s$ regardless of the security parameter $\lambda$.
    Thus, we would find that ${\rm Pr}\{\mathcal{D}^*_c(G, [x]G)=x\}$ also becomes non-negligible even when $\lambda$ is sufficiently large, 
    because $m$ is a finite integer and $m \ll 2^\lambda$.
    This violates the ECDLP assumption. 
    Therefore, the probability of finding $t$ and $t'$ that satisfy $[tr]G = [t'r']G$ is negligible.
  \end{proof}
  
  Therefore, there is a contradication to the assumption, where we assumed that 
  $Acct \in d_{\emptyset}(S^j(\fAP{attacker}))$. 
  This shows every $\mathcal{U\!W\!S}^{auth}$ is secure in the sense of Property A.
  
  \subsection{Proof of Property B}
  As stated above, Property B is defined as follows:
  \begin{definition}\label{def:B}
    Let $\uppressoauthwebsystem$ be an \uppresso web system. We say that
    \emph{$\uppressoauthwebsystem$ is secure (with respect to Property B)} if
    for every run $\rho$ of $\uppressoauthwebsystem$, every state $(S^j, E^j, N^j)$
    in $\rho$, every $r\in \fAP{RP}$ that is honest in $S^j$, 
    every RP service token of the form $Acct$ recorded in
    $S^j(r).\str{serviceTokens}$, with the request corresponding to
    $\myangle{nonce, Acct}$ sent by some $b\in \fAP{B}$ which is honest in $S^j$, $b$ owns $Acct$.
  %Let $\mathcal{U\!W\!S}^{auth}$  be an UPPRESSO web system for authentication analysis. We say that $\mathcal{U\!W\!S}^{auth}$  is secure (with respect to Property A) if for every run $rho$ of $\mathcal{U\!W\!S}^{auth}$ , every state ($S^j$, $E^j$, $N^j$) in $rho$, every $r \in \mathtt{RP}$ that is honest in
  %$S^j$, every RP service token of the form $\langle IDToken$, $Acct \rangle$ recorded in $S^j$($r$).$\mathtt{serviceTokens}$, with the request corresponding to $\langle IDToken$, $Acct \rangle$ sent by some $b \in B$ which is honest in $S^j$, b owns Acct.
  \end{definition}
  
  First we call the request corresponding to $Acct$ (or service token request) $m$ and
  its response $m'$, and we refer to the state of $\uppressoauthwebsystem$ in the run 
  $\rho$ where $r$ processes $m$ by $s_l$. We are going to prove the $IDToken$ uploaded 
  by honest $b$ can only be related with the $Acct$ owned by $b$.
  
  %we follows the Lemma 10 and its proof in SPRESSO, which guarantees that the request corresponding to $\langle IDToken$, $Acct \rangle$ sent by honest $b$ is loaded from $script\_rp$. 
  %Then we are going to prove the $IDToken$ uploaded by honest $b$ can only be related with the $Acct$ owned by $b$ (which is quite different from SPRESSO).
  
  \begin{lemma}\label{lemma:request-m-is-from-script-rp}
    The request $m$ was sent by $\mi{script\_rp}$ loaded from 
    the origin $\an{d_r, \https}$ where $d_r$ is some domain of 
    $r$.
  \end{lemma}
  
  \begin{proof}
    The request $m$ is XSRF protected. In Algorithm~\ref{alg:rp}, 
    RP checks the presence of the Origin header and its value. 
    If the request $m$ was initiated by a document from a 
    different origin than $\an{d_r, \https}$, the honest browser 
    $b$ would have added an Origin header that would not 
    pass this test (or no Origin header at all), according to 
    the browser definition. The script $\mi{script\_rp}$ is the 
    only script that the honest party $r$ sends as a response 
    and that sends a request to $r$.
  \end{proof}
  
  \begin{lemma}
    For every $IDToken$ uploaded by honest $b$ during authentication, 
    the honest $r \in RP$ can always derive the service token of the form 
    $\myangle{n, \mi{Acct}}$ recorded in $S^j$($r$).$\mathtt{serviceTokens}$, where b owns Acct. 
  \end{lemma}
  \begin{proof}
    According to lemma~\ref{lemma:request-m-is-from-script-rp}, 
    we know that $m$ was sent by $\mi{script\_rp}$ loaded from an honest relying party. 
    The RP accepts the user's identity at line~\ref{line:add-service-token} in Algorithm~\ref{alg:rp}.
    And the user's identity at RP is generated at Line~\ref{line:gen-acct}, 
    based on the $PID_u$ retrieved from the IDToken and the trapdoor $t^{-1}$. 
    
    The $t^{-1}$ is generated and set at Line~\ref{line:gen-t}, 
    holding that $\mi{PID_{rp}}=[t]\mi{ID_{rp}}$.
    Originally, $t$ is chosen by $\mi{script\_idp}$ at Line~\ref{line:gen-t}, Algorithm~\ref{alg:uppresso-script-idp} 
    and transmit to RP through $\mi{script\_rp}$.
    \newc
    Similar to lemma~\ref{lemma:script-idp-to-script-rp} and lemma~\ref{lemma:script-rp-to-rp}, 
    we can prove that RP can only receive the $t$ from $\mi{script\_idp}$, so the $t$ in both parties is equal.
    \oldc

    The $\mi{script\_idp}$ also receives a cert signed by the IdP from the honest RP and verify the cert at Line~\ref{alg:script-idp-verify-cert}, Algorithm~\ref{alg:uppresso-script-rp}.
    After that, $\mi{script\_idp}$ calculates $\mi{PID_{rp}}$ using $t$ and $\mi{ID_{rp}}$ from the cert.
    \newc
    Since $t$ in both parties is equal and $\mi{ID_{rp}}$ is guaranteed by the cert's signature, 
    we can have that the $\mi{PID_{rp}}$ in $\mi{script\_idp}$ and RP is equal.
    \oldc

    Then $\mi{script\_idp}$ sends a request to IdP bringing $\mi{PID_{rp}}$ for the IDToken 
    at Line~\ref{line:send-pidrp} in Algorithm~\ref{alg:uppresso-script-idp}.
    IdP adds the $\mi{PID_{rp}}$ from $\mi{script\_idp}$ into the IDToken.
    \newc
    Therefore, we can see that 
    $\mi{IDToken}.\mi{PID_{rp}}=[t]\mi{ID_{rp}}$ meaning that it is the same as in RP.
    \oldc

    The IDToken is issued at Line~\ref{line:sign-token} in Algorithm~\ref{alg:idp}.
    The IdP generates the $PID_u$ based on the $PID_{rp}$ and $ID_u$.
    \newc
    According to lemma~\ref{lemma:b-trigger-request}, 
    the identities IdP accept and store is from the honest browser, 
    and the honest browser can only provide its own usernames and passwords 
    because only $\mi{script\_idp}$ has the access to these accounts 
    according to lemma~\ref{lemma:script-idp-trigger-request}.
    Therefore, attacker cannot alter the username and password sent to IdP.
    Since the username and password are owned by the honest browser and 
    $ID_u$ is mapped to the account at Line~\ref{line:get-idu} in Algorithm~\ref{alg:idp}. 
    Now we can see that the $ID_u$ used to generate IDToken must be owned by the browser 
    and $\mi{IDToken}.\mi{PID_{u}}=[\mi{ID_u}]\mi{PID_{rp}}$.
    

    %In summary, for every IDToken sent by honest $b$ and IdP, there must be 
    %$\mi{IDToken}.\mi{PID_{rp}} \equiv [t]\mi{ID_{rp}} \equiv \mi{PID_{rp}}$ and 
    %$\mi{IDToken}.\mi{PID_{u}} \equiv [\mi{ID_u}]\mi{PID_{rp}}$. 
    %According to lemma~\ref{user-identification}, 
    Therefore, the $Acct$ calculated by RP following Equation~\ref{equ:calc-acct} must uniquely identifies the user $ID_u$ which is owned by honest $b$.  
    \oldc
    \begin{equation}\label{equ:calc-acct}
      \begin{split}
      Acct=[t^{-1}]PID_u=[t^{-1}][ID_u]PID_{rp}=\\
      [t^{-1}utr]G =[ur]G=[ID_U]ID_{RP}
      \end{split}
    \end{equation}
  \end{proof}

  \begin{lemma}\label{user-identification}
    $PID_u= [ID_u]PID_{rp}$ in IDToken uniquely identifies an 
    account at the RP designated by $PID_{rp}$ if and only if 
    it receives $t$ where $PID_{rp} = [t]ID_{rp}$ holds, and 
    this account is uniquely mapped to a user with $ID_u$.
  \end{lemma}
  \begin{proof}
    To issue an identity token requested for $PID_{rp}$, the 
    honest IdP authenticates the user with $ID_u$ and calculates 
    $PID_u = [ID_u]PID_{rp}$, following Equation \ref{equ:PIDU}. 
    The RP designated by $PID_{rp}$ should have received a $t$ 
    from the user. Following Equation \ref{equ:AccountNotChanged}, 
    it can calculate $Acct = [t^{-1}]PID_{u} = [ID_u]ID_{rp}$, 
    which is a \emph{permanent} identifier determined by $ID_u$ 
    and $ID_{rp}$ after the user and the RP register at the IdP. 
    $ID_{rp} = [r]G$ is a generator on $\mathbb{E}$ of order $n$, 
    as $\mathbb{E}$ is a finite cyclic group. Therefore, given a 
    user with $ID_u$, $Acct$ is a \emph{unique} point on 
    $\mathbb{E}$ for any $u \in [1, n)$, and it is \emph{uniquely} 
    associated with $ID_u=u$. 
    %We first prove that $PID_{U}$ \emph{uniquely} identifies one account at the designated RP and one user in the system. 
    
    \begin{equation}\label{equ:PIDU}
      PID_{U} = [{ID_U}]{PID_{RP}} = [utr]G
    \end{equation}
  
    \begin{equation}\label{equ:AccountNotChanged}
      Acct =  [t^{-1}utr \bmod n]G = [ur]G = [ID_U]ID_{RP}
    \end{equation}
  
    This proves that $PID_u$ in IDToken identifies an account 
    $Acct$ at the designated RP, which is uniquely mapped to a 
    user with $ID_u$ in the system.
  
    Next, we consider two adversarial scenarios where the 
    attacker replays a token for another user to (1) the 
    designated RP but receiving $t'\neq t$ in this login, and 
    (2) any other honest RP with $ID_{rp'} = [r']G \neq [r]G$ 
    (i.e., $r' \neq r$). In the first case, the designated RP 
    would calculate an account as $[t'^{-1}]PID_u = [t'^{-1}ut]ID_{rp}$.
    Because a user's identity is randomly selected by the IdP 
    in $\mathbb{Z}_n$ and known only to the user (and the honest 
    IdP), the probability that $t'^{-1}ut$ happens to be the 
    identity of another user is negligible, when $n$ is 
    sufficiently large. As a result, $[t'^{-1}]PID_u$ is likely 
    not to identify any known account at the RP and therefore 
    would be treated as a new account by the RP. 
    Secondly, the attacker presents IDToken to $RP'$ in the 
    system, where $ID_{rp'} = [r']G \neq [r]G$. $RP'$ would 
    calculate the account as $Acct' = [\tilde{t}^{-1}]PID_{u} = 
    [\tilde{t}^{-1}utrr'^{-1}][r']G = 
    [\tilde{t}^{-1}utrr'^{-1}]ID_{rp'}$. 
    The probability that $\tilde{t}^{-1}utrr'^{-1}$ happens to 
    be the identity of another user at $RP'$ is also negligible, 
    when $n$ is sufficiently large. 
    %it identifies no account mapped to a user, at any RP not designated by $PID_{RP}$.
  \end{proof}
  
  With the above proofs, we now can guarantee that every 
  $\uppressoauthwebsystem$ system satisfies the requirements in 
  Definition~\ref{def:B}, therefore $\uppressoauthwebsystem$ 
  must be secure of Property B.
  
  These prove Theorem~\ref{thm:authentication}.\QED
  
  \section{Proof of Privacy against IdP-based Login Tracing}
  
  In our privacy analysis, we show that an identity provider in UPPRESSO cannot learn 
  where its users log in. We formalize this property as an indistinguishability 
  property: an identity provider (modeled as a web attacker) cannot distinguish 
  between a user logging in at one relying party and the same user logging in at 
  a different relying party.
  
  We will here first describe the precise model that we use for privacy.
  After that, we define an equivalence relation between configurations,
  which we will then use in the proof of privacy.
  
  \subsection{Formal Model of UPPRESSO for Privacy Analysis}
  
  \begin{definition}[Challenge Browser]
    Let $\mi{dr}$ some domain and $b(\mi{dr})$ a DY process. 
    We call $b(\mi{dr})$ a \emph{challenge browser} iff $b$
    is defined exactly the same as a browser with two exceptions: 
    (1) the state contains one more property, namely 
    $\mi{challenge}$, which initially contains the term $\top$. 
    (2) The broswer's algorithm is extended by the following at 
    its very beginning: It is checked if a message $m$ is 
    addressed to the domain $\str{CHALLENGE}$ (which we call the 
    challenger domain). If $m$ is addressed to this domain and 
    no other message $m'$ was addressed to this domain before 
    (i.e., $\mi{challenge} \not\equiv \bot$), then $m$ is changed 
    to be addressed to the domain $\mi{dr}$ and $\mi{challenge}$ 
    is set to $\bot$ to recorded that a message was addressed to 
    $\str{CHALLENGE}$.
  \end{definition}
  
  \begin{definition}[Deterministic DY Process]
    We call a DY process $p = (I^p,Z^p,R^p,s_0^p)$ \emph{deterministic} iff 
    the relation $R^p$ is a (partial) function.
  
    We call a script $R_\text{script}$ \emph{deterministic} iff the relation 
    $R_\text{script}$ is a (partial) function.
  \end{definition}
  
  \begin{definition}[\uppresso Web System for Privacy Analysis]\label{def:uppresso-ws-priv}
    Let $\uppressowebsystem = (\bidsystem, \scriptset, 
    \mathsf{script}, E^0)$ be an UPPRESSO web system with 
    $\bidsystem = \mathsf{Hon} \cup \mathsf{Web} \cup \mathsf{Net}$, 
    $\mathsf{Hon} = \fAP{B} \cup \fAP{RP} \cup \fAP{IDP}$.
    (as described in Appendix~\ref{app:outlineuppressomodel}).
    $\fAP{RP} = \{r_1,r_2\}$, $r_1$ and $r_2$ two (honest) relying parties,
    Let $\fAP{attacker} \in \mathsf{Web}$ be some web attacker.
    Let $\mi{dr}$ be a domain of $r_1$ or $r_2$ and $b(\mi{dr})$ a challenge browser. 
    Let $\mathsf{Hon}' := \{ b(\mi{dr}) \} \cup \fAP{RP}$, 
    $\mathsf{Web}' := \mathsf{Web}$, 
    and $\mathsf{Net}' := \emptyset$ (i.e., there is no network attacker).
    Let $\bidsystem' := \mathsf{Hon}' \cup \mathsf{Web}' \cup \mathsf{Net}'$.  
    Let $\scriptset' := \scriptset$ and $\mathsf{script}'$ be accordingly.
    We call $\uppressoprivwebsystem(\mi{dr}) = (\bidsystem', \scriptset', \mathsf{script}', E^0, \fAP{attacker})$ 
    an \emph{\uppresso web system for privacy analysis} 
    iff the domain $\mi{dr}_1$ the only domain assigned to $r_1$, and
    $\mi{dr}_2$ the only domain assigned to $r_2$. The browser
    $b(\mi{dr})$ owns exactly one identity and this identity
    is governed by some attacker.  All honest parties (in
    $\mathsf{Hon}$) are not corruptible, i.e., they ignore any
    $\str{CORRUPT}$ message. Identity providers are assumed to be
    dishonest, and hence, are subsumed by the web attackers (which
    govern all identities). %In the initial state $s_0^b$ of the (only)
    %browser in $\bidsystem'$ and in the initial states $s_0^{r_1}$,
    %$s_0^{r_2}$ of both relying parties, the DNS address is
    %$\mapAddresstoAP(\fAP{dns})$. Further, $\mi{wkCache}$ in the initial
    %states $s_0^{r_1}$, $s_0^{r_2}$ is equal and contains a public key
    %for each domain registered in the DNS server (i.e., 
    the relying
    parties already know some public key to verify \uppresso identity
    assertions from all domains known in the system and they do not have to fetch them from IdP.
  \end{definition}
  
  As all parties in an \uppresso web system for privacy analysis are either web 
  attackers, browsers, or deterministic processes and all scripting processes are 
  either the attacker script or deterministic, it is easy to see that in \uppresso 
  web systems for privacy analysis with configuration $(S,E,N)$ a command $\zeta$ 
  induces at most one processing step. We further note that, under a given infinite 
  sequence of nonces $N^0$, all schedules $\sigma$ induce at most one run 
  $\rho = ((S^0,E^0,N^0),\dots,(S^i,E^i,N^i),\dots,(S^{|\sigma|},E^{|\sigma|},N^{|\sigma|}))$ 
  as all of its commands induce at most one processing step for the $i$-th configuration.
  
  We will now define our privacy property for \uppresso:
  
  \begin{definition}[IdP-Privacy]\label{def:idp-privacy}
    Let 
    \begin{align*}
      \uppressoprivwebsystem_1 := \uppressoprivwebsystem(\mi{dr}_1) =
      (\bidsystem_1, \scriptset, \mathsf{script}, E^0, \fAP{attacker}_1)&\text{ and}\\
      \uppressoprivwebsystem_2 := \uppressoprivwebsystem(\mi{dr}_2) =
      (\bidsystem_2, \scriptset, \mathsf{script}, E^0, \fAP{attacker}_2)&
    \end{align*}
    be \uppresso web systems for privacy analysis.  Further, we require
    $\fAP{attacker}_1 = \fAP{attacker}_2 =: \fAP{attacker}$ and for $b_1
    := b(\mi{dr}_1)$, $b_2 := b(\mi{dr}_2)$ we require $S(b_1) = S(b_2)$
    and $\bidsystem_1 \setminus \{b_1\} = \bidsystem_2 \setminus
    \{b_2\}$ (i.e., the web systems are the same up to the parameter of
    the challenge browsers).  We say that $\uppressoprivwebsystem$ is
    \emph{IdP-private} iff $\uppressoprivwebsystem_1$ and
    $\uppressoprivwebsystem_2$ are indistinguishable.
  \end{definition}
  
  \subsection{Definition of Equivalent Configurations}\label{app:defin-equiv-stat}
  
  Let $\uppressoprivwebsystem_1 = (\bidsystem_1, \scriptset, \mathsf{script}, E^0, \fAP{attacker})$ 
  and $\uppressoprivwebsystem_2 = (\bidsystem_2, \scriptset, \mathsf{script}, E^0, \fAP{attacker})$ 
  be \uppresso web systems for privacy analysis. Let $(S_1,E_1,N_1)$ 
  be a configuration of $\uppressoprivwebsystem_1$ and $(S_2,E_2,N_2)$ 
  be a configuration of $\uppressoprivwebsystem_2$.
  
  \begin{definition}[Proto-Tags]
    We call a term of the form $[t]R$ with the variable
    $R$ as a placeholder for an $ID_{rp}$, and $t$ some nonces a
    \emph{proto-tag}.
  \end{definition}
  
  \begin{definition}[Term Equivalence up to Proto-Tags]
    Let $\theta = \{a_1, \ldots, a_l \}$ be a finite set of proto-tags.
    Let $t_1$ and $t_2$ be terms. We call $t_1$ and $t_2$
    \emph{term-equivalent under a set of proto-tags $\theta$} iff there
    exists a term $\tau \in \terms(\{x_1,\dots,x_l\})$ such that
    $t_1 = (\tau [ a_1 / x_1 , \dots , a_l / x_l ])[ ID_{\mi{dr}_1} / R ]$ and
    $t_2 = (\tau [ a_1 / x_1 , \dots , a_l / x_l ])[ ID_{\mi{dr}_2} / R ]$.
    We write $t_1 \prototagequiv{\theta} t_2$.
  
    We say that two finite sets of terms $D$ and $D'$ are
    \emph{term-equivalent under a set of proto-tags $\theta$} iff
    $|D| = |D'|$ and, given a lexicographic ordering of the elements in
    $D$ of the form $(d_1,\dots,d_{|D|})$ and the elements in $D'$ of
    the form $(d'_1,\dots,d_{|D'|})$, we have that for all
    $i \in \{1,\dots,|D|\}$: $d_i \prototagequiv{\theta} d'_i$. We then
    write $D \prototagequiv{\theta} D'$.
  \end{definition}
  
  %随机数t的情况
  \begin{definition}[Equivalence of HTTP Requests]
    Let $m_1$ and $m_2$ be (potentially encrypted) HTTP requests, 
    $L$ be a set of login session tokens and
    $\theta = \{a_1, \ldots, a_l \}$ be a finite set of proto-tags. 
    We call $m_1$ and $m_2$ \emph{$\delta$-equivalent under a set of proto-tags $\theta$} 
    iff $m_1 \prototagequiv{\theta} m_2$ or all subterms are equal with the following exceptions:
    \begin{enumerate}
    \item the Host value and the Origin/Referer headers in both requests
      are the same except that the domain $\mi{dr}_1$ in $m_1$ can be
      replaced by $\mi{dr}_2$ in $m_2$,
    \item If the cookie in both requests include $\str{loginSessionToken}$, 
      then there exists an $l' \in L$ such that $g_1[\str{loginSessionToken}] \equiv l'$, and
    \item the HTTP body $g_1$ of $m_1$ and the HTTP body $g_2$ of $m_2$
      are (I) term-equivalent under $\theta$, 
      (II) for $j\in \{1,2\}$ if
      $g_j[\str{IDToken}] \sim \myangle{PID_{dr_j}, [*]PID_{dr_j}, 
      \sig{\myangle{PID_{dr_j}, [*]PID_{dr_j}}}{*}}$
      and the origin (HTTP header) of HTTP message in $m_j$ is
      $\an{\mi{dr}_j,\https}$ then the receiver of this message is
      $r_j$, and 
    \item if $m_1$ is an encrypted HTTP request then and only then $m_2$
      is an encrypted HTTP request and the keys used to encrypt the
      requests have to be the correct keys for $\mi{dr}_1$ and
      $\mi{dr}_2$ respectively.
    \end{enumerate}
    We write $m_1 \httptagequiv{\theta} m_2$.
  \end{definition}
  
  %loginsessionrecord := <t, tag>
  \begin{definition}[Extracting Entries from Login Sessions]
    Let $t_1$, $t_2$ be dictionaries over $\nonces$ and $\terms$,
    $\theta$ be a finite set of proto-tags, and $d$ a domain. We call
    $t_1$ and $t_2$ \emph{$\eta$-equivalent} iff $t_2$ can be
    constructed from $t_1$ as follows: For every proto-tag
    $a \in \theta$, we remove the entry identified by the dictionary key
    $i$ for which it holds that $\proj{2}{t_1[i]} \equiv a[ID_r/ R]$, if
    any. We denote the set of removed entries by $D$. We write
    $\logsessminus{t_1}{t_2}{\theta}{r}{D}$.
  \end{definition}
  
  \begin{definition}
    Let $a$ be a proto-tag, $S_1$ and $S_2$ be states of \uppresso web
    systems for privacy analysis, and $l$ a nonce. We call $l$ a login
    session token for the proto-tag $a$, written
    $l \in \mathsf{loginSessionTokens}(a,S_1,S_2)$ iff for any
    $i \in \{1,2\}$ and any $j \in \{1,2\}$ we have that
    $\proj{2}{S_i(r_j).\str{loginSessions}[l]} = a[ID_{dr_j}/ R]$.
  \end{definition}
  
  \begin{definition}[Equivalence of States]\label{def:eq-of-states}
    Let $\theta$ be a set of proto-tags and 
    %$H$ be a set of nonces. 
    $L$ be a set of login session tokens.
    Let $T:=\{t\mid [t]R\in \theta\}$. 
    We call $S_1$ and $S_2$ \emph{$\gamma$-equivalent under 
    $(\theta, L)$} iff the following conditions are met:
    \begin{enumerate}
    \item\label{eqs:r1} $S_1(\fAP{r_1})$ equals $S_2(\fAP{r_1})$ except
      for the subterms $\str{loginSessions}$ and $\str{serviceTokens}$, and
    \item\label{eqs:r2} $S_1(\fAP{r_2})$ equals $S_2(\fAP{r_2})$ except
      for the subterms $\str{loginSessions}$ and $\str{serviceTokens}$, and
    \item\label{eqs:logsess} for two sets of terms $D$ and $D'$:
      $\logsessminus{S_1(\fAP{r_1}).\str{loginSessions}}{S_2(\fAP{r_1}).\str{loginSessions}}{\theta}{\mi{dr}_1}{D}$,
      $\logsessminus{S_2(\fAP{r_2}).\str{loginSessions}}{S_1(\fAP{r_2}).\str{loginSessions}}{\theta}{\mi{dr}_2}{D'}$,
      and $D \prototagequiv{\theta} D'$, and
    \item\label{eqs:att-not-t} $\forall t \in T$:
      $t \not\in d_\emptyset(\bigcup_{i \in \{1,2\},\ A\, \in\, \mathsf{Web}\, \cup \,
      \mathsf{Net}\, 
      %\cup\, \{\mathsf{dns}, \mathsf{fwd}
      \}}S_i(A))$
    \item\label{eqs:att} for each attacker $A$:
      $S_1(A) \prototagequiv{\theta} S_2(A)$, and
    \item\label{eqs:att-not-l} for all $a\in\theta$ and all attackers $A$ we have that
      $\nexists\ l \in \mathsf{loginSessionTokens}(a,S_1,S_2)$ such that
      $l$ is a subterm of $S_1(A)$ or $S_2(A)$.
    \item\label{eqs:b} $S_1(b_1)$ equals $S_2(b_2)$ except for for the
      subterms $\str{challenge}$, $\str{windows}$
      %, $\str{pendingDNS}$, $\str{pendingRequests}$ 
      and we have that
      \begin{enumerate}
      \item \label{eqs:b:challenge}
        $S_1(b_1).\str{challenge} = \mi{dr}_1 \wedge
        S_2(b_2).\str{challenge} = \mi{dr}_2$
        or $S_1(b_1).\str{challenge} = S_2(b_2).\str{challenge} = \bot$,
        and
      \item $S_1(b_1).\str{windows}$ equals $S_2(b_2).\str{windows}$ with
        the exception of the subterms $\str{location}$, $\str{referrer}$,
        $\str{scriptstate}$, and $\str{scriptinputs}$ of some document terms
        pointed to by $\mathsf{Docs}^+(S_1(b_1)) = \mathsf{Docs}^+(S_2(b_2)) =: J$. 
        For all $j \in J$ we have that: \label{eqs:b:w}
        \begin{enumerate}
        \item there is no $t \in T$ such that
          \begin{align*}
            t \in d_{\nonces \setminus \{t\}}(\{&S_1(b_1).j.\str{location}
            ,  S_2(b_2).j.\str{location},\\ & S_1(b_1).j.\str{referrer} , 
            S_2(b_2).j.\str{referrer}\})
          \end{align*}
        \item for $p \in \{$
          \begin{align*}
            & \an{\tXMLHTTPRequest,*,*},\\
            & \an{\tPostMessage,*,\an{\mapDomain(dr_j), \https},\an{\str{t}, *}},\\
            & \an{\tPostMessage,*,\an{\mapDomain(dr_j), \https},\an{\str{IDToken}, *}}\\
            & \an{\tPostMessage,*,\an{\mapDomain(idp), \https},\an{\str{Cert}, *}}
          \end{align*}
          $\}$ we have
          $S_1(b_1).j.\str{scriptinputs} |\, p \prototagequiv{\theta}
          S_2(b_2).j.\str{scriptinputs} |\, p$, and
        \item\label{eqs:b:w:script_rp} if
          $S_1(b_1).j.\str{origin} \in \{\an{\mi{dr}_1, \https},\an{\mi{dr}_2, \https}\}$
          then $S_1(b_1).j.\str{script} \equiv \str{script\_rp}$ and \
          \begin{enumerate}
          \item $S_1(b_1).j.\str{location}$ and $S_2(b_2).j.\str{location}$
            are term-equivalent under $\theta$ except for the host part,
            which is either equal or $\mi{dr}_1$ in $b_1$ and $\mi{dr}_2$ in
            $b_2$, and
          \item $S_1(b_1).j.\str{referrer}$ and $S_2(b_2).j.\str{referrer}$
            are term-equivalent under $\theta$ except for the host part,
            which is either equal or $\mi{dr}_1$ in $b_1$ and $\mi{dr}_2$ in
            $b_2$, and
          \item
            $S_1(b_1).j.\str{scriptstate} \prototagequiv{\theta}
            S_2(b_2).j.\str{scriptstate}$ and if $\exists\, l \in L$ such that $l$ is a subterm of $S_1(b_1).j.\str{scriptstate}$, then $S_1(b_1).j.\str{location}.\str{host} \equiv \mi{dr}_1$ and $S_2(b_2).j.\str{location}.\str{host} \equiv \mi{dr}_2$, and
          \item if $\exists\, l \in L$ such that $l$ is a subterm of
            $S_1(b_1).j.\str{scriptinputs}$, then
            $S_1(b_1).j.\str{location}.\str{host} \equiv \mi{dr}_1$ and
            $S_2(b_2).j.\str{location}.\str{host} \equiv \mi{dr}_2$, and
          %\item $\forall t \in N$: $t$ is not contained in any subterm of
          %  $S_1(b_1).j.\str{scriptstate}$ except for
          %  $S_1(b_1).j.\str{scriptstate}.\str{parameters}$, and
          %  \begin{itemize}
          %  \item
          %    $S_1(b_1).j.\str{origin} \not\equiv
          %    \an{\mi{dr}_1,\https}$\\$\implies
          %    t \not\equiv S_1(b_1).j.\str{scriptstate}.\str{parameters}$, and
          %  \item $S_1(b_1).j.\str{origin}
          %    \not\equiv
          %    \an{\mi{dr}_1,\https}$\\$\implies t \not\in
          %    d_\emptyset(S_1(b_1).j.\str{scriptinputs})$, and
          %  \item
          %    $S_2(b_2).j.\str{origin} \not\equiv
          %    \an{\mi{dr}_2,\https}$\\$\implies
          %    t \not\equiv S_2(b_2).j.\str{scriptstate}.\str{parameters}$, and
          %  \item $S_2(b_2).j.\str{origin}
          %    \not\equiv
          %    \an{\mi{dr}_2,\https}$\\$\implies t \not\in
          %    d_\emptyset(S_2(b_2).j.\str{scriptinputs})$,
          %    and \end{itemize}
          \end{enumerate}
        \item\label{eqs:b:w:att_script} if
          $S_1(b_1).j.\str{origin} \not\in
          \{\an{\mi{dr}_1,\https},\an{\mi{dr}_2,\https}\}$
          then $S_1(b_1).j.\str{script} \equiv \str{script\_idp}$ and \
          \begin{enumerate}
          \item
            $S_1(b_1).j.\str{location} \prototagequiv{\theta}
            S_2(b_2).j.\str{location}$, and
          \item
            $S_1(b_1).j.\str{referrer} \prototagequiv{\theta}
            S_2(b_2).j.\str{referrer}$, and
          \item\label{eqs:b:w:att_script:state}
            $S_1(b_1).j.\str{scriptstate} \prototagequiv{\theta}
            S_2(b_2).j.\str{scriptstate}$, and
          \item
            $S_1(b_1).j.\str{scriptinputs} \prototagequiv{\theta}
            S_2(b_2).j.\str{scriptinputs}$, and
          \item\label{eqs:b:w:att_script:t}
            $\forall t \in T$: $t$ is not contained in any subterm of 
            $S_1(b_1).j.\str{scriptstate}$ except for 
            $S_1(b_1).j.\str{scriptstate}.\mi{parameters}[\str{t}]$, and
          \item $\nexists\, l \in L$ such that $l$ is a subterm of
            $S_1(b_1).j.\str{scriptstate}$ or of
            $S_1(b_1).j.\str{scriptinputs}$, and
          \end{enumerate}
        \end{enumerate}
      \item\label{eqs:b:misc} for
        $x \in \{\str{cookies},\str{localStorage},\str{sessionStorage},\str{sts}\}$
        we have that $S_1(b_1).x \prototagequiv{\theta} S_2(b_2).x$. For the
        domains $\mi{dr}_1$ and $\mi{dr}_2$ there are no entries in the
        subterms $x$.
      \end{enumerate}
    \end{enumerate}
  \end{definition}
  
  \begin{definition}[Equivalence of Events]\label{def:Events}
    Let $\theta$ be a set of proto-tags, 
    $L$ be a set of login session tokens, 
    $H$ be a set of nonces, and
    $T:=\{t\mid [t]R\in \theta\}$. 
    We call $E_1 = (e_1^{(1)}, e_2^{(1)}\dots)$ and
    $E_2= (e_1^{(2)}, e_2^{(2)} \dots)$ 
    \emph{$\beta$-equivalent under $(\theta, L, H)$} 
    iff all of the following conditions are satisfied for every 
    $i \in \mathbb{N}$:
  
    \begin{enumerate}
      \item\label{eqe:distinction} One of the following conditions holds
        true:
        \begin{enumerate}
        \item\label{eqe:prototagequiv}
          $e_i^{(1)} \prototagequiv{\theta} e_i^{(2)}$ and if $e_i^{(1)}$
          contains an HTTP(S) message (i.e., HTTP(S) request or HTTP(S)
          response), then the HTTP nonce of this HTTP(S) message is not
          contained in $H$, or
        \item\label{eqe:http-req} $e_i^{(1)}$ is an HTTP request $m_1$
          from $b_1$ to $r_1$ and $e_i^{(2)}$ is an HTTP request $m_2$
          from $b_2$ to $r_2$, $m_1 \httptagequiv{\theta} m_2$, and both
          requests are unencrypted or encrypted (i.e., $m_1$ and $m_2$ are
          the content of the encryption) and $m_1.\str{nonce} \in H$, or
        \item\label{eqe:http-res} $e_i^{(1)}$ is an HTTP(S) response from
          $r_1$ to $b_1$ and $e_i^{(2)}$ is an HTTP(S) response from $r_2$
          to $b_2$, and their HTTP messages $m_1$ (contained in
          $e_i^{(1)}$) and $m_2$ (contained in $e_i^{(1)}$) are the same
          except for the HTTP body $g_1 := m_1.\str{body}$ and the HTTP
          body $g_2 := m_2.\str{body}$ which have to be
          $g_1 \prototagequiv{\theta} g_2$ and $m_1.\str{nonce} \in H$.
        \end{enumerate}
      %可能破坏PID_rp不可区分性的参数需要单独列出
      \item\label{eqe:pre:l} If there exists $l \in L$ such that $l$ is a
        subterm of $e_i^{(1)}$ or $e_i^{(2)}$ then we have that
        $e_i^{(1)}$ is a message from $b_1$ to $r_1$ and $e_i^{(2)}$ is a
        message from $b_2$ to $r_2$ or we have that $e_i^{(1)}$ is a
        message from $r_1$ to $b_1$ and $e_i^{(2)}$ is a message from
        $r_2$ to $b_2$.
      \item\label{eqe:pre:t} If there exists $t \in T$ such that
        $t \in d_{\nonces\setminus\{t\}}(\{e_i^{(1)}, e_i^{(2)}\})$ 
        then $e_i^{(1)}$ is an HTTP(S) request from $b_1$ to $r_1$ 
        and $e_i^{q(2)}$ is an HTTP(S) request from $b_2$ to $r_2$ 
        and the bodies of both HTTP messages are of the form
        $\an{\an{\str{t}, t}}$.
      \item\label{eqe:pre:rp-scripts} If $e_i^{(1)}$ or $e_i^{(2)}$ is an
        HTTP(S) response with body $g$ from a relying party, then it does
        not contain any $\str{Location}$ or $\cSTS$ header
        and if $\proj{1}{g}$ is a string representing a script, then
        $\proj{1}{g}$ is $\str{script\_rp}$.
      \item\label{eqe:pre:unencrypted-http} If $e_i^{(1)}$ or $e_i^{(2)}$
        is an unencrypted HTTP response, then the message was sent by some
        attacker.
    \end{enumerate}
  \end{definition}
  
  \begin{definition}[Equivalence of Configurations]
    We call $(S_1,E_1,N_1)$ and $(S_2,E_2,N_2)$
    \emph{$\alpha$-equivalent} iff there exists a set of proto-tags
    $\theta$ and a set of nonces $H$ such that $S_1$ and $S_2$ are
    $\gamma$-equivalent under $(\theta,H)$, $E_1$ and $E_2$ are
    $\beta$-equivalent under $(\theta,L,H)$ for
    $L := \bigcup_{a\in\theta} \mathsf{loginSessionTokens}(a,S_1,S_2)$,
    and $N_1 = N_2$.
  \end{definition}
  
  \subsection{Privacy Proof}
  
  \begin{theorem} \label{theorem:A}Every UPPRESSO web system for privacy analysis is IdP-private.
  \end{theorem}
  
  Let $\mathcal{U\!W\!S}^{priv}$ be UPPRESSO web system for privacy analysis.\par
  To prove Theorem \ref{theorem:A}, we have to show that the UPPRESSO web systems $\mathcal{U\!W\!S}^{priv}_1$ and $\mathcal{U\!W\!S}^{priv}_2$ 
  are indistinguishable. To show the indistinguishability of $\mathcal{U\!W\!S}^{priv}_1$ and $\mathcal{U\!W\!S}^{priv}_2$, 
  we show that they are indistinguishable under all schedules $\sigma$.
  For this , we first note that for all $\sigma$, there is only one run induced by each $\sigma$(as our web system, when scheduled, is deterministic).
  We now proceed to show that for all schedules $\sigma=(\zeta _1, \zeta_2,\dots)$, iff $\sigma$ induces a run $\sigma(\mathcal{U\!W\!S}^{priv}_1)$ there exists a run $\sigma(\mathcal{U\!W\!S}^{priv}_2)$ such that $\sigma(\mathcal{U\!W\!S}^{priv}_1)\approx\sigma(\mathcal{U\!W\!S}^{priv}_1)$\par
  We now show that if two configurations are $\alpha$-equivalent, then the view of the attacker is statically equivalent.
  
  \begin{lemma}
    Let $(S_1,E_1,N_1)$ and $(S_2,E_2,N_2)$ be two 
    $\alpha$-equivalent configurations. 
    Then $S_1(attacker)\approx S_2(attacker)$.
  \end{lemma}
  \begin{proof}
    From the $\alpha$-equivalence of $(S_1,E_1,N_1)$ and 
    $(S_2,E_2,N_2)$ it follows that $S_1(\fAP{attacker}) 
    \prototagequiv{\theta} S_2(\fAP{attacker})$.
    From Condition~\ref{eqs:att-not-t} for $\gamma$-equivalence 
    it follows that
    $t \not\in d_\emptyset(\bigcup_{i \in \{1,2\},\ A\, \in\, 
    \mathsf{Web}\, \cup \, \mathsf{Net}\, \}}S_i(A))$
    (i.e., the attacker does not know any keys for the tags 
    contained in its view), and with Lemma~\ref{thm-idp-untraceability-new} it is easy to see 
    that the views are statically equivalent.
  \end{proof}

  \newc
  \begin{lemma}\label{thm-idp-untraceability-new}
    Given a point on the elliptic curve denoted by $[r]G$, 
    an adversary cannot distinguish $[tr]G$ from a random variable on $\mathbb{E}$, 
    where $t$ is random in $\mathbb{Z}_n$ and unknown to the adversary.
  \end{lemma}
  \begin{proof}
    Consider a finite cyclic group $\mathbb{E}$ where the number of points on $\mathbb{E}$ is $n$. 
    Because $G$ is a generator of order $n$, $[r]G$ is also a generator on $\mathbb{E}$ of order $n$. 
    $t$ is randomly chosen in $\mathbb{Z}_n$ and always kept unknown to the adversary. 
    Therefore, $[tr]G$ is \emph{indistinguishable} from a point $Q$ that is randomly chosen on $\mathbb{E}$.\cite{oprf-proved,voprf-proved}.
  \end{proof}
  \oldc
  
  We now show that $\sigma(\uppressoprivwebsystem_1) \approx
  \sigma(\uppressoprivwebsystem_2)$ by induction over the length 
  of $\sigma$. We first, in Lemma~\ref{lemma:initial-config-private}, 
  show that $\alpha$-equivalence (and therefore, indistinguishability 
  of the views of $\fAP{attacker}$) holds for the initial 
  configurations of $\uppressoprivwebsystem_1$ and 
  $\uppressoprivwebsystem_2$. We then, in 
  Lemma~\ref{lemma:step-config-private}, show that for each 
  configuration induced by a processing step in $\zeta$,
  $\alpha$-equivalence still holds true.
  
  \begin{lemma}\label{lemma:initial-config-private}
    The initial configurations $(S_1^0,E^0,N^0)$ of 
    $\mathcal{U\!W\!S}^{priv}_1$ and $(S_2^0,E^0,N^0)$ of 
    $\mathcal{U\!W\!S}^{priv}_2$ are $\alpha$-equivalent.
  \end{lemma}
  \begin{proof}
    We now have to show that there exists a set of proto-tags $\theta$ and a set of nonces $H$
    such that $S_1^0$ and $S_2^0$ are $\gamma$-equivalent under
    $(\theta,H)$, $E_1^0 = E^0$ and $E_2^0 = E^0$ are $\beta$-equivalent
    under $(\theta,L,H)$ with $L := \bigcup_{a\in\theta} \mathsf{loginSessionTokens}(a,S_1,S_2)$, and $N_1^0 = N_2^0 = N^0$.
  
    Let $\theta = H = L = \emptyset$. Obviously, both latter conditions are
    true. For all parties $p \in \bidsystem_1 \setminus \{b_1\}$, it is
    clear that $S_1^0(p) = S_2^0(p)$. Also the states $S_1^0(b_1)$ and
    $S_2^0(b_2)$ are equal. Therefore, all conditions
    of Definition~\ref{def:eq-of-states} are fulfilled. Hence, the
    initial configurations are $\alpha$-equivalent.
  \end{proof}
  
  \begin{lemma}\label{lemma:step-config-private}
    Let $(S_1,E_1,N_1)$ and $(S_2,E_2,N_2)$ be two 
    $\alpha$-equivalent configurations of 
    $\uppressoprivwebsystem_1$ and $\uppressoprivwebsystem_2$, 
    respectively. Let $\zeta = \an{\mi{ci},\mi{cp}, 
    \tau_\text{process}, \mi{cmd}_\text{switch}, 
    \mi{cmd}_\text{window},\tau_\text{script},\mi{url}}$
    be a web system command. Then, $\zeta$ induces a processing 
    step in either both configurations or in none. In the latter 
    case, let $(S_1',E_1',N_1')$ and $(S_2',E_2',N_2')$ be 
    configurations induced by $\zeta$ such that
    \[(S_1,E_1,N_1) \xrightarrow{\zeta} (S_1',E_1',N_1') \quad 
    \text{and} \quad (S_2,E_2,N_2) \xrightarrow{\zeta} 
    (S_2',E_2',N_2') \ .\]
    Then, $(S_1',E_1',N_1')$ and $(S_2',E_2',N_2')$ are
    $\alpha$-equivalent.  
  \end{lemma}
  \begin{proof}
    Let $\theta$ be a set of proto-tags and $H$ be a set of 
    nonces for which $\alpha$-equivalence holds and let 
    $L:=\bigcup_{a\in\theta}\text{loginSessionTokens}(a,S_1,S_2)$,
    $T:=\{t\mid [t]R\in \theta\}$.
    
    To induce a processing step, the ci-th message from $E_1$ or 
    $E_2$, respectively, is selected.Following Definition 
    \ref{def:Events}, we denote these messages by $e_i^{(1)}$ or 
    $e_i^{(2)}$, respectively. We now differentiate between the 
    receivers of the messages by denoting the induced processing 
    steps by
    \begin{equation}
      \begin{aligned}
        (S_1,E_1,N_1)\xrightarrow[p_1\rightarrow E_{out}^{(1)}]{\left \langle a_1,f_1,m_1\right \rangle\rightarrow p_1}(S_1\prime,E_1\prime,N_1\prime)\\
        (S_2,E_2,N_2)\xrightarrow[p_2\rightarrow E_{out}^{(2)}]{\left \langle a_2,f_2,m_2\right \rangle\rightarrow p_2}(S_2\prime,E_2\prime,N_2\prime)
      \end{aligned}
    \end{equation}
    \paragraph{\underline{Case $p_1=r_1$:}}
    In this case, we only distinct several cases of HTTP(S) requests that can happen. The others are ignored the same as SPRESSO.\par
    There are four possible types of HTTP requests that are accepted by $r_1$ in Algorithm \ref{alg:rp}:
    \begin{itemize}
      \item path=$\str{/script}$(get the rp-script), Line~\ref{line:rp-script};
      \item path=$\str{/loginSSO}$(start a login), Line~\ref{line:rp-loginSSO};
      \item path=$\str{/startNegotiation}$(derive a $PID_{rp}$), Line~\ref{line:rp-startNegotiation};
      \item path=$\str{/uploadToken}$(verify ID token, calculate Acct), Line~\ref{line:rp-uploadToken}.
    \end{itemize}
    \par From the cases in Definition \ref{def:Events}, only two 
    can possibly apply here:Case~\ref{eqe:prototagequiv} and 
    Case~\ref{eqe:http-req}. For both cases, we will now analyze 
    each of the HTTP requests listed above separately.
  
    \noindent \emph{Definition~\ref{def:Events}, Case~\ref{eqe:prototagequiv}:}
    $e_i^{(1)}\rightleftharpoons e_i^{(2)}$. This case implies 
    $p_2=r_1=p_1$. As we see below, for the output events 
    $E_{out}^{(1)}$ and $E_{out}^{(2)}$ (if any) only 
    Case~\ref{eqe:prototagequiv} of Definition \ref{def:Events} 
    applies. This implies the nonce of both the incoming HTTP 
    requests and HTTP responses cannot be in $H$.
    \begin{itemize}
      \item path=$\str{/script}$ 
        In this case, the same output 
        event is produced whose message is 
        \begin{equation}
          \begin{aligned}
            \left\langle HTTPResp,n,200,\left\langle\right\rangle,RPScript\right\rangle
          \end{aligned}
        \end{equation}
        We can note that Condition~\ref{eqe:pre:rp-scripts} of 
        Definition \ref{def:Events} 
        holds true and.The remaining conditions are trivially 
        fulfilled and $E_1\prime$ and $E_2\prime$ are 
        $\beta$-equivalent under $(\theta,H,L)$.As there are no 
        changes to any state, we have that $S_1\prime$ and 
        $S_2\prime$ are $\gamma$-equivalent under $(\theta,H)$. 
        No new nonces are chosen, hence $N_1\prime=N_1=N_2=N_2\prime$.
      \item path=$\str{/loginSSO}$ 
        In this case, the reason for holding equivalence is 
        similar to the case above since the same output event 
        is produced.
      \item path=$\str{/startNegotiation}$ 
        In both processing steps, a tag is constructed exactly 
        the same. The same HTTP response (which does not contain 
        a $t \in T$ or a $l \in L$) is put in both 
        $E^{(1)}_\text{out}$ and $E^{(2)}_\text{out}$. The first 
        element of the response's body is not a string and 
        therefore Condition~\ref{eqe:pre:rp-scripts} holds true. 
        The tag is only created on $r_1$ in both runs and hence, 
        $\theta$ does not have to be altered. 
        Analogously to above, we have that $E_1'$ and $E_2'$ are 
        $\beta$-equivalent under $(\theta,H,L)$. The subterm 
        $\str{loginSessions}$ of the state of $r_1$ is extended 
        exactly the same. Thus, we have that $S_1'$ and $S_2'$ 
        are $\gamma$-equivalent under $(\theta,H)$. In both
        processing steps exactly one nonce is chosen, and we 
        have that $N_1' = N_2'$.
      \item path=$\str{/uploadToken}$ 
        First, we note that there is no $l \in L$ contained in 
        either $m_1$ or $m_2$ (by the Defintion of 
        $\beta$-equivalence). We further note that there are 
        four checks at Algorithm~\ref{alg:rp} in
        Line~\ref{line:alg-rp-stop1}, \ref{line:alg-rp-stop3}, 
        \ref{line:alg-rp-stop3} and \ref{line:alg-rp-stop4}.
        The script either emits an empty message if failed in
        these checks or accept the request and response with a 
        nonce.
  
        From Condition~\ref{eqs:logsess} of
        Definition~\ref{def:eq-of-states}, we know that
        $S_2(r_1).\str{loginSessions}$ can be constructed from
        $S_1(r_1).\str{loginSessions}$ without removing the 
        entry with the dictionary key 
        $\mi{headers}[\str{Cookie}][\str{loginSessionToken}]$ 
        (as this key is not in $L$). Thus, both dictionaries 
        either contain the same entry for the dictionary key
        $\str{loginSessionToken}$ or they both contain no
        such entry and if they contain such entry, the contents
        will be equal. Hence, we have that if the first two 
        checks fail in $s_1$ then and only then they fail in $s_2$.
  
        If the first two checks pass, since $m_1$ equals to $m_2$ 
        from condition~\ref{eqe:prototagequiv} of 
        Definition~\ref{def:Events}, we have that if the third 
        check fails in $s_1$ then and only then it fails in $s_2$. 
        The same holds true for the fourth check.
  
        if they both accept the IDToken, exactly the 
        same outputs are emitted (without containing any 
        $l\in L$ or $t \in T$), no state is changed and exactly 
        one new nonce is chosen. We therefore trivially have
        $\alpha$-equivalence of the new configurations.
    \end{itemize}
  
    \noindent \emph{Definition~\ref{def:Events}, Case~\ref{eqe:http-req}:} 
    $e_i^{(1)}$ is an HTTP(S) request from $b_1$ to $r_1$ and 
    $e_i^{(2)}$ is an HTTP(S) request from $b_2$ to $r_2$. 
    This case implies $p_2 = \fAP{r_2}$.
  
    We note that Condition~\ref{eqe:pre:rp-scripts} of 
    Definition~\ref{def:Events} holds for the same reasons as in
    the previous case. As the response is always addressed to 
    the IP address of $b_1$ or $b_2$, respectively,
    Condition~\ref{eqe:pre:rp-scripts} of
    Definition~\ref{def:Events} is fulfilled. 
  
    As we see below, for the output events $E^{(1)}_\text{out}$ 
    and $E^{(2)}_\text{out}$ (if any) only Case~\ref{eqe:http-res} 
    of Definition~\ref{def:Events} applies. This implies that the
    output events must contain an HTTP nonce contained in $H$. As 
    we know that the HTTP nonce of the incoming HTTP requests is 
    contained in $H$ and the output HTTP responses (if any) of the 
    RP reuses the same HTTP nonce, the nonce of the HTTP responses 
    is in $H$.
  
    \begin{itemize}
      \item $\mi{path} = \str{/script}$ In this case, 
        the output events contain no $l\in L$ or $t\in T$ 
        meaning that $E_1'$ and $E_2'$ being $\beta$-equivalent
        under $(\theta,H,L)$ according to 
        Definition~\ref{def:Events}, Case~\ref{eqe:http-res}. As
        there are no changes to any state, we have that $S_1'$ 
        and $S_2'$ are $\gamma$-equivalent under $(\theta,H)$. 
        No new nonces are chosen, hence, 
        $N_1 = N'_1 = N_2 = N'_2$.
      \item $\mi{path} = \str{/loginSSO}$ This case is analogue
        to the case above.
      \item $\mi{path} = \str{/startNegotiation}$ In this case, 
        an HTTP response is created. We denote the HTTP response generated by $r_1$ as $m_1'$ and the one
        generated by $r_2$ as $m_2'$. We then have that
        \begin{align*}
          m_1' = \encs{\an{\cHttpResp,n,200,\an{\mi{setCookie}},g_1}}{k} \\
          m_2' = \encs{\an{\cHttpResp,n,200,\an{\mi{setCookie}},g_2}}{k}
        \end{align*}
        with
        \begin{align*}
          \mi{setCookie} := \myangle{\cSetCookie, \myangle{\myangle{\str{loginSessionToken}, \nu_1, \True, \True, \True}}} \\
        \end{align*}
        and 
        \begin{align*}
          g_1 = \an{\an{\str{Cert_{RP}},S_1(r_1).\str{IdPConfig}.Cert_{RP}}} \\
          g_2 = \an{\an{\str{Cert_{RP}},S_2(r_2).\str{IdPConfig}.Cert_{RP}}}
        \end{align*}
  
        Obviously, $m_1'$ equals $m_2'$. For $N_1 = N_2 = 
        (n_1, n_2, \dots)$, We set $\theta' = \theta \cup 
        \{ [t]S_j(r_j).ID_{RP} \}$ for $j\in \{1, 2\}$, 
        $N_1' = N_2' = (n_2, \dots)$ (as exactly one nonce is 
        chosen in both processing steps) and 
        $L' = L \cup \{n_1\}$. 
        The receiver of both messages is the browser $b_1$ or 
        $b_2$, respectively. Obviously, it holds that
        $L' = \bigcup_{a\in\theta'} 
        \mathsf{loginSessionTokens}(a,S_1',S_2')$
        and there exists an $l' \in L'$ such that
        $g_1[\str{loginSessionToken}] \equiv l'$. As
        Conditions~\ref{eqe:http-res} and~\ref{eqe:pre:t} of
        Definition~\ref{def:Events} hold, $E_1'$ and $E_2'$ are
        $\beta$-equivalent under $(\theta',H,L')$. The subterm
        $\str{loginSessions}$ of $S_1(r_1)$ is extended exactly 
        the same as the subterm $\str{loginSessions}$ of 
        $S_2(r_2)$. Thus, we have that $S_1'$ and $S_2'$ are
        $\gamma$-equivalent under $(\theta',H)$.
      \item $\mi{path} = \str{/uploadToken}$ In this case, 
        there are four checks at Algorithm~\ref{alg:rp} in
        Line~\ref{line:alg-rp-stop1}, \ref{line:alg-rp-stop3}, 
        \ref{line:alg-rp-stop3} and \ref{line:alg-rp-stop4}.
        
        From Condition~\ref{eqs:logsess} of 
        Definition~\ref{def:eq-of-states} we know that for 
        $\mi{ls}_1 := S_1(r_1).\str{loginSessions}[l]$ and 
        $\mi{ls}_2 := S_2(r_2).\str{loginSessions}[l]$, 
        we have that $\mi{ls}_1 \prototagequiv{\theta} \mi{ls}_2$.
        Therefore, we have that if the first two checks fail in
        $r_1$ then and only then they fail in $r_2$.
  
        As we know that $m_1 \httptagequiv{\theta} m_2$, we have 
        that if the third check fails in $r_1$ then and only 
        then it fails in $r_2$. The same holds true for the 
        fourth check.
  
        if $r_1$ and $r_2$ both accept the IDToken, they will 
        generate HTTP responses with service Token.We denote 
        the HTTP response generated by $r_1$ as $m_1'$ and the
        one generated by $r_2$ as $m_2'$. We then have that
        \begin{align*}
          m_1' = \encs{\an{\cHttpResp,n,200,\an{},g_1}}{k} \\
          m_2' = \encs{\an{\cHttpResp,n,200,\an{},g_2}}{k}
        \end{align*}
        with
        \begin{align*}
          g_1 = \an{\an{\str{nonce}, \nu_1}} \\
          g_2 = \an{\an{\str{nonce}, \nu_1}}
        \end{align*}
        Same as above, $m_1'$ equals $m_2'$, $N_1' = N_2' = 
        (n_2, \dots)$ and $L' = L$. 
        The receiver of both messages is the browser $b_1$ or 
        $b_2$, respectively. As Conditions~\ref{eqe:http-res} 
        and~\ref{eqe:pre:l} of Definition~\ref{def:Events} hold, 
        $E_1'$ and $E_2'$ are $\beta$-equivalent under 
        $(\theta,H,L)$. The subterm $\str{loginSessions}$ of 
        $S_1(r_1)$ is extended exactly the same as the subterm 
        $\str{loginSessions}$ of $S_2(r_2)$. Thus, we have that 
        $S_1'$ and $S_2'$ are $\gamma$-equivalent under 
        $(\theta,H)$.
    \end{itemize}
  
    \paragraph{\underline{Case $p_1 = \fAP{r_2}$:}} This case is
    analogue to the case $p_1 = \fAP{r_1}$ above. Note that the
    Case~\ref{eqe:http-req} of Definition~\ref{def:Events} 
    cannot occur by definition.
  
    \paragraph{\underline{Case $p_1 = \fAP{b_1}$:}} 
    $\implies p_2 = \fAP{b_2}$ 
  
    %We now do a case distinction over the types of messages a 
    %browser can receive.
  
    \begin{description}
      %\item[DNS response]
      \newc
      \item[HTTP response] In this case, it is clear that
        the HTTP(s) response nonce is the same in both
        messages $m_1$ and $m_2$. 
        We can now distinguish between two cases: 
        In both browsers, \ref{browser-http-response-normal}
        the $\mi{reference}$ that is stored along with the HTTP 
        nonce is a window reference (in this case, the request 
        was a ``normal'' HTTP(S) request), 
        or \ref{browser-http-response-xhr} this reference is a 
        pairing of a document nonce and an XHR reference chosen 
        by the script that sent the request, which is an XHR.
        
        \begin{enumerate}[I.]
        \item\label{browser-http-response-normal} 
          In Case~(I), we can distinguish between the following two cases:
          \begin{enumerate}
            \item The HTTP nonce in $m_1$ is in $H$: 
              In this case, only Case~\ref{eqe:http-res} of Definition~\ref{def:Events} can apply. 
              We therefore have that the expected sender in $e_i^{(1)}$ is $r_1$ and in $e_i^{(2)}$ is $r_2$. 
              We also have that there is no Location, Set-Cookie or Strict-Transport-Security header in the response, 
              and the responses $m_1$ and $m_2$ are both $\str{script\_rp}$ as Case~\ref{eqe:pre:rp-scripts} of Definition~\ref{def:Events} holds.
      
              With this, we observe that both browsers either accept and
              successfully decrypt the messages and call the function
              $\mathsf{PROCESSRESPONSE}$, or both browsers stop with not
              state change and no output event (in which case the
              $\alpha$-equivalence is given trivially). In particular we
              note that the expected sender in both cases matches precisely
              the sender the message has (compare Case~\ref{eqe:http-res} of
              Definition~\ref{def:Events}).
      
              In $\mathsf{PROCESSRESPONSE}$, we see that no changes in the
              browsers' cookies are performed (as no cookies are in the
              response), the $\str{sts}$ subterm is not changed, and no
              redirection is performed (as no Location header is present).
      
              Now, new documents are created in each browser. These have the
              form
              \[ \an{\nu_7, \mi{location}, \mi{referrer}, \mi{script},
                \mi{scriptstate}, \an{}, \an{}, \True} \] with
              \[ \mi{location} = \an{\cUrl, \mi{protocol}, \mi{host},
                \mi{path}, \mi{parameters}}\ .\]
      
            
              Here, $\mi{script}$, $\mi{scriptstate}$ are the same and
              $\mi{protocol}$, $\mi{path}$, $\mi{parameters}$ are taken from
              the requests, which means that these subterms are equal or
              term-equivalent up to proto-tags $\theta$. 
              The host and the referrer are the same in both states up to exchange of domains, 
              which can be $\mi{dr}_1$ in $b_1$ and $\mi{dr}_2$ in $b_2$.
      
              The browser now attaches these newly created documents to its
              window tree, and we have to check that the
              Condition~\ref{eqs:b:w} of Definition~\ref{def:eq-of-states}
              is satisfied.
      
              As we have that both incoming messages were encrypted messages
              (see Case~\ref{eqe:pre:unencrypted-http} of
              Definition~\ref{def:Events}) and both messages come from
              $r_1$ and $r_2$, respectively, and therefore $\mi{script}$ is
              either $\str{script\_rp}$ (see
              Case~\ref{eqe:pre:rp-scripts} of
              Definition~\ref{def:Events}) we have to check
              Conditions~\ref{eqs:b:w:script_rp} of
              Definition~\ref{def:Events} in particular.
      
              The scriptstate is initially equal and the script inputs are empty. The document's
              referer is constructed from the referer header of the request,
              which is equal in both cases or has the host $\mi{dr}_1$ in
              $b_1$ and $\mi{dr}_2$ in $b_2$.
      
              To sum up, $\gamma$-equivalence under $(\theta, H)$ is
              preserved. $\alpha$-equivalence is preserved as no output
              event is generated and the exact same number of nonces are
              chosen.
      
            
            \item The HTTP nonce in $m_1$ is not in $H$: In this case we
              have that $e_i^{(1)} \prototagequiv{\theta} e_i^{(2)}$
              (Case~\ref{eqe:prototagequiv} of
              Definition~\ref{def:Events}), and that the HTTP nonces,
              senders, encryption keys (if any) and original requests in the
              pending requests of both browsers are either equal or
              equivalent up to proto-tags $\theta$. There can be no
              $t \in T$ as a subterm (except in tags) of the input.
      
              With this, we observe that both browsers either accept and
              successfully decrypt the messages and call the function
              $\mathsf{PROCESSRESPONSE}$, or both browsers stop with no
              state change and no output event (in which case the
              $\alpha$-equivalence is given trivially). In particular we
              note that the expected sender in both cases matches precisely
              the sender of the message (as it is equal).
      
              If there is a Set-Cookie header in one of the responses, a new
              entry in the cookies of each browsers is created (which
              obviously is term-equivalent up to $\theta$, and therefore is
              in compliance with the requirements for $\gamma$-equivalence).
              The same holds true for any Strict-Transport-Security headers.
      
              Now, if there is a Location header in $m_1$ (and therefore
              also in $m_2$), a new request is generated and a HTTP(S) request is sent out. 
              The new HTTP(S) request contains the method, body, and Origin
              header of the original request (which were equivalent up to
              proto-tags $\theta$), where the Origin header is amended by
              the host and protocol of the original request.
      
              Also, we know from
              $e_i^{(1)} \prototagequiv{\theta} e_i^{(2)}$ that neither
              event may contain a subterm $l\in L$ or $t \in T$. Hence, the
              transferred (initial) scriptstate (or a request generated by a
              Location header, see below) cannot contain a subterm $l \in L$
              or $t \in T$.
      
              Now, assuming that the domain in the Location header was not
              $\str{CHALLENGE}$, then the new request is term-equivalent
              under $\theta$ between both browsers. A new HTTP(S) request is
              generated (which conforms to Condition~\ref{eqe:prototagequiv}
              of Definition~\ref{def:Events}). It is clear that in this case, the conditions for
              $\gamma$-equivalence under $(\theta, H)$ are satisfied. 
              The same number of nonces is chosen. Altogether, $\alpha$-equivalence is given.
      
              If, however, the domain is $\str{CHALLENGE}$ (and the browser
              has not started a request to $\str{CHALLENGE}$ before; in this
              case the browser would behave as above), then the domain is
              $\mi{dr}_1$ in $b_1$ and $\mi{dr}_2$ in $b_2$. In particular,
              in the resulting requests, the Host header is exchanged in
              this way. For alpha equivalence to hold for the new
              configuration, we have $H' = H \cup \{n\}$, where $n$ is the
              nonce chosen for the HTTP(S) request. A new HTTP(S) request is
              generated. Therefore, we have
              $\gamma$-equivalence under $(\theta, H')$ and
              $\beta$-equivalence under $(\theta, H', L)$. The same number
              of nonces is chosen, and we indeed have $\alpha$-equivalence.
      
              If there is no Location header in $m_1$ (and therefore none in
              $m_2$), a new document is constructed just as in the case when
              the nonce in $m_1$ is in $H$.
      
              The scriptstate is initially equal, and the script inputs are
              empty. The document's referer is constructed from the referer
              header of the request, which is equal in both cases (up to
              proto-tags in $\theta$).
      
              To sum up, $\gamma$-equivalence under $(\theta, H)$ is
              preserved in this case as well. $\alpha$-equivalence is
              preserved as no output event is generated and the exact same
              number of nonces are chosen.
          \end{enumerate}
        \item\label{browser-http-response-xhr}
          In Case~(II), i.e., the response is a response to an XHR, 
          we have that $\mi{reference}$ is a tupel, say,
          $\mi{reference} = \an{\mi{docnonce}, \mi{xhrref}}$, 
          and we again distinguish between the two cases as above:
          \begin{enumerate}
            \item The HTTP nonce in $m_1$ is in $H$: In this case, only
              Case~\ref{eqe:http-res} of Definition~\ref{def:Events}
              can apply. We therefore have that there is no Location,
              Set-Cookie or Strict-Transport-Security header in the
              response, and that the responses $m_1$ and $m_2$ are equal up
              to proto-tags in $\theta$.
      
              With this, we observe that both browsers either accept and
              successfully decrypt the messages and call the function
              $\mathsf{PROCESSRESPONSE}$, or both browsers stop with not
              state change and no output event (in which case the
              $\alpha$-equivalence is given trivially). In particular we
              note that the expected sender in both cases matches precisely
              the sender of the message (compare Case~\ref{eqe:http-res} of
              Definition~\ref{def:Events}).
      
              In $\mathsf{PROCESSRESPONSE}$, we see that no changes in the
              browsers' cookies are performed (as no cookies are in the
              response), the $\str{sts}$ subterm is not changed, and no
              redirection is performed (as no Location header is present).
      
              A new input is constructed for the document that is identified
              by $\mi{docnonce}$. We note that such a document exists either
              in both browsers or in none (in which, again, both browsers
              stop with no output or state change). As the input events may
              contain a subterm $l \in L$ (as we know from HTTP nonce in
              $m_1$ being in $H$ that the host of this document is
              $\mi{dr}_1$ in $b_1$ and $\mi{dr}_2$ in $b_2$), the
              constructed scriptinput may also contain a subterm $l \in L$.
      
              For $j \in \{1,2\}$, we have that the $\str{scriptinput}$ term
              for the document in $b_j$ is $\an{\tXMLHTTPRequest,
                \mi{g_j}.\str{body}, \mi{xhrref}}$, where $g_j$ is the HTTP
              body of $m_j$.  With $g_1 \prototagequiv{\theta} g_2$ and
              $\mi{xhrref} \in \nonces \cup \{\bot\}$, it is easy to see
              that the resulting $\str{scriptinput}$ term of the document is
              term-equivalent under proto-tags $\theta$ (as it was before).
              This satisfies $\gamma$-equivalence on the new browser state.
      
              No output event is generated, and no nonces are chosen.
              Therefore we have $\alpha$-equivalence on the new
              configuration.
      
            \item The HTTP nonce in $m_1$ is not in $H$: In this case we
              have that $e_i^{(1)} \prototagequiv{\theta} e_i^{(2)}$
              (Case~\ref{eqe:prototagequiv} of
              Definition~\ref{def:Events}), and that the HTTP nonces,
              senders, encryption keys (if any) and original requests of both browsers are either equal or
              equivalent up to proto-tags $\theta$.

              With this, we observe that both browsers either accept and
              successfully decrypt the messages and call the function
              $\mathsf{PROCESSRESPONSE}$, or both browsers stop with not
              state change and no output event (in which case the
              $\alpha$-equivalence is given trivially). In particular we
              note that the expected sender in both cases matches precisely
              the sender the message has (as it is equal).
      
              If there is a Set-Cookie header in one of the responses, a new
              entry in the cookies of each browsers is created (which
              obviously is term-equivalent up to $\theta$, and therefore is
              in compliance with the requirements for $\gamma$-equivalence).
              The same holds true for any Strict-Transport-Security headers.
      
              Now, if there is a Location header in $m_1$ (and therefore
              also in $m_2$), both browsers stop with not state change and
              no output event (in which case the $\alpha$-equivalence is
              given trivially), as XHR cannot be redirected in the browser.
      
              If there is no Location header in $m_1$ (and therefore none in
              $m_2$), a new input is constructed for the document that is
              identified by $\mi{docnonce}$. We note that such a document
              exists either in both browsers or in none. For $j \in
              \{1,2\}$, we have that the $\str{scriptinput}$ for the
              document in $b_j$ is $\an{\tXMLHTTPRequest,
                \mi{g_j}.\str{body}, \mi{xhrref}}$, where $g_j$ is the HTTP
              body of $m_j$. With $e_i^{(1)} \prototagequiv{\theta}
              e_i^{(2)}$ (which may not contain a subterm $l \in L$ or $t \in T$), it is
              easy to see that the resulting $\str{scriptinput}$ term of the
              document is term-equivalent under proto-tags $\theta$ (as it
              was before). This satisfies $\gamma$-equivalence on the new
              browser state.
      
              No output event is generated, and no nonces are chosen.
              Therefore we have $\alpha$-equivalence on the new
              configuration.
          \end{enumerate}
        \end{enumerate}
      \oldc
      
      \item[TRIGGER] We now distinguish between the possible 
        values for $\mi{cmd}_\text{switch}$.
        \begin{description}
        \item[1 (trigger script):] In this case, the script in 
          the window indexed by $\mi{cmd}_\text{window}$ is 
          triggered. Let $j$ be a pointer to that window.
  
          We first note that such a window exists in $b_1$ iff 
          it exists in $b_2$ and that $S_1(b_1).j.\str{script} 
          \equiv S_2(b_2).j.\str{script}$. We now distinguish 
          between the following cases, which cover all possible 
          states of the windows/documents:
  
          \begin{enumerate}
          \item
            $S_1(b_1).j.\str{origin} \in \{\an{\mi{dr}_1, 
            \https}, \an{\mi{dr}_2, \https}\}$ and 
            $S_1(b_1).j.\str{script} \equiv \str{script\_rp}$.
  
            Similar to the following scripts, the main 
            distinction in this script is between the script's 
            internal states (named $\str{phase}$). With the 
            term-equivalence under proto-tags $\theta$ we have 
            that either 
            $S_1(b_1).j.\str{scriptstate}.\str{phase} =
             S_2(b_2).j.\str{scriptstate}.\str{phase}$ or the 
            script's state contains a tag and is therefore in an 
            illegal state (in which case the script will stop 
            without producing output or changing its state).
  
            We can therefore now distinguish between the 
            possible values of
            $S_1(b_1).j.\str{scriptstate}.\str{phase} =
             S_2(b_2).j.\str{scriptstate}.\str{phase}$:
            \begin{description}
            \item[start:] In this case, the script open a blank
              page addressed to its own origin, which is either 
              (a) equal and $\an{\mi{dr}_1,\https}$ or 
              $\an{\mi{dr}_2,\https}$ or it is 
              (b) $\an{\mi{dr}_1,\https}$ in $b_1$ and
              $\an{\mi{dr}_2,\https}$ in $b_2$. The path is the 
              (static) string $\str{/loginSSO}$. The script 
              saves a (static) value for $\str{phase}$ in its 
              scriptstate.
  
              In both Cases, we have that the command is 
              term-equivalent under proto-tags $\theta$ and 
              hence, the browser emits a HTTP request which is 
              term-equivalent.Hence, we have 
              $\gamma$-equivalence under $(\theta,H)$ for the 
              new states, $\beta$-equivalence under 
              $(\theta,H,L)$ for the new events, and 
              $\alpha$-equivalence for the new configuration.
            
            \item[expectt:] In this case, the script retrieves 
              the result of a \pm from $\mi{scriptinputs}$. As 
              we know that $S_1(b_1).j.\str{scriptstate} 
              \prototagequiv{\theta} 
              S_2(b_2).j.\str{scriptstate}$ and that for all 
              matching \pms that they also have to be 
              term-equivalent up to $\theta$ and that the window 
              structure is equal in both browsers, we have that 
              either the same \pm is retrieved from 
              $\mi{scriptinputs}$ or none in both browsers.
  
              Then the script saves a (static) value for 
              $\str{phase}$ in its scriptstate, and we set 
              $H' := H \cup \{n\}$ with $n$ being the nonce 
              that the browser chooses for $\lambda_1$. 
              Therefore, we have $\gamma$-equivalence under 
              $(\theta,H')$ for the new states. We also have
              $\beta$-equivalence under $(\theta,H',L)$ for the 
              new events, and $\alpha$-equivalence for the new 
              configuration.
            \item[expectCert:] In this case, the script
              retrieves the result of an \xhr from 
              $\mi{scriptinputs}$ that matches the reference 
              contained in $\mi{scriptstate}$. From
              Condition~\ref{eqs:b:w:script_rp} of
              Definition~\ref{def:eq-of-states} we know that all 
              results from \xhr{}s in $\mi{scriptinput}$ are 
              term-equivalent up to $\theta$ and that 
              $\mi{scriptstate}$ is term-equivalent up to $\theta$. 
              Hence, in both browsers, both scripts stop with an 
              empty command or both continue as they successfully
              retrieved such an \xhr.
  
              The script now constructs a \pm that is sent to 
              exactly the same window in both browsers and that 
              requires that the receiver origin has to be 
              $\an{\str{IdPdomain},\https}$ The postMessage is 
              only sent to this origin, we have that 
              $\gamma$-equivalence cannot be violated.
  
              We now have that $S_1'$ and $S_2'$ are 
              $\gamma$-equivalent under $(\theta,H)$, 
              $E_1'$ and $E_2'$ are $\beta$-equivalent under 
              $(\theta,H,L)$, and as exactly none of nonces is 
              chosen, we have that the new configuration is 
              $\alpha$-equivalent.
            \item[expectToken:] This case is the same as
              $\str{expectt}$ and we have that the new 
              configuration is $\alpha$-equivalent.
            \end{description}
  
          \item $S_1(b_1).j.\str{origin} \not\in 
            \{\an{\mi{dr}_1,\https},\an{\mi{dr}_2,\https}\}$. 
            $S_1(b_1).j.\str{script} \equiv \str{script\_idp}$.
            
            Unlike SPRESSO, $\str{script_{idp}}$ is trustful in
            UPPRESSO due to the use of SRI (Subresource Integrity).
            Because of this check, browsers can control the 
            content of $\str{script_{idp}}$ downloaded from 
            the Identity Provider. Hence, we now analyze every 
            internal state just like in $\str{script_{rp}}$.
  
            \begin{description}
            \item[start:] In this case, the script chooses a
              new nonce for $t$. Since $t$ is only stored in
              $S_1(b_1).j.\str{scriptstate}.\mi{parameters}$,
              the condition~\ref{eqs:att-not-t} and 
              condition~\ref{eqs:b:w:att_script:t} of 
              Definiton~\ref{def:eq-of-states} hold. Hence, 
              we have $\gamma$-equivalence under $(\theta,H)$ 
              for the new states.
  
              From the equivalence definition of states
              (Definition~\ref{def:eq-of-states}) we can see 
              that the window tree has the same structure in 
              both processing steps. 
              So the script now constructs a \pm that is sent to 
              exactly the same window in both browsers and that 
              requires that the receiver has to be the opener
              of this window. Since the new tag hasn't been 
              generated and Condition~\ref{eqe:prototagequiv} 
              of Definition~\ref{def:Events} holds, we have 
              $\beta$-equivalence under $(\theta,H,L)$.
            \item[expectCert:] The same as above, we can have 
              that either the same \pm is retrieved from 
              $\mi{scriptinputs}$ or none in both browsers and 
              the result of $checksig$ is same as well. The 
              state $Cert_{rp}$ is equal in both 
              $\str{scriptstate}$ and the state $PID_{rp}$ is
              term-equivalent under $\theta$. As the 
              condition~\ref{eqs:b:w:att_script:state} of 
              Defition~\ref{def:eq-of-states} holds, we have 
              $\gamma$-equivalence under $(\theta,L)$ for the 
              new states.
  
              Obviously, we have $\beta$-equivalence under 
              $(\theta,H,L)$ as Condition~\ref{eqe:prototagequiv} 
              of Definition~\ref{def:Events} holds.
  
            \item[expectReqToken:] In this case, the script
              retrieves the result of an \xhr from 
              $\mi{scriptinputs}$ that matches the reference 
              contained in $\mi{scriptstate}$. From
              Condition~\ref{eqs:b:w:script_rp} of
              Definition~\ref{def:eq-of-states} we know that all 
              results from \xhr{}s in $\mi{scriptinput}$ are 
              term-equivalent up to $\theta$ and that 
              $\mi{scriptstate}$ is term-equivalent up to 
              $\theta$. Hence, in both browsers, both scripts 
              will reach the same if-else branch.
  
              Since there aren't any new states stored and the
              requests' destination are fixed, we can have 
              $\gamma$-equivalence under $(\theta,L)$ for the 
              new states and $\beta$-equivalence under 
              $(\theta,H,L)$. 
              \item[expectLoginResult:] This case is the same as 
                the second branch of $\str{expectReqToken}$.
              \item[expectToken:] This case is the same as 
                the third branch of $\str{expectReqToken}$.
            \end{description}
          \end{enumerate}
        \item[2 (navigate to URL):] 
        In this case, a new window is opened
        in the browser and a document is loaded from $\mi{url}$.
  
        The states of both browsers are changed in the same way except
        if the domain of the URL is $\str{CHALLENGE}$. In both cases, a
        new (at this point empty) window is created and appended the
        $\str{windows}$ subterm of the browsers. This subterm is
        therefore changed in exactly the same way.
  
        A new HTTP request is created and generated requests in 
        both processing
        steps can only differ in the host part iff the domain is
        $\str{CHALLENGE}$. In this case, in $b_1$ the domain is replaced
        by $\mi{dr}_1$ and in $b_2$ by $\mi{dr}_2$ and the
        $\alpha$-equivalence in the following holds for $H' := H \{n\}$,
        where $n$ is the nonce of the generated HTTP request.
  
        The request cannot contain any $l \in L$ or $t \in T$.
        and 
        
        the Condition~\ref{eqe:prototagequiv} of 
        Definition~\ref{def:Events}.
  
        In both processing steps, three nonces are chosen.
  
        Therefore, we have $\alpha$-equivalence for $(S_1',E_1',N_1')$
        and $(S_2',E_2',N_2')$.
        \item[3 (reload document):]
        Here, an existing document is
        retrieved from its original location again. From the definition
        of $\gamma$-equivalence under $(\theta,L)$ we can see that
        whatever document is reloaded, its location is either (I)
        term-equivalent under $\theta$, or (II) it is term-equivalent
        under $\theta$ except for the domain, which is $\mi{dr}_1$ in
        $b_1$ and $\mi{dr}_2$ in $b_2$. 
  
        We note that in either case, the requests are constructed from
        the location and referrer properties of the document that is to
        be reloaded, and therefore, cannot contain any $t\in T$.
  
        In Case~(I), we note that the domain cannot be
        $\str{CHALLENGE}$. If the document is reloaded, the same 
        request is issued in both browsers (therefore,
        $\beta$-equivalence under $(\theta, H, L)$ is given), and 
        none states are changed such that we have
        $\gamma$-equivalence under $(\theta, L)$. The same number of
        nonces is chosen in both runs, and we have
        $\alpha$-equivalence.
  
        Case~(II) is similar, but we have $H' := H \cup \{n\}$, where
        $n$ is the nonce of the HTTP request. 
        Then we have $\beta$-equivalence under
        $(\theta,H',L)$. Again, the same number of nonces is chosen and
        we have $\alpha$-equivalence. 
        \end{description}
      \item[Other] Any other message is discarded by the browsers 
        without any change to state or output events.
    \end{description}
  
  %lemma放在后面
  %需要在with case5部分添加补充说明,说明每个case具体说明了什么。
  %在详细说说semi-honest的IdP
  %只要保留web attacker
  %lemma16纯数学表达
    \paragraph{\underline{Case $p_1$ is some attacker:}}

    \newc
    When $p_1$ is some attacker, the most noticeable party we talk about here is IdP.
    In UPPRESSO, there is a centralized identity provider so we think that IdP is always honest but curious.
    Therefore, when we analyze IdP's processing steps, 
    we assume that the output events from IdP follow our models' design.
    However, IdP may be curious and derive something in its states that destroys our privacy property, 
    so what we try to do is to prove that with the equivalent states and events, 
    two systems can only process another equivalent states and events.
    \oldc

    Here, only Case~\ref{eqe:prototagequiv} from Definition~\ref{def:Events} can apply to the input events,
    i.e., the input events are term-equivalent under proto-tags $\theta$. 
    This implies that the message was delivered to the same attacker process in both processing steps. 
    
    \newc
    Let $A$ be that attacker process. 
    With Case~\ref{eqs:att} of Definition~\ref{def:eq-of-states} we have that $S_1(A) \prototagequiv\theta S_2(A)$.
    With Case~\ref{eqe:pre:t} of Definition~\ref{def:Events} and Case~\ref{eqs:att-not-t} of Definition~\ref{def:eq-of-states}, 
    we have that neither the states of A, i.e, $S_1(A)$ and $S_2(A)$ contain the nonce t, nor does the events $e_i^{(1)}$ and $e_i^{(2)}$.
    Hence with lemma~\ref{thm-idp-untraceability-new}, it follows immediately that the attacker cannot distinguish any of the tags in $\theta$ in its knowledge.
    Therefore, if two states or events are term-equivalent, the attacker process $A$ cannot distinguish between them.
    \oldc

    Obviously, there are no variables (from $V_\text{process}$) in the attackers' states.
    With the output term being a fixed term (with variables)
    $\tau_{\text{process}} \in \terms(\{x\} \cup V_\text{process})$ 
    and $x$ being $S_1(A)$ or $S_2(A)$ respectively, 
    there is no subterm $l \in L$ contained in either $S_1(A)$ or $S_2(A)$ 
    (Case~\ref{eqs:att-not-l} of Definition~\ref{def:eq-of-states}), 
    it is easy to see that only Case~\ref{eqe:prototagequiv} from Definition~\ref{def:Events} can apply to the output events 
    and the output events are $\beta$-equivalent under $\theta$, 
    i.e., $E ^{(1)}_\text{out} \prototagequiv\theta E^{(2)}_\text{out}$. 
    There are not any $t \in T$ contained in the output events meaning that 
    the new state of the attacker in both processing steps is $\beta$-equivalent under proto-tags $\theta$.
    The used nonces are the same, i.e., $N_1' = N_2'$. 
    Therefore we have $\alpha$-equivalence on the new configurations.
  \end{proof}

  %\begin{lemma}\label{thm-idp-untraceability}
  %  Given a point on the elliptic curve denoted by $[r]G$, 
  %  an adversary cannot distinguish $[tr]G$ from a random variable on $\mathbb{E}$, 
  %  where $t$ is random in $\mathbb{Z}_n$ and unknown to the adversary.
  %\end{lemma}
  %\begin{proof}
  %  Consider a finite cyclic group $\mathbb{E}$ where the number of points on $\mathbb{E}$ is $n$. 
  %  Because $G$ is a generator of order $n$, $[r]G$ is also a generator on $\mathbb{E}$ of order $n$. 
  %  $t$ is randomly chosen in $\mathbb{Z}_n$ and always kept unknown to the adversary. 
  %  Therefore, $[tr]G$ is \emph{indistinguishable} from a point $Q$ that is randomly chosen on $\mathbb{E}$.\cite{oprf-proved,voprf-proved}.
  %\end{proof}
  
  This proves Theorem~\ref{theorem:A}.\QED
  
  \section{Proof of Privacy against RP-based Identity Linkage}
  
  \subsection{Formal Model of UPPRESSO for Privacy Analysis}
  
  In our privacy analysis, we show that malicious RPs colluded 
  with each other in UPPRESSO cannot infer the identities of 
  honest users. We formalize this property as an 
  indistinguishability property: two relying parties (modeled as 
  a web attacker) cannot distinguish whether a user logging in 
  at one relying party also logs in the other relying party.
  
  We will here first describe the precise model that we use for 
  privacy analysis. After that, we define an equivalence 
  relation between configurations, which we will then use in the 
  proof of privacy.
  
  %privacy证明中,虽然会corrupt,但是不会打破authentication property
  %把rp归入web attacker,不需要加malicious
  %1.对于两个malicious rp,任何user的login instance不能区分是同一个人的还是不是
  %2.malicious user告诉rp是同一个login instance,
  %benign user对于rp是不可区分的
  %IDToken equivalence:由同一个user对两个RP分别产生的两个IDToken不可区分
  \begin{definition}[Challenge IdP]
    Let $\mi{dr}$ some domain, $u$ some identity and $\mi{idp}(\mi{dr}, u)$ a DY process. 
    We call it a \emph{challenge IdP} iff $\mi{idp}$ is defined exactly the same as a identity server with two exceptions: 
    (1) the state contains one more property, namely $\mi{challenge}$, which initially contains the term $\top$. 
    (2) The IdP's algorithm is modified by the following at line~\ref{line:uppresso-idp-set-pidu} in algorithm~\ref{alg:idp}: 
    It is checked if the login request $m$ is addressed to the domain $\mi{dr}$ and no other message $m'$ was addressed to this domain before.(i.e., $\mi{challenge} \not\equiv \bot$)
    Then the $PID_u$ is generated using the given $u$ and $\mi{challenge}$ is set to $\bot$ to recorded that a message was addressed to $\mi{dr}$. 
  \end{definition}
  
  %B->b1
  \begin{definition}[\uppresso Web System for Privacy Analysis]\label{def:uppresso-ws-priv}
    Let $\uppressowebsystem = (\bidsystem, \scriptset, \mathsf{script}, E^0)$ be an UPPRESSO web system with 
    $\bidsystem = \mathsf{Hon} \cup \mathsf{Web} \cup \mathsf{Net}$, 
    $\mathsf{Hon} = \fAP{B} \cup \fAP{RP} \cup \fAP{IDP}$. (as described in Appendix~\ref{app:outlineuppressomodel}).
    $\fAP{RP} = \{r_1,r_2\}$, $r_1$ and $r_2$ two relying parties, 
    Let $\fAP{attacker} = \{r_1,r_2\}$ be some web attacker.
    Let $\mi{dr}$ be the domain of $r_2$, $u_x$ an identity owned by the browser $b$ 
    and $u_y$ an identity only known by the IdP, then $\mi{idp}_c = \mi{idp}(\mi{dr}, u_x/u_y)$ is a challenge IdP. 
    Let $\mathsf{Hon}' := \fAP{B} \cup \{\mi{idp}_c\}$, $\mathsf{Web}' := \mathsf{Web}$, and $\mathsf{Net}' := \emptyset$ (i.e., there is no network attacker).
    Let $\bidsystem' := \mathsf{Hon}' \cup \mathsf{Web}' \cup \mathsf{Net}'$. 
    Let $\scriptset' := \scriptset$ and $\mathsf{script}'$ be accordingly.
    We call $\uppressoprivwebsystem(\mi{dr}, u_x, u_y) = (\bidsystem', \scriptset', \mathsf{script}', E^0, \fAP{attacker})$ 
    an \emph{\uppresso web system for privacy analysis} 
    iff the domain $\mi{dr}$ the only domain assigned to $r_2$. 
    All honest parties (in $\mathsf{Hon}$) are not corruptible, i.e., they ignore any $\str{CORRUPT}$ message. 
    Relying Parties are assumed to be dishonest, and hence, are subsumed by the web attackers.
  \end{definition}
  
  \begin{definition}[RP-Privacy]\label{def:rp-privacy}
    Let 
    \begin{align*}
      \uppressoprivwebsystem_1 := \uppressoprivwebsystem(\mi{dr}, u_x) =
      (\bidsystem_1, \scriptset, \mathsf{script}, E^0, \fAP{attacker}_1)&\text{ and}\\
      \uppressoprivwebsystem_2 := \uppressoprivwebsystem(\mi{dr}, u_y) =
      (\bidsystem_2, \scriptset, \mathsf{script}, E^0, \fAP{attacker}_2)&
    \end{align*}
    be \uppresso web systems for privacy analysis. 
    Further, we require $\fAP{attacker}_1 = \fAP{attacker}_2 =: \fAP{attacker}$ 
    and for $idp_1 := idp(\mi{dr}, u_x)$, $idp_2 := idp(\mi{dr}, u_y)$, 
    we require $S(idp_1) = S(idp_2)$ and $\bidsystem_1 \setminus \{idp_1\} = \bidsystem_2 \setminus \{idp_2\}$ 
    (i.e., the web systems are the same up to the parameter of the challenge idps).  
    We say that $\uppressoprivwebsystem$ is \emph{RP-private} iff $\uppressoprivwebsystem_1$ and $\uppressoprivwebsystem_2$ are indistinguishable.
  \end{definition}
  
  \subsection{Definition of Equivalent Configurations}\label{app:rp:defin-equiv-stat}
  
  Let $\uppressoprivwebsystem_1 = (\bidsystem_1, \scriptset, \mathsf{script}, E^0, \fAP{attacker})$ 
  and $\uppressoprivwebsystem_2 = (\bidsystem_2, \scriptset, \mathsf{script}, E^0, \fAP{attacker})$ 
  be \uppresso web systems for privacy analysis. 
  Let $(S_1,E_1,N_1)$ be a configuration of $\uppressoprivwebsystem_1$ 
  and $(S_2,E_2,N_2)$ be a configuration of $\uppressoprivwebsystem_2$.
  
  \begin{definition}[Proto-Accts]
    We call a term of the form $[u]ID_{r_2}$ with the variable
    $u$ as a placeholder for a nonce, and 
    $ID_{r_2}\in K_\text{point}$ as the identity of 
    relying party $r_2$ \emph{proto-acct}.
  \end{definition}
  
  \begin{definition}[Term Equivalence up to Proto-Accts]
    Let $\theta = \{a_1, \ldots, a_l \}$ be a finite set of proto-accts.
    Let $t_1$ and $t_2$ be terms. We call $t_1$ and $t_2$
    \emph{term-equivalent under a set of proto-accts $\theta$} 
    iff there exists a term $\tau \in \terms(\{x_1,\dots,x_l\})$ such that
    $t_1 = (\tau [ a_1 / x_1 , \dots , a_l / x_l ])[ u_x / u ]$ and
    $t_2 = (\tau [ a_1 / x_1 , \dots , a_l / x_l ])[ u_y / u ]$.
    We write $t_1 \prototagequiv{\theta} t_2$.
  
    We say that two finite sets of terms $D$ and $D'$ are
    \emph{term-equivalent under a set of proto-tags $\theta$} 
    iff $|D| = |D'|$ and, given a lexicographic ordering of the 
    elements in $D$ of the form $(d_1,\dots,d_{|D|})$ and the 
    elements in $D'$ of the form $(d'_1,\dots,d_{|D'|})$, we 
    have that for all 
    $i \in \{1,\dots,|D|\}$: $d_i \prototagequiv{\theta} d'_i$. 
    We then write $D \prototagequiv{\theta} D'$.
  \end{definition}

  \begin{definition}
    Let $a$ be a proto-acct, $S_1$ and $S_2$ be states of \uppresso web systems for privacy analysis, and $l$ a nonce. 
    We call $l$ a login session token for the proto-acct $a$, written $l \in \mathsf{loginSessionTokens}(a,S_1,S_2)$ 
    iff for any $i \in \{1,2\}$ we have that $\proj{2}{S_i(\mi{idp}).\str{sessions}[l].\mi{IDToken}} = a[ID_{u_{x/y}}/u]$.
  \end{definition}
  
  \begin{definition}[Equivalence of States]\label{def:rp:eq-of-states}
    Let $\theta$ be a set of proto-tags. %and 
    %$H$ be a set of nonces. 
    %$L$ be a set of login session tokens.
    %Let $T:=\{t\mid [t]R\in \theta\}$. 
    We call $S_1$ and $S_2$ \emph{$\gamma$-equivalent under 
    $\theta$} iff the following conditions are met:
    \begin{enumerate}
    \item\label{eqs:rp:idp} 
      $S_1(\mi{idp_1})$ equals $S_2(\mi{idp_2})$ except
      for the subterms $\str{sessions}$, and
    \item\label{eqs:rp:idp-sessions} 
      $S_1(\mi{idp_1}).\str{sessions} \prototagequiv{\theta}$ 
      $S_2(\mi{idp_2}).\str{sessions}$, and
    \item\label{eqs:rp:att-unknown}
      $u_x, u_y, r_1, r_2 \not\in $
      $d_\emptyset(\bigcup_{i\in\{1,2\},A\,\in\,\mathsf{Web}}S_i(A))$, and
    \item\label{eqs:rp:att} 
      for each attacker $A$:
      $S_1(A) \prototagequiv{\theta} S_2(A)$, and
    \item\label{eqs:rp:att-not-l} 
      for all $a\in\theta$ and all attackers $A$ we have that
      $\nexists\ l \in \mathsf{loginSessionTokens}(a,S_1,S_2)$ 
      such that $l$ is a subterm of $S_1(A)$ or $S_2(A)$.
    \item\label{eqs:rp:b} 
      $S_1(b_1)$ equals $S_2(b_1)$ except for the subterms 
      $\str{windows}$ and we have that
      \begin{enumerate}
      \item\label{eqs:rp:b:w}
        $S_1(b_1).\str{windows}$ equals $S_2(b_2).\str{windows}$ 
        with the exception of the subterms $\str{location}$, $\str{referrer}$, $\str{scriptstate}$ 
        and $\str{scriptinputs}$ of some document terms pointed 
        to by $\mathsf{Docs}^+(S_1(b_1)) = \mathsf{Docs}^+(S_2(b_1)) =: J$. 
        For all $j \in J$ we have that: 
        \begin{enumerate}
        \item there is no $x\in\myangle{u_x, u_y, r_1, r_2}$ such that
          \begin{align*}
            x \in d_{\nonces \setminus \{x\}}(\{&S_1(b_1).j.\str{location}
            ,  S_2(b_2).j.\str{location},\\ & S_1(b_1).j.\str{referrer} , 
            S_2(b_2).j.\str{referrer}\})
          \end{align*}
        \item \label{eqs:rp:b:w:scriptinputs} for $p \in \{$
          \begin{align*}
            & \an{\tXMLHTTPRequest,*,*},\\
            & \an{\tPostMessage,*,\an{\mapDomain(r_{1,2}), \https},\an{\str{t}, *}},\\
            & \an{\tPostMessage,*,\an{\mapDomain(r_{1,2}), \https},\an{\str{IDToken}, *}}\\
            & \an{\tPostMessage,*,\an{\mapDomain(idp), \https},\an{\str{Cert}, *}}
          \end{align*}
          $\}$ we have

          $S_1(b_1).j.\str{scriptinputs} |\, p \prototagequiv{\theta}
          S_2(b_2).j.\str{scriptinputs} |\, p$, and
        \item\label{eqs:rp:b:w:script}
          $S_1(b_1).j.\str{scriptstate} \prototagequiv{\theta}
          S_2(b_2).j.\str{scriptstate}$, and 
        \end{enumerate}
      \item\label{eqs:b:misc} for
        $x \in \{\str{cookies},\str{localStorage},\str{sessionStorage},\str{sts}\}$
        we have that $S_1(b_1).x \prototagequiv{\theta} S_2(b_2).x$.
      \end{enumerate}
    \end{enumerate}
  \end{definition}
  
  \begin{definition}[Equivalence of Events]\label{def:rp:Events}
    Let $\theta$ be a set of proto-tags, 
    $L$ be a set of login session tokens, 
    %$H$ be a set of nonces, and
    %$T:=\{t\mid [t]R\in \theta\}$. 
    We call $E_1 = (e_1^{(1)}, e_2^{(1)}\dots)$ and
    $E_2= (e_1^{(2)}, e_2^{(2)} \dots)$ 
    \emph{$\beta$-equivalent under $(\theta, L)$} 
    iff all of the following conditions are satisfied for every 
    $i \in \mathbb{N}$:
  
    \begin{enumerate}
      \item\label{eqe:rp:distinction} 
        $e_i^{(1)} \prototagequiv{\theta} e_i^{(2)}$, and
      \item\label{eqe:rp:pre:l} If there exists $l \in L$ such that $l$ is a
        subterm of $e_i^{(1)}$ or $e_i^{(2)}$ then we have that
        $e_i^{(1)}$ is a message from $b_1$ to $\mi{idp}$ and $e_i^{(2)}$ is a
        message from $b_2$ to $\mi{idp}$ or we have that $e_i^{(1)}$ is a
        message from $\mi{idp}$ to $b_1$ and $e_i^{(2)}$ is a message from
        $\mi{idp}$ to $b_2$.
      \item\label{eqe:rp:pre:t} $u_x, u_y, r_1, r_2 \not\in e_i^{(1)}, e_i^{(2)}$
      \item\label{eqe:rp:pre:rp-scripts} If $e_i^{(1)}$ or $e_i^{(2)}$ is an
        HTTP(S) response with body $g$ from a identity provider, then it does
        not contain any $\str{Location}$ or $\cSTS$ header
        and if $\proj{1}{g}$ is a string representing a script, then
        $\proj{1}{g}$ is $\str{script\_idp}$.
      \item\label{eqe:rp:pre:unencrypted-http} 
        If $e_i^{(1)}$ or $e_i^{(2)}$ is an unencrypted HTTP 
        response, then the message was sent by some attacker.
    \end{enumerate}
  \end{definition}
  
  \begin{definition}[Equivalence of Configurations]
    We call $(S_1,E_1,N_1)$ and $(S_2,E_2,N_2)$
    \emph{$\alpha$-equivalent} iff there exists a set of 
    proto-accts $\theta$ such that $S_1$ and $S_2$ are
    $\gamma$-equivalent under $\theta$, $E_1$ and $E_2$ are
    $\beta$-equivalent under $(\theta,L)$ 
    for $L := \bigcup_{a\in\theta} \mathsf{loginSessionTokens}(a,S_1,S_2)$, 
    and $N_1 = N_2$.
  \end{definition}
  
  \subsection{Privacy Proof}
  
  \begin{theorem} \label{theorem:rp-privacy}
    Every UPPRESSO web system for privacy analysis is RP-private.
  \end{theorem}
  
  Let $\mathcal{U\!W\!S}^{priv}$ be UPPRESSO web system for privacy analysis.
  To prove Theorem \ref{theorem:rp-privacy}, first we have to show that 
  the UPPRESSO web systems $\mathcal{U\!W\!S}^{priv}_1$ and 
  $\mathcal{U\!W\!S}^{priv}_2$ are indistinguishable. To show 
  the indistinguishability of $\mathcal{U\!W\!S}^{priv}_1$ and 
  $\mathcal{U\!W\!S}^{priv}_2$, we show that they are 
  indistinguishable under all schedules $\sigma$. For this, 
  we first note that for all $\sigma$, there is only one run 
  induced by each $\sigma$(as our web system, when scheduled, is deterministic).
  We now proceed to show that for all schedules $\sigma=(\zeta _1, \zeta_2,\dots)$, 
  iff $\sigma$ induces a run $\sigma(\mathcal{U\!W\!S}^{priv}_1)$ 
  there exists a run $\sigma(\mathcal{U\!W\!S}^{priv}_2)$ 
  such that $\sigma(\mathcal{U\!W\!S}^{priv}_1)\approx\sigma(\mathcal{U\!W\!S}^{priv}_1)$
  
  We now show that if two configurations are $\alpha$-equivalent, 
  then the view of the attacker is statically equivalent.
  
  \begin{lemma}\label{lemma:statically-equivalent}
    Let $(S_1,E_1,N_1)$ and $(S_2,E_2,N_2)$ be two 
    $\alpha$-equivalent configurations. 
    Then $S_1(attacker)\approx S_2(attacker)$.
  \end{lemma}
  \begin{proof}
    From the $\alpha$-equivalence of $(S_1,E_1,N_1)$ and 
    $(S_2,E_2,N_2)$ it follows that $S_1(\fAP{attacker}) 
    \prototagequiv{\theta} S_2(\fAP{attacker})$.
    From Condition~\ref{eqs:rp:att-unknown} for 
    $\gamma$-equivalence it follows that
    $u_x, u_y, r_1, r_2 \not\in $
    $d_\emptyset(\bigcup_{i\in\{1,2\},A\,\in\,\mathsf{Web}}S_i(A))$ 
    (i.e., the attacker does not know any keys for the accts 
    contained in its view), and therefore it is easy to see 
    that the views are statically equivalent.
  \end{proof}
  
  We now show that $\sigma(\uppressoprivwebsystem_1) \approx
  \sigma(\uppressoprivwebsystem_2)$ by induction over the length 
  of $\sigma$. 
  We first, in Lemma~\ref{lemma:rp:initial-config-private}, 
  show that $\alpha$-equivalence (and therefore, 
  indistinguishability of the views of $\fAP{attacker}$) holds 
  for the initial configurations of $\uppressoprivwebsystem_1$ 
  and $\uppressoprivwebsystem_2$. 
  We then, in Lemma~\ref{lemma:rp:step-config-private}, 
  show that for each configuration induced by a processing step 
  in $\zeta$, $\alpha$-equivalence still holds true.
  
  \begin{lemma}\label{lemma:rp:initial-config-private}
    The initial configurations $(S_1^0,E^0,N^0)$ of 
    $\mathcal{U\!W\!S}^{priv}_1$ and $(S_2^0,E^0,N^0)$ of 
    $\mathcal{U\!W\!S}^{priv}_2$ are $\alpha$-equivalent.
  \end{lemma}
  \begin{proof}
    We now have to show that there exists a set of proto-accts 
    $\theta$ such that $S_1^0$ and $S_2^0$ are 
    $\gamma$-equivalent under $\theta$, $E_1^0 = E^0$ and 
    $E_2^0 = E^0$ are $\beta$-equivalent under $\theta$.
  
    Let $\theta = \emptyset$. Obviously, both latter conditions 
    are true. For all parties $p \in \bidsystem_1 \setminus \{idp_1\}$, 
    it is clear that $S_1^0(p) = S_2^0(p)$. Also the states 
    $S_1^0(idp_1)$ and $S_2^0(b_2)$ are equal. Therefore, 
    all conditions of Definition~\ref{def:rp:eq-of-states} are 
    fulfilled. Hence, the initial configurations are 
    $\alpha$-equivalent.
  \end{proof}
  
  \begin{lemma}\label{lemma:rp:step-config-private}
    Let $(S_1,E_1,N_1)$ and $(S_2,E_2,N_2)$ be two 
    $\alpha$-equivalent configurations of 
    $\uppressoprivwebsystem_1$ and $\uppressoprivwebsystem_2$, 
    respectively. Let $\zeta = \an{\mi{ci},\mi{cp}, 
    \tau_\text{process}, \mi{cmd}_\text{switch}, 
    \mi{cmd}_\text{window},\tau_\text{script},\mi{url}}$
    be a web system command. Then, $\zeta$ induces a processing 
    step in either both configurations or in none. In the latter 
    case, let $(S_1',E_1',N_1')$ and $(S_2',E_2',N_2')$ be 
    configurations induced by $\zeta$ such that
    \[(S_1,E_1,N_1) \xrightarrow{\zeta} (S_1',E_1',N_1') \quad 
    \text{and} \quad (S_2,E_2,N_2) \xrightarrow{\zeta} 
    (S_2',E_2',N_2') \ .\]
    Then, $(S_1',E_1',N_1')$ and $(S_2',E_2',N_2')$ are
    $\alpha$-equivalent.  
  \end{lemma}
  \begin{proof}
    Let $\theta$ be a set of proto-tags for which 
    $\alpha$-equivalence holds.
    
    To induce a processing step, the ci-th message from $E_1$ or 
    $E_2$, respectively, is selected.Following Definition 
    \ref{def:Events}, we denote these messages by $e_i^{(1)}$ or 
    $e_i^{(2)}$, respectively. We now differentiate between the 
    receivers of the messages by denoting the induced processing 
    steps by
    \begin{equation}
      \begin{aligned}
        (S_1,E_1,N_1)\xrightarrow[p_1\rightarrow E_{out}^{(1)}]{\left \langle a_1,f_1,m_1\right \rangle\rightarrow p_1}(S_1\prime,E_1\prime,N_1\prime)\\
        (S_2,E_2,N_2)\xrightarrow[p_2\rightarrow E_{out}^{(2)}]{\left \langle a_2,f_2,m_2\right \rangle\rightarrow p_2}(S_2\prime,E_2\prime,N_2\prime)
      \end{aligned}
    \end{equation}
    \paragraph{\underline{Case $p_1 = \fAP{idp_1}$}}
    $\implies p_2 = \fAP{idp_2}$

    In this case, we distinct several cases of HTTP(S) requests that can happen.
    There are four possible types of HTTP requests that are accepted by $\mi{idp}_1$ in Algorithm \ref{alg:idp}:
    
    \begin{itemize}
      \item path=$\str{/script}$(get the idp-script), Line~\ref{line:idp-script};
      \item path=$\str{/authentication}$(set the login session), Line~\ref{line:idp-authentication};
      \item path=$\str{/reqToken}$(retrieve the IDToken), Line~\ref{line:idp-reqToken};
      \item path=$\str{/authorize}$(construct the IDToken), Line~\ref{line:idp-authorize}.
    \end{itemize}

    Here, Condition~\ref{eqe:rp:distinction} from Definition~\ref{def:rp:Events} applys to the input events,
    i.e., the input events are term-equivalent under proto-accts $\theta$. 
    \begin{itemize}
      \item path=$\str{/script}$ 
        In this case, the same output event is produced whose message is 
        \begin{equation}
          \begin{aligned}
            \myangle{HTTPResp,n,200,\myangle{},IdPScript}
          \end{aligned}
        \end{equation}
        We can note that Condition~\ref{eqe:rp:pre:rp-scripts} of Definition \ref{def:rp:Events} holds true and the remaining conditions are trivially fulfilled.
        Therefore, $E_1\prime$ and $E_2\prime$ are $\beta$-equivalent under $(\theta, L)$. 
        As there are no changes to any state, we have that $S_1\prime$ and $S_2\prime$ are $\gamma$-equivalent under $\theta$. 
        No new nonces are chosen, hence $N_1\prime=N_1=N_2=N_2\prime$.
      \item path=$\str{/authentication}$ 
        Here, since $e_i^{(1)} \prototagequiv{\theta} e_i^{(2)}$, we can see that the username and password in the HTTP body is equal. 
        Hence, the check at Line~\ref{line:uppresso-idp-check-login-state} in Algorithm~\ref{alg:idp} either both fail or not . 
        If the username and password are correct, IdP will set up a login session and output a respones whose message is 
        \begin{align*}
          \myangle{\cHttpResp,n,200,\an{\mi{setCookie}},\str{LoginSuccess}}
        \end{align*}
        with
        \begin{align*}
          \mi{setCookie} := \myangle{\cSetCookie, \myangle{\myangle{\str{sessionid}, \nu_3, \True, \True, \True}}} \\
        \end{align*}
        We can note that Condition~\ref{eqs:rp:idp-sessions} of Definition~\ref{def:rp:eq-of-states} holds true for the new sates of $\mi{idp}_1$ and $\mi{idp}_2$ because only a username is added to the sessions.
        As there are not any other changes to the state and $\theta\prime = \theta$, we have that $S_1\prime$ and $S_2\prime$ are $\gamma$-equivalent under $\theta$. 
        
        For $N_1 = N_2 = (n_1, n_2, \dots)$, we set $N_1' = N_2' = (n_2, \dots)$ (as exactly one nonce is chosen in both processing steps) and $L' = L \cup \{n_1\}$. 
        So Condition~\ref{eqe:rp:pre:l} of Definition~\ref{def:rp:Events} holds true the output event.
        Obviously we can have that Condition~\ref{eqe:rp:pre:t} of Definition~\ref{def:rp:Events} holds true, so $E_1'$ and $E_2'$ are $\beta$-equivalent under $(\theta, L')$.
      \item path=$\str{/reqToken}$ 
        Here, since $e_i^{(1)} \prototagequiv{\theta} e_i^{(2)}$, 
        we can see that the $\str{sessionid}$ in the HTTP headers either both exist or not.
        Therefore, the check at Line~\ref{line:check-sessionid} in Algorithm~\ref{alg:idp} either both fail or not. 
        Obviously, we can have that if the $\str{sessionid}$ exists, it must be the same in both systems.
        Since $S_1(\mi{idp}_1).\str{sessions} \prototagequiv{\theta} S_2(\mi{idp}_2).\str{sessions}$, 
        the second check at Line~\ref{line:check-session-pidrp} also either both fail or not.
        If the two checks both pass, the same output event is produced whose message is 
        \begin{align*}
          \myangle{\cHttpResp,n,200,\myangle{},\mi{IDToken}}
        \end{align*}
        Since the IDToken has already been constructed, there are not any changes to the new states of IdP.
        We can easily have that $S_1\prime$ and $S_2\prime$ are $\gamma$-equivalent under $\theta$. 
        
        As $L' = L$ and $u_x, u_y, r_1, r_2$ do not present directly in the IDToken, $E_1'$ and $E_2'$ are $\beta$-equivalent under $(\theta, L')$.
        No new nonces are chosen, hence $N_1\prime=N_1=N_2=N_2\prime$.
      \item path=$\str{/authorize}$ 
        Here, the first check is the same as $\str{/reqToken}$.
        Since $e_i^{(1)} \prototagequiv{\theta} e_i^{(2)}$, the second and third check either both pass or not.
        If IdP accepts the request, it will start to sign an IDToken.
        There are two cases for $\theta\prime$, if the $\mi{PID_{rp}}$ is mapped to $r_2$ and $S_i(\mi{idp_i}).\str{challenge} = \True$, 
        $\theta\prime = \theta \cup \{ [u_{x/y}]\mi{PID_{r_2}} \}$, otherwise, $\theta\prime = \theta$.
        In the first case, $S_1(\mi{idp_1}).\str{sessions} \prototagequiv{\theta\prime} S_2(\mi{idp_2}).\str{sessions}$, 
        while in the second case, the newly added sessions are equal.
        Therefore, we can see that Condition~\ref{eqs:rp:idp-sessions} of Definition~\ref{def:rp:eq-of-states} holds true for the new states, 
        and $S_1\prime$ and $S_2\prime$ are $\gamma$-equivalent under $\theta\prime$

        In this case, the output event is as same as $\str{/reqToken}$, 
        so we can have that $E_1'$ and $E_2'$ are $\beta$-equivalent under $(\theta\prime, L)$.
        No new nonces are chosen, hence $N_1\prime=N_1=N_2=N_2\prime$.
    \end{itemize}

    \paragraph{\underline{Case $p_1 = \fAP{b_1}$}}

    \begin{description}
      \item[TRIGGER] We now distinguish between the possible values for $\mi{cmd}_\text{switch}$.
        \begin{description}
        \item[1 (trigger script):] In this case, the script in the window indexed by $\mi{cmd}_\text{window}$ is triggered. Let $j$ be a pointer to that window.
            
          We first note that such a window exists in $b_1$ iff it exists in $b_2$ and that $S_1(b_1).j.\str{script} \equiv S_2(b_2).j.\str{script}$. 
          We now distinguish between the following cases, which cover all possible states of the windows/documents:
            
          \begin{enumerate}
          \item $S_1(b_1).j.\str{origin} \in \{\an{r_1,\https},\an{r_2,\https}\}$ and $S_1(b_1).j.\str{script} \equiv \str{script\_rp}$.
            Similar to the following scripts, the main distinction in this script is between the script's internal states (named $\str{phase}$). 
            With the term-equivalence under proto-accts $\theta$ we have that $S_1(b_1).j.\str{scriptstate}.\str{phase} = S_2(b_2).j.\str{scriptstate}.\str{phase}$ 
              
            We can therefore now distinguish between the possible values of $S_1(b_1).j.\str{scriptstate}.\str{phase} = S_2(b_2).j.\str{scriptstate}.\str{phase}$:
            \begin{description}
            \item[start:] In this case, the script open a blank page addressed to its own origin which is equal in both systems.
              The path is the (static) string $\str{/loginSSO}$. 
              The script saves a (static) value for $\str{phase}$ in its scriptstate.
      
              Obviously, we can have that the command is term-equivalent under proto-accts $\theta$ 
              and hence, the browser emits a HTTP request which is term-equivalent.
              Therefore, we have $\gamma$-equivalence under $\theta$ for the new states, 
              $\beta$-equivalence under $(\theta,L)$ for the new events, and $\alpha$-equivalence for the new configuration.
            \item[expectt:] In this case, the script retrieves the result of a \pm from $\mi{scriptinputs}$. 
              As we know that $S_1(b_1).j.\str{scriptstate} \prototagequiv{\theta} S_2(b_2).j.\str{scriptstate}$ 
              and that for all matching \pms that they also have to be term-equivalent up to $\theta$ 
              and that the window structure is equal in both browsers, 
              we have that either the same \pm is retrieved from $\mi{scriptinputs}$ or none in both browsers.
      
              Then the script saves a (static) value for $\str{phase}$ in its scriptstate. 
              Therefore, we have $\gamma$-equivalence under $\theta$ for the new states. 
              We also have $\beta$-equivalence under $(\theta, L)$ for the new events, 
              and $\alpha$-equivalence for the new configuration.
            \item[expectCert:] In this case, the script retrieves the result of an \xhr from $\mi{scriptinputs}$ that matches the reference contained in $\mi{scriptstate}$. 
              From Condition~\ref{eqs:rp:b:w:scriptinputs} and \ref{eqs:rp:b:w:script} of Definition~\ref{def:rp:eq-of-states}, 
              we know that all results from \xhr{}s in $\mi{scriptinput}$ are term-equivalent up to $\theta$ 
              and that $\mi{scriptstate}$ is term-equivalent up to $\theta$. 
              Hence, in both browsers, both scripts stop with an empty command or both continue as they successfully retrieved such an \xhr.
      
              The script now constructs a \pm that is sent to exactly the same window in both browsers 
              and that requires that the receiver origin has to be $\an{\str{IdPdomain},\https}$. 
              The postMessage is only sent to this origin, we have that $\gamma$-equivalence cannot be violated.
      
              We now have that $S_1'$ and $S_2'$ are $\gamma$-equivalent under $\theta$, 
              $E_1'$ and $E_2'$ are $\beta$-equivalent under $(\theta,L)$, 
              and as exactly none of nonces is chosen, 
              we have that the new configuration is $\alpha$-equivalent.
            \item[expectToken:] In this case, the script retrieves the result of an \xhr from $\mi{scriptinputs}$ that matches the reference contained in $\mi{scriptstate}$. 
              From Condition~\ref{eqs:rp:b:w:scriptinputs} and \ref{eqs:rp:b:w:script} of Definition~\ref{def:rp:eq-of-states}, 
              we know that all results from \xhr{}s in $\mi{scriptinput}$ are term-equivalent up to $\theta$ 
              and that $\mi{scriptstate}$ is term-equivalent up to $\theta$. 
              Hence, in both browsers, both scripts stop with an empty command or both continue as they successfully retrieved such an \xhr.

              Then the script saves a (static) value for $\str{phase}$ in its scriptstate. 
              Therefore, we have $\gamma$-equivalence under $\theta$ for the new states. 
              The $IDToken$ in the output event's message is term-equivalent up to $\theta$, 
              so we also have $\beta$-equivalence under $(\theta, L)$ for the new events, 
              Hence, we have $\alpha$-equivalence for the new configuration.
            \end{description}
          \item $S_1(b_1).j.\str{script} \equiv \str{script\_idp}$.
            \begin{description}
            \item[start:] In this case, the script chooses a new nonce for $t$ which does not violate the Definition~\ref{def:rp:eq-of-states}.
              Hence, we have $\gamma$-equivalence under $\theta$ for the new states.
    
              From the equivalence definition of states (Definition~\ref{def:rp:eq-of-states}), 
              we can see that the window tree has the same structure in both processing steps. 
              So the script now constructs a \pm that is sent to exactly the same window in both browsers 
              and that requires that the receiver has to be the opener of this window. 
              Since there are no $l\in L$ or $u_x, u_y, r_1, r_2$ contained in the message, 
              Condition~\ref{eqe:rp:pre:l} and \ref{eqe:rp:pre:t} of Definition~\ref{def:rp:Events} holds, 
              we have $\beta$-equivalence under $(\theta, L)$.
            \item[expectCert:] The same as above, we can have that either the same \pm is retrieved from $\mi{scriptinputs}$ or none in both browsers 
              and the result of $checksig$ is same as well. 
              As The state $Cert_{rp}$ and $PID_{rp}$ are equal in both $\str{scriptstate}$, 
              we have $\gamma$-equivalence under $(\theta,L)$ for the new states.
    
              We can note that regardless of whether the browser has logined in IdP or not, 
              the Condition~\ref{eqe:rp:pre:l} of Definition~\ref{def:rp:Events} always holds.
              Therefore, Condition~\ref{eqe:rp:distinction} of Definition~\ref{def:rp:Events} holds, 
              and we have $\beta$-equivalence under $(\theta, L)$.   
            \item[expectReqToken:] In this case, the script retrieves the result of an \xhr from $\mi{scriptinputs}$ that matches the reference contained in $\mi{scriptstate}$. 
              From Condition~\ref{eqs:rp:b:w:scriptinputs} and \ref{eqs:rp:b:w:script} of Definition~\ref{def:rp:eq-of-states}, 
              we know that all results from \xhr{}s in $\mi{scriptinput}$ are term-equivalent up to $\theta$ 
              and that $\mi{scriptstate}$ is term-equivalent up to $\theta$. 
              Hence, in both browsers, both scripts will reach the same if-else branch.
    
              Since there aren't any new states stored, we can have $\gamma$-equivalence under $\theta$ for the new states.
              The output events's destination are fixed and $IDToken$ in the message is term-equivalent up to $\theta$.
              Therefore, we have $\beta$-equivalence under $(\theta,L)$ for the new events. 
            \item[expectLoginResult:] This case is as same as the second branch in $\str{expectReqToken}$.
            \item[expectToken:] This case is as same as the third branch in $\str{expectReqToken}$.
            \end{description}
          \end{enumerate}
        \item[2 (navigate to URL):] In this case, a new window is opened in the browser and a document is loaded from $\mi{url}$.
          The states of both browsers are changed in the same way, where a new (at this point empty) window is created and appended the $\str{windows}$ subterm of the browsers. 
          This subterm is therefore changed in exactly the same way.
            
          A new HTTP request is created and the request cannot contain any $l \in L$. 
          The Condition~\ref{eqe:rp:pre:l} of Definition~\ref{def:rp:Events} holds true.
            
          In both processing steps, three nonces are chosen. 
          Therefore, we have $\alpha$-equivalence for $(S_1',E_1',N_1')$ and $(S_2',E_2',N_2')$.

        \item[3 (reload document):]
          Here, an existing document is retrieved from its original location again. 
          From the definition of $\gamma$-equivalence under $\theta$ we can see that
          whatever document is reloaded, its location is term-equivalent under $\theta$.
      
          We note that the requests are constructed from
          the location and referrer properties of the document that is to
          be reloaded, and therefore, cannot contain any $u_x, u_y, r_1, r_2$.
      
          We also note that if the document is reloaded, the same 
          request is issued in both browsers (therefore,
          $\beta$-equivalence under $(\theta, L)$ is given), and 
          none states are changed such that we have
          $\gamma$-equivalence under $\theta$. The same number of
          nonces is chosen in both runs, and we have $\alpha$-equivalence.
        \end{description}
      \item[Other] Any other message is discarded by the browsers without any change to state or output events.
    \end{description}
  
    \paragraph{\underline{Case $p_1$ is some attacker:}}
    
    Here, when we talk about attackers, 
    we mean two colluding relying parties.
    They act as web attackers and try to bind browser's identities 
    based on what they get during the login process, 
    especially the $IDToken$.
    
    We can note that Case~\ref{eqe:rp:distinction} from Definition~\ref{def:rp:Events} applys to the input events,
    i.e., the input events are term-equivalent under proto-accts $\theta$. 
    This implies that the message was delivered to the 
    same attacker process in both processing steps. 
    Let $A$ be that attacker process. 
    With Case~\ref{eqs:rp:att} of Definition~\ref{def:rp:eq-of-states}, 
    we have that $S_1(A) \prototagequiv\theta S_2(A)$. 
    With Case~\ref{eqe:rp:pre:t} of Definition~\ref{def:rp:Events} and Case~\ref{eqs:rp:att-unknown} of Definition~\ref{def:rp:eq-of-states}, 
    we have that neither the states of A, i.e, $S_1(A)$ and $S_2(A)$ contain $u_x, u_y, r_1, r_2$, 
    nor do the events $e_i^{(1)}$ and $e_i^{(2)}$.
    Further with lemma~\ref{thm-rp-untraceability}, it follows immediately that the attacker cannot distinguish any of the accts in $\theta$ in its knowledge.
    Therefore, if two states or events are term-equivalent, the attacker process $A$ cannot distinguish between them.

    Obviously, we can have that in the attackers state, 
    there are no variables (from $V_\text{process}$). 
    With the output term being a fixed term (with variables)
    $\tau_{\text{process}} \in \terms(\{x\} \cup V_\text{process})$ 
    and $x$ being $S_1(A)$ or $S_2(A)$, respectively.
    and there is no subterm $l\in L$ contained in  
    $e_i^{(1)}$ or $e_i^{(2)}$ (Condition~\ref{eqe:rp:pre:l} of 
    Definition~\ref{def:rp:Events}), 
    it is easy to see that the output events are 
    $\beta$-equivalent under $\theta$, i.e., 
    $E ^{(1)}_\text{out} \prototagequiv\theta E^{(2)}_\text{out}$. 
    The new state of the attacker in both processing steps 
    consists of the input events, the output events, and the 
    former state of the event, and, as such, is 
    either $\beta$-equivalent or $\gamma$-equivalent 
    under proto-accts $\theta$. 
    Hence, the new states are $\gamma$-equivalent under $\theta$
    The used nonces are the same, i.e., $N_1' = N_2'$. 
    Therefore we have $\alpha$-equivalence on the new configurations.
  \end{proof}

  \begin{lemma}\label{thm-rp-untraceability}
    Given two points on the elliptic curve denoted by $[u_xr_1]G$ and $[u_yr_2]G$, 
    an adversary cannot tell whether $u_x\equiv u_y$, where $u_x, u_y, r_1, r_2$ is random in $\mathbb{Z}_n$ and unknown to the adversary.
  \end{lemma}
  \begin{proof}
    Consider a finite cyclic group $\mathbb{E}$ where the number of points on $\mathbb{E}$ is $n$. 
    Because $G$ is a generator of order $n$, $[r]G$ is also a generator on $\mathbb{E}$ of order $n$. 
    $t$ is randomly chosen in $\mathbb{Z}_n$ and always kept unknown to the adversary. 
    Therefore, $[tr]G$ is \emph{indistinguishable} from a point $Q$ that is randomly chosen on $\mathbb{E}$.\cite{oprf-proved,voprf-proved}.
  \end{proof}

  Now we have shown that the UPPRESSO web systems $\mathcal{U\!W\!S}^{priv}_1$ and $\mathcal{U\!W\!S}^{priv}_2$ are indistinguishable.
  However, it is not enough to prove Theorem~\ref{theorem:rp-privacy}.
  We only have two relying-parties in out models $\mathcal{U\!W\!S}^{priv}_1$ and $\mathcal{U\!W\!S}^{priv}_2$, 
  so we need to further prove that it is still RP-private in the scenario of three or more RPs.
  
  What's more, It is necessary to broaden the knowledge available to the attackers. 
  We introduce some malicious users in our model which act as web attackers. 
  These users login in the RPs and give their identities to the RPs.
  Therefore, colluding RPs can bind these users' login instances. 
  So what we need to prove is that even if colluding RPs and users share $PID_U$s and other information observed in all the logins, they still cannot link any login from an honest user to any other logins from any other honest users to these RPs.

  \newc
  It should be pointed out that we omit the model analysis here after we add extra parties into the privacy model $\mathcal{U\!W\!S}^{priv}$ because the proof doesn't change too much.
  What really changes is that Lemma~\ref{lemma:statically-equivalent} doesn't seem obvious anymore.
  i.e., We cannot easily see that the views of RPs are statically-equivalent without RPs' knowing any keys for the accts contained in its view.
  However, the proof is out of formal analysis's scope, so we use provable security here.
  \oldc

  With the trapdoor $t$, $PID_{RP}$ and $PID_U$ can be easily transformed into $ID_{RP}$ and $Acct$, respectively, and vice versa. Therefore, we denote the information that an RP learns from a login as a tuple $L$, where $L =(ID_{RP}, t, Acct)=(ID_{RP}, t, [ID_{U}]ID_{RP})=([r]G, t, [ur]G)$.

  When $c$ malicious RPs collude with each other, they create a shared view of all their logins, denoted as $\mathbb{L}$.
  %some of which are initiated by honest users and denoted as $\mathbb{L}^h$, and the others by $v$ malicious users  are $\mathbb{L}^m = \mathbb{L} \setminus \mathbb{L}^h$.
  When they collude further with $v$ malicious users, the logins initiated by these malicious users are picked out and linked together as
  $\mathfrak{L}^m=\left \{ \begin{matrix}
  L^m_{1,1},&L^m_{1,2},&\cdots,&L^m_{1,c}\\
  L^m_{2,1},& L^m_{2,2},&\cdots,&L^m_{2,c}\\
  \cdots,&\cdots,&L^m_{i,j},&\cdots\\
  L^m_{v,1},&L^m_{v,2},&\cdots,&L^m_{v,c}
  \end{matrix}\right\}$,
  where $L^m_{i, j}=([r_j]G, t_{i,j}, [u_ir_j]G)$ for $1 \le i \le v$ and $1 \le j \le c$, and $L^m_{i,j} \in \mathbb{L}$. Any login in $\mathbb{L}$ but not linked in $\mathfrak{L}^m$ is initiated by an honest user to one of the $c$ malicious RPs.


  \begin{theorem}
  \emph{In \usso, given $\mathbb{L}$ and $\mathfrak{L}^m$, $c$ malicious RPs and $v$ malicious users cannot link any login from an honest user to a malicious RP to any subset of logins from honest users to any other malicious RPs.}
  \end{theorem}


  \begin{proof} 
  From the logins in $\mathbb{L}$,
  we randomly choose one login $L' \neq L^m_{i,j}$,
  which is from an (unknown) honest user with $ID_{U'}=u'$ to a malicious $RP_a$ and $a \in [1,c]$.
  Then, we randomly choose another malicious $RP_b$, where $b \in [1,c]$ and $b \neq a$.
  Consider any subset $\mathbb{L}''$ of $w$ logins visiting $RP_b$ by unknown honest users,
  we denote the identities of the honest users who initiate these logins as $\mathbf{u}_w=\{{u''_1}, {u''_2}, \cdots, {u''_w}\}$.
  Next, we prove that the colluding adversaries cannot decide if $u'$ is in $\mathbf{u}_w$ or randomly selected from the universal user set.
  This indicates the colluding adversaries cannot link $L'$ to another login visiting $RP_b$
  or to another subset of logins visiting $RP_b$.

  We first define an RP-based linkage game $\mathcal{G}_r$ between an adversary and a challenger, which describes this login linkage privacy threat: the adversary receives $\mathfrak{L}^m$, $L'$, and $\mathbb{L}''$ from the challenger and outputs $s$, where $s = 1$ if it decides $u'$ is in $\mathbf{u}_w$ %$\{{U''_1}, {U''_2}, \cdots, {U''_w}\}$
  and $s=0$ if it believes $u'$ is randomly chosen from the universal user set.
  Thus, the adversary succeeds in $\mathcal{G}_r$ with an advantage $\mathbf{Adv}$:
  \begin{align*}
  %&{\rm Pr}_1={\rm Pr}\{\mathcal{G}_r(\mathfrak{L}, L', \mathbb{L}'' | u' \in \{{u''_1}, {u''_2}, \cdots, {u''_w}\})=1\} \\
  &{\rm Pr}_1={\rm Pr}(\mathcal{G}_r(\mathfrak{L}^m, L', \mathbb{L}'')=1 \;| \; u' \in \mathbf{u}_w)  \\
  %&{\rm Pr}_2={\rm Pr}\{\mathcal{G}_r(\mathfrak{L}, L', \mathbb{L}'' | u' \in \mathbb{Z}_n)=1\}\\
  &{\rm Pr}_2={\rm Pr}(\mathcal{G}_r(\mathfrak{L}^m, L', \mathbb{L}'')=1 \; | \; u' \in \mathbb{Z}_n)\\
  &{\mathbf{Adv}}=|{\rm Pr}_1-{\rm Pr}_2|
  \end{align*}

  As depicted in Figure \ref{fig:dalgorithm}, we design a PPT algorithm $\mathcal{D}^*_r$ based on $\mathcal{G}_r$ to solve the elliptic curve decisional Diffie-Hellman (ECDDH) problem: given $(G, [x]G$, $[y]G$, $[z]G)$, decide whether $z$ is equal to $xy$ or randomly chosen in $\mathbb{Z}_n$, where $G$ is a point on an elliptic curve $\mathbb{E}$ of order $n$, and $x$ and $y$ are integers randomly and independently chosen in $\mathbb{Z}_n$.

  \begin{figure}[tb]
    \centering
    \includegraphics[width=1.0\linewidth]{fig/rp-linkage-game.pdf}
    \caption{The PPT algorithm $\mathcal{D}^*_r$ constructed based on the RP-based linkage game to solve the ECDDH problem.}
    \label{fig:dalgorithm}
  \end{figure}


  The algorithm $\mathcal{D}^*_r$ works as below. (1) Upon receiving an input $(G, Q_1=[x]G, Q_2=[y]G, Q_3=[z]G)$, %of $\mathcal{D}^*_r$
  the challenger
  chooses random numbers in $\mathbb{Z}_n$ to construct $\{u_i\}$, $\{r_j\}$, and $\{t_{i, j}\}$ for $1 \le i \le v$ and $1 \le j \le c$, with which it assembles $L^m_{i, j}=([r_j]G, t_{i,j}, [u_ir_j]G)$.
  In this process, it ensures $[r_{j}]G \neq Q_2$ so that $r_j \neq y$.  % 这个应该反过来讲;因为y是离散对数。
  (2) It randomly chooses $a \in [1, c]$ and $t' \in \mathbb{Z}_n$, to assemble $L' = ([r_{a}]G, t', [r_{a}]Q_1) = ([r_{a}]G, t', [xr_{a}]G)$.
  (3)
  % Here, $L'$ represents the knowledge of the login visiting $RP_{j'}$ by a user with $ID_U = x$.
  Next, the challenger randomly chooses $b \in [1, c]$ and $b \neq a$, and replaces $ID_{RP_b}$ with $Q_2 = [y]G$.
  Hence, for $1 \le i \le v$, the challenger replaces $L^m_{i, b}=([r_b]G, t_{i,b}, [u_ir_b]G)$ with $(Q_2, t_{i,b}, [u_i]Q_2) = ([y]G, t_{i,b}, [u_iy]G)$, and then constructs $\mathfrak{L}^m$.
  (4) the challenger chooses random numbers in $\mathbb{Z}_n$ to construct $\{u''_k\}$ and $\{t''_k\}$ for $1 \leq k \leq w$,
  with which it assembles $\mathbb{L}'' = \{L''_{k; 1\leq k \leq w}\} = \{(Q_2, t''_k, [u''_k]Q_2)\} = \{([y]G, t''_k, [u''_ky]G)\}$.
  In this process, it ensures that $[u''_k]G \neq Q_1$ (i.e., $u''_k \neq x$) and $u''_k \neq u_i$,
  for $1 \le i \le v$ and $1 \le k \le w$.
  Finally, it randomly chooses $d \in [1, w]$ and replaces $L''_{d}$ with $(Q_2, t''_d, Q_3) = ([y]G, t''_d, [z]G)$.
  Thus, $\mathbb{L}'' = \{L''_{k;1\leq k \leq w}\}$ represents the logins initiated by $w$ honest users, i.e., $\mathbf{u}_w=\{u''_1, u''_2, \cdots, u''_{d-1}, z/y, u''_{d+1}, \cdots, u''_w\}$.
  (5) When the adversary of $\mathcal{G}_r$ receives $\mathfrak{L}^m$, $L'$, and $\mathbb{L}''$ from the challenger, it returns $s$ as the output of $\mathcal{D}^*_r$.

  According to the above construction, % of $\mathfrak{L}$, $L'$ and $\mathbb{L}''$,
  $x$ is embedded as $ID_{U'}$ in the login $L'$ visiting the RP with $ID_{RP_{a}} = [r_{a}]G$,
  and $z/y$ is embedded as $ID_{U''_d}$ in $\mathbb{L}''$ visiting the RP with $ID_{RP_{b}}=[y]G$,
  together with $\{u''_1, \cdots, u''_{d-1}, u''_{d+1}, \cdots, u''_w\}$.
  Meanwhile, $[r_{a}]G$ and $[y]G$ are two malicious RPs' identities in $\mathfrak{L}^m$.
  Because $x \neq u''_{k; 1\leq k \leq w, k \neq d}$ and then $x$ is not in $\{u''_1, \cdots, u''_{d-1}, u''_{d+1}, \cdots, u''_w\}$, the adversary outputs $s=1$ and succeeds in the game \emph{only if} $x = z/y$.
  % 这里不是if and only if. "if, 就变成了the adversary必胜了;并不是,而是“有显著的概率”"
  % 当“the adversary outputs s=1 且 succeeds in the game”,=> "x = z/y"
  % 但是,"x = z/y"  => 不能推导得到“the adversary outputs s=1 且 succeeds in the game”。因为adversary有时候fail、不总是succeed
  Therefore, using $\mathcal{D}^*_r$ to solve the ECDDH problem, we have an advantage $\mathbf{Adv}^*=|{\rm Pr}^*_1 - {\rm Pr}^*_2|$, where
  \begin{align*}
  &{\rm Pr}^*_1 =  {\rm Pr}(\mathcal{D}^*_r(G, [x]G, [y]G, [xy]G)=1) \\
  =&{\rm Pr}(\mathcal{G}_r(\mathfrak{L}^m, L', \mathbb{L}'')=1 \; | \; u' \in \mathbf{u}_w) = {\rm Pr}_1 \\
  &{\rm Pr}^*_2= {\rm Pr}(\mathcal{D}^*_r(G, [x]G, [y]G, [z]G)=1) \\
  =&{\rm Pr}(\mathcal{G}_r(\mathfrak{L}^m, L', \mathbb{L}'')=1 \; | \; u' \in \mathbb{Z}_n) = {\rm Pr}_2 \\
  &\mathbf{Adv}^*=|{\rm Pr}^*_1-{\rm Pr}^*_2|=|{\rm Pr}_1-{\rm Pr}_2|={\mathbf{Adv}}
  \end{align*}

  If in $\mathcal{G}_r$ the adversary has a non-negligible advantage, then $\mathbf{Adv}^*={\mathbf{Adv}}$ is also non-negligible regardless of the security parameter $\lambda$. This violates the ECDDH assumption. Therefore, the adversary has no advantage in $\mathcal{G}_r$ and cannot decide whether $L'$ is initiated by some user with an identity in $\mathbf{u}_w$ or by a user in the universal user set.
  Moreover, because $RP_b$ is any malicious RP, this proof can be easily extended from $RP_b$ to more colluding malicious RPs.
  \end{proof}
  
  These prove Theorem~\ref{theorem:rp-privacy}.\QED
  
\end{document}
  

%%%%%%%%%%%%%%%%%%%%%%%%%%%%%%%%%%%%%%%%%%%%%%%%%%%%%%%%%%%%%%%%%%%%%%%%%%%%%%%%%%%%%%%%%

%\renewcommand{\algorithmicrequire}{\textbf{Input:}}
\newcommand{\deflet}{\textbf{let}}
\newcommand{\mystate}[1]{\STATE \textbf{let} {{}#1}}
\newcommand{\mystop}[1]{\STATE \textbf{stop} \myss{\myangle{{{}#1}}, s'}}
\newcommand{\mystopp}[1]{\STATE \textbf{stop} \myss{\myangle{{{}#1}}}}
\newcommand{\myss}[1]{${{}#1}$}
\newcommand{\myangle}[1]{\langle {{}#1} \rangle}
\newcommand{\myif}[1]{\IF{\myss{{{}#1}}}}
\newcommand{\myelse}[1]{\ELSIF{\myss{{{}#1}}}}

\newcommand{\aaa}[1]{\STATE \textbf{if} #1 \textbf{then} \begin{ALC@g}}
\newcommand{\bbb}[1]{\end{ALC@g} \STATE \textbf{else if} #1 \textbf{then} \begin{ALC@g}}
\newcommand{\ccc}{\end{ALC@g} \STATE \textbf{else} \textbf{then} \begin{ALC@g}}
\newcommand{\ddd}{\end{ALC@g} \STATE \textbf{endif}}

\newcommand{\SWITCH}[1]{\STATE \textbf{switch} #1\ \textbf{do} \begin{ALC@g}}
\newcommand{\ENDSWITCH}{\end{ALC@g}\STATE \textbf{end switch}}
\newcommand{\CASE}[1]{\STATE \textbf{case} #1\textbf{:} \begin{ALC@g}}
\newcommand{\ENDCASE}{\end{ALC@g}}
\newcommand{\CASELINE}[1]{\STATE \textbf{case} #1\textbf{:} }
\newcommand{\DEFAULT}{\STATE \textbf{default:} \begin{ALC@g}}
\newcommand{\ENDDEFAULT}{\end{ALC@g}}
\newcommand{\DEFAULTLINE}[1]{\STATE \textbf{default:} }

\setcounter{section}{0}
\renewcommand{\thesection}{\Alph{section}}

\section{Preparation}

%所有和SPRESSO的不同点放在同一个章节中一起说明。

Our formal security analysis of UPPRESSO is based on 
the general Dolev-Yao web model in SPRESSO. 
To facilitate the definition of UPPRESSO, however, 
we have some difference from SPRESSO. 
In particular, we remove some processes and 
add some function symbols for asymmetric encryption/decryption.

\subsection{Functions Symbols}

Since our model is using ECC(Elliptic Curve Cryptography) to encrypt/decrypt the data,
we add the following symbols to the signature $\Sigma$ for the terms and messages:

\begin{itemize}
  \item $\mathbb{E}$ is an elliptic curve over a finite field $\mathbb{F}_q$, $G$ is a base point(or generator) of $\mathbb{E}$ and the order of $G$ is a prime number n.
  \item $[t]P$ means using asymmetric key $t$ to encrypt the point $P=[p]G$ on the elliptic curve where $p$ is the actual plaintext.
  \item $[t^{-1}]C$ means using the reverse of $t$ to decrypt the point $C=[c]G=[tm]G$ on the elliptic curve where $c$ is the cipertext.  
  \item $\str{isValid}(P)$ checks whether $P$ is a valid point on the elliptic curve. That is to say whether $P=[m]G$ for the base point $G$ and some nonce $m$.
\end{itemize}

\subsection{DNS servers}

%如果是自己编写的脚本引入了DNS请求,那就需要在模型中考虑DNS服务器,UPPRESSO可以不需要考虑。

In SPRESSO, when receiving an e-mail address, 
RP needs to send DNS requests to DNS servers manually 
to fetch the information of the IdP server. 
Since there may be various DNS servers in SPRESSO, 
DNS server security issues need to be given special consideration.
As a result, DNS servers are added into the formal model of SPRESSO.

In UPPRESSO, however, we only have one centralized IdP server, and 
all RPs know the relevant information of the IdP in advance.
So all DNS requests are generated spontaneously by the browser, 
not introduced by our scripts.
Therefore, we remove DNS servers from the formal model of UPPRESSO.

\section{Formal Model of UPPRESSO}
\label{app:model-uppresso}
We here present the full details of our formal model of UPPRESSO. For our analysis regarding our authentication and privacy properties below, we will further restrict this generic model to suit the setting of respective analysis.\par
We model UPPRESSO as a web system. We call a web system $\uppressowebsystem=(\bidsystem, \scriptset, \mathsf{script}, E^0)$ an UPPRESSO web system if it is of the form described in what follows.

\subsection{Outline}\label{app:outlineuppressomodel}
The system $\bidsystem=\mathsf{Hon}\cup \mathsf{Web} \cup \mathsf{Net}$ consists of 
web attacker processes (in $\mathsf{Web}$), network attacker processes (in $\mathsf{Net}$), 
a finite set $\fAP{B}$ of web browsers, 
a finite set $\fAP{RP}$ of web servers for the relying parties, 
a finite set $\fAP{IDP}$ of web servers containing only one identity provider 
with $\mathsf{Hon} := \fAP{B} \cup \fAP{RP} \cup \fAP{IDP}$. 
More details on the processes in $\mathpzc{W}$ are provided below. 
%
Figure~\ref{fig:scripts-in-w} shows the set of scripts $\scriptset$ 
and their respectice string representations that are defined by the 
mapping $\mathsf{script}$. 
%
The set $E^0$ contains only the trigger events.

\begin{figure}[htb]
    \centering
    \begin{tabular}{|@{\hspace{1ex}}l@{\hspace{1ex}}|@{\hspace{1ex}}l@{\hspace{1ex}}|}\hline 
      \hfill $s \in \scriptset$\hfill  &\hfill $\mathsf{script}(s)$\hfill  \\\hline\hline
      $\Rasp$ & $\str{att\_script}$  \\\hline
      $\mi{script\_rp}$ & $\str{script\_rp}$  \\\hline
      $\mi{script\_idp}$ &  $\str{script\_idp}$  \\\hline
    \end{tabular}
    
    \caption{List of scripts in $\scriptset$ and their respective string
      representations.}
    \label{fig:scripts-in-w}
  \end{figure}
  
  This outlines $\uppressowebsystem$. We will define the DY processes in 
  $\uppressowebsystem$ and their addresses, domain names, and secrets in more detail. 
  The scripts are defined in detail in Appendix~\ref{app:uppresso-scripts}
  
  \subsection{Addresses and Domain Names}
  The set $\addresses$ contains for every web attacker in $\fAP{Web}$, 
  every network attacker in $\fAP{Net}$, 
  every relying party in $\fAP{RP}$, 
  the only one identity provider in $\fAP{IDP}$, 
  and every browser in $\fAP{B}$ a finite set of addresses each. 
  By $\mapAddresstoAP$ we denote the corresponding
  assignment from a process to its address. 
  The set $\dns$ contains a finite set of domains for 
  every relying party in $\fAP{RP}$, 
  the only one identity provider in $\fAP{IDP}$, 
  every web attacker in $\fAP{Web}$, and 
  every network attacker in $\fAP{Net}$. 
  Browsers (in $\fAP{B})$ do not have a domain.
  
  By $\mapAddresstoAP$ and $\mapDomain$ we denote the assignments from
  atomic processes to sets of $\addresses$ and $\dns$, respectively.
  
  %需不需要为椭圆曲线的点单独设立一个集合?
  \subsection{Keys and Secrets} 
  The set $\nonces$ of nonces is partitioned into four sets, 
  an infinite sequence $N$, 
  an infinite set $K_\text{SSL}$, 
  an infinite set $K_\text{sign}$, 
  an infinite set $K_\text{id}$, 
  an infinite set $K_\text{point}$, 
  and a finite set $\RPSecrets$. 
  We thus have
  \begin{align*}
  \def\hereMaxHeightPhantom{\vphantom{K_{\text{p}}^\bidsystem}}
  \nonces = 
  \underbrace{N\hereMaxHeightPhantom}_{\text{infinite sequence}} 
  \dot\cup \underbrace{K_{\text{SSL}}\hereMaxHeightPhantom}_{\text{finite}} 
  \dot\cup \underbrace{K_{\text{sign}}\hereMaxHeightPhantom}_{\text{finite}}
  \dot\cup \underbrace{K_{\text{point}}\hereMaxHeightPhantom}_{\text{finite}}  
  \dot\cup \underbrace{\RPSecrets\hereMaxHeightPhantom}_{\text{finite}}\ .
  \end{align*}
  The set $N$ contains the nonces that are available for each DY process
  in $\bidsystem$ (it can be used to create a run of $\bidsystem$). 
  
  The set $K_\text{SSL}$ contains the keys that will be used for SSL
  encryption. Let $\mapTLSKey\colon \dns \to K_\text{SSL}$ be an injective
  mapping that assigns a (different) private key to every domain.
  
  The set $K_\text{sign}$ contains the keys that will be used by IdPs
  for signing IDToken. Let $\mapSignKey\colon \fAP{IdPs} \to K_\text{sign}$
  be an injective mapping that assigns a (different) private key to every identity
  provider.
  
  The set $K_\text{point}$ contains all valid points on the curve. 
  The set $K_\text{point}$ will be used to generate identities of $\fAP{B}$ and $\fAP{RP}$.
  
  The set $\RPSecrets$ is the set of passwords (secrets) 
  the browsers share with the identity providers. 
  
  %用户的id表示:<id, username, idp-domain, <acct1, rp1-domain>, <acct2, rp2-domain>, <acct3, rp3-domain>, ...>
  %RP的id表示:,<id, rp-commonname, idp-domain>
  
  %def1.用户的id表示:<id, username, idp-domain>
  %def2.用户的account:<<acct1, rp1-domain>, <acct2, rp2-domain>, <acct3, rp3-domain>, ...>
  %def3.RP的id表示,<id, rp-commonname, idp-domain>
  
  %直接使用password
  \subsection{Identities}\label{app:uppresso-identities}
  There are many different types of identities in UPPRESSO. 
  The first is browsers' identities at the IdP. Browsers share
  a $username\in\mathbb{S}$ with IdP to identify an user 
  uniquely, and not like SPRESSO, UPPRESSO doesn't need a 
  domain to identify the IdP.
  
  By $\NToS:\mathbb{S} \to \RPSecrets$ we denote the bijective 
  mapping that assigns secrets to all usernames. 
  
  Let $\mapPLItoOwner: \RPSecrets \to \fAP{B}$ denote the 
  mapping that assigns to each secret a browser that 
  \emph{owns} this secret. Now, we define the mapping 
  $\mapIDtoOwner: \mathbb{S} \to \fAP{B}$, $username \mapsto
  \mapPLItoOwner(\NToS(username))$, which assigns to each 
  identity the browser that owns this identity (we say that the 
  identity belongs to the browser).
  
  Besides, the browsers' identities also have IDs 
  $\mi{ID_u}:=u\in[1,n)$ which is only known to the IdP. By $\NToID:\mathbb{S} \to N$ we denote 
  the bijective mapping that assigns IDs to all usernames. 
  
  The second type of identities is relying parties' identities 
  at the IdP, which are IDs 
  $\mi{ID_{rp}}:=[r]G \in K_{\text{point}}$ in which $r\in[1,n)$ 
  is unknown to any party.
  
  The third type of identities is browsers' identities at the
  Relying Parties which is called $\mi{Acct} := [\mi{ID_u}]
  \mi{ID_{rp}} = [ur]G \in K_{\text{point}}$. 
  
  \subsection{Tags, Identity Tokens and Service Tokens}\label{app:identity-assertions}
  
  \begin{definition}\label{def:tag}
    A \emph{tag} is a term of the form $\mi{PID_{rp}}=[t]ID_{rp}=[tr]G$ for a nonce 
    (here used as a asymmetric key) $t$.
  \end{definition}
  \begin{definition}
    An \emph{identity Tokens (IDToken)} is a term of the form 
    $\an{PID_{rp}, PID_u, ver}$ for a tag $PID_{rp}$, an encrypted identity 
    $PID_u=[u]PID_{rp}=[utr]G$ and a signature $ver=\sig{\an{PID_{rp},PID_u}}{k}$ 
    for a nonce $k\in K_{\text{sign}}$.
  \end{definition}
  %使用<n,i>来定义 service token;Acct or pidu?
  \begin{definition}
    A \emph{service token} is a term of the form 
    $\myangle{\mi{nonce}, \mi{Acct}}$ with 
    $\mi{Acct} = [t^{-1}]PID_u=[t^{-1}][utr]G=[ur]G$ 
    for a nonce $t$.
  \end{definition}
  
  %是否足够描述semi-honest的状态?
  %\subsection{Curiosity}
  %同样,也会有人给idp发送命令,让他进入curiosity状态
  %进入curiosity状态之后,就可以开始把历史上所有的信息,开始用来推导各种事情。
  %或者IdP始终honest-but-curious的状态,在安全性证明时不考虑curious的状态。
  
  \subsection{Corruption}
  RPs can become corrupted: If they receive the message
  $\corrupt$, they start collecting all incoming messages in their state
  and (upon triggering) send out all messages that are derivable from
  their state and collected input messages, just like the attacker
  process. We say that an RP is \emph{honest} if the according
  part of their state ($s.\str{corrupt}$) is $\bot$, and that they are
  corrupted otherwise. IdP is always honest-but-curious and never corrupted.
  
  We are now ready to define the processes in $\websystem$ as well as
  the scripts in $\scriptset$ in more detail. 
  
  \subsection{Processes in $\bidsystem$ (Overview)}
  
  We first provide an overview of the processes in $\bidsystem$. All
  processes in $\websystem$ contain in their initial states all public
  keys and the private keys of their respective domains (if any). We
  define $I^p=\mapAddresstoAP(p)$ for all $p\in \mathsf{Hon} \cup \mathsf{Web}$.
  
  %两个attacker是否足够描述corrupt的程度?
  \subsubsection{Web Attackers.}  Each $\mi{wa} \in \mathsf{Web}$  is a
  web attacker who uses only his own addresses for sending and listening. 
  
  \subsubsection{Network Attackers.}  Each $\mi{na} \in \mathsf{Net}$  is a
  network attacker who uses all addresses for sending and listening. 
  
  \subsubsection{Browsers.} Each $b \in \fAP{B}$ is a web browser. 
  The initial state contains all secrets owned by $b$, stored under the origin of the
  respective IdP. See Appendix~\ref{app:browsers-uppresso} for details.
  
  \subsubsection{Relying Parties.} 
  A relying party $r \in \fAP{RP}$ is a web server. RP knows four distinct paths: 
  $\mathtt{/script}$, where it serves $\str{script\_rp}$ to open a new window 
  and facilitate the login flow.
  $\mathtt{/loginSSO}$, where it only accepts GET requests and sends 
  redirect response to redirect the browser to the IdP to download $\str{script\_IdP}$
  $\mathtt{/startNegotiation}$, where it only accepts POST requests logically sent 
  from $\str{script\_rp}$ using postMessge and checks whether the data $t\in K_\text{id}$.
  If the request valid, it send back a certificate.
  $\mathtt{/uploadToken}$ running in the browser. It checks the ID token and, 
  if the data is deemed ``valid'', it issues a service token (again, for details, see below). 
  Intuitively, a client having such a token can use the service of the RP 
  (for a specific identity record along with the token). 
  Just like IdPs, RPs can become corrupted.
  
  \subsubsection{Identity Providers.} Each IdP is a web server, 
  users can authenticate to the IdP with their credentials. 
  IdP tracks the state of the users with sessions. 
  Authenticated users can receive IDTokens from the IdP. 
  %When receiving a special message ($\corrupt$) IdPs can become corrupted. 
  %Similar to the definition of corruption for the browser,
  %IdPs then start sending out all messages that are derivable from their state.
  
  %\subsubsection{DNS.} Each $\mi{dns} \in \fAP{DNS}$ is a DNS server.
  %Their state contains the allocation of domain names to IP addresses.
  
  \subsection{TLS Key Mapping}\label{app:common-data-structures}
  Before we define the atomic DY processes in more detail, we first
  define the common data structure that holds the mapping of domain
  names to public TLS keys: For an atomic DY process $p$ we define
  \[\mi{tlskeys}^p = \an{\left\{\an{d, \mapTLSKey(d)} \mid d \in \mapDomain(p)\right\}}.\]
  %ssl改成tls
  
  \subsection{Web Attackers}\label{app:webattackers-uppresso}
  Each $\mi{wa} \in \fAP{Web}$ is a web attacker. 
  The initial state of each $\mi{wa}$ is 
  $s_0^\mi{wa} = \an{\mi{attdoms}, \mi{tlskeys}, \mi{signkeys}}$, 
  where $\mi{attdoms}$ is a sequence of all domains along with 
  the corresponding private keys owned by $\mi{wa}$, 
  $\mi{tlskeys}$ is a sequence of all domains and 
  the corresponding public keys, and 
  $\mi{signkeys}$ contains the public signing key for the IdP. 
  %All other parties use the attacker as a DNS server.
  
  \subsection{Network Attackers}\label{app:networkattackers-uppresso}
  As mentioned, each network attacker $\mi{na}$ is modeled to 
  be a network attacker. We allow it to listen to/spoof all 
  available IP addresses, and hence, define 
  $I^\mi{na} = \addresses$. 
  The initial state is $s_0^\mi{na} = 
  \an{\mi{attdoms}, \mi{tlskeys}, \mi{signkeys}}$, 
  where $\mi{attdoms}$ is a sequence of all domains along with 
  the corresponding private keys owned by the attacker 
  $\mi{na}$, $\mi{tlskeys}$ is a sequence of all domains 
  and the corresponding public keys, and 
  $\mi{signkeys}$ contains the public signing key for the IdP. 
  
  \subsection{Browsers}\label{app:browsers-uppresso} 
  Each $b \in \fAP{B}$ is a web browser with 
  $I^b := \mapAddresstoAP(b)$ being its addresses.
  
  To define the inital state, first let $U^b := 
  \mapIDtoOwner^{-1}(b)$ be the set of all usernames of $b$, 
  %$\mi{ID}^{b,d} := \{i \mid \exists\, x,n:\ i = \an{id, n, d} \in \mi{ID}^b\}$ 
  %be the set of IDs of $b$ for a domain $d$, and 
  %$\mi{SecretDomains}^b := \{d \mid \mi{ID}^{b,d} \neq \emptyset \}$ 
  %be the set of all domains that $b$ owns identities for.
  Then, the initial state $s_0^b$ is defined as follows: the key mapping
  maps every domain to its public (tls) key, according to the mapping
  $\mapTLSKey$; the DNS address is $\mapAddresstoAP(p)$ with $p \in \bidsystem$;
  $\mi{ids}$ is $\an{U^b}$; $\mi{sts}$ is empty.
  
  \subsection{Relying Parties} \label{app:relying-parties-uppresso}
  
  A relying party $r \in \fAP{RP}$ is a web server modeled as an atomic
  DY process $(I^r, Z^r, R^r, s^r_0)$ with the addresses $I^r :=
  \mapAddresstoAP(r)$. Its initial state $s^r_0$ contains its domains,
  the private keys associated with its domains.
  The full state additionally contains the sets of service tokens and login 
  session identifiers the RP has issued. RP only accepts HTTPS requests.
  
  RP manages two kinds of sessions: The \emph{login sessions}, which are
  only used during the login phase of a user, and the \emph{service
    sessions} (we call the session identifier of a service session a
  \emph{service token}). Service sessions allow a user to use RP's
  services. The ultimate goal of a login flow is to establish such a
  service session.
  
  In a typical flow with one client, $r$ will first receive an HTTP GET
  request for the path $\str{/script}$. In this case, $r$ returns the script
  $\str{script\_rp}$ (see below).
  
  After the user loaded the script in his browser, $r$ will receive an 
  HTTP GET request for the path $\str{/loginSSO}$ sent from the new window opened
  by $\str{script\_rp}$. In this request, $r$ will send back a redirect response  
  for downloading $\str{script\_IdP}$ from IdP.
  
  When the IdP document in the browser generates a number $t$,
  $r$ will receive the third request for the path $\str{/startNegotiate}$.
  $r$ will verify $t$ and if valid, $r$ will create the corresponding 
  login session with a $\mi{loginSessionToken}$ as the identifier. After that,
  $r$ will use $t$ to generate $PID_{rp}$ and bind it with the login session.
  After all these are down, $r$ send its certificate signed by the specific IdP that browser selected.
  
  Finally, $r$ receives a last request in the login flow. This POST request 
  contains the IDToken. To conclude the login, $r$ looks up the user's login session, 
  compare the $\mi{IDToken}.\str{PID_{rp}}$ with the $\mi{PID_{rp}}$ in the login session, and checks 
  whether $\mi{IDToken}.\str{PID_{ver}}$ is a correct signature. If successful, $r$ calculates the 
  service token and returns it, which is also stored in the state of $r$.
  
  If $r$ receives a corrupt message, it becomes corrupt and acts like
  the attacker from then on.
  
  We now provide the formal definition of $r$ as an atomic DY process
  $(I^r, Z^r, R^r, s^r_0)$. As mentioned, we define $I^r =
  \mapAddresstoAP(r)$. Next, we define the set $Z^r$ of states of
  $r$ and the initial state $s^r_0$ of $r$.
  
  %和tag是否一致?
  \begin{definition}
    A \emph{login session record} is a term of the form 
    $\an{\mi{t}, \mi{PID_{rp}}}$ with 
    $\mi{t}\in N, \mi{PID_{rp}}\in K_{\text{point}}$.
  \end{definition}
  
  \begin{sloppypar}
    \begin{definition}\label{def:relying-parties}
      A \emph{state $s\in Z^r$ of an RP} is a term of the form
      $\langle\mi{keyMapping}$, 
      $\mi{tlskeys}$, 
      $\mi{loginSessions}$, 
      $\mi{serviceTokens}$, 
      $\mi{corrupt}$, 
      $\mi{IdPConfig}$, 
      $\mi{rp}\rangle$ where 
      $\mi{keyMapping} \in \dict{\mathbb{S}}{\nonces}$,
      $\mi{tlskeys}=\mi{tlskeys}^r$,
      $\mi{serviceTokens} \in \dict{\nonces}{K_{\text{point}}}$,
      $\mi{loginSessions} \in \dict{\nonces}{\terms}$ 
      is a dictionary of login session records,
      $\mi{corrupt} \in \terms$,
      $\mi{IdPConfig} \in \terms$ 
      is the configuration retrieved from IdP server,
      $\mi{rp} \in K_{\text{point}}$ is the identity of the RP, 
      see details in Appendix~\ref{app:uppresso-identities}.
  
      The \emph{initial state $s^r_0$ of $r$} is a state of 
      $r$ with $s^r_0.\str{serviceTokens} = 
      s^r_0.\str{loginSessions} = \an{}$,
      $s^r_0.\str{corrupt} = \bot$, 
      $s^r_0.\str{keyMapping}$ 
      is the same as the keymapping for browsers above,
      $s^r_0.\str{IdPConfig} = \an{\mi{pubkey},\mi{scriptUrl},\mi{Cert_{rp}}}$ and
      $s^r_0.\str{rp} = [r]G$ with $r\in N$.
    \end{definition}
  \end{sloppypar}
  
  We now specify the relation $R^r$. We describe this relation by a non-deterministic algorithm. 
  
  \captionof{algorithm}{\label{alg:rp} Relation of a Relying Party $R^r$}
  \begin{algorithmic}[1]
    \REQUIRE \myss{\myangle{a, b, m}, s}
    \mystate{\myss{s':=s}}
    \myif{s'.\str{corrupt} \not\equiv \bot \vee m \equiv \corrupt}
      \mystate{\myss{s'.\str{corrupt} := \an{\an{a, f, m}, s'.\str{corrupt}}}}
      \mystate{\myss{m' := d_{V}(s')}\label{line:usage-of-signkey-corrupt-uppresso}}
      \mystate{\myss{a' := \addresses}}
      \mystop{a',a,m'}
    \ENDIF
    \mystate{\myss{m_{dec},k,k',\mi{inDomain}} \textbf{such that} \breakalgohook{0}
      \myss{\an{m_{\text{dec}}, k} \equiv \dec{m}{k'} \wedge \an{inDomain,k'} \in s'.\str{sslkeys}}\breakalgohook{0}
      \textbf{if possible; otherwise stop} \myss{\myangle{}, s'}}
    \mystate{\myss{n, method, path, parameters, headers, body} \textbf{such that} \breakalgohook{0}
      \myss{\myangle{\mathtt{HTTPReq},n,method,path,parameters,headers,body} \equiv m_{dec}}\breakalgohook{0}
      \textbf{if possible; otherwise stop} \myss{\myangle{}, s'}}
    \myif{path \equiv /script}\label{line:rp-script}
      \mystate{\myss{m':=\encs{\myangle{\mathtt{HTTPResp},n,200, \myangle{}, \str{script\_rp}}}{k}}}
      \mystop{b, a, m'}
    \myelse{path \equiv /loginSSO}\label{line:rp-loginSSO}
      \mystate{\myss{m':=\encs{\myangle{\mathtt{HTTPResp},n,302,\myangle{\myangle{\mathtt{Location}, s'.\str{IdPConfig}.\mi{scriptUrl}}}, \myangle{}}}{k}}}
      \mystop{b, a, m'}
    \myelse{path \equiv /startNegotiation}\label{line:rp-startNegotiation}
      \mystate{\myss{\mi{loginSessionToken} := \nu_1}}
     %\mystate{\myss{cookie := headers[Cookie]}}
     %\mystate{\myss{session := s'.SessionList[cookie]}}
      \mystate{\myss{\mi{t} := body[t]}}\label{line:gen-t}
     %\mystate{\myss{t^{-1}:= \mathtt{Inverse}(t)}}
      \mystate{\myss{\mi{ID_{rp}} := s'.\str{rp}}}
      \mystate{\myss{\mi{PID_{rp}} := [\mi{t}]\mi{ID_{rp}}}}
      \mystate{\myss{\mi{state} := \str{expectToken}}}
      \mystate{\myss{\mi{Cert_{rp}} := s'.\str{IdPConfig}.\mi{Cert_{RP}}}}
      \mystate{\myss{s'.\str{loginSessions}[\mi{loginSessionToken}] := \an{\mi{t}, \mi{PID_{rp}}, \mi{state}}}}
     %\mystate{\myss{session[t] := t}}
     %\mystate{\myss{session[t^{-1}] := t^{-1}}}
     %\mystate{\myss{session[state] := expectToken}}
      \mystate{\myss{\mi{setCookie} := \myangle{\cSetCookie, \myangle{\myangle{\str{sessionid}, \mi{loginSessionToken}, \True, \True, \True}}}}}
      \mystate{\myss{m' := \encs{\myangle{\mathtt{HTTPResp}, n, 200, \myangle{\mi{setCookie}}, \myangle{\mathtt{Cert_{RP}}, \mi{Cert_{rp}}}}}{k}}}
      \mystop{b, a, m'}
    \myelse{path \equiv /uploadToken}\label{line:rp-uploadToken}
      \mystate{\myss{\mi{cookie} := headers[\str{Cookie}]}}
      \myif{\mi{headers}[\str{Origin}] \not\equiv \an{\mi{inDomain}, \https} \vee \mi{cookie}[\str{sessionid}] \equiv \myangle{}}
        \mystop{}\label{line:alg-rp-stop1}
      \ENDIF
      \mystate{\myss{\mi{loginSessions} := s'.\str{loginSessions}[\mi{cookie}[\str{sessionid}]]}}
      %\mystate{\myss{cookie := headers[Cookie]}}
      %\myif{\mi{loginSessions} \equiv \an{}}
      %  \mystop{}
      %\ENDIF
      %\mystate{\myss{session := s'.SessionList[cookie]}}
      %\myif{session[state] \not\equiv expectToken}
      \myif{\mi{loginSessions}.\str{state} \not\equiv expectToken}
        \mystate{\myss{m' := \encs{\myangle{\mathtt{HTTPResp}, n, 200, \myangle{}, \mathtt{Fail}}}{k}}}
        \mystop{b, a, m'}\label{line:alg-rp-stop2}
      \ENDIF
      \mystate{\myss{s'.\str{loginSessions} := s'.\str{loginSessions} - body[\mi{loginSessionToken}]}}
      \mystate{\myss{\mi{IDToken} := body[\str{IDToken}]}}
      \myif{\mi{IDToken}.\str{PID_{rp}} \not\equiv \mi{loginSessions}.\str{PID_{rp}}}
        \mystate{\myss{m' := \encs{\myangle{\mathtt{HTTPResp}, n, 200, \myangle{}, \mathtt{Fail}}}{k}}}
        \mystop{b, a, m'}\label{line:alg-rp-stop3}
      \ENDIF
      \myif{\checksigThree{\mi{IDToken}.\str{ver}}{\an{\mi{IDToken}.\str{PID_{rp}}, \mi{IDToken}.\str{PID_{u}}}}{s'.\str{IdPConfig}.\mi{pubkey}} \equiv \bot}
        \mystate{\myss{m' := \encs{\myangle{\mathtt{HTTPResp}, n, 200, \myangle{}, \mathtt{Fail}}}{k}}}
        \mystop{b, a, m'}\label{line:alg-rp-stop4}
      \ENDIF
     %\mystate{\myss{Time := \mathtt{CurrentTime}()}}
     %\mystate{\myss{PIDValidity := session[PIDValidity]}}
     %\mystate{\myss{Content := Token.Content}}
     %\myif{Time>Content.Validity}
       %\mystate{\myss{m' := \myangle{\mathtt{HTTPResp}, n, 200, \myangle{}, \mathtt{Fail}}}}
       %\mystop{b, a, m'}
     %\ENDIF
      \mystate{\myss{\mi{PID_u} := \mi{IDToken}.\str{PID_{u}}}}
      \mystate{\myss{\mi{Acct} := [\mi{loginSessions}.\str{t}]\mi{PID_u}}}\label{line:gen-acct}
     %\mystate{\myss{Acct := \mathtt{Multiply}(PID_U, t^{-1})}}
     %\myif{Acct \not\in \mathtt{ListOfUser}()}
       %\mystate{\myss{\mathtt{AddUser}(Acct)}}
     %\ENDIF
     %\mystate{\myss{session[user] := Acct}}
      \mystate{\myss{\mi{nonce} := \nu_2}}
      \mystate{\myss{s'.\str{serviceTokens} := s'.\str{serviceTokens} + ^{\myangle{}}\myangle{\mi{nonce}, \mi{Acct}}}}\label{line:add-service-token}
     %\mystate{\myss{s'.serviceTokens := s'.serviceTokens + ^{\myangle{}}\myangle{IDToken, Acct}}}
      \mystate{\myss{\mi{setCookie} := \myangle{\cSetCookie, \myangle{\myangle{\str{sessionid}, \mi{nonce}, \True, \True, \True}}}}}
      \mystate{\myss{m' := \encs{\myangle{\mathtt{HTTPResp}, n, 200, \myangle{\mi{setCookie}}, \mathtt{LoginSuccess}}}{k}}}
      \mystop{b, a, m'}
    \ENDIF
    \mystop{}
  \end{algorithmic}\setlength{\parindent}{1em}
  
  \subsection{Identity Providers} \label{app:idps}
  
  The identity provider $\mathsf{IdP}$ is a web server 
  modeled as an atomic process $(I, Z, R, s_0)$ with 
  the addresses $I := \mapAddresstoAP(\mathsf{IdP})$. 
  Its initial state $s_0$ contains a list of its 
  domains and (private) TLS keys, 
  a list of users and identites, and a private key 
  for signing IDTokens. Besides this, 
  the full state of $\mathsf{IdP}$ further contains a list 
  of used nonces, and information about active sessions.
  
  $\mathsf{IdP}$ react to four types of requests:
  
  First, they provide the $\str{script\_idp}$, where a $t$ 
  will be chosen and following requests to $\mathsf{IdP}$ 
  will be sent. $\mathsf{IdP}$ will transfer the data
  to RP by the communicating between two scripts $\str{script\_idp}$ 
  and $\str{script\_rp}$ using $\tPostMessage$.
  
  Second, they provide $\mi{IDToken}$ when receiving $\mi{PID_{rp}}$ and this 
  $\mi{PID_{rp}}$ has already first. If not, IdPs will redirect to the login dialog.
  
  After the user enter his username and password(secret) in the login dialog, a login
  request will send to $\str{/authentication}$. IdPs will check the parameters and 
  set the login session.
  
  The last type of requests IdPs react to is authorize requests with $\mi{PID_{rp}}$ and attribute
  scopes as parameters. After receving consent from browsers, IdPs will calculate 
  $\mi{PID_{u}}$ and construct $\mi{IDToken}$.
  
  \subsubsection{Formal description.} In the following, we 
  will first define the (initial) state of IdP formally and 
  afterwards present the definition of the relation $R$.
  
  To define the initial state, we will need a term that 
  represents the ``user database'' of the IdP. We will 
  call this term $\mi{userset}$. This database defines, 
  which secret and $\mi{ID_u}$ is valid for which identity. 
  It is encoded as a mapping of username to secrets and $\mi{ID_u}$. 
  For example, if the secret $\mi{secret}_1$ and $\mi{ID_{u_1}}$ is 
  valid for the username $u_1$ and the secret $\mi{secret}_2$ 
  and $\mi{ID_{u_2}}$ is valid for the identity $u_2$, the 
  $\mi{userset}^i$ looks as follows:
  \begin{align*}
  \mi{userset} = [u_1{:}\myangle{\mi{ID_{u_1}}, \mi{secret}_1}, 
    u_2{:}\myangle{\mi{ID_{u_2}}, \mi{secret}_2}]
  \end{align*}
  
  We define $\mi{userset}$ as $\mi{userset} = \an{\{\an{\mi{username}, \myangle{\mi{id}=\NToID(\mi{username}), \mi{secret}=\NToS(\mi{username})}}\, |\, username \in \mathbb{S}\}}$.
  
  \begin{definition}\label{def:initial-state-idp}
    A \emph{state $s\in Z$ of the IdP} is a term of the form
    $\langle\mi{tlskeys}$, $\mi{users}$, $\mi{signkey}$,
    $\mi{sessions}$, $\mi{corrupt}\rangle$ where 
    $\mi{tlskeys} = \mi{tlskeys} $, 
    $\mi{users} = \mi{userset}$, 
    $\mi{signkey} \in \nonces$ 
    (the key used by the IdP to sign IDTokens),
    $\mi{sessions}\in\dict{\nonces}{\terms}$, $\mi{corrupt} \in \terms$.
  
    An \emph{initial state $s_0$ of IdP} is a state of the form 
    $\an{\mi{tlskeys}, \mi{userset}, \mapSignKey(\mathsf{IdP}), \an{}, \bot}$.
  \end{definition}
  
  The relation $R$ that defines the behavior of the IdP is defined as follows:
  
  %和代码基本保持一致
  \captionof{algorithm}{\label{alg:idp} Relation of IdP $R$}
  \begin{algorithmic}[1]
    \REQUIRE \myss{\myangle{a, b, m}, s}
    \mystate{\myss{s':=s}}
    \myif{s'.\str{corrupt} \not\equiv \bot \vee m \equiv \corrupt}
      \mystate{\myss{s'.\str{corrupt} := \an{\an{a, f, m}, s'.\str{corrupt}}}}
      \mystate{\myss{m' := d_{V}(s')}\label{line:usage-of-signkey-corrupt-uppresso}}
      \mystate{\myss{a' := \addresses}}
      \mystop{a', a, m'}
    \ENDIF
    \mystate{\myss{m_{dec},k,k',\mi{inDomain}} \textbf{such that} \breakalgohook{0}
      \myss{\an{m_{\text{dec}}, k} \equiv \dec{m}{k'} \wedge \an{inDomain,k'} \in s'.\str{sslkeys}}\breakalgohook{0}
      \textbf{if possible; otherwise stop} \myss{\myangle{}, s'}}
    \mystate{\myss{n, method, path, parameters, headers, body} \textbf{such that} \breakalgohook{0}
      \myss{\myangle{\mathtt{HTTPReq},n,method,path,parameters,headers,body} \equiv m_{dec}}\breakalgohook{0}
      \textbf{if possible; otherwise stop} \myss{\myangle{}, s'}}
    \myif{path \equiv /script}\label{line:idp-script}
      \mystate{\myss{m':=\encs{\myangle{\mathtt{HTTPResp},n,200, \myangle{}, \str{script\_idp}}}{k}}}
      \mystop{b, a, m'}
    \myelse{path \equiv /authentication}\label{line:idp-authentication}
      \mystate{\myss{\mi{username} := \mi{body}[\str{username}]}}
      \mystate{\myss{\mi{password} := \mi{body}[\str{password}]}}
      \myif{\mi{password} \not\equiv s'.\str{userset}[\mi{username}].\mi{secret}}\label{line:check-username-password}
        \mystate{\myss{m':=\encs{\myangle{\mathtt{HTTPResp},n,200,\myangle{},\mathtt{LoginFailure}}}{k}}}
        \mystop{b, a, m'}
      \ENDIF
      \mystate{\myss{\mi{sessionid} := \nu_3}}
      \mystate{\myss{s'.\str{sessions}[\mi{sessionid}] := \mi{username}}}
      \mystate{\myss{\mi{setCookie} := \myangle{\cSetCookie, \myangle{\myangle{\str{sessionid}, \mi{sessionid}, \True, \True, \True}}}}}
      \mystate{\myss{m' :=\myangle{\mathtt{HTTPResp},n,200,\myangle{\mi{setCookie}},\mathtt{LoginSucess}}}}
      \mystop{b, a, m'}
    \myelse{path \equiv /reqToken}\label{line:idp-reqToken}
      \mystate{\myss{\mi{cookie} := headers[\str{Cookie}]}}
      \myif{\mi{cookie}[\str{sessionid}] \equiv \myangle{}}\label{line:check-sessionid}
        \mystate{\myss{m' := \encs{\myangle{\mathtt{HTTPResp},n,200,\myangle{},\mathtt{Unauthenticated}}}{k}}}
        \mystop{b, a, m'}
      \ENDIF
      \mystate{\myss{\mi{sessionid} := \mi{cookie}[\str{sessionid}]}}
      \mystate{\myss{\mi{PID_{rp}} := \mi{parameters}[\str{PID_{rp}}]}}
      \myif{s'.\str{sessions}[\mi{sessionid}].\mi{IDToken}[\mi{PID_{rp}}] \equiv \myangle{}}\label{line:check-session-pidrp}
        \mystate{\myss{m' := \encs{\myangle{\mathtt{HTTPResp},n,200,\myangle{},\mathtt{Unauthorized}}}{k}}}
        \mystop{b, a, m'}
      \ENDIF
      \mystate{\myss{\mi{IDToken} := s'.\str{sessions}[\mi{sessionid}].\mi{IDToken}[\mi{PID_{rp}}]}}
      \mystate{\myss{m' := \encs{\myangle{\mathtt{HTTPResp},n,200,\myangle{}, \mi{IDToken}}}{k}}}
      \mystop{b, a, m'}
    \myelse{path \equiv /authorize}\label{line:idp-authorize}
      \mystate{\myss{\mi{cookie} := headers[\str{Cookie}]}}
      \myif{\mi{cookie}[\str{sessionid}] \equiv \myangle{}}\label{line:uppresso-idp-check-login-state}
        \mystate{\myss{m' := \encs{\myangle{\mathtt{HTTPResp},n,200,\myangle{},\mathtt{Unauthenticated}}}{k}}}
        \mystop{b, a, m'}
      \ENDIF
      \mystate{\myss{\mi{sessionid} := \mi{cookie}[\str{sessionid}]}}
      \mystate{\myss{\mi{PID_{RP}} := \mi{parameters}[\str{PID_{RP}}]}}
      \myif{\mathtt{IsValid}(PID_{RP}) \equiv \bot}
        \mystate{\myss{m' := \encs{\myangle{\mathtt{HTTPResp}, n, 200, \myangle{}, \mathtt{Fail}}}{k}}}
        \mystop{b, a, m'}
      \ENDIF
      \myif{\mathtt{IsInScope}(uid, \mi{body}[\str{Attr}]) \equiv \bot}
        \mystate{\myss{m' := \encs{\myangle{\mathtt{HTTPResp}, n, 200, \myangle{}, \mathtt{Fail}}}{k}}}
        \mystop{b, a, m'}
      \ENDIF
      
      \mystate{\myss{\mi{username} := s'.\str{sessions}[\mi{sessionid}].username}}
      \mystate{\myss{\mi{ID_u} := s'.\str{userset}[\mi{username}].\mi{id}}}\label{line:get-idu}
      \mystate{\myss{\mi{PID_u} := [\mi{ID_u}]\mi{PID_{rp}}}}\label{line:uppresso-idp-set-pidu}
      %\mystate{\myss{Validity := \mathtt{CurrentTime} ()+ s'.Validity}}
      %\mystate{\myss{Content := \myangle{PID_{RP}, PID_U, s'.Issuer, Validity}}}
      \mystate{\myss{\mi{content} := \myangle{PID_{rp}, PID_u}}}
      \mystate{\myss{\mi{ver} := \sig{\mi{content}}{s'.\str{signkey}}}}\label{line:sign-token}
      \mystate{\myss{\mi{IDToken} := \myangle{\mi{content}, \mi{ver}}}}
      \mystate{\myss{s'.\str{sessions}[\mi{IDToken}]:=s'.\str{sessions}[\mi{IDTokens}]+^{\myangle{}}\myangle{PID_{rp}, IDToken}}}
      \mystate{\myss{m':=\encs{\myangle{\mathtt{HTTPResp}, n, 200, \myangle{}, \mi{IDToken}}}{k}}}
      \mystop{b, a, m'}
    \ENDIF
    \mystop{}
  \end{algorithmic}\setlength{\parindent}{1em}
  
  \subsection{UPPRESSO Scripts}\label{app:uppresso-scripts}
  As already mentioned in Appendix~\ref{app:outlineuppressomodel}, the set $\scriptset$ 
  of the web system $\uppressowebsystem=(\bidsystem, \scriptset, \mathsf{script}, E^0)$ 
  consists of the scripts $\Rasp$, $\mi{script\_rp}$, $\mi{script\_idp}$, and with their 
  string representations being $\str{att\_script}$, $\str{script\_rp}$, $\str{script\_idp}$, 
  and (defined by $\mathsf{script}$). 
  
  In what follows, the scripts $\mi{script\_rp}$ and $\mi{script\_idp}$ are
  defined formally.
  
  \subsubsection{Relying Party Page (script\_rp).}\label{app:uppresso-script-rp}
  As defined in SPRESSO, a script is a relation that takes a termas input and outputs 
  a new term. The input term is provided by the browser. It contains the current 
  internal state of the script (which we call \emph{scriptstate} in what follows) and
  additional information containing all browser state information the
  script has access to, such as the input the script has obtained so far
  via \xhrs and \pms, information about windows, etc. The browser
  expects the output term to contain, among other information, the new internal \emph{scriptstate}.
  
  We first describe the structure of the internal scriptstate
  of the script $\mi{script\_rp}$.
  
  \begin{definition} \label{def:scriptstaterp} 
  A \emph{scriptstate $s$ of $\mi{script\_rp}$} is a term of the form $\langle 
  \mi{phase}$, 
  %$\mi{loginSessionToken}$, 
  $\mi{refXHR}\rangle$, 
  where $phase \in \mathbb{S}$, 
  %$\mi{loginSessionToken}$,
  $\mi{refXHR}\in \nonces \cup \{\bot\}$. 
  
  The \emph{initial scriptstate $\mi{initState_{rp}}$} of $\mi{script\_rp}$ is 
  $\an{\str{start},
  %\bot,
  \bot}$.
  \end{definition}
  
  We now specify the relation $\mi{script\_rp}$ formally. We describe this relation
  by a non-deterministic algorithm.
  
  \captionof{algorithm}{\label{alg:uppresso-script-rp} Relation of $\mi{script\_rp}$}
  \begin{algorithmic}[1]
  \REQUIRE \myss{\langle\mi{tree},\mi{docnonce},\mi{scriptstate},\mi{scriptinputs},\mi{cookies},\mi{localStorage},\mi{sessionStorage},}
  \breakalgohook{-1}\myss{\mi{ids},\mi{secret}\rangle}
  \mystate{\myss{ s' := \mi{scriptstate}}}
  \mystate{\myss{\mi{command} := \myangle{}}}
  \mystate{\myss{\mi{origin} := \mathsf{GETORIGIN}(\mi{tree},\mi{docnonce})}}
  \mystate{\myss{\mi{RPDomain} := \mi{origin}.\str{host}}}
  \SWITCH{\myss{s'.\str{phase}}}
  \CASE{\myss{\str{start}}}
    \mystate{\myss{\mi{url} := \an{\tUrl, \https, \mi{RPDomain}, \str{/loginSSO}, \myangle{}}}}
    \mystate{\myss{\mi{command} := \an{\tHref,\mi{url},\wBlank,\an{}}}}
    \mystate{\myss{s'.\str{phase} := \str{expectt}}}
  \ENDCASE
  \CASE{\myss{\str{expectt}}}
    \mystate{\myss{\mi{pattern} := \myangle{\tPostMessage, target, *, \myangle{\str{t}, *}}}}
    \mystate{\myss{\mi{input} := \textsf{CHOOSEINPUT}(\mi{scriptinputs}, \mi{pattern})}}
    \myif{\mi{input} \not\equiv \bot}
      \mystate{\myss{t := \pi_2(\pi_4(\mi{input}))}}\label{line:receive-t}
      %\mystate{\myss{\mi{url} := \myangle{\tUrl, \https, \mi{RPDomain}, \str{/startNegotiation}, \myangle{}}}}
      \mystate{\myss{\mi{body} := \myangle{\myangle{\str{t},t}}}}
      \mystate{\myss{\mi{command} := \langle\tXMLHTTPRequest,\textsf{URL}^{\mi{RPDomain}}_\str{/startNegotiation},\mPost,\mi{body},}\breakalgohook{0}\myss{s'.\str{refXHR}\rangle}}
      \mystate{\myss{s'.\str{phase} := \str{expectCert}}}
    \ENDIF
  \ENDCASE
  \CASE{\myss{\str{expectCert}}}
    \mystate{\myss{pattern := \myangle{\tXMLHTTPRequest,*,s'.\str{refXHR}}}}
    \mystate{\myss{\mi{input} := \textsf{CHOOSEINPUT}(\mi{scriptinputs}, \mi{pattern})}}
    \myif{\mi{input} \not\equiv \bot}
      \mystate{\myss{\mi{Cert_{rp}} := \pi_2(\mi{input}).\str{Cert_{rp}}}}
      \mystate{\myss{\mi{IdPWindowNonce} := \pi_1(\textsf{SUBWINDOWS}(\mi{tree},\mi{docnonce})).\str{nonce}}}
      \mystate{\myss{\mi{IdPOrigin} := \mathsf{GETORIGIN}(\mi{tree}, \mi{IdPWindowNonce})}}
      \mystate{\myss{\mi{command} := \langle\tPostMessage, \mi{IdPWindowNonce}, \myangle{\str{Cert}, \mi{Cert_{rp}}},}\breakalgohook{0}\myss{\mi{IdPOrigin}\rangle}}
      \mystate{\myss{s'.\str{phase} := \str{expectToken}}}
    \ENDIF
  \ENDCASE
  \CASE{\myss{\str{expectToken}}}
    \mystate{\myss{\mi{pattern} := \myangle{\tPostMessage, target, *, \myangle{\str{IDToken}, *}}}}
    \mystate{\myss{\mi{input} := \textsf{CHOOSEINPUT}(\mi{scriptinputs}, \mi{pattern})}}
    \myif{input \not\equiv \bot}
      \mystate{\myss{\mi{IDToken} := \pi_2(\pi_4(\mi{input}))}}
      %\mystate{\myss{\mi{url} := \myangle{\tUrl, \https, \mi{RPDomain}, \str{/uploadToken}, \myangle{}}}}
      \mystate{\myss{\mi{body} := \myangle{\myangle{\str{IDToken},\mi{IDToken}}}}}
      \mystate{\myss{\mi{command} := \langle\tXMLHTTPRequest,\textsf{URL}^{\mi{RPDomain}}_\str{/uploadToken},\mPost,\mi{body},}\breakalgohook{0}\myss{s'.\str{refXHR}\rangle}}
      \mystate{\myss{s'.\str{phase} := \str{expectLoginResult}}}
    \ENDIF
  \ENDCASE
  \CASE{\myss{\str{expectLoginResult}}}
    \mystate{\myss{pattern := \myangle{\tXMLHTTPRequest,*,s'.\str{refXHR}}}}
    \mystate{\myss{\mi{input} := \textsf{CHOOSEINPUT}(\mi{scriptinputs}, \mi{pattern})}}
    \myif{input \not\equiv \bot}
      \myif{\pi_2(input) \equiv \str{LoginSuccess}}
      \mystate{Load Homepage}
      \ENDIF
    \ENDIF
  \ENDCASE
  \ENDSWITCH\\
  \mystopp{s',\mi{cookies},\mi{localStorage},\mi{sessionStorage},\mi{command}}
  \end{algorithmic}\setlength{\parindent}{1em}
  
  \subsubsection{Identity Provider Page (script\_idp).}\label{app:uppresso-script-Idp}
  
  \begin{definition}\label{def:scriptstateidp}
    A \emph{scriptstate $s$ of $\mi{script\_idp}$} is a term of the form
    $\langle \mi{phase}$, $\mi{user}$, $\mi{parameters} \rangle$ with $\mi{phase} \in
    \mathbb{S}$, $\mi{user} \in \IDs \cup \{\an{}\} \in \gterms$ and $\mi{parameters} \in \dict{\mathbb{S}}{\terms}$,. The 
    \emph{initial scriptstate} of $\mi{script\_idp}$ is $\an{\str{start},*,\myangle}$.
  \end{definition}
  
  We now formally specify the relation of $\mi{script\_idp}$
  
  \captionof{algorithm}{\label{alg:uppresso-script-idp} Relation of $\mi{script\_idp}$ }
  \begin{algorithmic}[1]
  \REQUIRE \myss{\langle\mi{tree},\mi{docnonce},\mi{scriptstate},\mi{scriptinputs},\mi{cookies},\mi{localStorage},\mi{sessionStorage},}
  \breakalgohook{-1}\myss{\mi{ids},\mi{secret}\rangle}
  \mystate{\myss{s' := scriptstate}}
  \mystate{\myss{\mi{command} := \myangle{}}}
  \mystate{\myss{\mi{target} := \textsf{OPENERWINDOW}(\mi{tree},\mi{docnonce})}}
  \mystate{\myss{\mi{origin} := \mathsf{GETORIGIN}(\mi{tree},\mi{docnonce})}}
  \mystate{\myss{\mi{IdPDomain} := \mi{origin}.\str{host}}}
  \SWITCH{\myss{s'.\str{phase}}}
  \CASE{\myss{start}}
    \mystate{\myss{t := \str{random}()}}
    \mystate{\myss{\mi{command} := \myangle{\tPostMessage, \mi{target}, \myangle{\str{t}, t}, \myangle{}}}}\label{line:send-t}
    \mystate{\myss{s'.\str{parameters}[t] := t}}
    \mystate{\myss{s'.\str{phase} := \str{expectCert}}}
  \ENDCASE
  \CASE{\myss{\str{expectCert}}}
    \mystate{\myss{\mi{pattern} := \myangle{\tPostMessage, target, *, \myangle{\str{Cert}, *}}}}
    \mystate{\myss{\mi{input} := \textsf{CHOOSEINPUT}(\mi{scriptinputs}, \mi{pattern})}}
    \myif{input \not\equiv \bot}
      \mystate{\myss{\mi{Cert_{rp}} := \pi_2(\pi_4(input))}}
      \myif{\checksigThree{\mi{Cert_{rp}}.\str{ver}}{\mi{Cert_{rp}}.\str{content}}{s'.\str{IdPConfig}.\mi{pubkey}} \equiv \True}\label{alg:script-idp-verify-cert}
        \mystate{\myss{s'.\str{parameters}[\mi{cert}] := \mi{Cert_{rp}}}}
        \mystate{\myss{t := s'.\str{parameters}[t]}}
        \mystate{\myss{\mi{PID_{rp}} := [t]\mi{Cert_{rp}}.\str{content}[\mi{ID_{rp}}]}}\label{line:gen-pidrp}
        \mystate{\myss{s'.\str{parameters}[\mi{PID_{rp}}] := \mi{PID_{rp}}}}
        \mystate{\myss{\mi{body} := \myangle{\myangle{\str{PID_{rp}},\mi{PID_{rp}}}}}}
        \mystate{\myss{\mi{command} := \langle\tXMLHTTPRequest,\textsf{URL}^{\mi{IdPDomain}}_\str{/reqToken},\mPost,\mi{body},}\breakalgohook{0}\myss{s'.\str{refXHR}\rangle}}
        \mystate{\myss{s'.\str{phase} := \str{expectReqToken}}}
      \ENDIF
    \ENDIF
  \ENDCASE
  \CASE{\myss{\str{expectReqToken}}}
    \mystate{\myss{pattern := \myangle{\tXMLHTTPRequest,*,s'.\str{refXHR}}}}
    \mystate{\myss{\mi{input} := \textsf{CHOOSEINPUT}(\mi{scriptinputs}, \mi{pattern})}}
    \myif{input \not\equiv \bot}
      \myif{\pi_2(input) \equiv \str{Unanthenticated}}
        \mystate{\myss{s'.\str{user} \gets \mi{ids}}}
        \mystate{\myss{\mi{username} := s'.\str{user}.\mi{name}}}
        \mystate{\myss{\mi{password} := \textsf{secretOfID}(s'.\str{user})}}
        \mystate{\myss{\mi{body} := \myangle{\myangle{\str{username}, \mi{username}}, \myangle{\str{password}, \mi{password}}}}}
        \mystate{\myss{\mi{command} := \langle\tXMLHTTPRequest,\textsf{URL}^{\mi{IdPDomain}}_\str{/authentication},\mPost,\mi{body},}\breakalgohook{0}\myss{s'.\str{refXHR}\rangle}}
        \mystate{\myss{s'.\str{phase} := \str{expectLoginResult}}}
      \myelse{\pi_2(input) \equiv \str{Unauthorized}}
        \mystate{\myss{\mi{PID_{rp}} := s'.\str{parameters}[\mi{PID_{rp}}]}}
        \mystate{\myss{\mi{Attr} := \textsf{GETPARAMETERS}(\mi{tree}, \mi{docnonce})[\str{iaKey}]}}
        \mystate{\myss{\mi{body} := \myangle{\myangle{\str{PID_{rp}}, \mi{PID_{rp}}}, \myangle{\str{Attr}, \mi{Attr}}}}}
        \mystate{\myss{\mi{command} := \langle\tXMLHTTPRequest,\textsf{URL}^{\mi{IdPDomain}}_\str{/authorize},\mPost,\mi{body},}\breakalgohook{0}\myss{s'.\str{refXHR}\rangle}}
        \mystate{\myss{s'.\str{phase} := \str{expectToken}}}
      \myelse{}
        \mystate{\myss{IDToken := \pi_2(input)[\str{IDToken}]}}
        \mystate{\myss{RPOringin := \myangle{s'.\str{parameters}[\mi{cert}].\mi{Content}[\str{Enpt}], \mathtt{S}}}}
        \mystate{\myss{\mi{command} := \myangle{\tPostMessage, \mi{target}, \myangle{\str{IDToken},\mi{IDToken}}, RPOrigin}}}
        \mystate{\myss{s'.\str{phase} := \str{stop}}}
      \ENDIF
    \ENDIF
  \ENDCASE
  \CASE{\myss{\str{expectLoginResult}}}
    \mystate{\myss{pattern := \myangle{\tXMLHTTPRequest,*,s'.\str{refXHR}}}}
    \mystate{\myss{\mi{input} := \textsf{CHOOSEINPUT}(\mi{scriptinputs}, \mi{pattern})}}
    \myif{input \not\equiv \bot}
      \myif{\pi_2(input) \equiv \str{LoginSuccess}}
        \mystate{\myss{\mi{PID_{rp}} := s'.\str{parameters}[\mi{PID_{rp}}]}}
        \mystate{\myss{\mi{Attr} := \textsf{GETPARAMETERS}(\mi{tree}, \mi{docnonce})[\str{iaKey}]}}
        \mystate{\myss{\mi{body} := \myangle{\myangle{\str{PID_{rp}}, \mi{PID_{rp}}}, \myangle{\str{Attr}, \mi{Attr}}}}}\label{line:send-pidrp}
        \mystate{\myss{\mi{command} := \langle\tXMLHTTPRequest,\textsf{URL}^{\mi{IdPDomain}}_\str{/authorize},\mPost,\mi{body},}\breakalgohook{0}\myss{s'.\str{refXHR}\rangle}}
        \mystate{\myss{s'.\str{phase} := \str{expectToken}}}
      \ENDIF
    \ENDIF
  \ENDCASE
  \CASE{\myss{\str{expectToken}}}
    \mystate{\myss{pattern := \myangle{\tXMLHTTPRequest,*,s'.\str{refXHR}}}}
    \mystate{\myss{\mi{input} := \textsf{CHOOSEINPUT}(\mi{scriptinputs}, \mi{pattern})}}
    \myif{input \not\equiv \bot}
      \mystate{\myss{IDToken := \pi_2(input)[\str{IDToken}]}}
      \mystate{\myss{RPOringin := \myangle{s'.\str{parameters}[\mi{cert}].\mi{Content}[\str{Enpt}], \mathtt{S}}}}
      \mystate{\myss{\mi{command} := \myangle{\tPostMessage, \mi{target}, \myangle{\str{IDToken},\mi{IDToken}}, RPOrigin}}}\label{line:token-send}
      \mystate{\myss{s'.\str{phase} := \str{stop}}}
    \ENDIF
  \ENDCASE
  \ENDSWITCH
  \mystopp{s',\mi{cookies},\mi{localStorage},\mi{sessionStorage},\mi{command}}
  \end{algorithmic}\setlength{\parindent}{1em}
  
  
  \section{Proof of Security}
  
  To state the security properties for \uppresso, we first
  define an \emph{\uppresso web system for authentication analysis}. This
  web system is based on the \uppresso web system and only considers one
  network attacker (which subsumes all web attackers and further network
  attackers).
  
  \begin{definition}
    Let $\uppressoauthwebsystem = (\bidsystem, \scriptset, \mathsf{script}, E^0)$
    an \uppresso web system. We call $\uppressoauthwebsystem$ an
    \emph{\uppresso web system for authentication analysis} iff
    $\bidsystem$ contains only one network attacker process
    $\fAP{attacker}$ and no other attacker processes (i.e.,
    $\mathsf{Net} = \{\fAP{attacker}\}$, $\mathsf{Web} = \emptyset$).
    %Further, $\bidsystem$ contains no DNS servers. DNS servers are
    %assumed to be dishonest, and hence, are subsumed by
    %$\fAP{attacker}$. In the initial state $s_0^b$ of each browser $b$
    %in $\bidsystem$, the DNS address is
    %$\mapAddresstoAP(\fAP{attacker})$. Also, in the initial state
    %$s_0^r$ of each relying party $r$, the DNS address is
    %$\mapAddresstoAP(\fAP{attacker})$.
  \end{definition}
  
  The security properties for \uppresso are formally defined 
  as follows. First note that every $Acct$ recorded in RP was
  calculated by RP as the result of an HTTPS $\mPost$ request 
  $m$. We refer to $m$ as the 
  \emph{request corresponding to $Acct$}. 
  
  In the following definition, when we say a browser 
  $b\in \fAP{B}$ owns an $Acct$, we holds that for some relying 
  party $rp\in \fAP{RP}$ that calculate the $Acct$ and 
  a $\mi{username} \in \mathbb{S}$ with $\mapIDtoOwner(username) = b$.
  \[\mi{Acct}=[\NToID(username)]ID_{rp}=[ID_u]ID_{rp}=[ur]G\]

  \newc
  Here we can treat $Acct$ a \emph{permanent} identifier 
  determined by $ID_u$ and $ID_{rp}$. 
  $ID_{rp} = [r]G$ is a generator on $\mathbb{E}$ of order $n$, 
  as $\mathbb{E}$ is a finite cyclic group. 
  Therefore, given a user with $ID_u$ owned by browser $b$, $Acct$ is a \emph{unique} point on 
  $\mathbb{E}$ for any $u \in [1, n)$, and it is \emph{uniquely} 
  associated with $ID_u=u$.
  That's why we can say a browser $b$ owns an $Acct$.
  \oldc
  
  We now define the similar security properties as the definition 52 in SPRESSO. 
  
  \begin{definition}\label{def:uppresso-security-property} 
    Let $\uppressoauthwebsystem$ be an \uppresso web system for authentication analysis. 
    We say that \emph{$\uppressoauthwebsystem$ is secure} if for every run $\rho$ of
    $\uppressoauthwebsystem$, every state $(S^j, E^j, N^j)$ in $\rho$,
    every $r\in \fAP{RP}$ that is honest in $S^j$, every RP service token of the form 
    $\myangle{nonce, Acct}$ recorded in $S^j(r).\str{serviceTokens}$, the following two conditions are
    satisfied:
  
    \textbf{(A)} If $\myangle{nonce, Acct}$ is derivable from the attackers knowledge
    in $S^j$ (i.e., $\myangle{nonce, Acct} \in d_{\emptyset}(S^j(\fAP{attacker}))$),
    then it follows that the browser $b$ owning $Acct$ is fully corrupted
    in $S^j$ (i.e., the value of $\mi{isCorrupted}$ is $\fullcorrupt$).
  
    \textbf{(B)} If the request corresponding to $\myangle{nonce, Acct}$ was sent by
    some $b\in \fAP{B}$ which is honest in $S^j$, then $b$ owns $Acct$.
  \end{definition}
  
  %First note that the RP service token should be defined as $\langle IDToken$, $Acct \rangle$ 
  %which is $\langle n$, $i \rangle$ in SPRESSO. That is,  
  
  %let  $\mathcal{U\!W\!S}^{auth}$ be an UPPRESSO web system for authentication analysis. We say that $\mathcal{U\!W\!S}^{auth}$ is secure if for every run $\rho$ of $\mathcal{U\!W\!S}^{auth}$, every state ($S^j$, $E^j$, $N^j$) in $\rho$, every $r \in$ $\mathtt{RP}$ that is honest, every RP service token of the form $\langle IDToken$, $Acct \rangle$ recorded in $S^j$($r$).$\mathtt{serviceTokens}$, the following two conditions are satisfied:
  
  %(A) If $\langle IDToken$, $Acct \rangle$ is derivable from the attackers knowledge in $S^j$ (i.e., $\langle IDToken$, $Acct \rangle \in d_{\emptyset}$($S^j$($\mathtt{attacker}$))), then it follows that the browser b owning $Acct$ is fully corrupted in $S^j$ (i.e., the value of $isCorrupted$ is $\mathtt{FULLCORRUPT}$) or $\mathtt{governor}$($Acct$) is not an honest IdP (in $S^j$).
  
  %(B) If the request corresponding to $\langle IDToken$, $Acct \rangle$ was sent by some $b \in \mathtt{B}$ which is honest in $S^j$, then b owns the $ID_U$ which satisfies $Acct=[ID_U]S^j(r).ID_{RP}$.
  
  \begin{theorem}\label{thm:authentication}
    Let $\uppressoauthwebsystem$ be an \uppresso web system as
    defined above. Then $\uppressoauthwebsystem$ is secure
    w.r.t.~authentication.
  \end{theorem}
  
  To prove Theorem~\ref{thm:authentication}, we are going to prove the following Lemmas.
  
  \begin{lemma}\label{lemma:k-does-not-leak-from-honest-rp} 
    If in the processing step $s_i \rightarrow s_{i+1}$ of a run $\rho$
    of $\uppressoauthwebsystem$ an honest relying party $r$ (I) emits an HTTPS
    request of the form
  
    \[ m = \ehreqWithVariable{\mi{req}}{k}{\pub(k')} \]
  %
    (where $\mi{req}$ is an HTTP request, $k$ is a nonce (symmetric
    key), and $k'$ is the private key of some other DY process $u$), and (II) in the
    initial state $s_0$ the private key $k'$ is only known to $u$, and
    (III) $u$ never leaks $k'$, then all of the following
    statements are true:
    \begin{enumerate}
    \item There is no state of $\uppressoauthwebsystem$ where any party except
      for $u$ knows $k'$, thus no one except for $u$ can
      decrypt $\mi{req}$.
      \label{prop:attacker-cannot-decrypt-spresso}
    \item If there is a processing step $s_j \rightarrow s_{j+1}$ where
      the RP $r$ leaks $k$ to $\bidsystem \setminus \{u, r\}$ there
      is a processing step $s_h \rightarrow s_{h+1}$ with $h < j$
      where $u$ leaks the symmetric key $k$ to $\bidsystem \setminus
      \{u,r\}$ or $r$ is corrupted in
      $s_j$. \label{prop:k-doesnt-leak-spresso}
    \item The value of the host header in $\mi{req}$ is the domain that
      is assigned the public key $\pub(k')$ in RP's keymapping
      $s_0.\str{keyMapping}$ (in its initial
      state). \label{prop:host-header-matches-spresso}
    \item If $r$ accepts a response (say, $m'$) to $m$ in a processing step $s_j
      \rightarrow s_{j+1}$ and $r$ is honest in $s_j$ and $u$ did not
      leak the symmetric key $k$ to $\bidsystem \setminus \{u,r\}$ prior
      to $s_j$, then $u$ created the HTTPS response $m'$ to the HTTPS
      request $m$, i.e., the nonce of the HTTP request $\mi{req}$ is not known to
      any atomic process $p$, except for the atomic DY processes $r$ and
      $u$.\label{prop:only-owner-answers-spresso}
    \end{enumerate}
  \end{lemma}
  
  %\begin{lemma}\label{lemma:wkcache-never-lies}
  %  For every honest relying party $r \in \fAP{RP}$, every $s \in \rho$, every
  %  $\an{\mi{host}, \mi{wkDoc}} \inPairing S(r).\str{wkCache}$ it holds
  %  that $\mi{wkDoc}[\str{signkey}] \equiv
  %  \pub(\mathsf{signkey}(\mapDomain^{-1}(\mi{host})))$ if
  %  $\mapDomain^{-1}(\mi{host})$ is an honest IdP.
  %\end{lemma}
  
  \begin{lemma}\label{lemma:uppresso-request-exists}
    In a run $\rho$ of $\uppressoauthwebsystem$, for every 
    state $s_j \in\rho$, every RP $r \in \fAP{RP}$ that is 
    honest in $s_j$, every $\myangle{nonce, Acct} \inPairing 
    S^j(r).\str{serviceTokens}$, the following properties hold:
  
    \begin{enumerate}
    \item There exists exactly one $l' < j$ such that there exists a
      processing step in $\rho$ of the form
      \[ s_{l'} \xrightarrow[r \rightarrow \an{\an{a',f',m'}}]{e'
        \rightarrow r} s_{l'+1}\]
      with $e'$ being some events, $a'$ and $f'$
      being addresses and $m'$ being a service token response for $Acct$.
  
    \item There exists exactly one $l < j$ such that there exists a
      processing step in $\rho$ of the form 
      \[ s_{l} \xrightarrow[r \rightarrow e]{\an{a,f,m} \rightarrow r}
      s_{l+1} \] with $e$ being some events, $a$ and $f$ being
      addresses and $m$ being a service token request for $Acct$.
  
    \item The processing steps from (1) and (2) are the same, i.e., $l = l'$.
  
    \item \label{lemma:item:form}The service token request for $Acct$, $m$ in (2), is an HTTPS message of the following form:
      \[ \mathsf{enc}_\mathsf{a}(\langle \hreq{ 
            nonce=n_\text{req}, 
            method=\mPost,
            xhost=d_r,
            path=\str{/authorize}, 
            parameters=x, 
            headers=h,
            xbody=b}, k\rangle, \pub(\mapTLSKey(d_r))) \]  
      for $d_r \in \mapDomain(r)$, some terms $x$, $h$, $n_\text{req}$, and a dictionary $b$ such that 
      \[ b[\str{IDToken}] \equiv \myangle{\mi{PID_{rp}, \mi{PID_u}, ver}} \]
      with 
      \[ \mi{PID_{rp}} \equiv [S^l(r).\str{loginSessions}[t]]S^l(r).\str{rp}, \]
      \[ \mi{PID_{u}} \equiv [u]\mi{PID_{rp}}, \]
      \[ \mi{ver} \equiv \sig{\an{PID_{rp},PID_u}}{k_{sign}} \]
      for some nonces $u$, and $k_\text{sign}$.
    \item If the IdP $i$ is honest, we have that $k_\text{sign} = S^l(i).\str{signkey}$.
    \end{enumerate}
  \end{lemma}
  
  We define the Lemma~\ref{lemma:k-does-not-leak-from-honest-rp} 
  and ~\ref{lemma:uppresso-request-exists}, which prove 
  that the data transmitted through HTTPS is secure and the 
  IdP's public key used for generating IDToken is secure. 
  In UPPRESSO, only the single IdP is trusted, so that the 
  public key is guaranteed to be always trusted. Therefore, 
  we can also follow the proofs in SPRESSO.
  
  %补充证明逻辑的描述
  %lemma8的描述需要明确malicious party是哪一个,honest party是哪一个
  %lemma8采用纯数学描述,去除UPPRESSO表述
  \subsection{Proof of Property A}
  Then we prove the Property $A$ is satisfied in UPPRESSO.
  As stated above, the Property $A$ is defined as follows:
  \begin{definition}\label{def:uppresso-security-property} 
    Let $\uppressoauthwebsystem$ be an \uppresso web system 
    for authentication analysis. We say that 
    \emph{$\uppressoauthwebsystem$ is secure 
    (with respect to Property A)} if for every run $\rho$ of 
    $\uppressoauthwebsystem$, every state $(S^j, E^j, N^j)$ in 
    $\rho$, every $r\in \fAP{RP}$ that is honest in $S^j$, 
    every RP service token of the form $\myangle{n, Acct}$ 
    recorded in $S^j(r).\str{serviceTokens}$ and derivable 
    from the attackers knowledge in $S^j$ (i.e., 
    $\myangle{n, Acct} \in 
    d_{\emptyset}(S^j(\fAP{attacker}))$), it follows that the 
    browser $b$ owning $Acct$ is fully corrupted in $S^j$ 
    (i.e., the value of $\mi{isCorrupted}$ is $\fullcorrupt$). 
  \end{definition}
  
  %\begin{definition}
  %Let $\mathcal{U\!W\!S}^{auth}$  be an UPPRESSO web system for authentication analysis. We say that $\mathcal{U\!W\!S}^{auth}$  is secure (with respect to Property A) if for every run $rho$ of $\mathcal{U\!W\!S}^{auth}$ , every state ($S^j$, $E^j$, $N^j$) in $rho$, every $r \in \mathtt{RP}$ that is honest in
  %$S^j$, every RP service token of the form $\langle IDToken$, $Acct \rangle$ recorded in $S^j$($r$).$\mathtt{serviceTokens}$ and derivable from the attackers knowledge in $S^j$ (i.e., $\langle IDToken$, $Acct \rangle \in d_{\emptyset}$($S^j$($\mathtt{attacker}$))), it follows that the browser b owning $Acct$ is fully corrupted in $S^j$ (i.e., the value of $isCorrupted$ is $\mathtt{FULLCORRUPT}$) or $\mathtt{governor}$($Acct$) is not an honest IdP (in $S^j$).
  %\end{definition}
  
  Same as the proof in SPRESSO, we want to show that every UPPRESSO web system is secure with regard to Property A and therefore assume that there exists an UPPRESSO web system that is not secure. We will lead this to a contradication and thereby show that all UPPRESSO web systems are secure (with regard to Property A).
  
  %In detail, we assume: \emph{There is an UPPRESSO web system for authentication analysis $\mathcal{U\!W\!S}^{auth}$. We say that $\mathcal{U\!W\!S}^{auth}$  is secure (with respect to Property A) if for every run $rho$ of $\mathcal{U\!W\!S}^{auth}$ , every state ($S^j$, $E^j$, $N^j$) in $rho$, every $r \in \mathtt{RP}$ that is honest in
  %$S^j$, every RP service token of the form $\langle IDToken$, $Acct \rangle$ recorded in $S^j$($r$).$\mathtt{serviceTokens}$ and derivable from the attackers knowledge in $S^j$ (i.e., $\langle IDToken$, $Acct \rangle \in d_{\emptyset}$($S^j$($\mathtt{attacker}$))), it follows that the browser b owning $Acct$ is not fully corrupted in $S^j$ and $\mathtt{governor}$($Acct$) is an honest IdP (in $S^j$).}
  
  In detail, we assume: \emph{There exists an \uppresso web 
  system $\uppressoauthwebsystem$, a run $\rho$ of 
  $\uppressoauthwebsystem$, a state $s_j = (S^j, E^j, N^j)$ 
  in $\rho$, a RP $r\in \fAP{RP}$ that is honest in $S^j$, 
  an RP service token of the form $\myangle{n, Acct}$
  recorded in $S^j(r).\str{serviceTokens}$ and derivable from 
  the attackers knowledge in $S^j$ (i.e., $\myangle{n, Acct} \in
  d_{\emptyset}(S^j(\fAP{attacker}))$), and the browser $b$ 
  owning $Acct$ is not fully corrupted (in $S^j$).}
  
  We now proceed to prove that this is a contradiction. 
  First, we can see that for $\an{nonce, Acct}$ and $s_j$, 
  the conditions in Lemma~\ref{lemma:uppresso-request-exists} 
  are fulfilled, i.e., a service token request $m$ and a 
  service token response $m'$ to/from $r$ exist, and $m'$ is 
  of form shown in Property~\ref{lemma:item:form} where there is an IDToken.
  Let $I$ be the Identity Provider. 
  We know that $I$ is an honest IdP.
  As such, it never leaks its signing key (see Algorithm~\ref{alg:idp}). 
  Therefore, the signed subterm $\mi{IDToken} := \myangle{\mi{Content}, \mi{ver}}$ in which
  $\mi{Content} := \myangle{PID_{rp}, PID_u}$ and 
  $\mi{ver} := \sig{\an{PID_{rp},PID_u}}{k_{sign}}$ 
  had to be created by the IdP $I$. 
  An (honest) IdP creates signatures only in Line~\ref{line:sign-token} of Algorithm~\ref{alg:idp}.
  
  %nonce从哪些获得?
  %当且仅当从系统中的任何一个角色获得token,攻击者可以实现攻击。
  %假设获得token,但无法获得,推翻假设。
  %spresso definition
  \begin{lemma}\label{lemma:b-trigger-request}%(Same as Lemma 4 in SPRESSO) 
    Under the assumption above, only the browser b can issue a 
    request $req$ (say, $m_{auth}$)that triggers the IdP I to 
    create the signed term IDToken. The request was sent by b 
    over HTTPS using I's public HTTPS key.
  \end{lemma}
  \begin{proof}
    We have to consider two cases for the request $m_{auth}$:
  
    \textbf{(A).} First, if the user is not logged in with the 
    $username$ at $I$ (i.e., the browser $b$ has no session 
    cookie that carries a nonce which is a session id at $I$ for 
    which the identitiy is marked as being logged in, compare 
    Line~\ref{line:uppresso-idp-check-login-state} of 
    Algorithm~\ref{alg:idp}), then the request has to carry (in
    the request body) the password matching the $username$ 
    ($\NToS(username)$) to the path $\str{/authentication}$ to 
    retrieve the session cookie. This secret is only known to 
    $b$ initially. Depending on the corruption status of $b$, we 
    can now have two cases:
    \begin{enumerate}
    \item[a)] If $b$ is honest in $s_j$, it has not sent the 
      password to any party except over HTTPS to $I$ (as defined 
      in the definition of browsers). 
    \item[b)] If $b$ is close-corrupted, it has not sent it to 
      any other party while it was honest (case a). When 
      becoming close-corrupted, it discarded the secret.
    \end{enumerate}
  
    I.e., the password has been sent only to $I$ over HTTPS or 
    to nobody at all. The IdP $I$ cannot send it to any other 
    party. Therefore we know that only the browser $b$ can send 
    the request $m_\text{attr}$ in this case.
  
    \textbf{(B).} Second, if the user is logged in at $I$, the 
    browser provides a session id to $I$ that refers to a 
    logged in session at $I$. This session id can only be 
    retrieved from $I$ by logging in, i.e., case (A) applies, 
    in particular, $b$ has to provide the proper password, 
    which only itself and $I$ know (see above). The session id 
    is sent to $b$ in the form of a cookie, which is set to 
    secure (i.e., it is only sent back to $I$ over HTTPS, and
    therefore not derivable by the attacker) and httpOnly 
    (i.e., it is not accessible by any scripts). The browser $b$ 
    sends the cookie only to $I$. The IdP $I$ never sends the 
    session id to any other party than $b$. The session id 
    therefore only leaks to $b$ and $I$, and never to the 
    attacker. Hence, the browser $b$ is the only atomic DY 
    process which can send the request $m_\text{auth}$ in this case.
  
    We can see that in both cases, the request was sent by $b$ 
    using HTTPS and $I$'s public key: If the browser would 
    intend to sent the request without encryption, the request 
    would not contain the password in case (A) or the cookie in 
    case (B). The browser always uses the ``correct'' encryption 
    key for any domain (as defined in $\uppressoauthwebsystem$).
  %The proof is same as the Lemma 4's proof in SPRESSO.
  %It can be proved that the $IDToken$ only contains the $PID_U:=[ID_U]PID_{RP}$ while $PID_U$ is provided by $b$, and $b$ owns the password of $ID_U$.
  \end{proof}
  
  \begin{lemma}\label{lemma:script-idp-trigger-request} %(Same as Lemma 5 in SPRESSO) 
    In the browser $b$, the request $m_{auth}$ was triggered by $\mi{script\_idp}$ 
    loaded from the origin $\myangle{d, S}$ for some $d \in \mathtt{dom}(I)$.
  \end{lemma}
  \begin{proof}
    First, $\an{d,\https}$ for some $d \in \mapDomain(I)$ is the 
    only origin that has access to the password $\NToS(u)$ for 
    the username $u$.
    (as defined in Appendix~\ref{app:browsers-uppresso}).
  
    With the general properties defined in~\cite{BrowserID} and the
    definition of Identity Providers in Appendix~\ref{app:idps}, in
    particular their property that they only send out one script,
    $\mi{script\_idp}$, we can see that this is the only script that can
    trigger a request containing the password.
  %The proof follows the Lemma 5's proof in SPRESSO.
  %It can be proved that only the IdP's script $script\_idp$ owns the password of $ID_U$ can request the $IDToken$ from $I$.
  \end{proof}

  \newc
  We now know that only the $\mi{script\_idp}$ in an honest browser 
  loaded from the IdP can issue a request for the IDToken. 
  That is to say, the attacker cannot request for the correct IDToken 
  by himself. Therefore, next we are going to discuss whether the attacker
  can steal the IDToken stored in honest parties.

  Obviously, the attacker is unable to get the IDToken from IdP because IdP 
  is always honest and never leak it.
  \oldc
  
  \begin{lemma} \label{lemma:idp-to-script-idp} %(Same as Lemma 6 in SPRESSO)
    In the browser $b$, the script $\mi{script\_idp}$ receives 
    the response to the request $m_{auth}$ (and no other script), 
    and at this point, the browser is still honest.
  \end{lemma}
  \begin{proof}
    From the definition of browser corruption, we can see that 
    the browser $b$ discards any information about pending 
    requests in its state when it becomes close-corrupted, in 
    particular any TLS keys. It can therefore not decrypt the 
    response if it becomes close-corrupted before receiving the 
    response.
  
    The rest follows from the general properties defined
    in~\cite{BrowserID}.
  %The proof follows Lemma 6's proof in SPRESSO.
  %It is proved that only the closed-corrupted browser cannot receive the $IDToken$ responding to the $req$ started by the honest browser $b$.
  \end{proof}
  
  We now know that only the script $\mi{script\_idp}$ received 
  the response containing the IDToken. For the following lemmas, 
  we will assume that the browser $b$ is honest. In the other 
  case (the browser is close-corrupted), the IDToken and any 
  information would be discarded from the browser's state 
  (as seen in the proof for Lemma~\ref{lemma:idp-to-script-idp}). 
  This would be a contradiction to the assumption 
  (which requires that the IDToken arrived at the RP).
  
  %Lemma 7 in SPRESSO is not useful here because there is no FWD server in UPPRESSO.
  
  \begin{lemma}\label{lemma:script-idp-to-script-rp} %(Same as Lemma 8 in SPRESSO) 
    The script $\mi{script\_idp}$ forwards the IDToken only to 
    the script $\mi{script\_rp}$ loaded from the origin 
    $\langle d_r, \https\rangle$.
  \end{lemma}
  \begin{proof}
    It is clear that, the IDToken held by the honest 
    $script\_idp$ is only sent to the origin 
    $\langle Cert_{rp}.Enpt_{rp}, \https\rangle$, 
    while the $IDToken.PID_{rp} \equiv [t]Cert_{rp}.ID_{rp}$, 
    and $t$ is the one-time random number. The relation of 
    $Cert_{rp}.ID_{rp}$ and $Cert_{rp}.Enpt_{rp}$ is guaranteed 
    by the signature $Cert_{rp}.ver$ generated by IdP $I$. 
    The process is shown at Line~\ref{line:token-send}
    Algorithm~\ref{alg:uppresso-script-idp}.
  %The proof is same as proof of Lemma 8 in SPRESSO.
  %It can be proved that, the $IDToken$ held by the honest $script\_idp$ is only sent to the origin $\langle Cert_{RP}.Enpt_{RP}, S \rangle$, while the $IDToken.PID_{RP} \equiv [t]Cert_{RP}.ID_{RP}$, and $t$ is the one-time random number.  The relation of $ID_RP$ and $Enpt$ is guaranteed by the signature generated by IdP $I$. The process is shown at Line 9, 16, 19, 21, 38, 39, 59, 60  in Algorithm~\ref{alg:script_idp}.
  \end{proof}
  
  \begin{lemma}\label{lemma:script-rp-to-rp} %(Same as Lemma 9 in SPRESSO) 
    From the RP document, the IDToken is only sent to the RP r 
    and over HTTPS
  \end{lemma}
  \begin{proof}
    It is proved that $script\_rp$ of the origin 
    $\langle Cert_{rp}.Enpt_{rp}, \https\rangle$ 
    would only sent to the corresponding RP $r$, 
    which is shown in Algorithm~\ref{alg:uppresso-script-rp}.
  %The proof follows the proof of Lemma 9 in SPRESSO.
  \end{proof}
  
  \newc
  The proofs show that the IDToken is only sent to the honest 
  browser and target RP. It cannot be known to the attacker or 
  any corrupted party, as none of the listed parties leak it to 
  any corrupted party or the attacker.
  \oldc
  
  These proofs are enough for SPRESSO system to show its 
  security, however, they are not enough for UPPRESSO. So far, 
  the proofs only guarantee that the $IDToken$ is never leaked 
  to the attacker. Obviously, the $tag$ in SPRESSO can be only 
  decrypted to a unique domain of RP. However, in UPPRESSO, this 
  statement is not easy to see, an attacker may misuse a wrong 
  $IDToken$ to retrieve the service token from an honest RP, 
  i.e., the attacker can use an IDToken from a corrupted RP. 
  He may try find the $t^{adversary}$ 
  satisfied $IDToken.PID_{RP} \equiv [t^{adversary}]ID_{RP}^{honest}$.
  Therefore, the following Lemmas should be proved.

  \begin{lemma}
    The $t^{adversary}$ is not derivable from the attackers knowledge in $S^j$ (i.e., $t^{adversary} \in d_{\emptyset}$($S^j$($\mathtt{attacker}$))), which satisfies that $IDToken.PID_{RP} \equiv [t^{adversary}]ID_{RP}^{honest}$.
  \end{lemma}
  \begin{proof}
    In UPPRESSO, $PID_{RP}=[t]ID_{RP}$ is generated by a user based on the target RP's identity $ID_{RP}$ and a user-selected random number $t \in [1,n)$.
    %The target RP with $ID_{RP}$ receives $t$, and it will also calculate $PID_{RP}=[t]ID_{RP}$ to match $PID_{RP}$ extracted from a token received.
    %It is computationally easy for any party who knows $ID_{RP}$ and $t$ to validate the $PID_{RP}$ in an identity token. A valid
    Thus, $PID_{RP}$ always specifies an RP, i.e., %$PID_{RP}$ sent by a user in her identity-token request is calculated as $PID_{RP} = [t]ID_{RP}$, where $ID_{RP}$ is the target RP's identity and $t$ is a random number selected by the user and shared with this RP.
    designates the target RP that knows $t$. 
    Moreover, according to Lemma \ref{lemma-rp}, given $PID_{RP} = [t]ID_{RP}$, the probability that $PID_{RP}$ designates another RP with $ID_{RP'}$ is \emph{negligible}. %This means that $PID_{RP}$ cannot be associated with any other RPs in the system.
    Therefore, $PID_{RP}$ designates only the target RP with $ID_{RP}$ in the system, 
    so attacker cannot find a number $t^{adversary}$\hfill
  \end{proof}
  
  \begin{lemma}\label{lemma-rp}
    Given any two points on an elliptic curve denoted by $[r]G$ and $[r']G$ 
    where $r$ and $r'$ are different numbers unknown to an adversary, 
    and $G$ is a generator on $\mathbb{E}$ of order $n$, 
    the probability that the adversary finds different numbers $t$ and $t' \in [1,n)$ 
    satisfying $[tr]G = [t'r']G$ is negligible.
  \end{lemma}
  \begin{proof}
    Finding $t$ and $t'$ that satisfy $[tr]G = [t'r']G$ can be described as 
    a collision game $\mathcal{G}_c$ between an adversary and a challenger: 
    the adversary receives from the challenger a finite set of points, 
    i.e., $[r_1]G$, ..., $[r_m]G$, where $m$ is the number of points, 
    and outputs $(a, b, t, t')$ where $a \neq b$. If $[tr_a]G=[t'r_b]G$, 
    which occurs with a probability ${\rm Pr}_s$, the adversary succeeds in this game.
    
    As depicted in Figure \ref{fig:ecdlp_algorithm}, 
    we design a probabilistic polynomial time (PPT) algorithm $\mathcal{D}^*_c$ based on $\mathcal{G}_c$, 
    to solve the ECDLP: find a number $x \in \mathbb{Z}_n$ satisfying $Q = [x]G$, 
    where $Q$ is a point on $\mathbb{E}$ and $G$ is a generator on $\mathbb{E}$ of order $n$.
  
    \begin{figure}[tb]
      \centering
      \includegraphics[width=0.96\linewidth]{fig/ecdlp_algorithm.pdf}
      \caption{The PPT algorithm $\mathcal{D}^*_c$ constructed based on the $PID_{RP}$ collision game to solve the ECDLP.}
      \label{fig:ecdlp_algorithm}
    \end{figure}
  
    The algorithm $\mathcal{D}^*_c$ works as below.
    The input of $\mathcal{D}^*_c$ is in the form of ($G, Q$). 
    On receiving an input ($G$, $Q$), 
    the challenger first randomly chooses $r_1, \cdots, r_m$ in $\mathbb{Z}_n$ to calculate $[r_1]G, \cdots, [r_m]G$.
    Then, it randomly chooses $j \in [1,m]$, replaces $[r_j]G$ with $Q$, and sends $m$ points to the adversary, 
    which returns the result ($a$, $b$, $t$, $t'$). 
    Finally, the challenger calculates $s = t^{-1}t'r_b \bmod n$ and returns $s$ as the output of $\mathcal{D}^*_c$.
  
    If the adversary succeeds in $\mathcal{G}_c$ and $[r_a]G$ happens to be replaced with $Q$, 
    then $\mathcal{D}^*_c$ outputs $s=t^{-1}t'r_b =x$ because $[tr_a]G = [t]Q = [t'r_b]G$. 
    For the adversary, $Q$ is indistinguishable from any other points in the input set, 
    as $[r_j]G$ is randomly replaced by the challenger.
    Hence, the probability of solving the ECDLP using $\mathcal{D}^*_c$ is formulated as:
    \begin{equation*}
      {\rm Pr}\{\mathcal{D}^*_c(G, [x]G)=x\} = {\rm Pr}\{s = x\}={\rm Pr}\{a=j\}{\rm Pr}_s=\frac{1}{m}{\rm Pr}_s
    \end{equation*}
  
    If the probability of finding $t$ and $t'$ satisfying $[tr]G = [t'r']G$ is non-negligible, 
    the adversary would also have non-negligible advantages in $\mathcal{G}_c$ and ${\rm Pr}_s$ regardless of the security parameter $\lambda$.
    Thus, we would find that ${\rm Pr}\{\mathcal{D}^*_c(G, [x]G)=x\}$ also becomes non-negligible even when $\lambda$ is sufficiently large, 
    because $m$ is a finite integer and $m \ll 2^\lambda$.
    This violates the ECDLP assumption. 
    Therefore, the probability of finding $t$ and $t'$ that satisfy $[tr]G = [t'r']G$ is negligible.
  \end{proof}
  
  Therefore, there is a contradication to the assumption, where we assumed that 
  $Acct \in d_{\emptyset}(S^j(\fAP{attacker}))$. 
  This shows every $\mathcal{U\!W\!S}^{auth}$ is secure in the sense of Property A.
  
  \subsection{Proof of Property B}
  As stated above, Property B is defined as follows:
  \begin{definition}\label{def:B}
    Let $\uppressoauthwebsystem$ be an \uppresso web system. We say that
    \emph{$\uppressoauthwebsystem$ is secure (with respect to Property B)} if
    for every run $\rho$ of $\uppressoauthwebsystem$, every state $(S^j, E^j, N^j)$
    in $\rho$, every $r\in \fAP{RP}$ that is honest in $S^j$, 
    every RP service token of the form $Acct$ recorded in
    $S^j(r).\str{serviceTokens}$, with the request corresponding to
    $\myangle{nonce, Acct}$ sent by some $b\in \fAP{B}$ which is honest in $S^j$, $b$ owns $Acct$.
  %Let $\mathcal{U\!W\!S}^{auth}$  be an UPPRESSO web system for authentication analysis. We say that $\mathcal{U\!W\!S}^{auth}$  is secure (with respect to Property A) if for every run $rho$ of $\mathcal{U\!W\!S}^{auth}$ , every state ($S^j$, $E^j$, $N^j$) in $rho$, every $r \in \mathtt{RP}$ that is honest in
  %$S^j$, every RP service token of the form $\langle IDToken$, $Acct \rangle$ recorded in $S^j$($r$).$\mathtt{serviceTokens}$, with the request corresponding to $\langle IDToken$, $Acct \rangle$ sent by some $b \in B$ which is honest in $S^j$, b owns Acct.
  \end{definition}
  
  First we call the request corresponding to $Acct$ (or service token request) $m$ and
  its response $m'$, and we refer to the state of $\uppressoauthwebsystem$ in the run 
  $\rho$ where $r$ processes $m$ by $s_l$. We are going to prove the $IDToken$ uploaded 
  by honest $b$ can only be related with the $Acct$ owned by $b$.
  
  %we follows the Lemma 10 and its proof in SPRESSO, which guarantees that the request corresponding to $\langle IDToken$, $Acct \rangle$ sent by honest $b$ is loaded from $script\_rp$. 
  %Then we are going to prove the $IDToken$ uploaded by honest $b$ can only be related with the $Acct$ owned by $b$ (which is quite different from SPRESSO).
  
  \begin{lemma}\label{lemma:request-m-is-from-script-rp}
    The request $m$ was sent by $\mi{script\_rp}$ loaded from 
    the origin $\an{d_r, \https}$ where $d_r$ is some domain of 
    $r$.
  \end{lemma}
  
  \begin{proof}
    The request $m$ is XSRF protected. In Algorithm~\ref{alg:rp}, 
    RP checks the presence of the Origin header and its value. 
    If the request $m$ was initiated by a document from a 
    different origin than $\an{d_r, \https}$, the honest browser 
    $b$ would have added an Origin header that would not 
    pass this test (or no Origin header at all), according to 
    the browser definition. The script $\mi{script\_rp}$ is the 
    only script that the honest party $r$ sends as a response 
    and that sends a request to $r$.
  \end{proof}
  
  \begin{lemma}
    For every $IDToken$ uploaded by honest $b$ during authentication, 
    the honest $r \in RP$ can always derive the service token of the form 
    $\myangle{n, \mi{Acct}}$ recorded in $S^j$($r$).$\mathtt{serviceTokens}$, where b owns Acct. 
  \end{lemma}
  \begin{proof}
    According to lemma~\ref{lemma:request-m-is-from-script-rp}, 
    we know that $m$ was sent by $\mi{script\_rp}$ loaded from an honest relying party. 
    The RP accepts the user's identity at line~\ref{line:add-service-token} in Algorithm~\ref{alg:rp}.
    And the user's identity at RP is generated at Line~\ref{line:gen-acct}, 
    based on the $PID_u$ retrieved from the IDToken and the trapdoor $t^{-1}$. 
    
    The $t^{-1}$ is generated and set at Line~\ref{line:gen-t}, 
    holding that $\mi{PID_{rp}}=[t]\mi{ID_{rp}}$.
    Originally, $t$ is chosen by $\mi{script\_idp}$ at Line~\ref{line:gen-t}, Algorithm~\ref{alg:uppresso-script-idp} 
    and transmit to RP through $\mi{script\_rp}$.
    \newc
    Similar to lemma~\ref{lemma:script-idp-to-script-rp} and lemma~\ref{lemma:script-rp-to-rp}, 
    we can prove that RP can only receive the $t$ from $\mi{script\_idp}$, so the $t$ in both parties is equal.
    \oldc

    The $\mi{script\_idp}$ also receives a cert signed by the IdP from the honest RP and verify the cert at Line~\ref{alg:script-idp-verify-cert}, Algorithm~\ref{alg:uppresso-script-rp}.
    After that, $\mi{script\_idp}$ calculates $\mi{PID_{rp}}$ using $t$ and $\mi{ID_{rp}}$ from the cert.
    \newc
    Since $t$ in both parties is equal and $\mi{ID_{rp}}$ is guaranteed by the cert's signature, 
    we can have that the $\mi{PID_{rp}}$ in $\mi{script\_idp}$ and RP is equal.
    \oldc

    Then $\mi{script\_idp}$ sends a request to IdP bringing $\mi{PID_{rp}}$ for the IDToken 
    at Line~\ref{line:send-pidrp} in Algorithm~\ref{alg:uppresso-script-idp}.
    IdP adds the $\mi{PID_{rp}}$ from $\mi{script\_idp}$ into the IDToken.
    \newc
    Therefore, we can see that 
    $\mi{IDToken}.\mi{PID_{rp}}=[t]\mi{ID_{rp}}$ meaning that it is the same as in RP.
    \oldc

    The IDToken is issued at Line~\ref{line:sign-token} in Algorithm~\ref{alg:idp}.
    The IdP generates the $PID_u$ based on the $PID_{rp}$ and $ID_u$.
    \newc
    According to lemma~\ref{lemma:b-trigger-request}, 
    the identities IdP accept and store is from the honest browser, 
    and the honest browser can only provide its own usernames and passwords 
    because only $\mi{script\_idp}$ has the access to these accounts 
    according to lemma~\ref{lemma:script-idp-trigger-request}.
    Therefore, attacker cannot alter the username and password sent to IdP.
    Since the username and password are owned by the honest browser and 
    $ID_u$ is mapped to the account at Line~\ref{line:get-idu} in Algorithm~\ref{alg:idp}. 
    Now we can see that the $ID_u$ used to generate IDToken must be owned by the browser 
    and $\mi{IDToken}.\mi{PID_{u}}=[\mi{ID_u}]\mi{PID_{rp}}$.
    

    %In summary, for every IDToken sent by honest $b$ and IdP, there must be 
    %$\mi{IDToken}.\mi{PID_{rp}} \equiv [t]\mi{ID_{rp}} \equiv \mi{PID_{rp}}$ and 
    %$\mi{IDToken}.\mi{PID_{u}} \equiv [\mi{ID_u}]\mi{PID_{rp}}$. 
    %According to lemma~\ref{user-identification}, 
    Therefore, the $Acct$ calculated by RP following Equation~\ref{equ:calc-acct} must uniquely identifies the user $ID_u$ which is owned by honest $b$.  
    \oldc
    \begin{equation}\label{equ:calc-acct}
      \begin{split}
      Acct=[t^{-1}]PID_u=[t^{-1}][ID_u]PID_{rp}=\\
      [t^{-1}utr]G =[ur]G=[ID_U]ID_{RP}
      \end{split}
    \end{equation}
  \end{proof}

  \begin{lemma}\label{user-identification}
    $PID_u= [ID_u]PID_{rp}$ in IDToken uniquely identifies an 
    account at the RP designated by $PID_{rp}$ if and only if 
    it receives $t$ where $PID_{rp} = [t]ID_{rp}$ holds, and 
    this account is uniquely mapped to a user with $ID_u$.
  \end{lemma}
  \begin{proof}
    To issue an identity token requested for $PID_{rp}$, the 
    honest IdP authenticates the user with $ID_u$ and calculates 
    $PID_u = [ID_u]PID_{rp}$, following Equation \ref{equ:PIDU}. 
    The RP designated by $PID_{rp}$ should have received a $t$ 
    from the user. Following Equation \ref{equ:AccountNotChanged}, 
    it can calculate $Acct = [t^{-1}]PID_{u} = [ID_u]ID_{rp}$, 
    which is a \emph{permanent} identifier determined by $ID_u$ 
    and $ID_{rp}$ after the user and the RP register at the IdP. 
    $ID_{rp} = [r]G$ is a generator on $\mathbb{E}$ of order $n$, 
    as $\mathbb{E}$ is a finite cyclic group. Therefore, given a 
    user with $ID_u$, $Acct$ is a \emph{unique} point on 
    $\mathbb{E}$ for any $u \in [1, n)$, and it is \emph{uniquely} 
    associated with $ID_u=u$. 
    %We first prove that $PID_{U}$ \emph{uniquely} identifies one account at the designated RP and one user in the system. 
    
    \begin{equation}\label{equ:PIDU}
      PID_{U} = [{ID_U}]{PID_{RP}} = [utr]G
    \end{equation}
  
    \begin{equation}\label{equ:AccountNotChanged}
      Acct =  [t^{-1}utr \bmod n]G = [ur]G = [ID_U]ID_{RP}
    \end{equation}
  
    This proves that $PID_u$ in IDToken identifies an account 
    $Acct$ at the designated RP, which is uniquely mapped to a 
    user with $ID_u$ in the system.
  
    Next, we consider two adversarial scenarios where the 
    attacker replays a token for another user to (1) the 
    designated RP but receiving $t'\neq t$ in this login, and 
    (2) any other honest RP with $ID_{rp'} = [r']G \neq [r]G$ 
    (i.e., $r' \neq r$). In the first case, the designated RP 
    would calculate an account as $[t'^{-1}]PID_u = [t'^{-1}ut]ID_{rp}$.
    Because a user's identity is randomly selected by the IdP 
    in $\mathbb{Z}_n$ and known only to the user (and the honest 
    IdP), the probability that $t'^{-1}ut$ happens to be the 
    identity of another user is negligible, when $n$ is 
    sufficiently large. As a result, $[t'^{-1}]PID_u$ is likely 
    not to identify any known account at the RP and therefore 
    would be treated as a new account by the RP. 
    Secondly, the attacker presents IDToken to $RP'$ in the 
    system, where $ID_{rp'} = [r']G \neq [r]G$. $RP'$ would 
    calculate the account as $Acct' = [\tilde{t}^{-1}]PID_{u} = 
    [\tilde{t}^{-1}utrr'^{-1}][r']G = 
    [\tilde{t}^{-1}utrr'^{-1}]ID_{rp'}$. 
    The probability that $\tilde{t}^{-1}utrr'^{-1}$ happens to 
    be the identity of another user at $RP'$ is also negligible, 
    when $n$ is sufficiently large. 
    %it identifies no account mapped to a user, at any RP not designated by $PID_{RP}$.
  \end{proof}
  
  With the above proofs, we now can guarantee that every 
  $\uppressoauthwebsystem$ system satisfies the requirements in 
  Definition~\ref{def:B}, therefore $\uppressoauthwebsystem$ 
  must be secure of Property B.
  
  These prove Theorem~\ref{thm:authentication}.\QED
  
  \section{Proof of Privacy against IdP-based Login Tracing}
  
  In our privacy analysis, we show that an identity provider in UPPRESSO cannot learn 
  where its users log in. We formalize this property as an indistinguishability 
  property: an identity provider (modeled as a web attacker) cannot distinguish 
  between a user logging in at one relying party and the same user logging in at 
  a different relying party.
  
  We will here first describe the precise model that we use for privacy.
  After that, we define an equivalence relation between configurations,
  which we will then use in the proof of privacy.
  
  \subsection{Formal Model of UPPRESSO for Privacy Analysis}
  
  \begin{definition}[Challenge Browser]
    Let $\mi{dr}$ some domain and $b(\mi{dr})$ a DY process. 
    We call $b(\mi{dr})$ a \emph{challenge browser} iff $b$
    is defined exactly the same as a browser with two exceptions: 
    (1) the state contains one more property, namely 
    $\mi{challenge}$, which initially contains the term $\top$. 
    (2) The broswer's algorithm is extended by the following at 
    its very beginning: It is checked if a message $m$ is 
    addressed to the domain $\str{CHALLENGE}$ (which we call the 
    challenger domain). If $m$ is addressed to this domain and 
    no other message $m'$ was addressed to this domain before 
    (i.e., $\mi{challenge} \not\equiv \bot$), then $m$ is changed 
    to be addressed to the domain $\mi{dr}$ and $\mi{challenge}$ 
    is set to $\bot$ to recorded that a message was addressed to 
    $\str{CHALLENGE}$.
  \end{definition}
  
  \begin{definition}[Deterministic DY Process]
    We call a DY process $p = (I^p,Z^p,R^p,s_0^p)$ \emph{deterministic} iff 
    the relation $R^p$ is a (partial) function.
  
    We call a script $R_\text{script}$ \emph{deterministic} iff the relation 
    $R_\text{script}$ is a (partial) function.
  \end{definition}
  
  \begin{definition}[\uppresso Web System for Privacy Analysis]\label{def:uppresso-ws-priv}
    Let $\uppressowebsystem = (\bidsystem, \scriptset, 
    \mathsf{script}, E^0)$ be an UPPRESSO web system with 
    $\bidsystem = \mathsf{Hon} \cup \mathsf{Web} \cup \mathsf{Net}$, 
    $\mathsf{Hon} = \fAP{B} \cup \fAP{RP} \cup \fAP{IDP}$.
    (as described in Appendix~\ref{app:outlineuppressomodel}).
    $\fAP{RP} = \{r_1,r_2\}$, $r_1$ and $r_2$ two (honest) relying parties,
    Let $\fAP{attacker} \in \mathsf{Web}$ be some web attacker.
    Let $\mi{dr}$ be a domain of $r_1$ or $r_2$ and $b(\mi{dr})$ a challenge browser. 
    Let $\mathsf{Hon}' := \{ b(\mi{dr}) \} \cup \fAP{RP}$, 
    $\mathsf{Web}' := \mathsf{Web}$, 
    and $\mathsf{Net}' := \emptyset$ (i.e., there is no network attacker).
    Let $\bidsystem' := \mathsf{Hon}' \cup \mathsf{Web}' \cup \mathsf{Net}'$.  
    Let $\scriptset' := \scriptset$ and $\mathsf{script}'$ be accordingly.
    We call $\uppressoprivwebsystem(\mi{dr}) = (\bidsystem', \scriptset', \mathsf{script}', E^0, \fAP{attacker})$ 
    an \emph{\uppresso web system for privacy analysis} 
    iff the domain $\mi{dr}_1$ the only domain assigned to $r_1$, and
    $\mi{dr}_2$ the only domain assigned to $r_2$. The browser
    $b(\mi{dr})$ owns exactly one identity and this identity
    is governed by some attacker.  All honest parties (in
    $\mathsf{Hon}$) are not corruptible, i.e., they ignore any
    $\str{CORRUPT}$ message. Identity providers are assumed to be
    dishonest, and hence, are subsumed by the web attackers (which
    govern all identities). %In the initial state $s_0^b$ of the (only)
    %browser in $\bidsystem'$ and in the initial states $s_0^{r_1}$,
    %$s_0^{r_2}$ of both relying parties, the DNS address is
    %$\mapAddresstoAP(\fAP{dns})$. Further, $\mi{wkCache}$ in the initial
    %states $s_0^{r_1}$, $s_0^{r_2}$ is equal and contains a public key
    %for each domain registered in the DNS server (i.e., 
    the relying
    parties already know some public key to verify \uppresso identity
    assertions from all domains known in the system and they do not have to fetch them from IdP.
  \end{definition}
  
  As all parties in an \uppresso web system for privacy analysis are either web 
  attackers, browsers, or deterministic processes and all scripting processes are 
  either the attacker script or deterministic, it is easy to see that in \uppresso 
  web systems for privacy analysis with configuration $(S,E,N)$ a command $\zeta$ 
  induces at most one processing step. We further note that, under a given infinite 
  sequence of nonces $N^0$, all schedules $\sigma$ induce at most one run 
  $\rho = ((S^0,E^0,N^0),\dots,(S^i,E^i,N^i),\dots,(S^{|\sigma|},E^{|\sigma|},N^{|\sigma|}))$ 
  as all of its commands induce at most one processing step for the $i$-th configuration.
  
  We will now define our privacy property for \uppresso:
  
  \begin{definition}[IdP-Privacy]\label{def:idp-privacy}
    Let 
    \begin{align*}
      \uppressoprivwebsystem_1 := \uppressoprivwebsystem(\mi{dr}_1) =
      (\bidsystem_1, \scriptset, \mathsf{script}, E^0, \fAP{attacker}_1)&\text{ and}\\
      \uppressoprivwebsystem_2 := \uppressoprivwebsystem(\mi{dr}_2) =
      (\bidsystem_2, \scriptset, \mathsf{script}, E^0, \fAP{attacker}_2)&
    \end{align*}
    be \uppresso web systems for privacy analysis.  Further, we require
    $\fAP{attacker}_1 = \fAP{attacker}_2 =: \fAP{attacker}$ and for $b_1
    := b(\mi{dr}_1)$, $b_2 := b(\mi{dr}_2)$ we require $S(b_1) = S(b_2)$
    and $\bidsystem_1 \setminus \{b_1\} = \bidsystem_2 \setminus
    \{b_2\}$ (i.e., the web systems are the same up to the parameter of
    the challenge browsers).  We say that $\uppressoprivwebsystem$ is
    \emph{IdP-private} iff $\uppressoprivwebsystem_1$ and
    $\uppressoprivwebsystem_2$ are indistinguishable.
  \end{definition}
  
  \subsection{Definition of Equivalent Configurations}\label{app:defin-equiv-stat}
  
  Let $\uppressoprivwebsystem_1 = (\bidsystem_1, \scriptset, \mathsf{script}, E^0, \fAP{attacker})$ 
  and $\uppressoprivwebsystem_2 = (\bidsystem_2, \scriptset, \mathsf{script}, E^0, \fAP{attacker})$ 
  be \uppresso web systems for privacy analysis. Let $(S_1,E_1,N_1)$ 
  be a configuration of $\uppressoprivwebsystem_1$ and $(S_2,E_2,N_2)$ 
  be a configuration of $\uppressoprivwebsystem_2$.
  
  \begin{definition}[Proto-Tags]
    We call a term of the form $[t]R$ with the variable
    $R$ as a placeholder for an $ID_{rp}$, and $t$ some nonces a
    \emph{proto-tag}.
  \end{definition}
  
  \begin{definition}[Term Equivalence up to Proto-Tags]
    Let $\theta = \{a_1, \ldots, a_l \}$ be a finite set of proto-tags.
    Let $t_1$ and $t_2$ be terms. We call $t_1$ and $t_2$
    \emph{term-equivalent under a set of proto-tags $\theta$} iff there
    exists a term $\tau \in \terms(\{x_1,\dots,x_l\})$ such that
    $t_1 = (\tau [ a_1 / x_1 , \dots , a_l / x_l ])[ ID_{\mi{dr}_1} / R ]$ and
    $t_2 = (\tau [ a_1 / x_1 , \dots , a_l / x_l ])[ ID_{\mi{dr}_2} / R ]$.
    We write $t_1 \prototagequiv{\theta} t_2$.
  
    We say that two finite sets of terms $D$ and $D'$ are
    \emph{term-equivalent under a set of proto-tags $\theta$} iff
    $|D| = |D'|$ and, given a lexicographic ordering of the elements in
    $D$ of the form $(d_1,\dots,d_{|D|})$ and the elements in $D'$ of
    the form $(d'_1,\dots,d_{|D'|})$, we have that for all
    $i \in \{1,\dots,|D|\}$: $d_i \prototagequiv{\theta} d'_i$. We then
    write $D \prototagequiv{\theta} D'$.
  \end{definition}
  
  %随机数t的情况
  \begin{definition}[Equivalence of HTTP Requests]
    Let $m_1$ and $m_2$ be (potentially encrypted) HTTP requests, 
    $L$ be a set of login session tokens and
    $\theta = \{a_1, \ldots, a_l \}$ be a finite set of proto-tags. 
    We call $m_1$ and $m_2$ \emph{$\delta$-equivalent under a set of proto-tags $\theta$} 
    iff $m_1 \prototagequiv{\theta} m_2$ or all subterms are equal with the following exceptions:
    \begin{enumerate}
    \item the Host value and the Origin/Referer headers in both requests
      are the same except that the domain $\mi{dr}_1$ in $m_1$ can be
      replaced by $\mi{dr}_2$ in $m_2$,
    \item If the cookie in both requests include $\str{loginSessionToken}$, 
      then there exists an $l' \in L$ such that $g_1[\str{loginSessionToken}] \equiv l'$, and
    \item the HTTP body $g_1$ of $m_1$ and the HTTP body $g_2$ of $m_2$
      are (I) term-equivalent under $\theta$, 
      (II) for $j\in \{1,2\}$ if
      $g_j[\str{IDToken}] \sim \myangle{PID_{dr_j}, [*]PID_{dr_j}, 
      \sig{\myangle{PID_{dr_j}, [*]PID_{dr_j}}}{*}}$
      and the origin (HTTP header) of HTTP message in $m_j$ is
      $\an{\mi{dr}_j,\https}$ then the receiver of this message is
      $r_j$, and 
    \item if $m_1$ is an encrypted HTTP request then and only then $m_2$
      is an encrypted HTTP request and the keys used to encrypt the
      requests have to be the correct keys for $\mi{dr}_1$ and
      $\mi{dr}_2$ respectively.
    \end{enumerate}
    We write $m_1 \httptagequiv{\theta} m_2$.
  \end{definition}
  
  %loginsessionrecord := <t, tag>
  \begin{definition}[Extracting Entries from Login Sessions]
    Let $t_1$, $t_2$ be dictionaries over $\nonces$ and $\terms$,
    $\theta$ be a finite set of proto-tags, and $d$ a domain. We call
    $t_1$ and $t_2$ \emph{$\eta$-equivalent} iff $t_2$ can be
    constructed from $t_1$ as follows: For every proto-tag
    $a \in \theta$, we remove the entry identified by the dictionary key
    $i$ for which it holds that $\proj{2}{t_1[i]} \equiv a[ID_r/ R]$, if
    any. We denote the set of removed entries by $D$. We write
    $\logsessminus{t_1}{t_2}{\theta}{r}{D}$.
  \end{definition}
  
  \begin{definition}
    Let $a$ be a proto-tag, $S_1$ and $S_2$ be states of \uppresso web
    systems for privacy analysis, and $l$ a nonce. We call $l$ a login
    session token for the proto-tag $a$, written
    $l \in \mathsf{loginSessionTokens}(a,S_1,S_2)$ iff for any
    $i \in \{1,2\}$ and any $j \in \{1,2\}$ we have that
    $\proj{2}{S_i(r_j).\str{loginSessions}[l]} = a[ID_{dr_j}/ R]$.
  \end{definition}
  
  \begin{definition}[Equivalence of States]\label{def:eq-of-states}
    Let $\theta$ be a set of proto-tags and 
    %$H$ be a set of nonces. 
    $L$ be a set of login session tokens.
    Let $T:=\{t\mid [t]R\in \theta\}$. 
    We call $S_1$ and $S_2$ \emph{$\gamma$-equivalent under 
    $(\theta, L)$} iff the following conditions are met:
    \begin{enumerate}
    \item\label{eqs:r1} $S_1(\fAP{r_1})$ equals $S_2(\fAP{r_1})$ except
      for the subterms $\str{loginSessions}$ and $\str{serviceTokens}$, and
    \item\label{eqs:r2} $S_1(\fAP{r_2})$ equals $S_2(\fAP{r_2})$ except
      for the subterms $\str{loginSessions}$ and $\str{serviceTokens}$, and
    \item\label{eqs:logsess} for two sets of terms $D$ and $D'$:
      $\logsessminus{S_1(\fAP{r_1}).\str{loginSessions}}{S_2(\fAP{r_1}).\str{loginSessions}}{\theta}{\mi{dr}_1}{D}$,
      $\logsessminus{S_2(\fAP{r_2}).\str{loginSessions}}{S_1(\fAP{r_2}).\str{loginSessions}}{\theta}{\mi{dr}_2}{D'}$,
      and $D \prototagequiv{\theta} D'$, and
    \item\label{eqs:att-not-t} $\forall t \in T$:
      $t \not\in d_\emptyset(\bigcup_{i \in \{1,2\},\ A\, \in\, \mathsf{Web}\, \cup \,
      \mathsf{Net}\, 
      %\cup\, \{\mathsf{dns}, \mathsf{fwd}
      \}}S_i(A))$
    \item\label{eqs:att} for each attacker $A$:
      $S_1(A) \prototagequiv{\theta} S_2(A)$, and
    \item\label{eqs:att-not-l} for all $a\in\theta$ and all attackers $A$ we have that
      $\nexists\ l \in \mathsf{loginSessionTokens}(a,S_1,S_2)$ such that
      $l$ is a subterm of $S_1(A)$ or $S_2(A)$.
    \item\label{eqs:b} $S_1(b_1)$ equals $S_2(b_2)$ except for for the
      subterms $\str{challenge}$, $\str{windows}$
      %, $\str{pendingDNS}$, $\str{pendingRequests}$ 
      and we have that
      \begin{enumerate}
      \item \label{eqs:b:challenge}
        $S_1(b_1).\str{challenge} = \mi{dr}_1 \wedge
        S_2(b_2).\str{challenge} = \mi{dr}_2$
        or $S_1(b_1).\str{challenge} = S_2(b_2).\str{challenge} = \bot$,
        and
      \item $S_1(b_1).\str{windows}$ equals $S_2(b_2).\str{windows}$ with
        the exception of the subterms $\str{location}$, $\str{referrer}$,
        $\str{scriptstate}$, and $\str{scriptinputs}$ of some document terms
        pointed to by $\mathsf{Docs}^+(S_1(b_1)) = \mathsf{Docs}^+(S_2(b_2)) =: J$. 
        For all $j \in J$ we have that: \label{eqs:b:w}
        \begin{enumerate}
        \item there is no $t \in T$ such that
          \begin{align*}
            t \in d_{\nonces \setminus \{t\}}(\{&S_1(b_1).j.\str{location}
            ,  S_2(b_2).j.\str{location},\\ & S_1(b_1).j.\str{referrer} , 
            S_2(b_2).j.\str{referrer}\})
          \end{align*}
        \item for $p \in \{$
          \begin{align*}
            & \an{\tXMLHTTPRequest,*,*},\\
            & \an{\tPostMessage,*,\an{\mapDomain(dr_j), \https},\an{\str{t}, *}},\\
            & \an{\tPostMessage,*,\an{\mapDomain(dr_j), \https},\an{\str{IDToken}, *}}\\
            & \an{\tPostMessage,*,\an{\mapDomain(idp), \https},\an{\str{Cert}, *}}
          \end{align*}
          $\}$ we have
          $S_1(b_1).j.\str{scriptinputs} |\, p \prototagequiv{\theta}
          S_2(b_2).j.\str{scriptinputs} |\, p$, and
        \item\label{eqs:b:w:script_rp} if
          $S_1(b_1).j.\str{origin} \in \{\an{\mi{dr}_1, \https},\an{\mi{dr}_2, \https}\}$
          then $S_1(b_1).j.\str{script} \equiv \str{script\_rp}$ and \
          \begin{enumerate}
          \item $S_1(b_1).j.\str{location}$ and $S_2(b_2).j.\str{location}$
            are term-equivalent under $\theta$ except for the host part,
            which is either equal or $\mi{dr}_1$ in $b_1$ and $\mi{dr}_2$ in
            $b_2$, and
          \item $S_1(b_1).j.\str{referrer}$ and $S_2(b_2).j.\str{referrer}$
            are term-equivalent under $\theta$ except for the host part,
            which is either equal or $\mi{dr}_1$ in $b_1$ and $\mi{dr}_2$ in
            $b_2$, and
          \item
            $S_1(b_1).j.\str{scriptstate} \prototagequiv{\theta}
            S_2(b_2).j.\str{scriptstate}$ and if $\exists\, l \in L$ such that $l$ is a subterm of $S_1(b_1).j.\str{scriptstate}$, then $S_1(b_1).j.\str{location}.\str{host} \equiv \mi{dr}_1$ and $S_2(b_2).j.\str{location}.\str{host} \equiv \mi{dr}_2$, and
          \item if $\exists\, l \in L$ such that $l$ is a subterm of
            $S_1(b_1).j.\str{scriptinputs}$, then
            $S_1(b_1).j.\str{location}.\str{host} \equiv \mi{dr}_1$ and
            $S_2(b_2).j.\str{location}.\str{host} \equiv \mi{dr}_2$, and
          %\item $\forall t \in N$: $t$ is not contained in any subterm of
          %  $S_1(b_1).j.\str{scriptstate}$ except for
          %  $S_1(b_1).j.\str{scriptstate}.\str{parameters}$, and
          %  \begin{itemize}
          %  \item
          %    $S_1(b_1).j.\str{origin} \not\equiv
          %    \an{\mi{dr}_1,\https}$\\$\implies
          %    t \not\equiv S_1(b_1).j.\str{scriptstate}.\str{parameters}$, and
          %  \item $S_1(b_1).j.\str{origin}
          %    \not\equiv
          %    \an{\mi{dr}_1,\https}$\\$\implies t \not\in
          %    d_\emptyset(S_1(b_1).j.\str{scriptinputs})$, and
          %  \item
          %    $S_2(b_2).j.\str{origin} \not\equiv
          %    \an{\mi{dr}_2,\https}$\\$\implies
          %    t \not\equiv S_2(b_2).j.\str{scriptstate}.\str{parameters}$, and
          %  \item $S_2(b_2).j.\str{origin}
          %    \not\equiv
          %    \an{\mi{dr}_2,\https}$\\$\implies t \not\in
          %    d_\emptyset(S_2(b_2).j.\str{scriptinputs})$,
          %    and \end{itemize}
          \end{enumerate}
        \item\label{eqs:b:w:att_script} if
          $S_1(b_1).j.\str{origin} \not\in
          \{\an{\mi{dr}_1,\https},\an{\mi{dr}_2,\https}\}$
          then $S_1(b_1).j.\str{script} \equiv \str{script\_idp}$ and \
          \begin{enumerate}
          \item
            $S_1(b_1).j.\str{location} \prototagequiv{\theta}
            S_2(b_2).j.\str{location}$, and
          \item
            $S_1(b_1).j.\str{referrer} \prototagequiv{\theta}
            S_2(b_2).j.\str{referrer}$, and
          \item\label{eqs:b:w:att_script:state}
            $S_1(b_1).j.\str{scriptstate} \prototagequiv{\theta}
            S_2(b_2).j.\str{scriptstate}$, and
          \item
            $S_1(b_1).j.\str{scriptinputs} \prototagequiv{\theta}
            S_2(b_2).j.\str{scriptinputs}$, and
          \item\label{eqs:b:w:att_script:t}
            $\forall t \in T$: $t$ is not contained in any subterm of 
            $S_1(b_1).j.\str{scriptstate}$ except for 
            $S_1(b_1).j.\str{scriptstate}.\mi{parameters}[\str{t}]$, and
          \item $\nexists\, l \in L$ such that $l$ is a subterm of
            $S_1(b_1).j.\str{scriptstate}$ or of
            $S_1(b_1).j.\str{scriptinputs}$, and
          \end{enumerate}
        \end{enumerate}
      \item\label{eqs:b:misc} for
        $x \in \{\str{cookies},\str{localStorage},\str{sessionStorage},\str{sts}\}$
        we have that $S_1(b_1).x \prototagequiv{\theta} S_2(b_2).x$. For the
        domains $\mi{dr}_1$ and $\mi{dr}_2$ there are no entries in the
        subterms $x$.
      \end{enumerate}
    \end{enumerate}
  \end{definition}
  
  \begin{definition}[Equivalence of Events]\label{def:Events}
    Let $\theta$ be a set of proto-tags, 
    $L$ be a set of login session tokens, 
    $H$ be a set of nonces, and
    $T:=\{t\mid [t]R\in \theta\}$. 
    We call $E_1 = (e_1^{(1)}, e_2^{(1)}\dots)$ and
    $E_2= (e_1^{(2)}, e_2^{(2)} \dots)$ 
    \emph{$\beta$-equivalent under $(\theta, L, H)$} 
    iff all of the following conditions are satisfied for every 
    $i \in \mathbb{N}$:
  
    \begin{enumerate}
      \item\label{eqe:distinction} One of the following conditions holds
        true:
        \begin{enumerate}
        \item\label{eqe:prototagequiv}
          $e_i^{(1)} \prototagequiv{\theta} e_i^{(2)}$ and if $e_i^{(1)}$
          contains an HTTP(S) message (i.e., HTTP(S) request or HTTP(S)
          response), then the HTTP nonce of this HTTP(S) message is not
          contained in $H$, or
        \item\label{eqe:http-req} $e_i^{(1)}$ is an HTTP request $m_1$
          from $b_1$ to $r_1$ and $e_i^{(2)}$ is an HTTP request $m_2$
          from $b_2$ to $r_2$, $m_1 \httptagequiv{\theta} m_2$, and both
          requests are unencrypted or encrypted (i.e., $m_1$ and $m_2$ are
          the content of the encryption) and $m_1.\str{nonce} \in H$, or
        \item\label{eqe:http-res} $e_i^{(1)}$ is an HTTP(S) response from
          $r_1$ to $b_1$ and $e_i^{(2)}$ is an HTTP(S) response from $r_2$
          to $b_2$, and their HTTP messages $m_1$ (contained in
          $e_i^{(1)}$) and $m_2$ (contained in $e_i^{(1)}$) are the same
          except for the HTTP body $g_1 := m_1.\str{body}$ and the HTTP
          body $g_2 := m_2.\str{body}$ which have to be
          $g_1 \prototagequiv{\theta} g_2$ and $m_1.\str{nonce} \in H$.
        \end{enumerate}
      %可能破坏PID_rp不可区分性的参数需要单独列出
      \item\label{eqe:pre:l} If there exists $l \in L$ such that $l$ is a
        subterm of $e_i^{(1)}$ or $e_i^{(2)}$ then we have that
        $e_i^{(1)}$ is a message from $b_1$ to $r_1$ and $e_i^{(2)}$ is a
        message from $b_2$ to $r_2$ or we have that $e_i^{(1)}$ is a
        message from $r_1$ to $b_1$ and $e_i^{(2)}$ is a message from
        $r_2$ to $b_2$.
      \item\label{eqe:pre:t} If there exists $t \in T$ such that
        $t \in d_{\nonces\setminus\{t\}}(\{e_i^{(1)}, e_i^{(2)}\})$ 
        then $e_i^{(1)}$ is an HTTP(S) request from $b_1$ to $r_1$ 
        and $e_i^{q(2)}$ is an HTTP(S) request from $b_2$ to $r_2$ 
        and the bodies of both HTTP messages are of the form
        $\an{\an{\str{t}, t}}$.
      \item\label{eqe:pre:rp-scripts} If $e_i^{(1)}$ or $e_i^{(2)}$ is an
        HTTP(S) response with body $g$ from a relying party, then it does
        not contain any $\str{Location}$ or $\cSTS$ header
        and if $\proj{1}{g}$ is a string representing a script, then
        $\proj{1}{g}$ is $\str{script\_rp}$.
      \item\label{eqe:pre:unencrypted-http} If $e_i^{(1)}$ or $e_i^{(2)}$
        is an unencrypted HTTP response, then the message was sent by some
        attacker.
    \end{enumerate}
  \end{definition}
  
  \begin{definition}[Equivalence of Configurations]
    We call $(S_1,E_1,N_1)$ and $(S_2,E_2,N_2)$
    \emph{$\alpha$-equivalent} iff there exists a set of proto-tags
    $\theta$ and a set of nonces $H$ such that $S_1$ and $S_2$ are
    $\gamma$-equivalent under $(\theta,H)$, $E_1$ and $E_2$ are
    $\beta$-equivalent under $(\theta,L,H)$ for
    $L := \bigcup_{a\in\theta} \mathsf{loginSessionTokens}(a,S_1,S_2)$,
    and $N_1 = N_2$.
  \end{definition}
  
  \subsection{Privacy Proof}
  
  \begin{theorem} \label{theorem:A}Every UPPRESSO web system for privacy analysis is IdP-private.
  \end{theorem}
  
  Let $\mathcal{U\!W\!S}^{priv}$ be UPPRESSO web system for privacy analysis.\par
  To prove Theorem \ref{theorem:A}, we have to show that the UPPRESSO web systems $\mathcal{U\!W\!S}^{priv}_1$ and $\mathcal{U\!W\!S}^{priv}_2$ 
  are indistinguishable. To show the indistinguishability of $\mathcal{U\!W\!S}^{priv}_1$ and $\mathcal{U\!W\!S}^{priv}_2$, 
  we show that they are indistinguishable under all schedules $\sigma$.
  For this , we first note that for all $\sigma$, there is only one run induced by each $\sigma$(as our web system, when scheduled, is deterministic).
  We now proceed to show that for all schedules $\sigma=(\zeta _1, \zeta_2,\dots)$, iff $\sigma$ induces a run $\sigma(\mathcal{U\!W\!S}^{priv}_1)$ there exists a run $\sigma(\mathcal{U\!W\!S}^{priv}_2)$ such that $\sigma(\mathcal{U\!W\!S}^{priv}_1)\approx\sigma(\mathcal{U\!W\!S}^{priv}_1)$\par
  We now show that if two configurations are $\alpha$-equivalent, then the view of the attacker is statically equivalent.
  
  \begin{lemma}
    Let $(S_1,E_1,N_1)$ and $(S_2,E_2,N_2)$ be two 
    $\alpha$-equivalent configurations. 
    Then $S_1(attacker)\approx S_2(attacker)$.
  \end{lemma}
  \begin{proof}
    From the $\alpha$-equivalence of $(S_1,E_1,N_1)$ and 
    $(S_2,E_2,N_2)$ it follows that $S_1(\fAP{attacker}) 
    \prototagequiv{\theta} S_2(\fAP{attacker})$.
    From Condition~\ref{eqs:att-not-t} for $\gamma$-equivalence 
    it follows that
    $t \not\in d_\emptyset(\bigcup_{i \in \{1,2\},\ A\, \in\, 
    \mathsf{Web}\, \cup \, \mathsf{Net}\, \}}S_i(A))$
    (i.e., the attacker does not know any keys for the tags 
    contained in its view), and with Lemma~\ref{thm-idp-untraceability-new} it is easy to see 
    that the views are statically equivalent.
  \end{proof}

  \newc
  \begin{lemma}\label{thm-idp-untraceability-new}
    Given a point on the elliptic curve denoted by $[r]G$, 
    an adversary cannot distinguish $[tr]G$ from a random variable on $\mathbb{E}$, 
    where $t$ is random in $\mathbb{Z}_n$ and unknown to the adversary.
  \end{lemma}
  \begin{proof}
    Consider a finite cyclic group $\mathbb{E}$ where the number of points on $\mathbb{E}$ is $n$. 
    Because $G$ is a generator of order $n$, $[r]G$ is also a generator on $\mathbb{E}$ of order $n$. 
    $t$ is randomly chosen in $\mathbb{Z}_n$ and always kept unknown to the adversary. 
    Therefore, $[tr]G$ is \emph{indistinguishable} from a point $Q$ that is randomly chosen on $\mathbb{E}$.\cite{oprf-proved,voprf-proved}.
  \end{proof}
  \oldc
  
  We now show that $\sigma(\uppressoprivwebsystem_1) \approx
  \sigma(\uppressoprivwebsystem_2)$ by induction over the length 
  of $\sigma$. We first, in Lemma~\ref{lemma:initial-config-private}, 
  show that $\alpha$-equivalence (and therefore, indistinguishability 
  of the views of $\fAP{attacker}$) holds for the initial 
  configurations of $\uppressoprivwebsystem_1$ and 
  $\uppressoprivwebsystem_2$. We then, in 
  Lemma~\ref{lemma:step-config-private}, show that for each 
  configuration induced by a processing step in $\zeta$,
  $\alpha$-equivalence still holds true.
  
  \begin{lemma}\label{lemma:initial-config-private}
    The initial configurations $(S_1^0,E^0,N^0)$ of 
    $\mathcal{U\!W\!S}^{priv}_1$ and $(S_2^0,E^0,N^0)$ of 
    $\mathcal{U\!W\!S}^{priv}_2$ are $\alpha$-equivalent.
  \end{lemma}
  \begin{proof}
    We now have to show that there exists a set of proto-tags $\theta$ and a set of nonces $H$
    such that $S_1^0$ and $S_2^0$ are $\gamma$-equivalent under
    $(\theta,H)$, $E_1^0 = E^0$ and $E_2^0 = E^0$ are $\beta$-equivalent
    under $(\theta,L,H)$ with $L := \bigcup_{a\in\theta} \mathsf{loginSessionTokens}(a,S_1,S_2)$, and $N_1^0 = N_2^0 = N^0$.
  
    Let $\theta = H = L = \emptyset$. Obviously, both latter conditions are
    true. For all parties $p \in \bidsystem_1 \setminus \{b_1\}$, it is
    clear that $S_1^0(p) = S_2^0(p)$. Also the states $S_1^0(b_1)$ and
    $S_2^0(b_2)$ are equal. Therefore, all conditions
    of Definition~\ref{def:eq-of-states} are fulfilled. Hence, the
    initial configurations are $\alpha$-equivalent.
  \end{proof}
  
  \begin{lemma}\label{lemma:step-config-private}
    Let $(S_1,E_1,N_1)$ and $(S_2,E_2,N_2)$ be two 
    $\alpha$-equivalent configurations of 
    $\uppressoprivwebsystem_1$ and $\uppressoprivwebsystem_2$, 
    respectively. Let $\zeta = \an{\mi{ci},\mi{cp}, 
    \tau_\text{process}, \mi{cmd}_\text{switch}, 
    \mi{cmd}_\text{window},\tau_\text{script},\mi{url}}$
    be a web system command. Then, $\zeta$ induces a processing 
    step in either both configurations or in none. In the latter 
    case, let $(S_1',E_1',N_1')$ and $(S_2',E_2',N_2')$ be 
    configurations induced by $\zeta$ such that
    \[(S_1,E_1,N_1) \xrightarrow{\zeta} (S_1',E_1',N_1') \quad 
    \text{and} \quad (S_2,E_2,N_2) \xrightarrow{\zeta} 
    (S_2',E_2',N_2') \ .\]
    Then, $(S_1',E_1',N_1')$ and $(S_2',E_2',N_2')$ are
    $\alpha$-equivalent.  
  \end{lemma}
  \begin{proof}
    Let $\theta$ be a set of proto-tags and $H$ be a set of 
    nonces for which $\alpha$-equivalence holds and let 
    $L:=\bigcup_{a\in\theta}\text{loginSessionTokens}(a,S_1,S_2)$,
    $T:=\{t\mid [t]R\in \theta\}$.
    
    To induce a processing step, the ci-th message from $E_1$ or 
    $E_2$, respectively, is selected.Following Definition 
    \ref{def:Events}, we denote these messages by $e_i^{(1)}$ or 
    $e_i^{(2)}$, respectively. We now differentiate between the 
    receivers of the messages by denoting the induced processing 
    steps by
    \begin{equation}
      \begin{aligned}
        (S_1,E_1,N_1)\xrightarrow[p_1\rightarrow E_{out}^{(1)}]{\left \langle a_1,f_1,m_1\right \rangle\rightarrow p_1}(S_1\prime,E_1\prime,N_1\prime)\\
        (S_2,E_2,N_2)\xrightarrow[p_2\rightarrow E_{out}^{(2)}]{\left \langle a_2,f_2,m_2\right \rangle\rightarrow p_2}(S_2\prime,E_2\prime,N_2\prime)
      \end{aligned}
    \end{equation}
    \paragraph{\underline{Case $p_1=r_1$:}}
    In this case, we only distinct several cases of HTTP(S) requests that can happen. The others are ignored the same as SPRESSO.\par
    There are four possible types of HTTP requests that are accepted by $r_1$ in Algorithm \ref{alg:rp}:
    \begin{itemize}
      \item path=$\str{/script}$(get the rp-script), Line~\ref{line:rp-script};
      \item path=$\str{/loginSSO}$(start a login), Line~\ref{line:rp-loginSSO};
      \item path=$\str{/startNegotiation}$(derive a $PID_{rp}$), Line~\ref{line:rp-startNegotiation};
      \item path=$\str{/uploadToken}$(verify ID token, calculate Acct), Line~\ref{line:rp-uploadToken}.
    \end{itemize}
    \par From the cases in Definition \ref{def:Events}, only two 
    can possibly apply here:Case~\ref{eqe:prototagequiv} and 
    Case~\ref{eqe:http-req}. For both cases, we will now analyze 
    each of the HTTP requests listed above separately.
  
    \noindent \emph{Definition~\ref{def:Events}, Case~\ref{eqe:prototagequiv}:}
    $e_i^{(1)}\rightleftharpoons e_i^{(2)}$. This case implies 
    $p_2=r_1=p_1$. As we see below, for the output events 
    $E_{out}^{(1)}$ and $E_{out}^{(2)}$ (if any) only 
    Case~\ref{eqe:prototagequiv} of Definition \ref{def:Events} 
    applies. This implies the nonce of both the incoming HTTP 
    requests and HTTP responses cannot be in $H$.
    \begin{itemize}
      \item path=$\str{/script}$ 
        In this case, the same output 
        event is produced whose message is 
        \begin{equation}
          \begin{aligned}
            \left\langle HTTPResp,n,200,\left\langle\right\rangle,RPScript\right\rangle
          \end{aligned}
        \end{equation}
        We can note that Condition~\ref{eqe:pre:rp-scripts} of 
        Definition \ref{def:Events} 
        holds true and.The remaining conditions are trivially 
        fulfilled and $E_1\prime$ and $E_2\prime$ are 
        $\beta$-equivalent under $(\theta,H,L)$.As there are no 
        changes to any state, we have that $S_1\prime$ and 
        $S_2\prime$ are $\gamma$-equivalent under $(\theta,H)$. 
        No new nonces are chosen, hence $N_1\prime=N_1=N_2=N_2\prime$.
      \item path=$\str{/loginSSO}$ 
        In this case, the reason for holding equivalence is 
        similar to the case above since the same output event 
        is produced.
      \item path=$\str{/startNegotiation}$ 
        In both processing steps, a tag is constructed exactly 
        the same. The same HTTP response (which does not contain 
        a $t \in T$ or a $l \in L$) is put in both 
        $E^{(1)}_\text{out}$ and $E^{(2)}_\text{out}$. The first 
        element of the response's body is not a string and 
        therefore Condition~\ref{eqe:pre:rp-scripts} holds true. 
        The tag is only created on $r_1$ in both runs and hence, 
        $\theta$ does not have to be altered. 
        Analogously to above, we have that $E_1'$ and $E_2'$ are 
        $\beta$-equivalent under $(\theta,H,L)$. The subterm 
        $\str{loginSessions}$ of the state of $r_1$ is extended 
        exactly the same. Thus, we have that $S_1'$ and $S_2'$ 
        are $\gamma$-equivalent under $(\theta,H)$. In both
        processing steps exactly one nonce is chosen, and we 
        have that $N_1' = N_2'$.
      \item path=$\str{/uploadToken}$ 
        First, we note that there is no $l \in L$ contained in 
        either $m_1$ or $m_2$ (by the Defintion of 
        $\beta$-equivalence). We further note that there are 
        four checks at Algorithm~\ref{alg:rp} in
        Line~\ref{line:alg-rp-stop1}, \ref{line:alg-rp-stop3}, 
        \ref{line:alg-rp-stop3} and \ref{line:alg-rp-stop4}.
        The script either emits an empty message if failed in
        these checks or accept the request and response with a 
        nonce.
  
        From Condition~\ref{eqs:logsess} of
        Definition~\ref{def:eq-of-states}, we know that
        $S_2(r_1).\str{loginSessions}$ can be constructed from
        $S_1(r_1).\str{loginSessions}$ without removing the 
        entry with the dictionary key 
        $\mi{headers}[\str{Cookie}][\str{loginSessionToken}]$ 
        (as this key is not in $L$). Thus, both dictionaries 
        either contain the same entry for the dictionary key
        $\str{loginSessionToken}$ or they both contain no
        such entry and if they contain such entry, the contents
        will be equal. Hence, we have that if the first two 
        checks fail in $s_1$ then and only then they fail in $s_2$.
  
        If the first two checks pass, since $m_1$ equals to $m_2$ 
        from condition~\ref{eqe:prototagequiv} of 
        Definition~\ref{def:Events}, we have that if the third 
        check fails in $s_1$ then and only then it fails in $s_2$. 
        The same holds true for the fourth check.
  
        if they both accept the IDToken, exactly the 
        same outputs are emitted (without containing any 
        $l\in L$ or $t \in T$), no state is changed and exactly 
        one new nonce is chosen. We therefore trivially have
        $\alpha$-equivalence of the new configurations.
    \end{itemize}
  
    \noindent \emph{Definition~\ref{def:Events}, Case~\ref{eqe:http-req}:} 
    $e_i^{(1)}$ is an HTTP(S) request from $b_1$ to $r_1$ and 
    $e_i^{(2)}$ is an HTTP(S) request from $b_2$ to $r_2$. 
    This case implies $p_2 = \fAP{r_2}$.
  
    We note that Condition~\ref{eqe:pre:rp-scripts} of 
    Definition~\ref{def:Events} holds for the same reasons as in
    the previous case. As the response is always addressed to 
    the IP address of $b_1$ or $b_2$, respectively,
    Condition~\ref{eqe:pre:rp-scripts} of
    Definition~\ref{def:Events} is fulfilled. 
  
    As we see below, for the output events $E^{(1)}_\text{out}$ 
    and $E^{(2)}_\text{out}$ (if any) only Case~\ref{eqe:http-res} 
    of Definition~\ref{def:Events} applies. This implies that the
    output events must contain an HTTP nonce contained in $H$. As 
    we know that the HTTP nonce of the incoming HTTP requests is 
    contained in $H$ and the output HTTP responses (if any) of the 
    RP reuses the same HTTP nonce, the nonce of the HTTP responses 
    is in $H$.
  
    \begin{itemize}
      \item $\mi{path} = \str{/script}$ In this case, 
        the output events contain no $l\in L$ or $t\in T$ 
        meaning that $E_1'$ and $E_2'$ being $\beta$-equivalent
        under $(\theta,H,L)$ according to 
        Definition~\ref{def:Events}, Case~\ref{eqe:http-res}. As
        there are no changes to any state, we have that $S_1'$ 
        and $S_2'$ are $\gamma$-equivalent under $(\theta,H)$. 
        No new nonces are chosen, hence, 
        $N_1 = N'_1 = N_2 = N'_2$.
      \item $\mi{path} = \str{/loginSSO}$ This case is analogue
        to the case above.
      \item $\mi{path} = \str{/startNegotiation}$ In this case, 
        an HTTP response is created. We denote the HTTP response generated by $r_1$ as $m_1'$ and the one
        generated by $r_2$ as $m_2'$. We then have that
        \begin{align*}
          m_1' = \encs{\an{\cHttpResp,n,200,\an{\mi{setCookie}},g_1}}{k} \\
          m_2' = \encs{\an{\cHttpResp,n,200,\an{\mi{setCookie}},g_2}}{k}
        \end{align*}
        with
        \begin{align*}
          \mi{setCookie} := \myangle{\cSetCookie, \myangle{\myangle{\str{loginSessionToken}, \nu_1, \True, \True, \True}}} \\
        \end{align*}
        and 
        \begin{align*}
          g_1 = \an{\an{\str{Cert_{RP}},S_1(r_1).\str{IdPConfig}.Cert_{RP}}} \\
          g_2 = \an{\an{\str{Cert_{RP}},S_2(r_2).\str{IdPConfig}.Cert_{RP}}}
        \end{align*}
  
        Obviously, $m_1'$ equals $m_2'$. For $N_1 = N_2 = 
        (n_1, n_2, \dots)$, We set $\theta' = \theta \cup 
        \{ [t]S_j(r_j).ID_{RP} \}$ for $j\in \{1, 2\}$, 
        $N_1' = N_2' = (n_2, \dots)$ (as exactly one nonce is 
        chosen in both processing steps) and 
        $L' = L \cup \{n_1\}$. 
        The receiver of both messages is the browser $b_1$ or 
        $b_2$, respectively. Obviously, it holds that
        $L' = \bigcup_{a\in\theta'} 
        \mathsf{loginSessionTokens}(a,S_1',S_2')$
        and there exists an $l' \in L'$ such that
        $g_1[\str{loginSessionToken}] \equiv l'$. As
        Conditions~\ref{eqe:http-res} and~\ref{eqe:pre:t} of
        Definition~\ref{def:Events} hold, $E_1'$ and $E_2'$ are
        $\beta$-equivalent under $(\theta',H,L')$. The subterm
        $\str{loginSessions}$ of $S_1(r_1)$ is extended exactly 
        the same as the subterm $\str{loginSessions}$ of 
        $S_2(r_2)$. Thus, we have that $S_1'$ and $S_2'$ are
        $\gamma$-equivalent under $(\theta',H)$.
      \item $\mi{path} = \str{/uploadToken}$ In this case, 
        there are four checks at Algorithm~\ref{alg:rp} in
        Line~\ref{line:alg-rp-stop1}, \ref{line:alg-rp-stop3}, 
        \ref{line:alg-rp-stop3} and \ref{line:alg-rp-stop4}.
        
        From Condition~\ref{eqs:logsess} of 
        Definition~\ref{def:eq-of-states} we know that for 
        $\mi{ls}_1 := S_1(r_1).\str{loginSessions}[l]$ and 
        $\mi{ls}_2 := S_2(r_2).\str{loginSessions}[l]$, 
        we have that $\mi{ls}_1 \prototagequiv{\theta} \mi{ls}_2$.
        Therefore, we have that if the first two checks fail in
        $r_1$ then and only then they fail in $r_2$.
  
        As we know that $m_1 \httptagequiv{\theta} m_2$, we have 
        that if the third check fails in $r_1$ then and only 
        then it fails in $r_2$. The same holds true for the 
        fourth check.
  
        if $r_1$ and $r_2$ both accept the IDToken, they will 
        generate HTTP responses with service Token.We denote 
        the HTTP response generated by $r_1$ as $m_1'$ and the
        one generated by $r_2$ as $m_2'$. We then have that
        \begin{align*}
          m_1' = \encs{\an{\cHttpResp,n,200,\an{},g_1}}{k} \\
          m_2' = \encs{\an{\cHttpResp,n,200,\an{},g_2}}{k}
        \end{align*}
        with
        \begin{align*}
          g_1 = \an{\an{\str{nonce}, \nu_1}} \\
          g_2 = \an{\an{\str{nonce}, \nu_1}}
        \end{align*}
        Same as above, $m_1'$ equals $m_2'$, $N_1' = N_2' = 
        (n_2, \dots)$ and $L' = L$. 
        The receiver of both messages is the browser $b_1$ or 
        $b_2$, respectively. As Conditions~\ref{eqe:http-res} 
        and~\ref{eqe:pre:l} of Definition~\ref{def:Events} hold, 
        $E_1'$ and $E_2'$ are $\beta$-equivalent under 
        $(\theta,H,L)$. The subterm $\str{loginSessions}$ of 
        $S_1(r_1)$ is extended exactly the same as the subterm 
        $\str{loginSessions}$ of $S_2(r_2)$. Thus, we have that 
        $S_1'$ and $S_2'$ are $\gamma$-equivalent under 
        $(\theta,H)$.
    \end{itemize}
  
    \paragraph{\underline{Case $p_1 = \fAP{r_2}$:}} This case is
    analogue to the case $p_1 = \fAP{r_1}$ above. Note that the
    Case~\ref{eqe:http-req} of Definition~\ref{def:Events} 
    cannot occur by definition.
  
    \paragraph{\underline{Case $p_1 = \fAP{b_1}$:}} 
    $\implies p_2 = \fAP{b_2}$ 
  
    %We now do a case distinction over the types of messages a 
    %browser can receive.
  
    \begin{description}
      %\item[DNS response]
      \newc
      \item[HTTP response] In this case, it is clear that
        the HTTP(s) response nonce is the same in both
        messages $m_1$ and $m_2$. 
        We can now distinguish between two cases: 
        In both browsers, \ref{browser-http-response-normal}
        the $\mi{reference}$ that is stored along with the HTTP 
        nonce is a window reference (in this case, the request 
        was a ``normal'' HTTP(S) request), 
        or \ref{browser-http-response-xhr} this reference is a 
        pairing of a document nonce and an XHR reference chosen 
        by the script that sent the request, which is an XHR.
        
        \begin{enumerate}[I.]
        \item\label{browser-http-response-normal} 
          In Case~(I), we can distinguish between the following two cases:
          \begin{enumerate}
            \item The HTTP nonce in $m_1$ is in $H$: 
              In this case, only Case~\ref{eqe:http-res} of Definition~\ref{def:Events} can apply. 
              We therefore have that the expected sender in $e_i^{(1)}$ is $r_1$ and in $e_i^{(2)}$ is $r_2$. 
              We also have that there is no Location, Set-Cookie or Strict-Transport-Security header in the response, 
              and the responses $m_1$ and $m_2$ are both $\str{script\_rp}$ as Case~\ref{eqe:pre:rp-scripts} of Definition~\ref{def:Events} holds.
      
              With this, we observe that both browsers either accept and
              successfully decrypt the messages and call the function
              $\mathsf{PROCESSRESPONSE}$, or both browsers stop with not
              state change and no output event (in which case the
              $\alpha$-equivalence is given trivially). In particular we
              note that the expected sender in both cases matches precisely
              the sender the message has (compare Case~\ref{eqe:http-res} of
              Definition~\ref{def:Events}).
      
              In $\mathsf{PROCESSRESPONSE}$, we see that no changes in the
              browsers' cookies are performed (as no cookies are in the
              response), the $\str{sts}$ subterm is not changed, and no
              redirection is performed (as no Location header is present).
      
              Now, new documents are created in each browser. These have the
              form
              \[ \an{\nu_7, \mi{location}, \mi{referrer}, \mi{script},
                \mi{scriptstate}, \an{}, \an{}, \True} \] with
              \[ \mi{location} = \an{\cUrl, \mi{protocol}, \mi{host},
                \mi{path}, \mi{parameters}}\ .\]
      
            
              Here, $\mi{script}$, $\mi{scriptstate}$ are the same and
              $\mi{protocol}$, $\mi{path}$, $\mi{parameters}$ are taken from
              the requests, which means that these subterms are equal or
              term-equivalent up to proto-tags $\theta$. 
              The host and the referrer are the same in both states up to exchange of domains, 
              which can be $\mi{dr}_1$ in $b_1$ and $\mi{dr}_2$ in $b_2$.
      
              The browser now attaches these newly created documents to its
              window tree, and we have to check that the
              Condition~\ref{eqs:b:w} of Definition~\ref{def:eq-of-states}
              is satisfied.
      
              As we have that both incoming messages were encrypted messages
              (see Case~\ref{eqe:pre:unencrypted-http} of
              Definition~\ref{def:Events}) and both messages come from
              $r_1$ and $r_2$, respectively, and therefore $\mi{script}$ is
              either $\str{script\_rp}$ (see
              Case~\ref{eqe:pre:rp-scripts} of
              Definition~\ref{def:Events}) we have to check
              Conditions~\ref{eqs:b:w:script_rp} of
              Definition~\ref{def:Events} in particular.
      
              The scriptstate is initially equal and the script inputs are empty. The document's
              referer is constructed from the referer header of the request,
              which is equal in both cases or has the host $\mi{dr}_1$ in
              $b_1$ and $\mi{dr}_2$ in $b_2$.
      
              To sum up, $\gamma$-equivalence under $(\theta, H)$ is
              preserved. $\alpha$-equivalence is preserved as no output
              event is generated and the exact same number of nonces are
              chosen.
      
            
            \item The HTTP nonce in $m_1$ is not in $H$: In this case we
              have that $e_i^{(1)} \prototagequiv{\theta} e_i^{(2)}$
              (Case~\ref{eqe:prototagequiv} of
              Definition~\ref{def:Events}), and that the HTTP nonces,
              senders, encryption keys (if any) and original requests in the
              pending requests of both browsers are either equal or
              equivalent up to proto-tags $\theta$. There can be no
              $t \in T$ as a subterm (except in tags) of the input.
      
              With this, we observe that both browsers either accept and
              successfully decrypt the messages and call the function
              $\mathsf{PROCESSRESPONSE}$, or both browsers stop with no
              state change and no output event (in which case the
              $\alpha$-equivalence is given trivially). In particular we
              note that the expected sender in both cases matches precisely
              the sender of the message (as it is equal).
      
              If there is a Set-Cookie header in one of the responses, a new
              entry in the cookies of each browsers is created (which
              obviously is term-equivalent up to $\theta$, and therefore is
              in compliance with the requirements for $\gamma$-equivalence).
              The same holds true for any Strict-Transport-Security headers.
      
              Now, if there is a Location header in $m_1$ (and therefore
              also in $m_2$), a new request is generated and a HTTP(S) request is sent out. 
              The new HTTP(S) request contains the method, body, and Origin
              header of the original request (which were equivalent up to
              proto-tags $\theta$), where the Origin header is amended by
              the host and protocol of the original request.
      
              Also, we know from
              $e_i^{(1)} \prototagequiv{\theta} e_i^{(2)}$ that neither
              event may contain a subterm $l\in L$ or $t \in T$. Hence, the
              transferred (initial) scriptstate (or a request generated by a
              Location header, see below) cannot contain a subterm $l \in L$
              or $t \in T$.
      
              Now, assuming that the domain in the Location header was not
              $\str{CHALLENGE}$, then the new request is term-equivalent
              under $\theta$ between both browsers. A new HTTP(S) request is
              generated (which conforms to Condition~\ref{eqe:prototagequiv}
              of Definition~\ref{def:Events}). It is clear that in this case, the conditions for
              $\gamma$-equivalence under $(\theta, H)$ are satisfied. 
              The same number of nonces is chosen. Altogether, $\alpha$-equivalence is given.
      
              If, however, the domain is $\str{CHALLENGE}$ (and the browser
              has not started a request to $\str{CHALLENGE}$ before; in this
              case the browser would behave as above), then the domain is
              $\mi{dr}_1$ in $b_1$ and $\mi{dr}_2$ in $b_2$. In particular,
              in the resulting requests, the Host header is exchanged in
              this way. For alpha equivalence to hold for the new
              configuration, we have $H' = H \cup \{n\}$, where $n$ is the
              nonce chosen for the HTTP(S) request. A new HTTP(S) request is
              generated. Therefore, we have
              $\gamma$-equivalence under $(\theta, H')$ and
              $\beta$-equivalence under $(\theta, H', L)$. The same number
              of nonces is chosen, and we indeed have $\alpha$-equivalence.
      
              If there is no Location header in $m_1$ (and therefore none in
              $m_2$), a new document is constructed just as in the case when
              the nonce in $m_1$ is in $H$.
      
              The scriptstate is initially equal, and the script inputs are
              empty. The document's referer is constructed from the referer
              header of the request, which is equal in both cases (up to
              proto-tags in $\theta$).
      
              To sum up, $\gamma$-equivalence under $(\theta, H)$ is
              preserved in this case as well. $\alpha$-equivalence is
              preserved as no output event is generated and the exact same
              number of nonces are chosen.
          \end{enumerate}
        \item\label{browser-http-response-xhr}
          In Case~(II), i.e., the response is a response to an XHR, 
          we have that $\mi{reference}$ is a tupel, say,
          $\mi{reference} = \an{\mi{docnonce}, \mi{xhrref}}$, 
          and we again distinguish between the two cases as above:
          \begin{enumerate}
            \item The HTTP nonce in $m_1$ is in $H$: In this case, only
              Case~\ref{eqe:http-res} of Definition~\ref{def:Events}
              can apply. We therefore have that there is no Location,
              Set-Cookie or Strict-Transport-Security header in the
              response, and that the responses $m_1$ and $m_2$ are equal up
              to proto-tags in $\theta$.
      
              With this, we observe that both browsers either accept and
              successfully decrypt the messages and call the function
              $\mathsf{PROCESSRESPONSE}$, or both browsers stop with not
              state change and no output event (in which case the
              $\alpha$-equivalence is given trivially). In particular we
              note that the expected sender in both cases matches precisely
              the sender of the message (compare Case~\ref{eqe:http-res} of
              Definition~\ref{def:Events}).
      
              In $\mathsf{PROCESSRESPONSE}$, we see that no changes in the
              browsers' cookies are performed (as no cookies are in the
              response), the $\str{sts}$ subterm is not changed, and no
              redirection is performed (as no Location header is present).
      
              A new input is constructed for the document that is identified
              by $\mi{docnonce}$. We note that such a document exists either
              in both browsers or in none (in which, again, both browsers
              stop with no output or state change). As the input events may
              contain a subterm $l \in L$ (as we know from HTTP nonce in
              $m_1$ being in $H$ that the host of this document is
              $\mi{dr}_1$ in $b_1$ and $\mi{dr}_2$ in $b_2$), the
              constructed scriptinput may also contain a subterm $l \in L$.
      
              For $j \in \{1,2\}$, we have that the $\str{scriptinput}$ term
              for the document in $b_j$ is $\an{\tXMLHTTPRequest,
                \mi{g_j}.\str{body}, \mi{xhrref}}$, where $g_j$ is the HTTP
              body of $m_j$.  With $g_1 \prototagequiv{\theta} g_2$ and
              $\mi{xhrref} \in \nonces \cup \{\bot\}$, it is easy to see
              that the resulting $\str{scriptinput}$ term of the document is
              term-equivalent under proto-tags $\theta$ (as it was before).
              This satisfies $\gamma$-equivalence on the new browser state.
      
              No output event is generated, and no nonces are chosen.
              Therefore we have $\alpha$-equivalence on the new
              configuration.
      
            \item The HTTP nonce in $m_1$ is not in $H$: In this case we
              have that $e_i^{(1)} \prototagequiv{\theta} e_i^{(2)}$
              (Case~\ref{eqe:prototagequiv} of
              Definition~\ref{def:Events}), and that the HTTP nonces,
              senders, encryption keys (if any) and original requests of both browsers are either equal or
              equivalent up to proto-tags $\theta$.

              With this, we observe that both browsers either accept and
              successfully decrypt the messages and call the function
              $\mathsf{PROCESSRESPONSE}$, or both browsers stop with not
              state change and no output event (in which case the
              $\alpha$-equivalence is given trivially). In particular we
              note that the expected sender in both cases matches precisely
              the sender the message has (as it is equal).
      
              If there is a Set-Cookie header in one of the responses, a new
              entry in the cookies of each browsers is created (which
              obviously is term-equivalent up to $\theta$, and therefore is
              in compliance with the requirements for $\gamma$-equivalence).
              The same holds true for any Strict-Transport-Security headers.
      
              Now, if there is a Location header in $m_1$ (and therefore
              also in $m_2$), both browsers stop with not state change and
              no output event (in which case the $\alpha$-equivalence is
              given trivially), as XHR cannot be redirected in the browser.
      
              If there is no Location header in $m_1$ (and therefore none in
              $m_2$), a new input is constructed for the document that is
              identified by $\mi{docnonce}$. We note that such a document
              exists either in both browsers or in none. For $j \in
              \{1,2\}$, we have that the $\str{scriptinput}$ for the
              document in $b_j$ is $\an{\tXMLHTTPRequest,
                \mi{g_j}.\str{body}, \mi{xhrref}}$, where $g_j$ is the HTTP
              body of $m_j$. With $e_i^{(1)} \prototagequiv{\theta}
              e_i^{(2)}$ (which may not contain a subterm $l \in L$ or $t \in T$), it is
              easy to see that the resulting $\str{scriptinput}$ term of the
              document is term-equivalent under proto-tags $\theta$ (as it
              was before). This satisfies $\gamma$-equivalence on the new
              browser state.
      
              No output event is generated, and no nonces are chosen.
              Therefore we have $\alpha$-equivalence on the new
              configuration.
          \end{enumerate}
        \end{enumerate}
      \oldc
      
      \item[TRIGGER] We now distinguish between the possible 
        values for $\mi{cmd}_\text{switch}$.
        \begin{description}
        \item[1 (trigger script):] In this case, the script in 
          the window indexed by $\mi{cmd}_\text{window}$ is 
          triggered. Let $j$ be a pointer to that window.
  
          We first note that such a window exists in $b_1$ iff 
          it exists in $b_2$ and that $S_1(b_1).j.\str{script} 
          \equiv S_2(b_2).j.\str{script}$. We now distinguish 
          between the following cases, which cover all possible 
          states of the windows/documents:
  
          \begin{enumerate}
          \item
            $S_1(b_1).j.\str{origin} \in \{\an{\mi{dr}_1, 
            \https}, \an{\mi{dr}_2, \https}\}$ and 
            $S_1(b_1).j.\str{script} \equiv \str{script\_rp}$.
  
            Similar to the following scripts, the main 
            distinction in this script is between the script's 
            internal states (named $\str{phase}$). With the 
            term-equivalence under proto-tags $\theta$ we have 
            that either 
            $S_1(b_1).j.\str{scriptstate}.\str{phase} =
             S_2(b_2).j.\str{scriptstate}.\str{phase}$ or the 
            script's state contains a tag and is therefore in an 
            illegal state (in which case the script will stop 
            without producing output or changing its state).
  
            We can therefore now distinguish between the 
            possible values of
            $S_1(b_1).j.\str{scriptstate}.\str{phase} =
             S_2(b_2).j.\str{scriptstate}.\str{phase}$:
            \begin{description}
            \item[start:] In this case, the script open a blank
              page addressed to its own origin, which is either 
              (a) equal and $\an{\mi{dr}_1,\https}$ or 
              $\an{\mi{dr}_2,\https}$ or it is 
              (b) $\an{\mi{dr}_1,\https}$ in $b_1$ and
              $\an{\mi{dr}_2,\https}$ in $b_2$. The path is the 
              (static) string $\str{/loginSSO}$. The script 
              saves a (static) value for $\str{phase}$ in its 
              scriptstate.
  
              In both Cases, we have that the command is 
              term-equivalent under proto-tags $\theta$ and 
              hence, the browser emits a HTTP request which is 
              term-equivalent.Hence, we have 
              $\gamma$-equivalence under $(\theta,H)$ for the 
              new states, $\beta$-equivalence under 
              $(\theta,H,L)$ for the new events, and 
              $\alpha$-equivalence for the new configuration.
            
            \item[expectt:] In this case, the script retrieves 
              the result of a \pm from $\mi{scriptinputs}$. As 
              we know that $S_1(b_1).j.\str{scriptstate} 
              \prototagequiv{\theta} 
              S_2(b_2).j.\str{scriptstate}$ and that for all 
              matching \pms that they also have to be 
              term-equivalent up to $\theta$ and that the window 
              structure is equal in both browsers, we have that 
              either the same \pm is retrieved from 
              $\mi{scriptinputs}$ or none in both browsers.
  
              Then the script saves a (static) value for 
              $\str{phase}$ in its scriptstate, and we set 
              $H' := H \cup \{n\}$ with $n$ being the nonce 
              that the browser chooses for $\lambda_1$. 
              Therefore, we have $\gamma$-equivalence under 
              $(\theta,H')$ for the new states. We also have
              $\beta$-equivalence under $(\theta,H',L)$ for the 
              new events, and $\alpha$-equivalence for the new 
              configuration.
            \item[expectCert:] In this case, the script
              retrieves the result of an \xhr from 
              $\mi{scriptinputs}$ that matches the reference 
              contained in $\mi{scriptstate}$. From
              Condition~\ref{eqs:b:w:script_rp} of
              Definition~\ref{def:eq-of-states} we know that all 
              results from \xhr{}s in $\mi{scriptinput}$ are 
              term-equivalent up to $\theta$ and that 
              $\mi{scriptstate}$ is term-equivalent up to $\theta$. 
              Hence, in both browsers, both scripts stop with an 
              empty command or both continue as they successfully
              retrieved such an \xhr.
  
              The script now constructs a \pm that is sent to 
              exactly the same window in both browsers and that 
              requires that the receiver origin has to be 
              $\an{\str{IdPdomain},\https}$ The postMessage is 
              only sent to this origin, we have that 
              $\gamma$-equivalence cannot be violated.
  
              We now have that $S_1'$ and $S_2'$ are 
              $\gamma$-equivalent under $(\theta,H)$, 
              $E_1'$ and $E_2'$ are $\beta$-equivalent under 
              $(\theta,H,L)$, and as exactly none of nonces is 
              chosen, we have that the new configuration is 
              $\alpha$-equivalent.
            \item[expectToken:] This case is the same as
              $\str{expectt}$ and we have that the new 
              configuration is $\alpha$-equivalent.
            \end{description}
  
          \item $S_1(b_1).j.\str{origin} \not\in 
            \{\an{\mi{dr}_1,\https},\an{\mi{dr}_2,\https}\}$. 
            $S_1(b_1).j.\str{script} \equiv \str{script\_idp}$.
            
            Unlike SPRESSO, $\str{script_{idp}}$ is trustful in
            UPPRESSO due to the use of SRI (Subresource Integrity).
            Because of this check, browsers can control the 
            content of $\str{script_{idp}}$ downloaded from 
            the Identity Provider. Hence, we now analyze every 
            internal state just like in $\str{script_{rp}}$.
  
            \begin{description}
            \item[start:] In this case, the script chooses a
              new nonce for $t$. Since $t$ is only stored in
              $S_1(b_1).j.\str{scriptstate}.\mi{parameters}$,
              the condition~\ref{eqs:att-not-t} and 
              condition~\ref{eqs:b:w:att_script:t} of 
              Definiton~\ref{def:eq-of-states} hold. Hence, 
              we have $\gamma$-equivalence under $(\theta,H)$ 
              for the new states.
  
              From the equivalence definition of states
              (Definition~\ref{def:eq-of-states}) we can see 
              that the window tree has the same structure in 
              both processing steps. 
              So the script now constructs a \pm that is sent to 
              exactly the same window in both browsers and that 
              requires that the receiver has to be the opener
              of this window. Since the new tag hasn't been 
              generated and Condition~\ref{eqe:prototagequiv} 
              of Definition~\ref{def:Events} holds, we have 
              $\beta$-equivalence under $(\theta,H,L)$.
            \item[expectCert:] The same as above, we can have 
              that either the same \pm is retrieved from 
              $\mi{scriptinputs}$ or none in both browsers and 
              the result of $checksig$ is same as well. The 
              state $Cert_{rp}$ is equal in both 
              $\str{scriptstate}$ and the state $PID_{rp}$ is
              term-equivalent under $\theta$. As the 
              condition~\ref{eqs:b:w:att_script:state} of 
              Defition~\ref{def:eq-of-states} holds, we have 
              $\gamma$-equivalence under $(\theta,L)$ for the 
              new states.
  
              Obviously, we have $\beta$-equivalence under 
              $(\theta,H,L)$ as Condition~\ref{eqe:prototagequiv} 
              of Definition~\ref{def:Events} holds.
  
            \item[expectReqToken:] In this case, the script
              retrieves the result of an \xhr from 
              $\mi{scriptinputs}$ that matches the reference 
              contained in $\mi{scriptstate}$. From
              Condition~\ref{eqs:b:w:script_rp} of
              Definition~\ref{def:eq-of-states} we know that all 
              results from \xhr{}s in $\mi{scriptinput}$ are 
              term-equivalent up to $\theta$ and that 
              $\mi{scriptstate}$ is term-equivalent up to 
              $\theta$. Hence, in both browsers, both scripts 
              will reach the same if-else branch.
  
              Since there aren't any new states stored and the
              requests' destination are fixed, we can have 
              $\gamma$-equivalence under $(\theta,L)$ for the 
              new states and $\beta$-equivalence under 
              $(\theta,H,L)$. 
              \item[expectLoginResult:] This case is the same as 
                the second branch of $\str{expectReqToken}$.
              \item[expectToken:] This case is the same as 
                the third branch of $\str{expectReqToken}$.
            \end{description}
          \end{enumerate}
        \item[2 (navigate to URL):] 
        In this case, a new window is opened
        in the browser and a document is loaded from $\mi{url}$.
  
        The states of both browsers are changed in the same way except
        if the domain of the URL is $\str{CHALLENGE}$. In both cases, a
        new (at this point empty) window is created and appended the
        $\str{windows}$ subterm of the browsers. This subterm is
        therefore changed in exactly the same way.
  
        A new HTTP request is created and generated requests in 
        both processing
        steps can only differ in the host part iff the domain is
        $\str{CHALLENGE}$. In this case, in $b_1$ the domain is replaced
        by $\mi{dr}_1$ and in $b_2$ by $\mi{dr}_2$ and the
        $\alpha$-equivalence in the following holds for $H' := H \{n\}$,
        where $n$ is the nonce of the generated HTTP request.
  
        The request cannot contain any $l \in L$ or $t \in T$.
        and 
        
        the Condition~\ref{eqe:prototagequiv} of 
        Definition~\ref{def:Events}.
  
        In both processing steps, three nonces are chosen.
  
        Therefore, we have $\alpha$-equivalence for $(S_1',E_1',N_1')$
        and $(S_2',E_2',N_2')$.
        \item[3 (reload document):]
        Here, an existing document is
        retrieved from its original location again. From the definition
        of $\gamma$-equivalence under $(\theta,L)$ we can see that
        whatever document is reloaded, its location is either (I)
        term-equivalent under $\theta$, or (II) it is term-equivalent
        under $\theta$ except for the domain, which is $\mi{dr}_1$ in
        $b_1$ and $\mi{dr}_2$ in $b_2$. 
  
        We note that in either case, the requests are constructed from
        the location and referrer properties of the document that is to
        be reloaded, and therefore, cannot contain any $t\in T$.
  
        In Case~(I), we note that the domain cannot be
        $\str{CHALLENGE}$. If the document is reloaded, the same 
        request is issued in both browsers (therefore,
        $\beta$-equivalence under $(\theta, H, L)$ is given), and 
        none states are changed such that we have
        $\gamma$-equivalence under $(\theta, L)$. The same number of
        nonces is chosen in both runs, and we have
        $\alpha$-equivalence.
  
        Case~(II) is similar, but we have $H' := H \cup \{n\}$, where
        $n$ is the nonce of the HTTP request. 
        Then we have $\beta$-equivalence under
        $(\theta,H',L)$. Again, the same number of nonces is chosen and
        we have $\alpha$-equivalence. 
        \end{description}
      \item[Other] Any other message is discarded by the browsers 
        without any change to state or output events.
    \end{description}
  
  %lemma放在后面
  %需要在with case5部分添加补充说明,说明每个case具体说明了什么。
  %在详细说说semi-honest的IdP
  %只要保留web attacker
  %lemma16纯数学表达
    \paragraph{\underline{Case $p_1$ is some attacker:}}

    \newc
    When $p_1$ is some attacker, the most noticeable party we talk about here is IdP.
    In UPPRESSO, there is a centralized identity provider so we think that IdP is always honest but curious.
    Therefore, when we analyze IdP's processing steps, 
    we assume that the output events from IdP follow our models' design.
    However, IdP may be curious and derive something in its states that destroys our privacy property, 
    so what we try to do is to prove that with the equivalent states and events, 
    two systems can only process another equivalent states and events.
    \oldc

    Here, only Case~\ref{eqe:prototagequiv} from Definition~\ref{def:Events} can apply to the input events,
    i.e., the input events are term-equivalent under proto-tags $\theta$. 
    This implies that the message was delivered to the same attacker process in both processing steps. 
    
    \newc
    Let $A$ be that attacker process. 
    With Case~\ref{eqs:att} of Definition~\ref{def:eq-of-states} we have that $S_1(A) \prototagequiv\theta S_2(A)$.
    With Case~\ref{eqe:pre:t} of Definition~\ref{def:Events} and Case~\ref{eqs:att-not-t} of Definition~\ref{def:eq-of-states}, 
    we have that neither the states of A, i.e, $S_1(A)$ and $S_2(A)$ contain the nonce t, nor does the events $e_i^{(1)}$ and $e_i^{(2)}$.
    Hence with lemma~\ref{thm-idp-untraceability-new}, it follows immediately that the attacker cannot distinguish any of the tags in $\theta$ in its knowledge.
    Therefore, if two states or events are term-equivalent, the attacker process $A$ cannot distinguish between them.
    \oldc

    Obviously, there are no variables (from $V_\text{process}$) in the attackers' states.
    With the output term being a fixed term (with variables)
    $\tau_{\text{process}} \in \terms(\{x\} \cup V_\text{process})$ 
    and $x$ being $S_1(A)$ or $S_2(A)$ respectively, 
    there is no subterm $l \in L$ contained in either $S_1(A)$ or $S_2(A)$ 
    (Case~\ref{eqs:att-not-l} of Definition~\ref{def:eq-of-states}), 
    it is easy to see that only Case~\ref{eqe:prototagequiv} from Definition~\ref{def:Events} can apply to the output events 
    and the output events are $\beta$-equivalent under $\theta$, 
    i.e., $E ^{(1)}_\text{out} \prototagequiv\theta E^{(2)}_\text{out}$. 
    There are not any $t \in T$ contained in the output events meaning that 
    the new state of the attacker in both processing steps is $\beta$-equivalent under proto-tags $\theta$.
    The used nonces are the same, i.e., $N_1' = N_2'$. 
    Therefore we have $\alpha$-equivalence on the new configurations.
  \end{proof}

  %\begin{lemma}\label{thm-idp-untraceability}
  %  Given a point on the elliptic curve denoted by $[r]G$, 
  %  an adversary cannot distinguish $[tr]G$ from a random variable on $\mathbb{E}$, 
  %  where $t$ is random in $\mathbb{Z}_n$ and unknown to the adversary.
  %\end{lemma}
  %\begin{proof}
  %  Consider a finite cyclic group $\mathbb{E}$ where the number of points on $\mathbb{E}$ is $n$. 
  %  Because $G$ is a generator of order $n$, $[r]G$ is also a generator on $\mathbb{E}$ of order $n$. 
  %  $t$ is randomly chosen in $\mathbb{Z}_n$ and always kept unknown to the adversary. 
  %  Therefore, $[tr]G$ is \emph{indistinguishable} from a point $Q$ that is randomly chosen on $\mathbb{E}$.\cite{oprf-proved,voprf-proved}.
  %\end{proof}
  
  This proves Theorem~\ref{theorem:A}.\QED
  
  \section{Proof of Privacy against RP-based Identity Linkage}
  
  \subsection{Formal Model of UPPRESSO for Privacy Analysis}
  
  In our privacy analysis, we show that malicious RPs colluded 
  with each other in UPPRESSO cannot infer the identities of 
  honest users. We formalize this property as an 
  indistinguishability property: two relying parties (modeled as 
  a web attacker) cannot distinguish whether a user logging in 
  at one relying party also logs in the other relying party.
  
  We will here first describe the precise model that we use for 
  privacy analysis. After that, we define an equivalence 
  relation between configurations, which we will then use in the 
  proof of privacy.
  
  %privacy证明中,虽然会corrupt,但是不会打破authentication property
  %把rp归入web attacker,不需要加malicious
  %1.对于两个malicious rp,任何user的login instance不能区分是同一个人的还是不是
  %2.malicious user告诉rp是同一个login instance,
  %benign user对于rp是不可区分的
  %IDToken equivalence:由同一个user对两个RP分别产生的两个IDToken不可区分
  \begin{definition}[Challenge IdP]
    Let $\mi{dr}$ some domain, $u$ some identity and $\mi{idp}(\mi{dr}, u)$ a DY process. 
    We call it a \emph{challenge IdP} iff $\mi{idp}$ is defined exactly the same as a identity server with two exceptions: 
    (1) the state contains one more property, namely $\mi{challenge}$, which initially contains the term $\top$. 
    (2) The IdP's algorithm is modified by the following at line~\ref{line:uppresso-idp-set-pidu} in algorithm~\ref{alg:idp}: 
    It is checked if the login request $m$ is addressed to the domain $\mi{dr}$ and no other message $m'$ was addressed to this domain before.(i.e., $\mi{challenge} \not\equiv \bot$)
    Then the $PID_u$ is generated using the given $u$ and $\mi{challenge}$ is set to $\bot$ to recorded that a message was addressed to $\mi{dr}$. 
  \end{definition}
  
  %B->b1
  \begin{definition}[\uppresso Web System for Privacy Analysis]\label{def:uppresso-ws-priv}
    Let $\uppressowebsystem = (\bidsystem, \scriptset, \mathsf{script}, E^0)$ be an UPPRESSO web system with 
    $\bidsystem = \mathsf{Hon} \cup \mathsf{Web} \cup \mathsf{Net}$, 
    $\mathsf{Hon} = \fAP{B} \cup \fAP{RP} \cup \fAP{IDP}$. (as described in Appendix~\ref{app:outlineuppressomodel}).
    $\fAP{RP} = \{r_1,r_2\}$, $r_1$ and $r_2$ two relying parties, 
    Let $\fAP{attacker} = \{r_1,r_2\}$ be some web attacker.
    Let $\mi{dr}$ be the domain of $r_2$, $u_x$ an identity owned by the browser $b$ 
    and $u_y$ an identity only known by the IdP, then $\mi{idp}_c = \mi{idp}(\mi{dr}, u_x/u_y)$ is a challenge IdP. 
    Let $\mathsf{Hon}' := \fAP{B} \cup \{\mi{idp}_c\}$, $\mathsf{Web}' := \mathsf{Web}$, and $\mathsf{Net}' := \emptyset$ (i.e., there is no network attacker).
    Let $\bidsystem' := \mathsf{Hon}' \cup \mathsf{Web}' \cup \mathsf{Net}'$. 
    Let $\scriptset' := \scriptset$ and $\mathsf{script}'$ be accordingly.
    We call $\uppressoprivwebsystem(\mi{dr}, u_x, u_y) = (\bidsystem', \scriptset', \mathsf{script}', E^0, \fAP{attacker})$ 
    an \emph{\uppresso web system for privacy analysis} 
    iff the domain $\mi{dr}$ the only domain assigned to $r_2$. 
    All honest parties (in $\mathsf{Hon}$) are not corruptible, i.e., they ignore any $\str{CORRUPT}$ message. 
    Relying Parties are assumed to be dishonest, and hence, are subsumed by the web attackers.
  \end{definition}
  
  \begin{definition}[RP-Privacy]\label{def:rp-privacy}
    Let 
    \begin{align*}
      \uppressoprivwebsystem_1 := \uppressoprivwebsystem(\mi{dr}, u_x) =
      (\bidsystem_1, \scriptset, \mathsf{script}, E^0, \fAP{attacker}_1)&\text{ and}\\
      \uppressoprivwebsystem_2 := \uppressoprivwebsystem(\mi{dr}, u_y) =
      (\bidsystem_2, \scriptset, \mathsf{script}, E^0, \fAP{attacker}_2)&
    \end{align*}
    be \uppresso web systems for privacy analysis. 
    Further, we require $\fAP{attacker}_1 = \fAP{attacker}_2 =: \fAP{attacker}$ 
    and for $idp_1 := idp(\mi{dr}, u_x)$, $idp_2 := idp(\mi{dr}, u_y)$, 
    we require $S(idp_1) = S(idp_2)$ and $\bidsystem_1 \setminus \{idp_1\} = \bidsystem_2 \setminus \{idp_2\}$ 
    (i.e., the web systems are the same up to the parameter of the challenge idps).  
    We say that $\uppressoprivwebsystem$ is \emph{RP-private} iff $\uppressoprivwebsystem_1$ and $\uppressoprivwebsystem_2$ are indistinguishable.
  \end{definition}
  
  \subsection{Definition of Equivalent Configurations}\label{app:rp:defin-equiv-stat}
  
  Let $\uppressoprivwebsystem_1 = (\bidsystem_1, \scriptset, \mathsf{script}, E^0, \fAP{attacker})$ 
  and $\uppressoprivwebsystem_2 = (\bidsystem_2, \scriptset, \mathsf{script}, E^0, \fAP{attacker})$ 
  be \uppresso web systems for privacy analysis. 
  Let $(S_1,E_1,N_1)$ be a configuration of $\uppressoprivwebsystem_1$ 
  and $(S_2,E_2,N_2)$ be a configuration of $\uppressoprivwebsystem_2$.
  
  \begin{definition}[Proto-Accts]
    We call a term of the form $[u]ID_{r_2}$ with the variable
    $u$ as a placeholder for a nonce, and 
    $ID_{r_2}\in K_\text{point}$ as the identity of 
    relying party $r_2$ \emph{proto-acct}.
  \end{definition}
  
  \begin{definition}[Term Equivalence up to Proto-Accts]
    Let $\theta = \{a_1, \ldots, a_l \}$ be a finite set of proto-accts.
    Let $t_1$ and $t_2$ be terms. We call $t_1$ and $t_2$
    \emph{term-equivalent under a set of proto-accts $\theta$} 
    iff there exists a term $\tau \in \terms(\{x_1,\dots,x_l\})$ such that
    $t_1 = (\tau [ a_1 / x_1 , \dots , a_l / x_l ])[ u_x / u ]$ and
    $t_2 = (\tau [ a_1 / x_1 , \dots , a_l / x_l ])[ u_y / u ]$.
    We write $t_1 \prototagequiv{\theta} t_2$.
  
    We say that two finite sets of terms $D$ and $D'$ are
    \emph{term-equivalent under a set of proto-tags $\theta$} 
    iff $|D| = |D'|$ and, given a lexicographic ordering of the 
    elements in $D$ of the form $(d_1,\dots,d_{|D|})$ and the 
    elements in $D'$ of the form $(d'_1,\dots,d_{|D'|})$, we 
    have that for all 
    $i \in \{1,\dots,|D|\}$: $d_i \prototagequiv{\theta} d'_i$. 
    We then write $D \prototagequiv{\theta} D'$.
  \end{definition}

  \begin{definition}
    Let $a$ be a proto-acct, $S_1$ and $S_2$ be states of \uppresso web systems for privacy analysis, and $l$ a nonce. 
    We call $l$ a login session token for the proto-acct $a$, written $l \in \mathsf{loginSessionTokens}(a,S_1,S_2)$ 
    iff for any $i \in \{1,2\}$ we have that $\proj{2}{S_i(\mi{idp}).\str{sessions}[l].\mi{IDToken}} = a[ID_{u_{x/y}}/u]$.
  \end{definition}
  
  \begin{definition}[Equivalence of States]\label{def:rp:eq-of-states}
    Let $\theta$ be a set of proto-tags. %and 
    %$H$ be a set of nonces. 
    %$L$ be a set of login session tokens.
    %Let $T:=\{t\mid [t]R\in \theta\}$. 
    We call $S_1$ and $S_2$ \emph{$\gamma$-equivalent under 
    $\theta$} iff the following conditions are met:
    \begin{enumerate}
    \item\label{eqs:rp:idp} 
      $S_1(\mi{idp_1})$ equals $S_2(\mi{idp_2})$ except
      for the subterms $\str{sessions}$, and
    \item\label{eqs:rp:idp-sessions} 
      $S_1(\mi{idp_1}).\str{sessions} \prototagequiv{\theta}$ 
      $S_2(\mi{idp_2}).\str{sessions}$, and
    \item\label{eqs:rp:att-unknown}
      $u_x, u_y, r_1, r_2 \not\in $
      $d_\emptyset(\bigcup_{i\in\{1,2\},A\,\in\,\mathsf{Web}}S_i(A))$, and
    \item\label{eqs:rp:att} 
      for each attacker $A$:
      $S_1(A) \prototagequiv{\theta} S_2(A)$, and
    \item\label{eqs:rp:att-not-l} 
      for all $a\in\theta$ and all attackers $A$ we have that
      $\nexists\ l \in \mathsf{loginSessionTokens}(a,S_1,S_2)$ 
      such that $l$ is a subterm of $S_1(A)$ or $S_2(A)$.
    \item\label{eqs:rp:b} 
      $S_1(b_1)$ equals $S_2(b_1)$ except for the subterms 
      $\str{windows}$ and we have that
      \begin{enumerate}
      \item\label{eqs:rp:b:w}
        $S_1(b_1).\str{windows}$ equals $S_2(b_2).\str{windows}$ 
        with the exception of the subterms $\str{location}$, $\str{referrer}$, $\str{scriptstate}$ 
        and $\str{scriptinputs}$ of some document terms pointed 
        to by $\mathsf{Docs}^+(S_1(b_1)) = \mathsf{Docs}^+(S_2(b_1)) =: J$. 
        For all $j \in J$ we have that: 
        \begin{enumerate}
        \item there is no $x\in\myangle{u_x, u_y, r_1, r_2}$ such that
          \begin{align*}
            x \in d_{\nonces \setminus \{x\}}(\{&S_1(b_1).j.\str{location}
            ,  S_2(b_2).j.\str{location},\\ & S_1(b_1).j.\str{referrer} , 
            S_2(b_2).j.\str{referrer}\})
          \end{align*}
        \item \label{eqs:rp:b:w:scriptinputs} for $p \in \{$
          \begin{align*}
            & \an{\tXMLHTTPRequest,*,*},\\
            & \an{\tPostMessage,*,\an{\mapDomain(r_{1,2}), \https},\an{\str{t}, *}},\\
            & \an{\tPostMessage,*,\an{\mapDomain(r_{1,2}), \https},\an{\str{IDToken}, *}}\\
            & \an{\tPostMessage,*,\an{\mapDomain(idp), \https},\an{\str{Cert}, *}}
          \end{align*}
          $\}$ we have

          $S_1(b_1).j.\str{scriptinputs} |\, p \prototagequiv{\theta}
          S_2(b_2).j.\str{scriptinputs} |\, p$, and
        \item\label{eqs:rp:b:w:script}
          $S_1(b_1).j.\str{scriptstate} \prototagequiv{\theta}
          S_2(b_2).j.\str{scriptstate}$, and 
        \end{enumerate}
      \item\label{eqs:b:misc} for
        $x \in \{\str{cookies},\str{localStorage},\str{sessionStorage},\str{sts}\}$
        we have that $S_1(b_1).x \prototagequiv{\theta} S_2(b_2).x$.
      \end{enumerate}
    \end{enumerate}
  \end{definition}
  
  \begin{definition}[Equivalence of Events]\label{def:rp:Events}
    Let $\theta$ be a set of proto-tags, 
    $L$ be a set of login session tokens, 
    %$H$ be a set of nonces, and
    %$T:=\{t\mid [t]R\in \theta\}$. 
    We call $E_1 = (e_1^{(1)}, e_2^{(1)}\dots)$ and
    $E_2= (e_1^{(2)}, e_2^{(2)} \dots)$ 
    \emph{$\beta$-equivalent under $(\theta, L)$} 
    iff all of the following conditions are satisfied for every 
    $i \in \mathbb{N}$:
  
    \begin{enumerate}
      \item\label{eqe:rp:distinction} 
        $e_i^{(1)} \prototagequiv{\theta} e_i^{(2)}$, and
      \item\label{eqe:rp:pre:l} If there exists $l \in L$ such that $l$ is a
        subterm of $e_i^{(1)}$ or $e_i^{(2)}$ then we have that
        $e_i^{(1)}$ is a message from $b_1$ to $\mi{idp}$ and $e_i^{(2)}$ is a
        message from $b_2$ to $\mi{idp}$ or we have that $e_i^{(1)}$ is a
        message from $\mi{idp}$ to $b_1$ and $e_i^{(2)}$ is a message from
        $\mi{idp}$ to $b_2$.
      \item\label{eqe:rp:pre:t} $u_x, u_y, r_1, r_2 \not\in e_i^{(1)}, e_i^{(2)}$
      \item\label{eqe:rp:pre:rp-scripts} If $e_i^{(1)}$ or $e_i^{(2)}$ is an
        HTTP(S) response with body $g$ from a identity provider, then it does
        not contain any $\str{Location}$ or $\cSTS$ header
        and if $\proj{1}{g}$ is a string representing a script, then
        $\proj{1}{g}$ is $\str{script\_idp}$.
      \item\label{eqe:rp:pre:unencrypted-http} 
        If $e_i^{(1)}$ or $e_i^{(2)}$ is an unencrypted HTTP 
        response, then the message was sent by some attacker.
    \end{enumerate}
  \end{definition}
  
  \begin{definition}[Equivalence of Configurations]
    We call $(S_1,E_1,N_1)$ and $(S_2,E_2,N_2)$
    \emph{$\alpha$-equivalent} iff there exists a set of 
    proto-accts $\theta$ such that $S_1$ and $S_2$ are
    $\gamma$-equivalent under $\theta$, $E_1$ and $E_2$ are
    $\beta$-equivalent under $(\theta,L)$ 
    for $L := \bigcup_{a\in\theta} \mathsf{loginSessionTokens}(a,S_1,S_2)$, 
    and $N_1 = N_2$.
  \end{definition}
  
  \subsection{Privacy Proof}
  
  \begin{theorem} \label{theorem:rp-privacy}
    Every UPPRESSO web system for privacy analysis is RP-private.
  \end{theorem}
  
  Let $\mathcal{U\!W\!S}^{priv}$ be UPPRESSO web system for privacy analysis.
  To prove Theorem \ref{theorem:rp-privacy}, first we have to show that 
  the UPPRESSO web systems $\mathcal{U\!W\!S}^{priv}_1$ and 
  $\mathcal{U\!W\!S}^{priv}_2$ are indistinguishable. To show 
  the indistinguishability of $\mathcal{U\!W\!S}^{priv}_1$ and 
  $\mathcal{U\!W\!S}^{priv}_2$, we show that they are 
  indistinguishable under all schedules $\sigma$. For this, 
  we first note that for all $\sigma$, there is only one run 
  induced by each $\sigma$(as our web system, when scheduled, is deterministic).
  We now proceed to show that for all schedules $\sigma=(\zeta _1, \zeta_2,\dots)$, 
  iff $\sigma$ induces a run $\sigma(\mathcal{U\!W\!S}^{priv}_1)$ 
  there exists a run $\sigma(\mathcal{U\!W\!S}^{priv}_2)$ 
  such that $\sigma(\mathcal{U\!W\!S}^{priv}_1)\approx\sigma(\mathcal{U\!W\!S}^{priv}_1)$
  
  We now show that if two configurations are $\alpha$-equivalent, 
  then the view of the attacker is statically equivalent.
  
  \begin{lemma}\label{lemma:statically-equivalent}
    Let $(S_1,E_1,N_1)$ and $(S_2,E_2,N_2)$ be two 
    $\alpha$-equivalent configurations. 
    Then $S_1(attacker)\approx S_2(attacker)$.
  \end{lemma}
  \begin{proof}
    From the $\alpha$-equivalence of $(S_1,E_1,N_1)$ and 
    $(S_2,E_2,N_2)$ it follows that $S_1(\fAP{attacker}) 
    \prototagequiv{\theta} S_2(\fAP{attacker})$.
    From Condition~\ref{eqs:rp:att-unknown} for 
    $\gamma$-equivalence it follows that
    $u_x, u_y, r_1, r_2 \not\in $
    $d_\emptyset(\bigcup_{i\in\{1,2\},A\,\in\,\mathsf{Web}}S_i(A))$ 
    (i.e., the attacker does not know any keys for the accts 
    contained in its view), and therefore it is easy to see 
    that the views are statically equivalent.
  \end{proof}
  
  We now show that $\sigma(\uppressoprivwebsystem_1) \approx
  \sigma(\uppressoprivwebsystem_2)$ by induction over the length 
  of $\sigma$. 
  We first, in Lemma~\ref{lemma:rp:initial-config-private}, 
  show that $\alpha$-equivalence (and therefore, 
  indistinguishability of the views of $\fAP{attacker}$) holds 
  for the initial configurations of $\uppressoprivwebsystem_1$ 
  and $\uppressoprivwebsystem_2$. 
  We then, in Lemma~\ref{lemma:rp:step-config-private}, 
  show that for each configuration induced by a processing step 
  in $\zeta$, $\alpha$-equivalence still holds true.
  
  \begin{lemma}\label{lemma:rp:initial-config-private}
    The initial configurations $(S_1^0,E^0,N^0)$ of 
    $\mathcal{U\!W\!S}^{priv}_1$ and $(S_2^0,E^0,N^0)$ of 
    $\mathcal{U\!W\!S}^{priv}_2$ are $\alpha$-equivalent.
  \end{lemma}
  \begin{proof}
    We now have to show that there exists a set of proto-accts 
    $\theta$ such that $S_1^0$ and $S_2^0$ are 
    $\gamma$-equivalent under $\theta$, $E_1^0 = E^0$ and 
    $E_2^0 = E^0$ are $\beta$-equivalent under $\theta$.
  
    Let $\theta = \emptyset$. Obviously, both latter conditions 
    are true. For all parties $p \in \bidsystem_1 \setminus \{idp_1\}$, 
    it is clear that $S_1^0(p) = S_2^0(p)$. Also the states 
    $S_1^0(idp_1)$ and $S_2^0(b_2)$ are equal. Therefore, 
    all conditions of Definition~\ref{def:rp:eq-of-states} are 
    fulfilled. Hence, the initial configurations are 
    $\alpha$-equivalent.
  \end{proof}
  
  \begin{lemma}\label{lemma:rp:step-config-private}
    Let $(S_1,E_1,N_1)$ and $(S_2,E_2,N_2)$ be two 
    $\alpha$-equivalent configurations of 
    $\uppressoprivwebsystem_1$ and $\uppressoprivwebsystem_2$, 
    respectively. Let $\zeta = \an{\mi{ci},\mi{cp}, 
    \tau_\text{process}, \mi{cmd}_\text{switch}, 
    \mi{cmd}_\text{window},\tau_\text{script},\mi{url}}$
    be a web system command. Then, $\zeta$ induces a processing 
    step in either both configurations or in none. In the latter 
    case, let $(S_1',E_1',N_1')$ and $(S_2',E_2',N_2')$ be 
    configurations induced by $\zeta$ such that
    \[(S_1,E_1,N_1) \xrightarrow{\zeta} (S_1',E_1',N_1') \quad 
    \text{and} \quad (S_2,E_2,N_2) \xrightarrow{\zeta} 
    (S_2',E_2',N_2') \ .\]
    Then, $(S_1',E_1',N_1')$ and $(S_2',E_2',N_2')$ are
    $\alpha$-equivalent.  
  \end{lemma}
  \begin{proof}
    Let $\theta$ be a set of proto-tags for which 
    $\alpha$-equivalence holds.
    
    To induce a processing step, the ci-th message from $E_1$ or 
    $E_2$, respectively, is selected.Following Definition 
    \ref{def:Events}, we denote these messages by $e_i^{(1)}$ or 
    $e_i^{(2)}$, respectively. We now differentiate between the 
    receivers of the messages by denoting the induced processing 
    steps by
    \begin{equation}
      \begin{aligned}
        (S_1,E_1,N_1)\xrightarrow[p_1\rightarrow E_{out}^{(1)}]{\left \langle a_1,f_1,m_1\right \rangle\rightarrow p_1}(S_1\prime,E_1\prime,N_1\prime)\\
        (S_2,E_2,N_2)\xrightarrow[p_2\rightarrow E_{out}^{(2)}]{\left \langle a_2,f_2,m_2\right \rangle\rightarrow p_2}(S_2\prime,E_2\prime,N_2\prime)
      \end{aligned}
    \end{equation}
    \paragraph{\underline{Case $p_1 = \fAP{idp_1}$}}
    $\implies p_2 = \fAP{idp_2}$

    In this case, we distinct several cases of HTTP(S) requests that can happen.
    There are four possible types of HTTP requests that are accepted by $\mi{idp}_1$ in Algorithm \ref{alg:idp}:
    
    \begin{itemize}
      \item path=$\str{/script}$(get the idp-script), Line~\ref{line:idp-script};
      \item path=$\str{/authentication}$(set the login session), Line~\ref{line:idp-authentication};
      \item path=$\str{/reqToken}$(retrieve the IDToken), Line~\ref{line:idp-reqToken};
      \item path=$\str{/authorize}$(construct the IDToken), Line~\ref{line:idp-authorize}.
    \end{itemize}

    Here, Condition~\ref{eqe:rp:distinction} from Definition~\ref{def:rp:Events} applys to the input events,
    i.e., the input events are term-equivalent under proto-accts $\theta$. 
    \begin{itemize}
      \item path=$\str{/script}$ 
        In this case, the same output event is produced whose message is 
        \begin{equation}
          \begin{aligned}
            \myangle{HTTPResp,n,200,\myangle{},IdPScript}
          \end{aligned}
        \end{equation}
        We can note that Condition~\ref{eqe:rp:pre:rp-scripts} of Definition \ref{def:rp:Events} holds true and the remaining conditions are trivially fulfilled.
        Therefore, $E_1\prime$ and $E_2\prime$ are $\beta$-equivalent under $(\theta, L)$. 
        As there are no changes to any state, we have that $S_1\prime$ and $S_2\prime$ are $\gamma$-equivalent under $\theta$. 
        No new nonces are chosen, hence $N_1\prime=N_1=N_2=N_2\prime$.
      \item path=$\str{/authentication}$ 
        Here, since $e_i^{(1)} \prototagequiv{\theta} e_i^{(2)}$, we can see that the username and password in the HTTP body is equal. 
        Hence, the check at Line~\ref{line:uppresso-idp-check-login-state} in Algorithm~\ref{alg:idp} either both fail or not . 
        If the username and password are correct, IdP will set up a login session and output a respones whose message is 
        \begin{align*}
          \myangle{\cHttpResp,n,200,\an{\mi{setCookie}},\str{LoginSuccess}}
        \end{align*}
        with
        \begin{align*}
          \mi{setCookie} := \myangle{\cSetCookie, \myangle{\myangle{\str{sessionid}, \nu_3, \True, \True, \True}}} \\
        \end{align*}
        We can note that Condition~\ref{eqs:rp:idp-sessions} of Definition~\ref{def:rp:eq-of-states} holds true for the new sates of $\mi{idp}_1$ and $\mi{idp}_2$ because only a username is added to the sessions.
        As there are not any other changes to the state and $\theta\prime = \theta$, we have that $S_1\prime$ and $S_2\prime$ are $\gamma$-equivalent under $\theta$. 
        
        For $N_1 = N_2 = (n_1, n_2, \dots)$, we set $N_1' = N_2' = (n_2, \dots)$ (as exactly one nonce is chosen in both processing steps) and $L' = L \cup \{n_1\}$. 
        So Condition~\ref{eqe:rp:pre:l} of Definition~\ref{def:rp:Events} holds true the output event.
        Obviously we can have that Condition~\ref{eqe:rp:pre:t} of Definition~\ref{def:rp:Events} holds true, so $E_1'$ and $E_2'$ are $\beta$-equivalent under $(\theta, L')$.
      \item path=$\str{/reqToken}$ 
        Here, since $e_i^{(1)} \prototagequiv{\theta} e_i^{(2)}$, 
        we can see that the $\str{sessionid}$ in the HTTP headers either both exist or not.
        Therefore, the check at Line~\ref{line:check-sessionid} in Algorithm~\ref{alg:idp} either both fail or not. 
        Obviously, we can have that if the $\str{sessionid}$ exists, it must be the same in both systems.
        Since $S_1(\mi{idp}_1).\str{sessions} \prototagequiv{\theta} S_2(\mi{idp}_2).\str{sessions}$, 
        the second check at Line~\ref{line:check-session-pidrp} also either both fail or not.
        If the two checks both pass, the same output event is produced whose message is 
        \begin{align*}
          \myangle{\cHttpResp,n,200,\myangle{},\mi{IDToken}}
        \end{align*}
        Since the IDToken has already been constructed, there are not any changes to the new states of IdP.
        We can easily have that $S_1\prime$ and $S_2\prime$ are $\gamma$-equivalent under $\theta$. 
        
        As $L' = L$ and $u_x, u_y, r_1, r_2$ do not present directly in the IDToken, $E_1'$ and $E_2'$ are $\beta$-equivalent under $(\theta, L')$.
        No new nonces are chosen, hence $N_1\prime=N_1=N_2=N_2\prime$.
      \item path=$\str{/authorize}$ 
        Here, the first check is the same as $\str{/reqToken}$.
        Since $e_i^{(1)} \prototagequiv{\theta} e_i^{(2)}$, the second and third check either both pass or not.
        If IdP accepts the request, it will start to sign an IDToken.
        There are two cases for $\theta\prime$, if the $\mi{PID_{rp}}$ is mapped to $r_2$ and $S_i(\mi{idp_i}).\str{challenge} = \True$, 
        $\theta\prime = \theta \cup \{ [u_{x/y}]\mi{PID_{r_2}} \}$, otherwise, $\theta\prime = \theta$.
        In the first case, $S_1(\mi{idp_1}).\str{sessions} \prototagequiv{\theta\prime} S_2(\mi{idp_2}).\str{sessions}$, 
        while in the second case, the newly added sessions are equal.
        Therefore, we can see that Condition~\ref{eqs:rp:idp-sessions} of Definition~\ref{def:rp:eq-of-states} holds true for the new states, 
        and $S_1\prime$ and $S_2\prime$ are $\gamma$-equivalent under $\theta\prime$

        In this case, the output event is as same as $\str{/reqToken}$, 
        so we can have that $E_1'$ and $E_2'$ are $\beta$-equivalent under $(\theta\prime, L)$.
        No new nonces are chosen, hence $N_1\prime=N_1=N_2=N_2\prime$.
    \end{itemize}

    \paragraph{\underline{Case $p_1 = \fAP{b_1}$}}

    \begin{description}
      \item[TRIGGER] We now distinguish between the possible values for $\mi{cmd}_\text{switch}$.
        \begin{description}
        \item[1 (trigger script):] In this case, the script in the window indexed by $\mi{cmd}_\text{window}$ is triggered. Let $j$ be a pointer to that window.
            
          We first note that such a window exists in $b_1$ iff it exists in $b_2$ and that $S_1(b_1).j.\str{script} \equiv S_2(b_2).j.\str{script}$. 
          We now distinguish between the following cases, which cover all possible states of the windows/documents:
            
          \begin{enumerate}
          \item $S_1(b_1).j.\str{origin} \in \{\an{r_1,\https},\an{r_2,\https}\}$ and $S_1(b_1).j.\str{script} \equiv \str{script\_rp}$.
            Similar to the following scripts, the main distinction in this script is between the script's internal states (named $\str{phase}$). 
            With the term-equivalence under proto-accts $\theta$ we have that $S_1(b_1).j.\str{scriptstate}.\str{phase} = S_2(b_2).j.\str{scriptstate}.\str{phase}$ 
              
            We can therefore now distinguish between the possible values of $S_1(b_1).j.\str{scriptstate}.\str{phase} = S_2(b_2).j.\str{scriptstate}.\str{phase}$:
            \begin{description}
            \item[start:] In this case, the script open a blank page addressed to its own origin which is equal in both systems.
              The path is the (static) string $\str{/loginSSO}$. 
              The script saves a (static) value for $\str{phase}$ in its scriptstate.
      
              Obviously, we can have that the command is term-equivalent under proto-accts $\theta$ 
              and hence, the browser emits a HTTP request which is term-equivalent.
              Therefore, we have $\gamma$-equivalence under $\theta$ for the new states, 
              $\beta$-equivalence under $(\theta,L)$ for the new events, and $\alpha$-equivalence for the new configuration.
            \item[expectt:] In this case, the script retrieves the result of a \pm from $\mi{scriptinputs}$. 
              As we know that $S_1(b_1).j.\str{scriptstate} \prototagequiv{\theta} S_2(b_2).j.\str{scriptstate}$ 
              and that for all matching \pms that they also have to be term-equivalent up to $\theta$ 
              and that the window structure is equal in both browsers, 
              we have that either the same \pm is retrieved from $\mi{scriptinputs}$ or none in both browsers.
      
              Then the script saves a (static) value for $\str{phase}$ in its scriptstate. 
              Therefore, we have $\gamma$-equivalence under $\theta$ for the new states. 
              We also have $\beta$-equivalence under $(\theta, L)$ for the new events, 
              and $\alpha$-equivalence for the new configuration.
            \item[expectCert:] In this case, the script retrieves the result of an \xhr from $\mi{scriptinputs}$ that matches the reference contained in $\mi{scriptstate}$. 
              From Condition~\ref{eqs:rp:b:w:scriptinputs} and \ref{eqs:rp:b:w:script} of Definition~\ref{def:rp:eq-of-states}, 
              we know that all results from \xhr{}s in $\mi{scriptinput}$ are term-equivalent up to $\theta$ 
              and that $\mi{scriptstate}$ is term-equivalent up to $\theta$. 
              Hence, in both browsers, both scripts stop with an empty command or both continue as they successfully retrieved such an \xhr.
      
              The script now constructs a \pm that is sent to exactly the same window in both browsers 
              and that requires that the receiver origin has to be $\an{\str{IdPdomain},\https}$. 
              The postMessage is only sent to this origin, we have that $\gamma$-equivalence cannot be violated.
      
              We now have that $S_1'$ and $S_2'$ are $\gamma$-equivalent under $\theta$, 
              $E_1'$ and $E_2'$ are $\beta$-equivalent under $(\theta,L)$, 
              and as exactly none of nonces is chosen, 
              we have that the new configuration is $\alpha$-equivalent.
            \item[expectToken:] In this case, the script retrieves the result of an \xhr from $\mi{scriptinputs}$ that matches the reference contained in $\mi{scriptstate}$. 
              From Condition~\ref{eqs:rp:b:w:scriptinputs} and \ref{eqs:rp:b:w:script} of Definition~\ref{def:rp:eq-of-states}, 
              we know that all results from \xhr{}s in $\mi{scriptinput}$ are term-equivalent up to $\theta$ 
              and that $\mi{scriptstate}$ is term-equivalent up to $\theta$. 
              Hence, in both browsers, both scripts stop with an empty command or both continue as they successfully retrieved such an \xhr.

              Then the script saves a (static) value for $\str{phase}$ in its scriptstate. 
              Therefore, we have $\gamma$-equivalence under $\theta$ for the new states. 
              The $IDToken$ in the output event's message is term-equivalent up to $\theta$, 
              so we also have $\beta$-equivalence under $(\theta, L)$ for the new events, 
              Hence, we have $\alpha$-equivalence for the new configuration.
            \end{description}
          \item $S_1(b_1).j.\str{script} \equiv \str{script\_idp}$.
            \begin{description}
            \item[start:] In this case, the script chooses a new nonce for $t$ which does not violate the Definition~\ref{def:rp:eq-of-states}.
              Hence, we have $\gamma$-equivalence under $\theta$ for the new states.
    
              From the equivalence definition of states (Definition~\ref{def:rp:eq-of-states}), 
              we can see that the window tree has the same structure in both processing steps. 
              So the script now constructs a \pm that is sent to exactly the same window in both browsers 
              and that requires that the receiver has to be the opener of this window. 
              Since there are no $l\in L$ or $u_x, u_y, r_1, r_2$ contained in the message, 
              Condition~\ref{eqe:rp:pre:l} and \ref{eqe:rp:pre:t} of Definition~\ref{def:rp:Events} holds, 
              we have $\beta$-equivalence under $(\theta, L)$.
            \item[expectCert:] The same as above, we can have that either the same \pm is retrieved from $\mi{scriptinputs}$ or none in both browsers 
              and the result of $checksig$ is same as well. 
              As The state $Cert_{rp}$ and $PID_{rp}$ are equal in both $\str{scriptstate}$, 
              we have $\gamma$-equivalence under $(\theta,L)$ for the new states.
    
              We can note that regardless of whether the browser has logined in IdP or not, 
              the Condition~\ref{eqe:rp:pre:l} of Definition~\ref{def:rp:Events} always holds.
              Therefore, Condition~\ref{eqe:rp:distinction} of Definition~\ref{def:rp:Events} holds, 
              and we have $\beta$-equivalence under $(\theta, L)$.   
            \item[expectReqToken:] In this case, the script retrieves the result of an \xhr from $\mi{scriptinputs}$ that matches the reference contained in $\mi{scriptstate}$. 
              From Condition~\ref{eqs:rp:b:w:scriptinputs} and \ref{eqs:rp:b:w:script} of Definition~\ref{def:rp:eq-of-states}, 
              we know that all results from \xhr{}s in $\mi{scriptinput}$ are term-equivalent up to $\theta$ 
              and that $\mi{scriptstate}$ is term-equivalent up to $\theta$. 
              Hence, in both browsers, both scripts will reach the same if-else branch.
    
              Since there aren't any new states stored, we can have $\gamma$-equivalence under $\theta$ for the new states.
              The output events's destination are fixed and $IDToken$ in the message is term-equivalent up to $\theta$.
              Therefore, we have $\beta$-equivalence under $(\theta,L)$ for the new events. 
            \item[expectLoginResult:] This case is as same as the second branch in $\str{expectReqToken}$.
            \item[expectToken:] This case is as same as the third branch in $\str{expectReqToken}$.
            \end{description}
          \end{enumerate}
        \item[2 (navigate to URL):] In this case, a new window is opened in the browser and a document is loaded from $\mi{url}$.
          The states of both browsers are changed in the same way, where a new (at this point empty) window is created and appended the $\str{windows}$ subterm of the browsers. 
          This subterm is therefore changed in exactly the same way.
            
          A new HTTP request is created and the request cannot contain any $l \in L$. 
          The Condition~\ref{eqe:rp:pre:l} of Definition~\ref{def:rp:Events} holds true.
            
          In both processing steps, three nonces are chosen. 
          Therefore, we have $\alpha$-equivalence for $(S_1',E_1',N_1')$ and $(S_2',E_2',N_2')$.

        \item[3 (reload document):]
          Here, an existing document is retrieved from its original location again. 
          From the definition of $\gamma$-equivalence under $\theta$ we can see that
          whatever document is reloaded, its location is term-equivalent under $\theta$.
      
          We note that the requests are constructed from
          the location and referrer properties of the document that is to
          be reloaded, and therefore, cannot contain any $u_x, u_y, r_1, r_2$.
      
          We also note that if the document is reloaded, the same 
          request is issued in both browsers (therefore,
          $\beta$-equivalence under $(\theta, L)$ is given), and 
          none states are changed such that we have
          $\gamma$-equivalence under $\theta$. The same number of
          nonces is chosen in both runs, and we have $\alpha$-equivalence.
        \end{description}
      \item[Other] Any other message is discarded by the browsers without any change to state or output events.
    \end{description}
  
    \paragraph{\underline{Case $p_1$ is some attacker:}}
    
    Here, when we talk about attackers, 
    we mean two colluding relying parties.
    They act as web attackers and try to bind browser's identities 
    based on what they get during the login process, 
    especially the $IDToken$.
    
    We can note that Case~\ref{eqe:rp:distinction} from Definition~\ref{def:rp:Events} applys to the input events,
    i.e., the input events are term-equivalent under proto-accts $\theta$. 
    This implies that the message was delivered to the 
    same attacker process in both processing steps. 
    Let $A$ be that attacker process. 
    With Case~\ref{eqs:rp:att} of Definition~\ref{def:rp:eq-of-states}, 
    we have that $S_1(A) \prototagequiv\theta S_2(A)$. 
    With Case~\ref{eqe:rp:pre:t} of Definition~\ref{def:rp:Events} and Case~\ref{eqs:rp:att-unknown} of Definition~\ref{def:rp:eq-of-states}, 
    we have that neither the states of A, i.e, $S_1(A)$ and $S_2(A)$ contain $u_x, u_y, r_1, r_2$, 
    nor do the events $e_i^{(1)}$ and $e_i^{(2)}$.
    Further with lemma~\ref{thm-rp-untraceability}, it follows immediately that the attacker cannot distinguish any of the accts in $\theta$ in its knowledge.
    Therefore, if two states or events are term-equivalent, the attacker process $A$ cannot distinguish between them.

    Obviously, we can have that in the attackers state, 
    there are no variables (from $V_\text{process}$). 
    With the output term being a fixed term (with variables)
    $\tau_{\text{process}} \in \terms(\{x\} \cup V_\text{process})$ 
    and $x$ being $S_1(A)$ or $S_2(A)$, respectively.
    and there is no subterm $l\in L$ contained in  
    $e_i^{(1)}$ or $e_i^{(2)}$ (Condition~\ref{eqe:rp:pre:l} of 
    Definition~\ref{def:rp:Events}), 
    it is easy to see that the output events are 
    $\beta$-equivalent under $\theta$, i.e., 
    $E ^{(1)}_\text{out} \prototagequiv\theta E^{(2)}_\text{out}$. 
    The new state of the attacker in both processing steps 
    consists of the input events, the output events, and the 
    former state of the event, and, as such, is 
    either $\beta$-equivalent or $\gamma$-equivalent 
    under proto-accts $\theta$. 
    Hence, the new states are $\gamma$-equivalent under $\theta$
    The used nonces are the same, i.e., $N_1' = N_2'$. 
    Therefore we have $\alpha$-equivalence on the new configurations.
  \end{proof}

  \begin{lemma}\label{thm-rp-untraceability}
    Given two points on the elliptic curve denoted by $[u_xr_1]G$ and $[u_yr_2]G$, 
    an adversary cannot tell whether $u_x\equiv u_y$, where $u_x, u_y, r_1, r_2$ is random in $\mathbb{Z}_n$ and unknown to the adversary.
  \end{lemma}
  \begin{proof}
    Consider a finite cyclic group $\mathbb{E}$ where the number of points on $\mathbb{E}$ is $n$. 
    Because $G$ is a generator of order $n$, $[r]G$ is also a generator on $\mathbb{E}$ of order $n$. 
    $t$ is randomly chosen in $\mathbb{Z}_n$ and always kept unknown to the adversary. 
    Therefore, $[tr]G$ is \emph{indistinguishable} from a point $Q$ that is randomly chosen on $\mathbb{E}$.\cite{oprf-proved,voprf-proved}.
  \end{proof}

  Now we have shown that the UPPRESSO web systems $\mathcal{U\!W\!S}^{priv}_1$ and $\mathcal{U\!W\!S}^{priv}_2$ are indistinguishable.
  However, it is not enough to prove Theorem~\ref{theorem:rp-privacy}.
  We only have two relying-parties in out models $\mathcal{U\!W\!S}^{priv}_1$ and $\mathcal{U\!W\!S}^{priv}_2$, 
  so we need to further prove that it is still RP-private in the scenario of three or more RPs.
  
  What's more, It is necessary to broaden the knowledge available to the attackers. 
  We introduce some malicious users in our model which act as web attackers. 
  These users login in the RPs and give their identities to the RPs.
  Therefore, colluding RPs can bind these users' login instances. 
  So what we need to prove is that even if colluding RPs and users share $PID_U$s and other information observed in all the logins, they still cannot link any login from an honest user to any other logins from any other honest users to these RPs.

  \newc
  It should be pointed out that we omit the model analysis here after we add extra parties into the privacy model $\mathcal{U\!W\!S}^{priv}$ because the proof doesn't change too much.
  What really changes is that Lemma~\ref{lemma:statically-equivalent} doesn't seem obvious anymore.
  i.e., We cannot easily see that the views of RPs are statically-equivalent without RPs' knowing any keys for the accts contained in its view.
  However, the proof is out of formal analysis's scope, so we use provable security here.
  \oldc

  With the trapdoor $t$, $PID_{RP}$ and $PID_U$ can be easily transformed into $ID_{RP}$ and $Acct$, respectively, and vice versa. Therefore, we denote the information that an RP learns from a login as a tuple $L$, where $L =(ID_{RP}, t, Acct)=(ID_{RP}, t, [ID_{U}]ID_{RP})=([r]G, t, [ur]G)$.

  When $c$ malicious RPs collude with each other, they create a shared view of all their logins, denoted as $\mathbb{L}$.
  %some of which are initiated by honest users and denoted as $\mathbb{L}^h$, and the others by $v$ malicious users  are $\mathbb{L}^m = \mathbb{L} \setminus \mathbb{L}^h$.
  When they collude further with $v$ malicious users, the logins initiated by these malicious users are picked out and linked together as
  $\mathfrak{L}^m=\left \{ \begin{matrix}
  L^m_{1,1},&L^m_{1,2},&\cdots,&L^m_{1,c}\\
  L^m_{2,1},& L^m_{2,2},&\cdots,&L^m_{2,c}\\
  \cdots,&\cdots,&L^m_{i,j},&\cdots\\
  L^m_{v,1},&L^m_{v,2},&\cdots,&L^m_{v,c}
  \end{matrix}\right\}$,
  where $L^m_{i, j}=([r_j]G, t_{i,j}, [u_ir_j]G)$ for $1 \le i \le v$ and $1 \le j \le c$, and $L^m_{i,j} \in \mathbb{L}$. Any login in $\mathbb{L}$ but not linked in $\mathfrak{L}^m$ is initiated by an honest user to one of the $c$ malicious RPs.


  \begin{theorem}
  \emph{In \usso, given $\mathbb{L}$ and $\mathfrak{L}^m$, $c$ malicious RPs and $v$ malicious users cannot link any login from an honest user to a malicious RP to any subset of logins from honest users to any other malicious RPs.}
  \end{theorem}


  \begin{proof} 
  From the logins in $\mathbb{L}$,
  we randomly choose one login $L' \neq L^m_{i,j}$,
  which is from an (unknown) honest user with $ID_{U'}=u'$ to a malicious $RP_a$ and $a \in [1,c]$.
  Then, we randomly choose another malicious $RP_b$, where $b \in [1,c]$ and $b \neq a$.
  Consider any subset $\mathbb{L}''$ of $w$ logins visiting $RP_b$ by unknown honest users,
  we denote the identities of the honest users who initiate these logins as $\mathbf{u}_w=\{{u''_1}, {u''_2}, \cdots, {u''_w}\}$.
  Next, we prove that the colluding adversaries cannot decide if $u'$ is in $\mathbf{u}_w$ or randomly selected from the universal user set.
  This indicates the colluding adversaries cannot link $L'$ to another login visiting $RP_b$
  or to another subset of logins visiting $RP_b$.

  We first define an RP-based linkage game $\mathcal{G}_r$ between an adversary and a challenger, which describes this login linkage privacy threat: the adversary receives $\mathfrak{L}^m$, $L'$, and $\mathbb{L}''$ from the challenger and outputs $s$, where $s = 1$ if it decides $u'$ is in $\mathbf{u}_w$ %$\{{U''_1}, {U''_2}, \cdots, {U''_w}\}$
  and $s=0$ if it believes $u'$ is randomly chosen from the universal user set.
  Thus, the adversary succeeds in $\mathcal{G}_r$ with an advantage $\mathbf{Adv}$:
  \begin{align*}
  %&{\rm Pr}_1={\rm Pr}\{\mathcal{G}_r(\mathfrak{L}, L', \mathbb{L}'' | u' \in \{{u''_1}, {u''_2}, \cdots, {u''_w}\})=1\} \\
  &{\rm Pr}_1={\rm Pr}(\mathcal{G}_r(\mathfrak{L}^m, L', \mathbb{L}'')=1 \;| \; u' \in \mathbf{u}_w)  \\
  %&{\rm Pr}_2={\rm Pr}\{\mathcal{G}_r(\mathfrak{L}, L', \mathbb{L}'' | u' \in \mathbb{Z}_n)=1\}\\
  &{\rm Pr}_2={\rm Pr}(\mathcal{G}_r(\mathfrak{L}^m, L', \mathbb{L}'')=1 \; | \; u' \in \mathbb{Z}_n)\\
  &{\mathbf{Adv}}=|{\rm Pr}_1-{\rm Pr}_2|
  \end{align*}

  As depicted in Figure \ref{fig:dalgorithm}, we design a PPT algorithm $\mathcal{D}^*_r$ based on $\mathcal{G}_r$ to solve the elliptic curve decisional Diffie-Hellman (ECDDH) problem: given $(G, [x]G$, $[y]G$, $[z]G)$, decide whether $z$ is equal to $xy$ or randomly chosen in $\mathbb{Z}_n$, where $G$ is a point on an elliptic curve $\mathbb{E}$ of order $n$, and $x$ and $y$ are integers randomly and independently chosen in $\mathbb{Z}_n$.

  \begin{figure}[tb]
    \centering
    \includegraphics[width=1.0\linewidth]{fig/rp-linkage-game.pdf}
    \caption{The PPT algorithm $\mathcal{D}^*_r$ constructed based on the RP-based linkage game to solve the ECDDH problem.}
    \label{fig:dalgorithm}
  \end{figure}


  The algorithm $\mathcal{D}^*_r$ works as below. (1) Upon receiving an input $(G, Q_1=[x]G, Q_2=[y]G, Q_3=[z]G)$, %of $\mathcal{D}^*_r$
  the challenger
  chooses random numbers in $\mathbb{Z}_n$ to construct $\{u_i\}$, $\{r_j\}$, and $\{t_{i, j}\}$ for $1 \le i \le v$ and $1 \le j \le c$, with which it assembles $L^m_{i, j}=([r_j]G, t_{i,j}, [u_ir_j]G)$.
  In this process, it ensures $[r_{j}]G \neq Q_2$ so that $r_j \neq y$.  % 这个应该反过来讲;因为y是离散对数。
  (2) It randomly chooses $a \in [1, c]$ and $t' \in \mathbb{Z}_n$, to assemble $L' = ([r_{a}]G, t', [r_{a}]Q_1) = ([r_{a}]G, t', [xr_{a}]G)$.
  (3)
  % Here, $L'$ represents the knowledge of the login visiting $RP_{j'}$ by a user with $ID_U = x$.
  Next, the challenger randomly chooses $b \in [1, c]$ and $b \neq a$, and replaces $ID_{RP_b}$ with $Q_2 = [y]G$.
  Hence, for $1 \le i \le v$, the challenger replaces $L^m_{i, b}=([r_b]G, t_{i,b}, [u_ir_b]G)$ with $(Q_2, t_{i,b}, [u_i]Q_2) = ([y]G, t_{i,b}, [u_iy]G)$, and then constructs $\mathfrak{L}^m$.
  (4) the challenger chooses random numbers in $\mathbb{Z}_n$ to construct $\{u''_k\}$ and $\{t''_k\}$ for $1 \leq k \leq w$,
  with which it assembles $\mathbb{L}'' = \{L''_{k; 1\leq k \leq w}\} = \{(Q_2, t''_k, [u''_k]Q_2)\} = \{([y]G, t''_k, [u''_ky]G)\}$.
  In this process, it ensures that $[u''_k]G \neq Q_1$ (i.e., $u''_k \neq x$) and $u''_k \neq u_i$,
  for $1 \le i \le v$ and $1 \le k \le w$.
  Finally, it randomly chooses $d \in [1, w]$ and replaces $L''_{d}$ with $(Q_2, t''_d, Q_3) = ([y]G, t''_d, [z]G)$.
  Thus, $\mathbb{L}'' = \{L''_{k;1\leq k \leq w}\}$ represents the logins initiated by $w$ honest users, i.e., $\mathbf{u}_w=\{u''_1, u''_2, \cdots, u''_{d-1}, z/y, u''_{d+1}, \cdots, u''_w\}$.
  (5) When the adversary of $\mathcal{G}_r$ receives $\mathfrak{L}^m$, $L'$, and $\mathbb{L}''$ from the challenger, it returns $s$ as the output of $\mathcal{D}^*_r$.

  According to the above construction, % of $\mathfrak{L}$, $L'$ and $\mathbb{L}''$,
  $x$ is embedded as $ID_{U'}$ in the login $L'$ visiting the RP with $ID_{RP_{a}} = [r_{a}]G$,
  and $z/y$ is embedded as $ID_{U''_d}$ in $\mathbb{L}''$ visiting the RP with $ID_{RP_{b}}=[y]G$,
  together with $\{u''_1, \cdots, u''_{d-1}, u''_{d+1}, \cdots, u''_w\}$.
  Meanwhile, $[r_{a}]G$ and $[y]G$ are two malicious RPs' identities in $\mathfrak{L}^m$.
  Because $x \neq u''_{k; 1\leq k \leq w, k \neq d}$ and then $x$ is not in $\{u''_1, \cdots, u''_{d-1}, u''_{d+1}, \cdots, u''_w\}$, the adversary outputs $s=1$ and succeeds in the game \emph{only if} $x = z/y$.
  % 这里不是if and only if. "if, 就变成了the adversary必胜了;并不是,而是“有显著的概率”"
  % 当“the adversary outputs s=1 且 succeeds in the game”,=> "x = z/y"
  % 但是,"x = z/y"  => 不能推导得到“the adversary outputs s=1 且 succeeds in the game”。因为adversary有时候fail、不总是succeed
  Therefore, using $\mathcal{D}^*_r$ to solve the ECDDH problem, we have an advantage $\mathbf{Adv}^*=|{\rm Pr}^*_1 - {\rm Pr}^*_2|$, where
  \begin{align*}
  &{\rm Pr}^*_1 =  {\rm Pr}(\mathcal{D}^*_r(G, [x]G, [y]G, [xy]G)=1) \\
  =&{\rm Pr}(\mathcal{G}_r(\mathfrak{L}^m, L', \mathbb{L}'')=1 \; | \; u' \in \mathbf{u}_w) = {\rm Pr}_1 \\
  &{\rm Pr}^*_2= {\rm Pr}(\mathcal{D}^*_r(G, [x]G, [y]G, [z]G)=1) \\
  =&{\rm Pr}(\mathcal{G}_r(\mathfrak{L}^m, L', \mathbb{L}'')=1 \; | \; u' \in \mathbb{Z}_n) = {\rm Pr}_2 \\
  &\mathbf{Adv}^*=|{\rm Pr}^*_1-{\rm Pr}^*_2|=|{\rm Pr}_1-{\rm Pr}_2|={\mathbf{Adv}}
  \end{align*}

  If in $\mathcal{G}_r$ the adversary has a non-negligible advantage, then $\mathbf{Adv}^*={\mathbf{Adv}}$ is also non-negligible regardless of the security parameter $\lambda$. This violates the ECDDH assumption. Therefore, the adversary has no advantage in $\mathcal{G}_r$ and cannot decide whether $L'$ is initiated by some user with an identity in $\mathbf{u}_w$ or by a user in the universal user set.
  Moreover, because $RP_b$ is any malicious RP, this proof can be easily extended from $RP_b$ to more colluding malicious RPs.
  \end{proof}
  
  These prove Theorem~\ref{theorem:rp-privacy}.\QED
  
\end{document}
  


% that's all folks
\end{document}
