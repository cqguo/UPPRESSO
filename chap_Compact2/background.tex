\section{Background and Related Work}
\label{sec:background}

\subsection{OpenID Connect}
\label{subsec:OIDC}
OIDC is one of the most popular SSO protocols. It supports different login flows: implicit flow, authorization code flow, and hybrid flow (a mix of the other two).
These flows differ in the steps for requesting, receiving and forwarding identity tokens, but have common security requirements for the tokens.
We present our designs in the implicit flow and discuss the support for the authorization code flow in Section \ref{sec:discussion}.

Users and RPs register at an OIDC IdP with their identities
and other information such as user credentials (e.g., passwords)
and RP endpoints (i.e., the URLs to receive tokens).
User operations in the login flows are conducted through a user agent (or browser typically).
To provide an in-context user experience,
    the SSO login flow of OIDC with pop-up UX \cite{dimvaLiM16,GoogleIdIntegrate,uber} is recommended.
As shown in Figure \ref{fig:OpenID},
when a user clicks the SSO login button in a browser window visiting the target RP 
(called the \emph{user-r window} in this paper),
 an identity-token request is constructed with the RP's identity and the scope of requested user attributes.
It pops up another window (i.e., the \emph{user-i window}),
    and this request is sent to the IdP.
After authenticating the user, the IdP issues an identity token that encloses the (pseudo-)identities of the authenticated user and the target RP, the authorized attributes, a validity period, etc.
The identity token is then forwarded to the RP's endpoint via the user-i window.
Finally, the RP verifies the token and allows the token holder to log in as the enclosed user (pseudo-)identity.
There are scripts in two browser windows (called the \emph{user-r script} and the \emph{user-i script}),
    responsible for communicating with the respective origin web servers
    and also the cross-origin
communications using the \verb+postMessage+ HTML5 API within the COTS browser.


%To process identity tokens retrieved from the IdP, which is carried with a URL following the fragment identifier \texttt{\#} instead of \texttt{?} due to security considerations \cite{de2014oauth},
%To process responses from an IdP and forward tokens to the target RP correctly,

Alternatively, if a browser works as the user agent of OIDC with only one window,
        user attributes are authorized in this window when redirected to the IdP \cite{OpenIDConnect,rfc6749,GoogleIdIntegrate}, 
        which is called \emph{redirect UX}.


\begin{table*}[tb]
    \caption{Privacy-preserving solutions of SSO and identity federation}
    \centering{
    \begin{tabular}{|c|c|c|c|c|c|}
  \hline
  & \multirow{3}*{\textbf{Solution}} &
  \multicolumn{2}{c|}{\textbf{Privacy Threat}$^{\sharp}$} & & \\ \cline{3-4}
  & & IdP-based & RP-based & \textbf{Extra Trusted} & \textbf{Browser Extension}\\
  & & Login Tracing & Identity Linkage & \textbf{Server}$^{\dag}$ & \textbf{with User Secret}$^{\ddag}$ \\\hline\hline

  \multirow{6}*{SSO} & OIDC w/ PPID \cite{NIST2017draft} & $\Circle$ & $\CIRCLE$ & $\CIRCLE$ & $\CIRCLE$ \\ \cline{2-6}
   & BrowserID \cite{BrowserID} & $\CIRCLE$ & $\Circle$ & $\CIRCLE$ & $\Circle$ \\ \cline{2-6}
   & SPRESSO \cite{SPRESSO} & $\CIRCLE$ & $\Circle$ & $\LEFTcircle$$^1$ & $\CIRCLE$ \\ \cline{2-6}
   & POIDC \cite{POIDC,save-flow} & $\CIRCLE$ & $\Circle$$^2$ & $\CIRCLE$ & $\CIRCLE$ \\ \cline{2-6}
   & MISO \cite{miso} & $\CIRCLE$ & $\CIRCLE$ & $\Circle$ & $\CIRCLE$ \\ \cline{2-6}
   & UP-SSO \cite{up-sso} & $\CIRCLE$ & $\CIRCLE$ & $\CIRCLE$ & $\Circle$ \\ \hline 
  \multirow{5}*{\makecell{Identity\\Federation}} & PRIMA \cite{prima} & $\CIRCLE$ & $\Circle$ & $\CIRCLE$ & $\Circle$ \\ \cline{2-6}
   & PseudoID \cite{PseudoID} & $\CIRCLE$ & $\CIRCLE$ & $\Circle$ & $\Circle$ \\ \cline{2-6}
   & Opaak \cite{Opaak} & $\CIRCLE$ & $\CIRCLE$ & $\CIRCLE$ & $\Circle$ \\ \cline{2-6}
 % U-Prove \cite{uprov} & $\CIRCLE$ & $\Circle$ & $\LEFTcircle$$^6$ & $\CIRCLE$ & $\CIRCLE$ & $\CIRCLE$ \\ \hline
 % UnlimitID \cite{UnlimitID} & $\CIRCLE$ & $\Circle$ & $\CIRCLE$ & $\CIRCLE$ & $\CIRCLE$ & $\CIRCLE$ \\ \hline
 % EL PASSO \cite{ELPASSO} & $\CIRCLE$ & $\Circle$ & $\CIRCLE$ & $\CIRCLE$ & $\CIRCLE$ & $\CIRCLE$ \\ \hline
   & PP-IDF \cite{ELPASSO,uprov,UnlimitID} & $\CIRCLE$ & $\CIRCLE$ & $\CIRCLE$ & $\Circle$ \\ \cline{2-6}
   & Fabric Idemix \cite{hyperledge-idemix} & $\CIRCLE$ & $\CIRCLE$ & $\CIRCLE$ & $\Circle$ \\ \hline
 % PrivacyPass \cite{privacypass,trusttoken} & $\Circle$ & $\Circle$ & $\Circle$ & $\CIRCLE$ & $\CIRCLE$ & $\Circle$ \\ \hline
  SSO & UPPRESSO & $\CIRCLE$ & $\CIRCLE$ & $\CIRCLE$ & $\CIRCLE$ \\ \hline
\end{tabular}}
    \label{tbl:comparison-protocol}
{
\begin{itemize}
\item[]
${\sharp}$ \textbf{Privacy Threat}: $\CIRCLE$ prevented, $\Circle$ not prevented.
\ \ \ \ \ \ \ \ \ \ \ \ \ \ \ \ \ \ \ \ \ \ \ \ \ \ \ ${\dag}$ \textbf{Extra Trusted Server}: $\CIRCLE$ no, $\Circle$ required.\\
${\ddag}$ \textbf{Browser Extension with User Secret}: $\CIRCLE$ no, $\Circle$ required.

\item[1.]
SPRESSO assumes a malicious IdP (but honest IdPs in others), and then a trusted forwarder server is needed to decrypt the RP identity and forward tokens to the RP.

\item[2.]
A variation of POIDC \cite{POIDC} proposes to also hide a user's identity in the commitment and prove this to the IdP in zero-knowledge,
        but it takes seconds or more to generate such a zero-knowledge proof (ZKP) even for a powerful server \cite{ZKP-BINF,zkp-benchmark,ZKP-GPU},
        which is impracticable for a user agent in SSO.
%\item[3.]
%Opaak supports two exclusive pseudonym options: (\emph{a}) linkable within an RP but unlinkable across multiple RPs and (\emph{b}) unlinkable for any pair of actions.
%
%\item[4.]
%Different from \cite{ELPASSO,UnlimitID}, a credential in U-Prove \cite{uprov} may contain some attributes that are \emph{invisible} to the IdP, in addition to the ones confirmed by the IdP.
%\item[5.]
%In the original design of Idemix \cite{idemix}, every user logs into an RP with a unique account, but Fabric Idemix \cite{hyperledge-idemix} implements completely-anonymous services.
\end{itemize}}
\end{table*}

\subsection{Privacy-Preserving SSO and Identity Federation}
\label{subsec-solutions}
To provide secure SSO services,
    an IdP signs identity tokens binding the identities of the authenticated user and the target RP \cite{OpenIDConnect,rfc6749}.
However,
    a user's login activities are leaked due to these identities:
        The curious IdP learns the visited RP from $ID_{RP}$ enclosed in an identity-token request (i.e., IdP-based login tracing),
        while colluding RPs link a user's logins visiting these RPs by $ID_{U}$ in the tokens (i.e., RP-based identity linkage).


%Privacy-preserving SSO is expected to offer the desirable features listed in Section \ref{subsec:OIDC}, while addressing privacy threats.
%Privacy-preserving identity federation offers more privacy protections, but brings extra complexity to users as described above.


\noindent\textbf{Privacy-preserving SSO.}
%Some schemes \cite{BrowserID, SPRESSO, NIST2017draft} prevent either IdP-based login tracing or RP-based identity linkage, but not both.
Existing solutions propose various mechanisms to hide (\emph{a}) RP identities from the IdP or/and (\emph{b}) user identities from the visited RPs.
Table \ref{tbl:comparison-protocol} compares these solutions,
and only UPPRESSO provides SSO services accessed from COTS browsers while preventing both of the two privacy threats and not introducing extra trusted servers in addition to the honest IdP.


Pairwise pseudonymous identifiers (PPIDs) are specified \cite{OpenIDConnect, SAMLIdentifier} and recommended \cite{NIST2017draft} in OIDC to protect user privacy against colluding RPs.
An IdP assigns a unique PPID for a user to log into every RP and encloses it in identity tokens, so colluding RPs cannot link the user by PPIDs.
It does not prevent IdP-based login tracing because the IdP needs the RP's identity to assign PPIDs.

Other privacy-preserving schemes prevent IdP-based login tracing but leave users vulnerable to RP-based identity linkage, due to the unique user identities enclosed in identity tokens.
For example, in BrowserID \cite{BrowserID} 
after authenticating a user,
    an IdP %(called the primary identity authority in BrowserID)
issues a ``user certificate'' binding the user's identity to an ephemeral public key.
The user then uses the corresponding private key to sign a subsidiary ``identity assertion'' that binds the target RP's identity and sends both of them to the RP.
In SPRESSO an RP creates a one-time tag (i.e., ephemeral pseudo-identity) for each login \cite{SPRESSO},
 and in POIDC \cite{POIDC,save-flow} a user sends an identity-token request with a hash commitment on the target RP's identity (but not $ID_{RP}$),
        which are enclosed in identity tokens along with the user's unique identity.

%As in SPRESSO the IdP is assumed to be malicious, there is no IdP-confirmed attribute provision.
% J. He, L. Lei \emph{et al}. \cite{ARPSSO} copied the identity transformations of UPPRESSO published in the preprint version \cite{uppresso-arxiv}, into a prime field $F_q$,\footnote{That is, $ID_U = u$, $ID_{RP} = h = g^r$, $PID_{RP} = h^t = g^{rt}$, $PID_U = h^{tu}=g^{rtu}$, and $Acct = h^u= g^{ru}$ in $F_q$ where $g$ is a generator. It seems that these identity transformations were called RP anonymization and user identity mix-up intentionally in ARPSSO \cite{ARPSSO}.} and attempted to implement UPPRESSO in the authorization code flow of OIDC using \emph{trusted} scripts without any extensions or plug-ins, which also follows the same principle as the UPPRESSO prototype does.

MISO \cite{miso} decouples the calculation of PPIDs from an honest IdP,
 to an extra fully-trusted mixer server that calculates a user's PPID based on $ID_U$, $ID_{RP}$ and a secret after it receives the authenticated user's identity from the IdP.
MISO prevents both RP-based identity linkage for the RPs receives only PPIDs,
    and IdP-based login tracing because $ID_{RP}$ is disclosed to the mixer but not the IdP.
It protects a user's online profile against even collusive attacks by the IdP and RPs,
    but the mixer server could track a user's all login activities.
Similarly, UP-SSO \cite{up-sso} runs a fully-trusted Intel SGX enclave on the user side,
 which is remotely attested by the IdP and then receives the secret to generate PPIDs for a user, and then both of two privacy threats are eliminated.

\noindent\textbf{Privacy-preserving identity federation.}
% SSO allows a user to log into an RP, without by herself maintaining an account at the RP or holding a long-term secret verified by the RP, so that SSO services are accessible from COTS browsers.
Identity federation enables a user registered at a trusted IdP to be accepted by other parties, potentially with different accounts,
but %authentication operations between the user and RPs are involved, so 
browser extensions are needed to handle a long-term user secret.
Although the term ``single sign-on (SSO)'' was used in some schemes \cite{PseudoID, Opaak, ELPASSO, WangWS13, HanCSTW18, HanCSTWW20}, this paper refers to them as \emph{identity federation} to emphasize this difference.

In PRIMA \cite{prima}, an IdP signs a credential
that binds user attributes and  a verification key.
Using the corresponding signing key, the user authenticates herself and provides selected attributes to RPs. The verification key works as the user's identity and exposes her to RP-based identity linkage.
PseudoID \cite{PseudoID} introduces another trusted server in addition to the IdP,
 to blindly sign \cite{blind-sign}
a token that binds a long-term user secret and a pseudonym.
The user then unblinds this token,
    and then authenticates herself to an RP using the secret.

Privacy-preserving schemes of identity federation are designed based on anonymous credentials \cite{anon-credential-2001, idemix, anon-credential}.
For instance, the IdP signs anonymous credentials in Opaak \cite{Opaak}, UnlimitID \cite{UnlimitID}, U-Prove \cite{uprov}, and EL PASSO \cite{ELPASSO}, each of which binds a long-term user secret. %, with which users can authenticate to an RP.
Then a user proves ownership of the anonymous credentials using her secret.
  %   and discloses IdP-confirmed attributes in the credentials in most schemes except Opaak.
Similarly, Fabric \cite{hyperledge-idemix} integrates Idemix anonymous credentials \cite{idemix} for unlinkable pseudonyms. % and IdP-confirmed attribute provision.


Some privacy-preserving identity federation solutions \cite{PseudoID,ELPASSO,UnlimitID,Opaak,uprov,hyperledge-idemix} prevent both IdP-based login tracing and RP-based identity linkage, for (\emph{a}) the RP's identity is not enclosed in the anonymous credentials and (\emph{b}) the user selects different pseudonyms to visit different RPs.
They even protect user privacy against collusive attacks by the IdP and RPs, as these pseudonyms cannot be linked through anonymous credentials \cite{anon-credential-2001, idemix, anon-credential} when the ownership of these credentials is proved to colluding RPs. % using a user secret.
%However, this protection brings extra complexity to users, and such services cannot be accessed from COTS browsers.
In privacy-preserving identity federation
this privacy protection depends on long-term user secrets to mask the relationship of among pseudonyms (or accounts),
        so a browser extension is always needed to handle the user secrets. 
In Section \ref{sec:discussion}
we particularly discuss the collusion of the IdP and RPs in privacy-preserving SSO and identity federation.

%Compared with privacy-preserving identity federation \cite{prima,PseudoID,Opaak,ELPASSO,uprov,UnlimitID,hyperledge-idemix}
%    where a browser plug-in or extension is required to handle a long-term user secret
%    but
%    more privacy threats are eliminated,
%UPPRESSO works with COTS browsers,
%    but collusive attacks by the IdP and RPs still break user privacy in UPPRESSO.
%Our design provides a convenient choice for some users not installing browser plug-ins or extensions.
%

\noindent\textbf{Anonymous identity federation.}
Such approaches offer the strongest privacy,
    where a user visits RPs with pseudonyms that cannot be used to link any two actions.
Anonymous identity federation was formalized \cite{WangWS13} and implemented using cryptographic primitives such as group signature and ZKP \cite{WangWS13, HanCSTWW20, HanCSTW18}.
Extended features including proxy re-verification \cite{HanCSTWW20}, designated verification \cite{HanCSTW18} and distributed IdP servers \cite{TSAPP}, are also considered. 

These completely-anonymous authentication services only work for special applications and do not support user identification (or account uniqueness) at each RP, a common requirement in most applications.

\subsection{OPRF-based Applications}
PrivacyPass and TrustToken \cite{privacypass,trusttoken} adopted the oblivious pseudo-random function (OPRF) protocol of Hash(EC)DH \cite{oprf-proved,voprf-proved} to generate anonymous tokens. %, with which a user ly accesses protected resources.
A user generates a random number $e$ to blind an unsigned token $T$ into $[e]T$. After authenticating the user, a token server signs $[e]T$ using its private key $k$ and returns $[k e]T$. The user then uses $e$ to convert it into ($T, [k]T$),
        which is redeemed when she anonymously accesses protected resources
 (see \cite{privacypass,trusttoken} for details).
In this procedure, the token server obliviously calculates a pseudo-random output as an anonymous token,
    because ($T, [k]T$) and ($[e]T, [ke]T$) cannot be linked.

% RPs do not identify each user if we use such anonymous tokens in SSO services, because these tokens are indistinguishable.

UPPRESSO employs a similar cryptographic technique.\footnote{We designed and implemented UPPRESSO in 2020, and in 2023 someone reminded us that the identity transformations are very similar to the Hash(EC)DH OPRF.} 
 A user selects $t$ to transform $ID_{RP}$ into $PID_{RP} = [t]ID_{RP}$.
  % protecting $ID_{RP}$ from the IdP.
Then, the IdP uses the user's permanent identity $u$ to calculate $PID_U = [u]PID_{RP} = [ut]ID_{RP}$,
which is similar to token signing by the token server in PrivacyPass/TrustToken.
Hence, \emph{IdP untraceability} in UPPRESSO, i.e., the IdP cannot link $ID_{RP}$ and $[t]ID_{RP}$,
 roughly corresponds to {\em unlinkability of token signing-redemption} in PrivacyPass/TrustToken, i.e., the token server cannot link $T$ and $[e]T$.

UPPRESSO leverages this cryptographic technique in very different ways.
First, 
it works among three parties, unlike the two-party PrivacyPass/TrustToken protocols.
UPPRESSO shares the user-selected random number $t$ with the target RP,
enabling it to derive the user's permanent account, i.e., $Acct = [t^{-1}]PID_{U}$. This account is ``obliviously'' determined by the IdP when it calculates the user's pseudo-identity.
% based on the ``blinded'' input (i.e., the visited RP's pseudo-identity).
Second, $u$ and $ID_{RP}$, which roughly correspond to $k$ and $T$ in PrivacyPass/TrustToken, are assigned as the permanent identities of the user and the RP, respectively, to establish the relationship among identities and accounts.
This is different from existing OPRF-based systems  \cite{privacypass, trusttoken, strong-oprf, oprf-bitcoin-wallet, pesto, oprf-ot-si, pp-ss, Private-Contact-Discovery, o-kms, oprf-deduplication} that always use $k$ as a server's secret key.
%The integration of SSO (pseudo-)identities and OPRF variables requires a deep understanding of both SSO and OPRF protocols.
Finally, UPPRESSO leverages randomness of Hash(EC)DH OPRFs to ensure RP unlinkability, but this property is not utilized in PrivacyPass/TrustToken.
This ensures colluding RPs cannot link any logins across RPs,
even by sharing their knowledge about the pseudo-identities and permanent accounts in UPPRESSO.
This property offers desirable indistinguishability of \emph{different users} visiting colluding RPs.
It roughly corresponds to the indistinguishability of \emph{different private keys} for signing anonymous tokens, not considered in PrivacyPass/TrustToken. 

Although UPPRESSO mathematically employs the same technique in the identity transformations as the Hash(EC)DH OPRF \cite{oprf-proved,voprf-proved}, we require more properties of these algorithms. % in Theorems 4, 6, and 7.
UPPRESSO essentially 
depends on the \emph{deterministicness} property of \emph{pseudo-random} functions to correctly derive the account for any $t$, \emph{obliviousness} to ensure IdP untraceability, 
and \emph{pseudo-randomness} to ensure RP unlinkability. 
In contrast, PrivacyPass and TrustToken \cite{privacypass,trusttoken} depend on only the properties of deterministicness and obliviousness.
Moreover, UPPRESSO requires additional properties for secure SSO services (i.e., user identification and RP designation; see Theorems \ref{thm-u-id} and \ref{thm-rp-designation} in Section \ref{analysis-security}),
    which are not discussed in OPRFs \cite{sok-oprf,oprf-proved,voprf-proved}.
% in the case of user identity leakage, \emph{collision-freeness} of $PID_{RP}$ (see Appendix \ref{proof-rp-collision}).
Thus, an OPRF protocol is not always ready to work as the identity transformations in UPPRESSO.
% unless no collision exists in the blinded inputs of the pseudo-random function \cite{oprf-proved,voprf-proved}.

%\subsection{Formal Analysis of SSO Protocols}
%Dolev-Yao style models are developed to analyze the communications among entities in an SSO system \cite{SPRESSO,BrowserID,FettKS14}, to ensure that all messages including identity tokens are delivered as expected in the system
%        and then to prove security and privacy of services 
%        based on only traditional public-key and symmetric cryptographic algorithms (e.g., RSA and AES in SPRESSO \cite{SPRESSO}).
%On the contrary, in this paper security and privacy of SSO services are analyzed based on more complicated cryptographic primitives,
%    while secure communications among entities (i.e., authenticity, confidentiality and integrity of identity tokens) in UPPRESSO are assumed (or ensured by common mechanisms in popular SSO services).
%
