\section{The Concepts of Web Model}
Our formal security analysis of UPPRESSO is based on the general Dolev-Yao web model in SPRESSO~\cite{SPRESSO}. 
This web model is designed independently of a specific web application and closely mimics published (de-facto) standards and specifications for the web, for instance, the HTTP/1.1 and HTML5 standards and associated (proposed) standards. 
So far, the web model has been employed to analyze both trace-based authentication and indistinguishability/privacy properties for web applications.
Therefore, we formalize UPPRESSO in the web model to prove the authentication and privacy properties for UPPRESSO.

To facilitate the definition of UPPRESSO, however, we have some difference from SPRESSO. 
In particular, we remove some processes and add some function symbols for asymmetric encryption/decryption.
In this section, we present the model of the web infrastructure, along with the following changes and additions compared to SPRESSO:
\begin{itemize}
\item For constants $C$, we add the $\mathbb{P}$ as the set of points on the chosen elliptic curve over a finite field, which has been described in section~\ref{sec:UPPRESSO}.
\item We add the function symbols $\funcMultiply{\cdot, \cdot}$, $\funcReverse{\cdot}$, and $\funcCheckValid{\cdot}$, representing the Elliptic Curve calculations.
\item We remove the DNS servers and messages in UPPRESSO. In UPPRESSO, however, we only have one centralized IdP server, and all RPs know the relevant information of the IdP in advance.
So all DNS requests are generated spontaneously by the browser, not introduced by UPPRESSO.
\end{itemize}

\subsection{Communication Model}
We follow the SPRESSO to describe the basic concepts of the DY-style communication model.

\subsubsection{Terms, Messages and Events}
The signature $\Sigma$ for the terms and messages in this work can be described as the following sets of symbols:
\begin{itemize}
\item constants $C = \addresses \cup \mathbb{S} \cup \mathbb{P} \cup \{ \True, \bot, \notdef \}$ where the four sets are pairwise disjoint, $\addresses$ is interpreted to be a set of (IP) addresses, and $\mathbb{S}$ is interpreted to be the set of ASCII strings (including the empty string $\varepsilon$), and $\mathbb{P}$ is interpreted to be the set of points on the chosen elliptic curve over a finite field where $G \in \mathbb{P}$ is the base point (or generator), 
\item function symbols for public keys, (a)symmetric  encryption/decryption, signatures, elliptic curve multiplication, reversing elliptic curve multiplication scalar, and checking point is valid on the chosen elliptic curve: 
	$\funcPub{\cdot}$, $\funcEncA{\cdot, \cdot}$,
	$\funcDecA{\cdot, \cdot}$, $\funcEncS{\cdot, \cdot}$,
	$\funcDecS{\cdot, \cdot}$, $\funcSig{\cdot, \cdot}$,
	$\funcCheckSig{\cdot, \cdot}$, $\funcExtractMsg{\cdot}$,
	$\funcMultiply{\cdot, \cdot}$, $\funcReverse{\cdot}$,
	and $funcCheckValid{\cdot}$,
\item $n$-ary sequences $\an{}, \an{\cdot}, \an{\cdot, \cdot}, \an{\cdot, \cdot, \cdot},$ etc., and
\item projection symbols $\pi_i(\cdot)$ for all $i \in \mathbb{N}$.
\end{itemize}

The equational theory associated with $\Sigma$ is shown as follows:
\begin{align*}
\dec{\enc x{\pub(y)}}{y} &= x\\
\decs{\encs x{y}}{y} &= x\\
\mathsf{multiply}(x, \mathsf{multiply}(y, P)) &= \mathsf{multiply}(y, \mathsf{multiply}(x, P))\\
\mathsf{multiply}(\mathsf{reverse}(x), \mathsf{multiply}(x, P)) &= P \text{\;\;if\ } P \in \mathbb{P}\\
\checksigThree{\sig{x}{y}}{x}{\pub(y)} &= \True\\
\mathsf{checkvalid}(\mathsf{multiply}(x, P)) &= \True  \text{\;\;if\ } P \in \mathbb{P}\\
\mathsf{checkvalid}(P) &= \bot  \text{\;\;if\ } P \not\in \mathbb{P}\\
\pi_i(\an{x_1,\dots,x_n}) &= x_i \text{\;\;if\ } 1 \leq i \leq n \\
\proj{j}{\an{x_1,\dots,x_n}} &= \notdef \text{\;\;if\ } j
\not\in \{1,\dots,n\}
\end{align*}

\begin{definition}[Terms]\label{def:terms}
  We denote the set of variables $X$ and an infinite set of nonces $\nonces$. For $N\subseteq\nonces$, the set of \textbf{terms} over $\Sigma\cup N\cup X$, $\gterms_N(X)$ can be defined as usual: 
  (1) If $t\in N\cup X$, then $t \in \gterms_N(X)$ is a term. 
  (2) If $f \in \Sigma$ is an $n$-ary function symbol in $\Sigma$ for some $n\ge 0$ and  $t_1,\ldots,t_n \in \gterms_N(X)$, then $f(t_1,\ldots,t_n) \in \gterms_N(X)$ is a term.
  (3) If for some $n\ge 0$ and  $t_1,\ldots,t_n \in \gterms_N(X)$, then n-ary sequence <$t_1,\ldots,t_n$> $\in \gterms_N(X)$ is a term. 
  (4) If the n-ary sequence <$t_1,\ldots,t_n$> $\in \gterms_N(X)$, then projection symbol $\pi_i(\an{x_1,\dots,x_n}) \in \gterms_N(X)$ is a term.
\end{definition}

\begin{definition}[Messages and Protomessages]\label{def:messages}
  The set $\messages$ of messages (over $\nonces$) is defined to be
  the set of terms without variables (ground terms), i.e., $\messages :=\gterms_{\nonces}$.
  The set of protomessages, $\messages^\nu$ is defined to be the set of terms with only the predefined set of variables $V_{\text{process}} = \{\nu_1, \nu_2, \dots\}$ (called placeholders), i.e., $\messages^\nu := \gterms_N(V_{\text{process}})$.
\end{definition}

\begin{definition}[Normal Form]
  Let $t$ be a term. The \emph{normal form} of $t$ is acquired by reducing the function symbols from left to right as far as possible  using the equational theory. For a term $t$, we denote its normal form as $t\nf$.
\end{definition}

\begin{definition}[Pattern Matching]
  Let $\mi{pattern} \in \terms(\{*\})$ be a term containing the
  wildcard (variable $*$). We say that a term $t$ \emph{matches}
  $\mi{pattern}$ iff $t$ can be acquired from $\mi{pattern}$ by
  replacing each occurrence of the wildcard with an arbitrary term
  (which may be different for each instance of the wildcard). We write
  $t \sim \mi{pattern}$.

  For a term $t'$ we write $t'|\, \mi{pattern}$ to denote the term
  that is acquired from $t'$ by removing all immediate subterms of
  $t'$ that do not match $\mi{pattern}$.
\end{definition}


  For example, for a pattern $p = \an{\top,*}$ we have that $\an{\top,42} \sim p$, $\an{\bot,42} \not\sim p$, and \[\an{\an{\bot,\top},\an{\top,23},\an{\str{a},\str{b}},\an{\top,\bot}} |\, p = \an{\an{\top,23},\an{\top,\bot}}\ .\]


\begin{definition}[Variable Replacement]
  Let $N\subseteq \nonces$, $\tau \in \gterms_N(\{x_1,\ldots,x_n\})$,
  and $t_1,\ldots,t_n\in \gterms_N$. By
  $\tau[t_1\!/\!x_1,\ldots,t_n\!/\!x_n]$ we denote the term obtained
  from $\tau$ by replacing all occurrences of $x_i$ in $\tau$ by
  $t_i$, for all $i\in \{1,\ldots,n\}$.
\end{definition}


\begin{definition}[Events and Protoevents]
  An \emph{event (over $\addresses$ and $\messages$)} is a term of the
  form $\an{a, f, m}$, for $a$, $f\in \addresses$ and $m \in
  \messages$, where $a$ is interpreted to be the receiver address and
  $f$ is the sender address. We denote by $\events$ the set of all
  events. Events over $\addresses$ and $\messages^\nu$ are called
  \emph{protoevents} and are denoted $\events^\nu$. By
  $2^{\events\an{}}$ (or $2^{\events^\nu\an{}}$, respectively) we
  denote the set of all sequences of (proto)events, including the
  empty sequence (e.g., $\an{}$, $\an{\an{a, f, m}, \an{a', f', m'},
    \dots}$, etc.). 
\end{definition}

\subsubsection{Atomic Processes, Systems and Runs}

\begin{definition}[Generic Atomic Processes]
The atomic process is the modelling web nodes, in the form p=($I^p$, $Z^p$, $R^p$, $s_0^p$). The definitions of these notations is described as below.
\begin{itemize}
\item $I^p$ is the set of addresses that the process listens to, where $I^p \subseteq \addresses$.
\item  $Z^p$ is the set of \textbf{states} that describes the process. The state is the formal \textbf{term}, where $Z^p \subseteq \terms(V_{\text{process}})$. For any new state $s$ and any sequence of nonces  $(\eta_1, \eta_2, \dots)$ we demand that $s[\eta_1/\nu_1,  \eta_2/\nu_2, \dots] \in Z^p$.
\item $R^p$ is the set of \textbf{relations} $R$ between the input event $e$, state $z \in Z^p$ and an output events sequence $E$, the new state $z' \in Z^p$. It means that, at one time point, while the process is on state $s$ and receive the event $e$, it would transition to state $s'$ and send the events $E$ to other process. There is $R^p\subseteq (\events \times Z^p) \times (2^{\events^\nu\an{}}
  \times \terms(V_{\text{process}}))$. 
\item $s_0^p$ is an initial state.
\end{itemize}
\end{definition}

\begin{definition}[Systems]
A \emph{system} $\process$ is a  (possibly infinite) set of processes \aps.
\end{definition}

\begin{definition}[Configurations]
 The configuration of system represents the instantaneous state of the system at the exact moment.
  A \emph{configuration of a system $\process$} is a tuple $(S, E, N)$
  where the state of the system $S$ maps every atomic process  $p\in \process$ to its current state $S(p)\in Z^p$, the sequence of  waiting events $E$ is an infinite sequence $(e_1, e_2, \dots)$ of events waiting to be delivered, and $N$ is an infinite sequence of nonces  $(n_1, n_2, \dots)$.
\end{definition}

\begin{definition}[Concatenating sequences]
  For a term $a = \an{a_1, \dots, a_i}$ and a sequence $b = (b_1, b_2,
  \dots)$, we define the \emph{concatenation} as $a \cdot b := (a_1,
  \dots, a_i, b_1, b_2, \dots)$.
  
\end{definition}

\begin{definition}[Subtracting from Sequences]
  For a sequence $X$ and a set or sequence $Y$ we define $X \setminus
  Y$ to be the sequence $X$ where for each element in $Y$, a
  non-deterministically chosen occurence of that element in $X$ is
  removed.
\end{definition}

\begin{definition}[Processing Steps]\label{def:processing-step}
  The processing step is the system migrating from configurations ($S, E, N$) to ($S', E', N'$) by processing an event $e \in E$.
  A \emph{processing step of the system $\process$} is of the form
  \[(S,E,N) \xrightarrow[p \rightarrow E_{\text{out}}]{e_\text{in}
    \rightarrow p} (S', E', N')\]
  where
  \begin{enumerate}
  \item $(S,E,N)$ and $(S',E',N')$ are configurations of $\process$,
  \item $e_\text{in} = \an{a, f, m} \in E$ is an event,
  \item $p \in \process$ is a process,
  \item $E_{\text{out}}$ is a sequence (term) of events
  \end{enumerate}
  such that there exists 
  \begin{enumerate}
  \item a sequence (term)
    $E^\nu_{\text{out}} \subseteq 2^{\events^\nu\an{}}$ of protoevents,
  \item a term $s^\nu \in \gterms_{\nonces}(V_{\text{process}})$, 
  \item a sequence $(v_1, v_2, \dots, v_i)$ of all placeholders appearing in $E^\nu_{\text{out}}$ (ordered lexicographically),
  \item a sequence $N^\nu = (\eta_1, \eta_2, \dots, \eta_i)$ of the first $i$ elements in $N$ 
  \end{enumerate}
  with
  \begin{enumerate}
  \item $((e_{\text{in}}, S(p)), (E^\nu_{\text{out}}, s^\nu)) \in R^p$
    and $a \in I^p$,
  \item $E_{\text{out}} = E^\nu_{\text{out}}[m_1/v_1, \dots, m_i/v_i]$
  \item $S'(p) = s^\nu[m_1/v_1, \dots, m_i/v_i]$ and $S'(p') = S(p')$ for all $p' \neq p$
  \item $E' = E_{\text{out}} \cdot (E \setminus \{e_{\text{in}}\})$ 
  \item $N' = N \setminus N^\nu$ 
  \end{enumerate}
\end{definition} 

\begin{definition}[Runs]
  Let $\process$ be a system, $E^0$ be sequence of events, and $N^0$ be
  a sequence of nonces. A \emph{run $\rho$ of a system $\process$
    initiated by $E^0$ with nonces $N^0$} is a finite sequence of
  configurations $((S^0, E^0, N^0),\dots,(S^n, E^n, N^n))$ or an infinite sequence
  of configurations $((S^0, E^0, N^0),\dots)$ such that $S^0(p) = s_0^p$ for
  all $p \in \process$ and $(S^i, E^i, N^i) \xrightarrow{} (S^{i+1},
  E^{i+1}, N^{i+1})$ for all $0 \leq i < n$ (finite run) or for all $i \geq 0$  
  (infinite run).

  We denote the state $S^n(p)$ of a process $p$ at the end of a run $\rho$ by $\rho(p)$.
\end{definition}

\subsubsection{Atomic Dolev-Yao Processes}  

We next define
atomic Dolev-Yao processes, for which we require that the
messages and states that they output can be computed (more
formally, derived) from the current input event and
state. For this purpose, we first define what it means to
derive a message from given messages.




\begin{definition}[Deriving Terms]
  Let $M$ be a set of ground terms. We say that \emph{a
    term $m$ can be derived from $M$ with placeholders $V$} if there
  exist $n\ge 0$, $m_1,\ldots,m_n\in M$, and $\tau\in
  \gterms_{\emptyset}(\{x_1,\ldots,x_n\} \cup V)$ such that $m\equiv
  \tau[m_1/x_1,\ldots,m_n/x_n]$. We denote by $d_V(M)$ the set of all
  messages that can be derived from $M$ with variables $V$.
\end{definition}
For example, $a\in d_{\{\}}(\{\enc{\an{a,b,c}}{\pub(k)}, k\})$.

\begin{definition}[Atomic Dolev-Yao Process] \label{def:adyp} An \emph{atomic Dolev-Yao process
    (or simply, a DY process)} is a tuple $p = (I^p, Z^p,$ $R^p,
  s^p_0)$ such that $(I^p, Z^p, R^p, s^p_0)$ is an atomic process and
  (1) $Z^p \subseteq \gterms_{\nonces}$ (and hence, $s^p_0\in
  \gterms_{\nonces}$), and (2) for all events $e \in \events$,
  sequences of protoevents $E$, $s\in \gterms_{\nonces}$, $s'\in
  \gterms_{\nonces}(V_{\text{process}})$, with $((e, s), (E, s')) \in R^p$ it holds
  true that $E$, $s' \in d_{V_{\text{process}}}(\{e,s\})$.
\end{definition}

\begin{definition}[Atomic Attacker Process]\label{def:atomicattacker}
  An \emph{(atomic) attacker process for a set of sender addresses
    $A\subseteq \addresses$} is an atomic DY process $p = (I, Z, R,
  s_0)$ such that for all events $e$, and $s\in \gterms_{\nonces}$ we
  have that $((e, s), (E,s')) \in R$ iff $s'=\an{e, E, s}$ and
  $E=\an{\an{a_1, f_1, m_1}, \dots, \an{a_n, f_n, m_n}}$ with $n \in
  \mathbb{N}$, $a_1,\dots,a_n\in \addresses$, $f_0,\dots,f_n\in A$,
  $m_1,\dots,m_n\in d_{V_{\text{process}}}(\{e,s\})$.
\end{definition}


\subsubsection{Scripting Processes}

\begin{definition}[Scripting Processes]
 A scripting process is the modelling JavaScript code, as a relation $R$, representing the server-defined functions, while the input and output are handled by the browser. There is, $R\subseteq \terms \times
  \terms(V_{\text{script}})$ such that for all $s \in \terms$, $s' \in \terms(V_{\text{script}})$ with $(s, s') \in R$ it follows that $s'\in d_{V_{\text{script}}}(s)$.
\end{definition}

\begin{definition}[Attacker Script]
  The attacker script $\Rasp$ outputs everything that is derivable
  from the input, i.e., $\Rasp=\{(s, s')\mid s\in \terms, s'\in
  d_{V_{\text{script}}}(s)\}$.
\end{definition}


\subsection{Web Systems Model}
\label{app:websystemmodel}


The architecture of the web and its applications are represented through an model known as a web system. This system encompasses a set of DY processes, which may be infinite in number, to depict various components such as web browsers, servers, and potential adversaries capable of compromising these elements.
It also contains a set of scripting processes, including the honest and malicious processes.

\begin{definition}[Web Systems]
We can formulate the web infrastructure as a web system of form $\mathcal{WS}$=($\mathcal{W}$, $\mathcal{S}$, $\mathtt{script}$, $E^0$), where $\mathcal{W}$ is the set of atomic processes containing both honest and malicious processes, $\mathcal{S}$ is the set of scripting processes including honest and malicious scripts, $\mathtt{script}$ is the set of concrete script codes related to specific scripting processes in $\mathcal{S}$, and $E^0$ is the set of events acceptable to the processes in $\mathcal{W}$.
\end{definition}

In the following sections, we will formulate UPPRESSO system based on the web system model. Based on the UPPRESSO web system model, we will give the formal analysis of security and privacy.  

